\section{Introduzione}
	\subsection{Scopo del documento}
		Questo documento di propone come guida per fli sviluppatori che andranno ad estendere e/o manutenere il prodotto ThiReMa. Nelle sezioni successive del documento sarà possibile trovare una descrizione dei linguaggi e degli strumenti utilizzati durante lo sviluppo del progetto oltre che ad un'analisi dell'architettura scelta per il prodotto.
			\subsection{Glossario e documenti esterni}
		Per evitare possibili ambiguità relative alle terminologie (che andranno indicate in \textsc{maiuscoletto}) utilizzate nei vari documenti, verranno utilizzate due simboli:
		\begin{itemize}
			\item una \textit{D} al pedice per indicare il nome di un particolare documento;
			\item una \textit{G} al pedice per indicare un termine che sarà presente nel \dext{Glossario v2.0.0}.
		\end{itemize}
		\subsection{Riferimenti}
		\subsubsection{Normativi}
			\begin{itemize}
				\item \textbf{norme di progetto: }\dext{Norme di Progetto v3.0.0} 
				\item \textbf{capitolato C6 - ThiReMa: }\url{https://www.math.unipd.it/~tullio/IS-1/2019/Progetto/C6.pdf}
			\end{itemize}
		\subsubsection{Informativi}
			\begin{itemize}
				\item \textbf{Docker: }\url{https://www.docker.com};
				\item \textbf{Maven: }\url{http://maven.apache.org};
				\item \textbf{Node Package manager: }\url{https://www.npmjs.com};
				\item \textbf{Composer: }\url{https://getcomposer.org};
				\item \textbf{Spring: }\url{https://spring.io/projects/spring-framework};
				\item \textbf{JSON Web Token: }\url{https://jwt.io};
				\item \textbf{Laravel: } \url{https://laravel.com};
				\item \textbf{Vue.js: }\url{https://vuejs.org};
				\item \textbf{Axios: }\url{https://axios.nuxtjs.org};
				\item \textbf{API Telegram: }\url{https://core.telegram.org/bots/api};
				\item \textbf{Telegraf: }\url{https://telegraf.js.org}. 

			\end{itemize}