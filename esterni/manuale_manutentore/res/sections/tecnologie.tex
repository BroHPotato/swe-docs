\section{Tecnologie interessate}
	Di seguito vengono descritte le diverse tecnologie che, dopo una prima fase di analisi, sono state scelte per lo sviluppo del prodotto.
	\subsection{Linguaggi}
		In questa sezione vengono trattati i linguaggi utilizzati per l'implementazione di ThiReMa.
		\subsubsection{Java}
			Il linguaggio è stato scelto perché permette di interfacciarsi con facilità con la piattaforma \glock{Kafka} richiesta dal capitolato. \glock{Java} permette inoltre di gestire le dipendenze in maniera automatica tramite Maven e di scrivere ed eseguire suite di test in maniera facile e veloce tramite \glock{JUnit}.
			\newline
			Il linguaggio è inoltre ben conosciuto e facile da apprendere per uno sviluppatore con una qualche esperienza nella programmazione orientata agli oggetti. 
		\subsubsection{PHP}
			Questo linguaggio è stato scelto per lo sviluppo dell'applicazione web perché utilizzabile assieme a framework che permettono la realizzazione di componenti e pagine web in maniera veloce ed intuitiva, utilizzando sempre una sintassi elegante.
		\subsubsection{JavaScript}
			\glock{JavaScript} è stato scelto perché permette la realizzazione della componente \glock{bot Telegram} con un numero limitato di righe di codice, utilizzando dei moduli presenti per il suddetto linguaggio.
			\newline
			Un altro punto a suo favore è che il suo tempo di apprendimento è ragionevolmente breve.
	\subsection{Strumenti}
		In questa sezione vengono trattati gli strumenti utilizzati per l'implementazione di ThiReMa.
		\subsubsection{Docker}
			Questo strumento è stato utilizzato sia perché fortemente consigliato dal capitolato, sia perché permette il rilascio delle varie componenti in ambienti isolati tra loro, detti \glock{Container}, che simulano degli ambienti virtuali dov'è possibile eseguire e testare le proprie applicazioni.
			\newline
			In questo modo è possibile simulare l'esecuzione di più sistemi operativi su una stessa macchina fisica, condividendone le risorse, in modo da ridurre i costi e aumentare l'efficienza generale.
			\newline
			A differenza delle macchine virtuali, i \glock{container} risultano essere più leggeri, in quanto vengono fatti su misura per le applicazioni che devono contenere; in questo modo occupano meno memoria sul disco e impiegano meno risorse hardware.
			\newline
			I container vengono configurati tramite dei \glock{Dockerfile}, dove vengono specificate le operazioni da eseguire all'avvio del container stesso, oltre che informazioni e parametri specifici dell'ambiente, come il sistema operativo  da utilizzare. Infine tramite dei \glock{docker-compose} vengono assemblate le diverse componenti, permettendo in pochi comandi di costruire l'intera architettura.
			\newline
			Per maggiori informazioni si consiglia di visitare il seguente link:
			\newline
			\begin{center}
				\url{https://www.docker.com}
			\end{center}
	\subsection{Framework e librerie}
		Le librerie esterne utilizzate sono state gestite tramite Maven per le componenti sviluppate in \glock{Java} (gateway, data connector ed \glock{API}); per i moduli del bot Telegram (sviluppato in \glock{Javascript}) è richiesto invece il gestore di pacchetti di \glock{Node.js} \glock{npm}. Infine per la componente web è necessario, oltre al già menzionato \textit{npm}, anche il gestore di pacchetti PHP \textit{Composer}.
		\subsubsection{Spring}
			Il framework \glock{Spring} è stato utilizzato per lo sviluppo della componente \glock{API}. Più precisamente dell'intera suite sono stati utilizzati Spring Boot, Spring Security, Spring Kafka e Spring Data JPA.
			\newline
			\begin{itemize}
				\item \textbf{Spring Boot} viene utilizzato per la creazione della componente API in quanto permette di creare un microservizio che è quindi indipendente dalle altre componenti del sistema; è inoltre fornito di ottima documentazione \href{https://spring.io/guides/gs/rest-service/}{per creare un servizio web RESTful};
				\item \textbf{Spring Security} viene utilizzato per gestire la sicurezza e le richieste fatte alle API dalle altre componenti del sistema; è stato scelto perché è lo standard de-facto per gestire la sicurezza di applicazioni basate su Spring.
				Questo framework permette inoltre un'alta personalizzazione dei servizi di autenticazione da fornire, nel nostro caso nella componente API. La documentazione è disponibile al \href{https://spring.io/projects/spring-security#learn}{seguente indirizzo}, dove è possibile trovare anche alcune guide molto utili;
				\item \textbf{Spring Kafka} viene utilizzato per gestire ad alto livello la comunicazione con uno o più broker kafka, permettendo quindi di inserire o reperire dei dati da uno o più topic.
				Nello specifico è stato impiegato per configurare sia per la connessione con il nostro servizio Kafka, che per inviare nuove configurazioni ai gateway, \href{https://docs.spring.io/spring-kafka/docs/2.4.6.RELEASE/reference/html/#kafka-template}{utilizzando un KafkaTemplate che permette di inviare messaggi ad un topic con una sintassi ad alto livello}.
				Una guida utile ad un'eventuale estensione di questo servizio è disponibile \href{https://docs.spring.io/spring-kafka/docs/2.4.6.RELEASE/reference/html/#reference}{al seguente puntatore};
				\item \textbf{Spring Data JPA} viene utilizzato all'interno del pacchetto repositories della componente API per gestire l'accesso al layer contenente i dati. Più nello specifico permette l'implementazione dell'accesso ai dati in maniera semplice, senza richiedere la scrittura di una grande quantità codice boilerplate. Per un'eventuale estensione del servizio o per eventuali chiarimenti si consiglia la seguente \href{https://spring.io/guides/gs/accessing-data-jpa/}{guida}.
			\end{itemize}
			Per maggiori informazioni di carattere generale visitare il seguente link:
			\newline
			\begin{center}
				\url{https://spring.io/projects/spring-framework}
			\end{center}
		\subsubsection{Jwt}
			Questa libreria permette di utilizzare lo standard JWT per trasmettere informazioni tra due componenti in formato \glock{JSON}. Viene utilizzata dalle componenti API per creare e gestire dei token personalizzati che permettono l'accesso (sia normale che a 2FA) e lo scambio di dati alla Web-app o al bot Telegram.
			\newline
			Per maggiori informazioni visitare il seguente link:
			\newline
			\begin{center}
				\url{https://jwt.io}
			\end{center}
		\subsubsection{Laravel}
			Questo framework permette di realizzare applicazioni web utilizzando una sintassi espressiva ed elegante. Nel progetto è stato utilizzato per sviluppare la componente \glock{Web-app}. Laravel è stato scelto anche per l'\href{https://laravel.com/docs/7.x}{ottima documentazione} e le \href{https://laravel.com/docs/7.x/routing}{numerose guide} che sono disponibili nel sito ufficiale.
			\newline
			Per maggiori informazioni di carattere generale visitare il seguente link:
			\newline
			\begin{center}
				\url{https://laravel.com}
			\end{center}
		\subsubsection{Vue.js}
			Questo framework permette di scrivere pagine web reattive e dalla sintassi elegante. Nel progetto è stato utilizzato per sviluppare la componente \glock{Web-app}. Più nello specifico Vue permette di suddividere in componenti una pagina HTML che può quindi variare la sua struttura in maniera dinamica in base al comportamento dell'utente. Un'altra funzionalità fornita da Vue è la possibilità di rendere la pagina reattiva collegando il DOM con i dati che dovranno essere renderizzati dal browser.
			Come i framework precedenti anche Vue è dotato di una \href{https://vuejs.org/v2/api/}{ricca documentazione} e di \href{https://vuejs.org/v2/guide/index.html}{numerose guide} che ne facilitano l'implementazione. 
			\newline
			Per maggiori informazioni di carattere generale visitare il seguente link:
			\newline
			\begin{center}
				\url{https://vuejs.org}
			\end{center}
		\subsubsection{Boostrap}
			Questo framework permette di sviluppare siti web e \glock{web application} responsive. Nel progetto è stato utilizzato nella creazione delle view della nostra Web-app.
			Più nel dettaglio è stata sfruttata la possibilità fornita da Bootstrap di creare tabelle e cards responsive che hanno permesso la gestione del contenuto delle pagine HTML in maniera ottimale.
			\newline
			Per maggiori informazioni, utili per un'eventuale estensione dell'applicazione web, visitare il seguente link:
			\newline
			\begin{center}
				\url{https://getbootstrap.com/docs/4.4/getting-started/introduction/}
			\end{center}	
		\subsubsection{Telegraf}
			Questo modulo JavaScript viene usato all'interno della componente \glock{bot Telegram} per creare ed interfacciarsi con le \href{https://core.telegram.org/bots/api}{API ufficiali di Telegram}.
			Più nello specifico questo modulo è stato impiegato per creare la variabile "bot", grazie alla quale è stato possibile implementare i comandi necessari al corretto funzionamento della componente bot Telegram.
			Telegraf mette inoltre a disposizione vari tipi e sottotipi di dato (quali ad esempio message, callback\_query, text o audio) che permettono una buona gestione delle risposte sulla base degli input dell'utente. Per questo modulo sono disponibili \href{https://telegraf.js.org/#/?id=getting-started}{alcune guide} direttamente sul sito ufficiale, dove è presente anche la \href{https://telegraf.js.org/#/?id=api-reference}{documentazione delle API} sia di \href{https://telegraf.js.org/#/?id=telegraf}{Telegraf} che di \href{https://telegraf.js.org/#/?id=telegram}{Telegram}.
			\newline
			Per maggiori informazioni di carattere generale visitare il link seguente:
			\newline
			\begin{center}
				\url{https://telegraf.js.org}
			\end{center}
	