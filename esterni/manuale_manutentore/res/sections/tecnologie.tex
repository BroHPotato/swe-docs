\section{Tecnologie interessate}
	Di seguito vengono descritte le diverse tecnologie che, dopo una prima fase di analisi, sono state scelte per lo sviluppo del prodotto.
	\subsection{Linguaggi}
		In questa sezione vengono trattati i linguaggi utilizzati per l'implementazione di ThiReMa.
		\subsubsection{Java}
			Il linguaggio è stato scelto perché permette di interfacciarsi con facilità con la piattaforma Kafka richiesta dal capitolato. Java permette inoltre di gestire le dipendenze in maniera automatica tramite Maven e di scrivere ed eseguire suite di test in maniera facile e veloce tramite \glock{JUnit}.
			\newline
			Il linguaggio è inoltre ben conosciuto e facile da apprendere per uno sviluppatore con una qualche esperienza nella programmazione orientata agli oggetti. 
		\subsubsection{PHP}
			Questo linguaggio è stato scelto per lo sviluppo dell'applicazione web perché utilizzabile assieme a framework che permettono la realizzazione di componenti e pagine web in maniera veloce ed intuitiva, utilizzando sempre una sintassi elegante.
		\subsubsection{JavaScript}
			JavaScript è stato scelto perché permette la realizzazione della componente bot Telegram con un numero limitato di righe di codice, utilizzando dei moduli presenti per il suddetto linguaggio.
			\newline
			Un altro punto a suo favore è che il suo tempo di apprendimento è ragionevolmente breve.
	\subsection{Strumenti}
		In questa sezione vengono trattati gli strumenti utilizzati per l'implementazione di ThiReMa.
		\subsubsection{Docker}
			Questo strumento è stato utilizzato sia perché fortemente consigliato dal capitolato, sia perché permette il rilascio delle varie componenti in ambienti isolati tra loro, detti \glock{Container}, che simulano degli ambienti virtuali dov'è possibile eseguire e testare le proprie applicazioni.
			\newline
			In questo modo è possibile simulare l'esecuzione di più sistemi operativi su una stessa macchina fisica, condividendone le risorse, in modo da ridurre i costi e aumentare l'efficienza generale.
			\newline
			A differenza delle macchine virtuali, i container risultano essere più leggeri, in quanto vengono fatti su misura per le applicazioni che devono contenere; in questo modo occupano meno memoria sul disco e impiegano meno risorse hardware.
			\newline
			I container vengono configurati tramite dei Dockerfile, dove vengono specificate le operazioni da eseguire all'avvio del container stesso, oltre che informazioni e parametri specifici dell'ambiente, come il sistema operativo  da utilizzare. Infine tramite dei docker-compose vengono assemblate le diverse componenti, permettendo in pochi comandi di costruire l'intera architettura.
			\newline
			Per maggiori informazioni si consiglia di visitare il seguente link:
			\newline
			\begin{center}
				\url{https://www.docker.com}
			\end{center}
		\subsubsection{PostgreSQL}
			% TODO
		\subsubsection{Timescale}
			% TODO
	\subsection{Framework e librerie}
		Le librerie esterne utilizzate sono state gestite tramite Maven per le componenti sviluppate in Java (gateway, data connector ed api); per i moduli del bot Telegram (sviluppato in Javascript) è richiesto invece il gestore di pacchetti di Node.js \textit{npm}. Infine per la componente web è necessario, oltre al già menzionato \textit{npm}, anche il gestore di pacchetti PHP \textit{Composer}.
		\subsubsection{Spring}
			Il framework Spring è stato utilizzato per lo sviluppo della componente API. Più precisamente dell'intera suite sono stati utilizzati Spring Boot, Spring Security, Spring Kafka e Spring Jpa.
			\newline
			Per maggiori informazioni visitare il seguente link:
			\newline
			\begin{center}
				\url{https://spring.io/projects/spring-framework}
			\end{center}
		\subsubsection{Jwt}
			Questa libreria permette di utilizzare lo standard JWT per trasmettere informazioni tra due componenti in formato \glock{JSON}.
			\newline
			Per maggiori informazioni visitare il seguente link:
			\newline
			\begin{center}
				\url{https://jwt.io}
			\end{center}
		\subsubsection{Laravel}
			Questo framework permette di realizzare applicazioni web utilizzando una sintassi espressiva ed elegante. Nel progetto è stato utilizzato per sviluppare la componente web app.
			\newline
			Per maggiori informazioni visitare il seguente link:
			\newline
			\begin{center}
				\url{https://laravel.com}
			\end{center}
		\subsubsection{Vue.js}
			Questo framework permette di scrivere pagine web reattive e dalla sintassi elegante. Nel progetto è stato utilizzato per sviluppare la componente web app.
			\newline
			Per maggiori informazioni visitare il seguente link:
			\newline
			\begin{center}
				\url{https://vuejs.org}
			\end{center}
		\subsubsection{Boostrap}
			Questo framework permette di sviluppare siti web e web appllication responsive. Nel progetto è stato utilizzato nella creazione delle view della nostra web app.
			\newline
			Per maggiori informazioni visitare il seguente link:
			\newline
			\begin{center}
				\url{https://getbootstrap.com/docs/4.4/getting-started/introduction/}
			\end{center}	
		\subsubsection{Telegraf}
			Questo modulo JavaScript viene usato all'interno della componente bot Telegram per creare ed interfacciarsi con le \href{https://core.telegram.org/bots/api}{API ufficiali di Telegram}.
			\newline
			Per maggiori informazioni visitare il link seguente:
			\newline
			\begin{center}
				\url{https://telegraf.js.org}
			\end{center}
	