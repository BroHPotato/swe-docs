\section{Installazione}
	In questa sezione viene spiegato come effettuare l'installazione delle componenti del progetto RIoT.

	\subsection{Installazione e gestione tramite script BASH}

	L'installazione e la gestione del prodotto è stata semplificata attraverso uno script BASH denominato \textit{mr\_wolf.sh}. Questo script si basa su Docker e permette di gestire alla stessa maniera di un docker-compose tutti i servizi attivi.
	In particolare, è possibile controllare le seguenti operazioni: 
	\begin{itemize}
		\item \textbf{help:}  	comandi di aiuto per lo script \textit{mr\_wolf};
    	\item \textbf{status:}  status dei servizi RIoT;
    	\item \textbf{init:}  	prima installazione e avvio di tutti i servizi RIoT;
    	\item \textbf{start:} 	avvio di tutti i servizi RIoT;
    	\item \textbf{stop:}  	stop di tutti i servizi RIoT;
    	\item \textbf{remove:}	stop e rimozione di tutti i servizi RIoT;
	\end{itemize}
	L'avvio dello script può essere effettuato nei sistemi operativi Linux e MacOs nel seguente modo:
	\begin{verbatim}
	$ chmod +x mr_wolf.sh
	$ ./mr_wolf.sh
	\end{verbatim}
	Notare che il primo comando va effettuato solo la prima volta.
	\newline
	Infine, l'installazione e gestione su Windows è disponibile solamente tramite Docker.

	\subsection{Installazione e avvio completo tramite Docker}
	
	In alternativa, è possibile eseguire l'installazione completa di tutto il prodotto sulla propria macchina tramite Docker. Per farlo, è sufficiente eseguire il file \textit{docker-compose.riot.yml} nel proprio terminale tramite il comando:
	\begin{verbatim}
	$ docker-compose -f docker-compose.riot.yml up -d
	\end{verbatim}

	\subsection{Gateway}
		
		\subsubsection{Avvio tramite Docker}
		Il gateway è stato predisposto con un \textit{Dockerfile} con cui avviare direttamente tramite un Docker container l'applicazione. Per fare ciò è sufficiente eseguire i seguenti comandi:
		\begin{verbatim}
		$ docker build --tag thirema-gateway:1.0 .
		$ docker run --publish 29092:29092 --detach --name thirema-gateway thirema-gateway:1.0
		\end{verbatim}
		Verrà creata una immagine contenente il software avviato.

		\subsubsection{Avvio manuale}
		Per installare localmente la componente gateway, è necessario eseguire nel terminale il maven build lifecycle all'interno della cartella della componente tramite il comando:
		\begin{verbatim}
		$ mvn install
		\end{verbatim}
		ed in seguito avviare il client.

	\subsection{Data collector}

		\subsubsection{Avvio tramite Docker}
		Il data collector è stato predisposto con un \textit{Dockerfile} con cui avviare direttamente tramite un Docker container l'applicazione. Per fare ciò è sufficiente eseguire i seguenti comandi:
		\begin{verbatim}
		$ docker build --tag thirema-dc:1.0 .
		$ docker run --publish 29092:29092 --detach --name thirema-data-collector thirema-dc:1.0
		\end{verbatim}
		Verrà creata una immagine contenente il software avviato.

		\subsubsection{Avvio manuale}
		Per installare localmente la componente data collector è necessario eseguire nel terminale il maven build lifecycle all'interno della cartella della componente tramite il comando:
		\begin{verbatim}
		$ mvn install
		\end{verbatim}
		ed in seguito avviare il client.

	\subsection{API}

		\subsubsection{Avvio tramite Docker}
		Le API sono state predisposte con un \textit{Dockerfile} con cui avviare direttamente tramite un Docker container l'applicazione. Per fare ciò è sufficiente eseguire i seguenti comandi:
		\begin{verbatim}
		$ docker build --tag thirema-api:1.0 .
		$ docker run -p 29092:29092 -p 9999:9999 -p 3000:3000 --detach --name thirema-api thirema-api:1.0
		\end{verbatim}
		Verrà creata una immagine contenente il software avviato.

		\subsubsection{Avvio manuale}
		Per installare localmente la componente API è necessario eseguire nel terminale il maven build lifecycle all'interno della cartella della componente tramite il comando:
		\begin{verbatim}
		mvn install
		\end{verbatim}
		ed in seguito avviare l'eseguibile ApirestApplication.java.

	\subsection{Web application}

		\subsubsection{Avvio tramite Docker}
		La \glock{web app} è stata predisposta con un \textit{Dockerfile} con cui avviare direttamente tramite un Docker container l'applicazione. Per fare ciò è sufficiente eseguire i seguenti comandi:
		\begin{verbatim}
		$ docker build --tag thirema-webapp:1.0 .
		$ docker run -p 80:8000 --detach --name thirema-webapp thirema-webapp:1.0
		\end{verbatim}
		Verrà creata una immagine contenente il software avviato.

		\subsubsection{Avvio manuale}
		Per installare localmente la componente \glock{web application} è necessario eseguire nel terminale i seguenti comandi, a partire dalla cartella della componente:
		\begin{verbatim}
		$ composer install
		$ npm install
		$ npm run dev
		$ php artisan serve
		\end{verbatim}
		ed in seguito collegarsi all'indirizzo specificato nel terminale.

	\subsection{Bot Telegram}

		\subsubsection{Avvio tramite Docker}
		Il \glock{bot telegram} è stata predisposta con un \textit{Dockerfile} con cui avviare direttamente tramite un Docker container l'applicazione. Per fare ciò è sufficiente eseguire i comandi riportati di seguito.

        Verrà creata una immagine contenente il software avviato.

		\begin{verbatim}
		$ docker build --tag thirema-telegram-bot:1.0 .
		$ docker run -p 3000:3000 -d --name thirema-telegram-bot thirema-telegram-bot:1.0
		\end{verbatim}




		\subsubsection{Avvio manuale}
		Per installare ed eseguire localmente il bot Telegram è sufficiente eseguire i seguenti comandi mentre ci si trova all'interno della cartella della componente:
		\begin{verbatim}
		$ npm install
		$ node main
		\end{verbatim}
		
