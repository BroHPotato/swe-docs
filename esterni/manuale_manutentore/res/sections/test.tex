\section{Test}
	In questa sezione viene spiegato come eseguire i test implementati per le componenti del progetto RIoT.
	Per ciascun componente, in base al linguaggio di programmazione con cui è stato sviluppato, sarà necessario eseguire un comando apposito che avvierà tutte le suite dei test integrati.
	A partire dalla cartella con tutte le componenti, sarà opportuno spostarsi sulla \textit{working directory} in cui sono contenuti i sorgenti. 

	\subsection{Gateway}
		
		\begin{itemize}
			\item \textit{ambiente:} Java (maven)
			\item \textit{working directory:} \verb!./gateway/gateway/!
			\item \textit{comando:} \verb!$ mvn verify!
		\end{itemize}
		

	\subsection{Data collector}

		\begin{itemize}
			\item \textit{ambiente:} Java (maven)
			\item \textit{working directory:} \verb!./kafka-db/kafka-data-collector/!
			\item \textit{comando:} \verb!$ mvn verify!
		\end{itemize}

	\subsection{API}

		\begin{itemize}
			\item \textit{ambiente:} Java (maven)
			\item \textit{working directory:} \verb!./api/apirest/!
			\item \textit{comando:} \verb!$ mvn verify!
		\end{itemize}


	\subsection{Web application}

		\begin{itemize}
			\item \textit{ambiente:} PHP (composer), NodeJs (npm)
			\item \textit{working directory:} \verb!./webapp/!
			\item \textit{comando per PHP:} \verb!$ composer install && vendor/bin/phpunit -c phpunit.xml!
			\item \textit{comando per NodeJs:} \verb!$ npm run build --if-present && npm test !
		\end{itemize}

	\subsection{Bot Telegram}

		\begin{itemize}
			\item \textit{ambiente:} NodeJs (npm)
			\item \textit{working directory:} \verb!./telegram-bot/!
			\item \textit{comando:} \verb!$ npm run build --if-present && npm test!
		\end{itemize}
		
