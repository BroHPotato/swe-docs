\appendix
\addcontentsline{toc}{section}{Appendice}

\section{Glossario}
\subsection{A}
\subsubsection{Alert} Segnale generato per comunicare un messaggio.
\subsubsection{API} Le API (acronimo di Application Programming Interface, ovvero Interfaccia di programmazione delle applicazioni) sono set di definizioni e protocolli con i quali vengono realizzati e integrati software applicativi.
\subsection{B}
\subsubsection{Bootstrap}  Bootstrap è una raccolta di strumenti liberi per la creazione di siti e applicazioni per il Web. Essa contiene modelli di progettazione basati su HTML e CSS, sia per la tipografia, che per le varie componenti dell'interfaccia, come moduli, pulsanti e navigazione, così come alcune estensioni opzionali di JavaScript.
\subsubsection{Bot}  I bot sono applicazioni di terze parti sviluppati da programmatori esterni per interagire con gli utenti tramite messaggi, comandi e richieste in linea tramite il servizio di messaggistica Telegram.
\subsection{C}
\subsubsection*{Chrome, Google}
Browser web sviluppato da Google.
\subsubsection*{Container, Docker}
Un container è una singola unità di software che contiene sia il codice che tutte le sue dipendenze in modo tale da poter essere eseguito velocemente in qualsiasi ambiente.
\subsection{D}
\subsubsection{Docker}  Il software Docker è una tecnologia di containerizzazione che consente la creazione e l'utilizzo dei container Linux.
\subsubsection{Dockerfile}  Un \textit{Dockerfile} è un file di configurazione che illustra i passaggi che devono essere realizzati per compilare un'immagine di un sistema operativo come base per eseguire un certo applicativo.
\subsubsection{Docker-compose}  Il docker-compose è un comando che può essere utilizzato per comporre, letteralmente, tutta l'infrastruttura in base ai servizi che si è scelto di integrare. Fa uso dei \textit{Dockerfile} per avviare l'eventuale build delle \textit{immagini Docker} e permette con un solo comando di attivare (\verb!docker-compose up -d!) o disattivare (\verb!docker-compose down!) tutti i servizi. L'intera configurazione è salvata su un file denominato per convenzione \verb!docker-compose.yml!.
\subsection{E}
\subsubsection*{Edge, Microsoft}
Browser web sviluppato da Microsoft.
\subsubsection*{Ente}
Con ente si intende una azienda o associazione a cui viene assegnato il monitoraggio di alcuni dispositivi.
\subsection{F}
\subsubsection*{Firefox, Mozilla}
Browser web libero e multipiattaforma mantenuto da Mozilla Foundation.
\subsubsection{Framework}  Un framework, è un'architettura logica di supporto (spesso un'implementazione logica di un particolare design pattern) su cui un software può essere progettato e realizzato.
\subsection{I}
\subsubsection{Immagine, Docker} Un'immagine Docker è un eseguibile che può essere istanziato in un container. L'immagine Docker viene prodotta a seguito della compilazione tramite il comando \verb!docker build! di un \textit{Dockerfile} o di un \textit{docker-compose}.
\subsection{J}
\subsubsection{Java}  Java è un linguaggio di programmazione ad alto livello, orientato agli oggetti e a tipizzazione statica, che si appoggia sull'omonima piattaforma software di esecuzione, specificamente progettato per essere il più possibile indipendente dalla piattaforma hardware di esecuzione tramite l'utilizzo di macchina virtuale.
\subsubsection{JavaScript}  JavaScript è un linguaggio di scripting orientato agli oggetti e agli eventi, comunemente utilizzato nella programmazione web lato client per la creazione, in siti web e applicazioni web, di effetti dinamici interattivi tramite funzioni di script invocate da eventi innescati a loro volta in vari modi dall'utente sulla pagina web in uso.
\subsubsection*{JSON}
Formato utilizzato per lo scambio di dati all'interno della componenti dell'applicazione.
\subsubsection*{JUnit}
Framework di unit testing per il linguaggio di programmazione Java.
\subsection{K}
\subsubsection{Kafka, Apache}  Apache Kafka è una piattaforma open source di stream processing scritta in Java e Scala e sviluppata dall'Apache Software Foundation.  Questo progetto viene usato principalmente per tutte le applicazioni di elaborazioni di stream di dati in tempo reale.
\subsection{L}
\subsubsection{Laravel}  Laravel è un framework open-source molto potente e usato per sviluppare applicazioni in PHP, facendo uso nativamente di Bootstrap e Vue.js. Permette di realizzare la parte back-end di un sito web seguendo il modello model-view-controller. 
\subsection{N}
\subsubsection{Node Package Manager (NPM)}  NPM (abbreviazione di Node Package Manager) è un gestore di pacchetti per il linguaggio di programmazione JavaScript. È il gestore di pacchetti predefinito per l'ambiente di runtime JavaScript Node.js.
\subsubsection{Node.js}  Node.js è una runtime di JavaScript Open source multipiattaforma orientato agli eventi per l'esecuzione di codice JavaScript.
\subsection{P}
\subsubsection{PostgreSQL}  PostgreSQL è un completo DBMS ad oggetti rilasciato con licenza libera.
\subsection{S}
\subsubsection*{Safari}
Browser web sviluppato da Apple Inc.
\subsubsection{Spring, framework}  Spring è un framework Java open-source che viene utilizzato per lo sviluppo di applicativi web e di servizi API.
\subsection{T}
\subsubsection{Telegram}  Telegram è un servizio di messaggistica istantanea e broadcasting basato su cloud.
\subsubsection{TimeScaleDB}  TimeScaleDB è un database open-source costruito per analizzare dati di serie storiche.
\subsubsection{Timeseries DB}  Con il termine timeseries DB si intende un database il cui scopo è la memorizzazione di serie storiche di dati.
\subsection{V}
\subsubsection{Vue.js}  Vue.js è un framework open-source di Javascript che permette di realizzare la parte front-end dinamica in unione con Laravel. Il framework è molto flessibile e altamente scalabile, nonché compatibile anche con Bootstrap.
\subsection{W}
\subsubsection*{Web application}
Con web app o applicazione web, si intende una applicazione fruibile via web per mezzo di un browser.