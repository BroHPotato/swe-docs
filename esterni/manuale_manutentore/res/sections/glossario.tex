\appendix
\addcontentsline{toc}{section}{Appendice}

\section{Glossario}
\subsection{C}
\subsubsection*{Chrome, Google}
Browser web sviluppato da Google.
\subsubsection*{Container}
Un container è una singola unità di software che contiene sia il codice che tutte le sue dipendenze in modo tale da poter essere eseguito velocemente in qualsiasi ambiente.
\subsection{E}
\subsubsection*{Edge, Microsoft}
Browser web sviluppato da Microsoft.
\subsubsection*{Ente}
Con ente si intende una azienda o associazione a cui viene assegnato il monitoraggio di alcuni dispositivi.
\subsection{F}
\subsubsection*{Firefox, Mozilla}
Browser web libero e multipiattaforma mantenuto da Mozilla Foundation.
\subsection{J}
\subsubsection*{JSON}
Formato utilizzato per lo scambio di dati all'interno della componenti dell'applicazione.
\subsubsection*{JUnit}
Framework di unit testing per il linguaggio di programmazione Java.
\subsection{S}
\subsubsection*{Safari}
Browser web sviluppato da Apple Inc.. 
\subsection{W}
\subsubsection*{Web Application}
Con webapp o applicazione web, si intende una applicazione fruibile via web per mezzo di un browser.