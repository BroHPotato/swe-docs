\section{Test}
	Per la classificazione dei test si fa riferimento alle sezioni Verifica e Validazione delle \dext{Norme di Progetto}.
	

	\subsection{Tipologie di test:}
		I test saranno di quattro tipologie differenti:
		\begin{itemize}

			\item \textbf{Test di Accettazione [TA]}
			\item \textbf{Test di Sistema [TS]}
			\item \textbf{Test di Integrazione [TI]}
			\item \textbf{Test di Unità [TU]}

		\end{itemize}

	\subsection{Test di Accettazione}
		
	\subsection{Test di Sistema}
		\begin{center}
		Tabella 1 - Riepilogo Test di Sistema
			\rowcolors{2}{lightest-grayest}{white}
			\begin{longtable}{|c|p{10cm}|c|}
			\hline
			\rowcolor{lighter-grayer}
			\textbf{Codice} & \textbf{Descrizione} & \textbf{Stato}  \\ %& \textbf{Risultato}

			\hline
			\endhead

			% ----- Modificare da qui -----
			\hline
			%Esempio
			 TSA-F-1 & Si verifichi che un utente con un account valido possa accedere alle sezioni private del sito inserendo la propria mail e password.
			  & NI \\
			 \hline
			 TSA-F-1.1 & Si verifichi che un utente senza un account valido non possa accedere ad alcuna sezione privata del sito. & NI \\
			 \hline
			 TSA-F-1.2 & Si verifichi che un utente con un account valido che ha effettuato l'accesso al sito, possa uscire dalla sessione mediante l'apposito bottone di logout. & NI \\
			 \hline
			 TSA-F-1.3 & Si verifichi che un utente che sia possesso di un account valido con attivata l'opzione di autenticazione a due fattori possa accedere alle sezioni private del sito inserendo la propria mail, password ed il codice per l'autenticazione a due fattori. & NI \\
			 \hline
			 TSA-F-1.3.1 & Si verifichi che un utente al quale in fase di login viene richiesto un codice di autenticazione lo riceva tramite Telegram. & NI \\
			 \hline
			 TSA-F-1.3.2 & Si verifichi che un utente al quale viene chiesto di inserire un codice di autenticazione a due fattori possa richiederne il reinvio tramite un bottone & NI \\
			 \hline
			 TSA-F-2 & Si verifichi che un utente autenticato abbia accesso alle sezioni private del sito in base ai suoi permessi e visualizzare di conseguenza un menù per navigare & NI \\
			 \hline
			 TSA-F-2.1 & Si verifichi che un utente autenticato possa accedere ad una dashboard che mostra:
			 \begin{itemize}
			 	\item Le statistiche generali del suo ente;
			 	\item Gli ultimi avvisi impostati per il suo ente;
			 	\item I principali contatti di supporto tecnico.
			 \end{itemize} & NI \\
			 \hline
			 TSA-F-2.2 & Si verifichi che un moderatore ente possa accedere ad una dashboard che mostra:
			 \begin{itemize}
			 	\item Le statistiche generali del suo ente;
			 	\item Gli ultimi avvisi del suo ente;
			 	\item I principali contatti di supporto tecnico.
			 \end{itemize} & NI \\
			 \hline
			 TSA-F-2.3 & Si verifichi che un amministratore tecnico abbia accesso ad una dashboard che mostra le statistiche generali del sito & NI \\
			 \hline
			 TSA-F-2.4 & Si verifichi che un utente autorizzato o un moderatore ente possano visualizzare la lista dei dispositivi e dei relativi sensori autorizzati per il proprio ente. & NI \\
			 \hline
			 TSA-F-2.5 & Si verifichi che un amministratore possa visualizzare la lista completa dei dispositivi e dei relativi sensori. & NI \\
			 \hline
			 TSA-F-2.6 & Si verifichi che un qualsiasi utente autenticato possa accedere alla sezione impostazioni del proprio account. & NI \\
			 \hline
			 TSA-F-2.7 & Si verifichi che un qualsiasi utente autenticato possa modificare le informazioni del proprio account tramite la sezione impostazioni. & NI \\
			 \hline
			 TSA-F-2.8 & Si verifichi che un utente autorizzato o un moderatore possono visualizzare le informazioni di uno dei dispositivi autorizzati per il proprio ente. & NI \\
			 \hline
			 TSB-F-2.8.1 & Si verifichi che un utente autorizzato o un moderatore ente possano visualizzare sotto forma di grafici le informazioni sui sensori a loro autorizzati. & NI \\
			 \hline
			 TSA-F-2.9 & Si verifichi che un amministratore possa visualizzare la lista completa delle informazioni di qualunque dispositivo censito. & NI \\
			 \hline
			 TSA-F-2.9.1 & Si verifichi che un amministratore possa assegnare un sensore di un qualisasi dispositivo censito ad un ente. & NI \\
			 \hline
			 TSA-F-2.9.2 & Si verifichi che un amministratore possa revocare un sensore di un qualunque dispositivo censito ad un ente. & NI \\
			 \hline
			 TSB-F-2.9.3 & Si verifichi che un amministratore possa visualizzare tramite grafici le informazioni sui sensori. & NI \\
			 \hline
			 TSA-F-2.10 &  Si verifichi che un moderatore ente possa visualizzare i membri appartenenti al proprio ente. & NI \\
			 \hline
			 TSA-F-2.10.1 & Si verifichi che un moderatore ente possa visualizzare le informazioni di un membro appartenente al suo ente. & NI \\
			 \hline
			 TSA-F-2.10.2 & Si verifichi che un moderatore ente possa resettare la password di un utente autorizzato appartenente al proprio ente. & NI \\
			 \hline
			 TSA-F-2.10.3 & Si verifichi che un moderatore ente possa modificare le informazioni principali di un utente autorizzato appartenente al suo ente. & NI \\
			 \hline
			 TSA-F-2.10.4 & Si verifichi che un moderatore ente possa rimuovere un utente autorizzato appartenente al suo ente. & NI \\
			 \hline 
			 TSA-2.10.5 & Si verifichi che un moderatore ente possa creare un nuovo account per un utente autorizzato che apparterrà al suo ente. & NI \\
			 \hline
			 TSA-F-2.11 & Si verifichi che un amministratore possa visualizzare la lista completa con tutti gli utenti. & NI \\
			 \hline
			 TSA-F-2.11.1 & Si verifichi che un amministratore possa visualizzare le informazioni di uno qualsiasi degli utenti. & NI \\
			 \hline
			 TSA-F-2.11.2 & Si verifichi che un amministratore possa disattivare un account di un utente autorizzato o di un moderatore ente. & NI \\
			 \hline
			 TSA-F-2.11.3 & Si verifichi che un amministratore possa modificare le informazioni di un utente autorizzato o di un moderatore ente. & NI \\
			 \hline
			 TSA-F-2.11.4 & Si verifichi che un amministratore possa resettare la password di un utente autorizzato o di un moderatore ente. & NI \\
			 \hline
			 TSA-F-2.11.5 & Si verifichi che un amministratore possa riattivare un utente non autorizzato. & NI \\
			 \hline
			 TSA-F-2.11 & Si verifichi che un amministratore possa assegnare un utente autorizzato o un moderatore ente ad un ente. & NI \\
			 \hline
			 TSA-F-2.11 & Si verifichi che un amministratore possa creare un account per un utente autorizzato o un moderatore ente. & NI \\  
			 \hline
			 TSA-F-2.12 & Si verifichi che un moderatore ente possa visualizzare la lista degli alert attivi per il proprio ente. & NI \\
			 \hline
			 TSA-F-2.12.1 & Si verifichi che un moderatore ente possa aggiungere un alert di un particolare sensore per il suo ente. & NI \\
			 \hline
			 TSA-F-2.12.2 & Si verifichi che un moderatore ente possa rimuovere uno degli alert attivi per il proprio ente. & NI \\
			 \hline
			 TSA-F-2.13 & Si verifichi che un amministratore possa visualizzare la lista degli alert attivi per tutti gli enti. & NI \\
			 \hline
			 TSA-F-2.13.1 & Si verifichi che un amministratore possa rimuovere un alert di un particolare sensore. & NI \\
			 \hline
			 TSA-F-2.14 & Si verifichi che un utente autorizzato o un moderatore ente possano ricevere notifiche sulla base degli alert impostati per il loro ente al raggiungimento dei valori soglia. & NI \\
			 \hline
			 TSB-F-3.1 & Si verifichi che un utente autorizzato o un moderatore ente abbiano a disposizione una sezione view dove possono creare delle proprie pagine personalizzate. & NI \\
			 \hline
			 TSB-F-3.2 & Si verifichi che un utente autorizzato o un moderatore ente possano rimuovere le proprie pagine personalizzate dalla view. & NI \\
			 \hline
			 TSB-F-3.3 & Si verifichi che un utente autorizzato o un moderatore ente abbiano a disposizione una sezione view dove possono aggiungere grafici in una pagina personalizzata della view. & NI \\
			 \hline
			 TSB-F-3.3.1 & Si verifichi che un utente autorizzato o un moderatore ente abbiano a disposizione una sezione view dove possono personalizzare un grafico vuoto per poter visualizzare diversi tipi di correlazione tra i dati. & NI \\
			 \hline
			 TSB-F-3.4 & Si verifichi che un utente autorizzato o un moderatore ente possano eliminare un grafico da una delle pagine personalizzate della view. & NI \\
			 \hline
			 TSB-F-3.5 & Si verifichi che un utente autorizzato o un moderatore ente possano spostare un grafico all'interno di una delle pagine personalizzate della view. & NI \\
			 \hline
			 TSB-F-4.1 & Si verifichi che il moderatore ente possa visualizzare la lista dei logs degli utenti autorizzati del suo ente. & NI \\
			 \hline
			 TSB-F-4.2 & Si verifichi che l'amministratore possa visualizzare la lista dei logs deigli utenti autorizzati e dei moderatori ente. & NI \\
			 \hline
			 TSA-F-5 & Si verifichi che l'utente autorizzato e il moderatore ente possano inviare comandi ai singoli dispositivi autorizzati per il loro ente. & NI \\
			 \hline
			 TSA-F-5.1 & Si verifichi che l'utente autorizzato e il moderatore ente possano visualizzare la lista dei dispositivi autorizzati all'invio dei comandi. & NI \\
			 \hline
			 TSB-F-5.2 & Si verifichi che l'invio dei comandi ai dispositivi avvenga tramite un bot Telegram. & NI \\
			 \hline
			 TSA-F-6.1 & Si verifichi che qualora si ottenga una nuova configurazione del gateway, il sistema esegua la rimozione automatica degli alert attivi di un dispositivo non più presente nella configurazione. & NI \\
			 \hline
			 TSA-F-6.2 & Si verifichi che qualora si ottenga una nuova configurazione del gateway, il sistema esegua la rimozione automatica dei sensori autorizzati agli enti che non sono più esistenti. & NI \\
			 \hline
			 TSA-F-6 & Si verifichi che qualora si ottenga una nuova configurazione del gateway, il sistema esegua il censimento automatico dei nuovi dispositivi. & NI \\
			 \hline
			 TSA-F-6 & Si verifichi che qualora si ottenga una nuova configurazione del gateway, il sistema esegua la rimozione automatica dei dispositivi non più esistenti. & NI \\
			 \hline
			 TSB-Q-9 & Si verifichi che la web app superi la validazione W3C & NI \\
			 \hline
			 TSB-Q-10 & Si verifichi che la web app sia stata sviluppata utilizzando il framework \glock{Bootstrap} & NI \\
			 \hline
			 TSA-V-1 & Si verifichi che la web app sia accessibile tramite i browser internet più aggiornati, quali Chrome, Edge o Firefox) & NI \\
			 \hline
			 TSA-V-2 & Si verifichi che la web app permetta l'accesso a tutte le sue funzionalità nella modalità desktop e tablet. & NI \\ 
			 \hline
			 TSA-V-2.1 & Si verifichi che la web app permetta le funzionalità di compilazione moduli da browser in modalità mobile. & NI \\
			 \hline
			 TSA-V-3 & Si verifichi che le istanze del sistema siano gestite tramite \glock{Docker} & NI \\
			 \hline 
			 TSA-V-4 & Si verifichi che la ricezione degli avvisi avvenga tramite un bot Telegram. & NI \\
			 \hline
			 TSA-V-5 & Si verifichi che un utente autorizzato o un moderatore utente possa inviare comandi ai dispositivi autorizzati tramite la web app o il bot Telegram. & NI \\
			 \hline
			 TSA-V-6 & Si verifichi che il sistema faccia uso dell'ecosistema \glock{Kafka}. & NI \\
			 \hline
			 TSA-V-7 & Si verifichi che il sistema faccia uso di un time-series database per la memorizzazione dei dati dei sensori. & NI \\
			 \hline 

			\end{longtable}
		\end{center}


	\subsection{Test di Integrazione}
		Le specifiche di questi test verranno scritte successivamente rispettando il \glock{modello a V}. 

	\subsection{Test di Unità}
	 	Le specifiche di questi test verranno scritte successivamente rispettando il \glock{modello a V}. 
