\section{Test}
	Per garantire la qualità del prodotto, il gruppo RedRoundRobin ha deciso di seguire il modello a V per gestire la verifica e la validazione nello sviluppo. Secondo questo modello le specifiche dei test verranno create durante l'analisi dei requisiti e la progettazione.
	Per elencare le specifiche dei test si è scelta una rappresentazione tabellare contenente il codice del componente da testare, la descrizione dei test ed infine il suo stato di avanzamento.
	Si è deciso inoltre di dare assegnare una sigla ad ogni test per decretarne lo stato di avanzamento:
	\begin{itemize}
		\item \textbf{I}: se il test è stato implementato;
		\item \textbf{NI}: se il test non è ancora stato implementato.
	\end{itemize}

	\subsection{Tipologie di test:}
		I test saranno di quattro tipologie differenti:
		\begin{itemize}
			
			\item \textbf{Test di Accettazione [TA]}: i test di accettazione (o collaudo) accertano il soddisfacimento dei requisiti dell'utente;
			\item \textbf{Test di Sistema [TS]}: essi verificano il comportamento dell'intero sistema, verificando che i requisiti software siano stati soddisfatti;
			\item \textbf{Test di Integrazione [TI]}: questi test verificano se sono rispettati i contratti di interfaccia tra più moduli o sub-system;
			\item \textbf{Test di Unità [TU]}: questi test sono del codice, prodotto dallo sviluppatore, che esercitano una funzionalità atomica del programma.

		\end{itemize}

	\subsection{Test di Accettazione}
		Il test di accettazione serve a confermare che i requisiti derivati dai casi d'uso specificati nel capitolato siano stati soddisfatti. Questo test richiede perciò la presenza del committente e del proponente.  
		Per classificare questo tipo di test verrà associata un codice ad ognuno di essi secondo il seguente modello:
		
		\begin{center}
		\textbf{TS[Priorità]-[Tipologia]-[Identificativo]}
		\end{center}
		dove: 
		
		\textbf{Priorità}: indica la priorità del requisito associato al test e potrà avere i seguenti valori:
		\begin{itemize}
		 	\item \textbf{A}: Obbligatorio, strattamente necessario;
		 	\item \textbf{B}: Desiderabile, non strettamente necessario;
		 	\item \textbf{C}: Relativamente utile o contrattabile in corso d'opera. 
		 \end{itemize} 
		 \textbf{Tipologia}: indica la tipologia del requisito associato al test e potrà avere i seguenti valori:
		 \begin{itemize}
		 	\item \textbf{F}: funzionale;
		 	\item \textbf{P}: prestazionale;
		 	\item \textbf{Q}: qualitativo;
		 	\item \textbf{V}: vincolo.
		 \end{itemize}
		\textbf{Identificativo}: numero progressivo il cui obiettivo sarà di contraddistinguere il singolo componente da testare.

		\begin{center}
		Tabella x - Riepilogo Test di Accettazione
			\rowcolors{2}{lightest-grayest}{white}
			\begin{longtable}{|c|p{8cm}|c|}
			\hline
			\rowcolor{lighter-grayer}
			\textbf{Codice} & \textbf{Descrizione} & \textbf{Stato} \\
			\hline
			\endfirsthead

			% ----- Modificare da qui -----
			\hline
			%Esempio
			% TAA-F-1 & L'utente deve poter accedere alla gestione del profilo & NI \\
			\hline

			\end{longtable}
		\end{center}

