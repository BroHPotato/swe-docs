\section{Test}
	Per garantire la qualità del prodotto, il gruppo RedRoundRobin ha deciso di seguire il modello a V per gestire la verifica e la validazione nello sviluppo. Secondo questo modello le specifiche dei test verranno create durante l'analisi dei requisiti e la progettazione.
	Per elencare le specifiche dei test si è scelta una rappresentazione tabellare contenente il codice del componente da testare, la descrizione dei test ed infine il suo stato di avanzamento.
	Si è deciso inoltre di dare assegnare una sigla ad ogni test per decretarne lo stato di avanzamento:
	\begin{itemize}
		\item \textbf{I}: se il test è stato implementato;
		\item \textbf{NI}: se il test non è ancora stato implementato.
	\end{itemize}

	\subsection{Tipologie di test:}
		I test saranno di quattro tipologie differenti:
		\begin{itemize}
			
			\item \textbf{Test di Accettazione [TA]}: i test di accettazione (o collaudo) accertano il soddisfacimento dei requisiti dell'utente;
			\item \textbf{Test di Sistema [TS]}: essi verificano il comportamento dell'intero sistema, verificando che i requisiti software siano stati soddisfatti;
			\item \textbf{Test di Integrazione [TI]}: questi test verificano se sono rispettati i contratti di interfaccia tra più moduli o sub-system;
			\item \textbf{Test di Unità [TU]}: questi test sono del codice, prodotto dallo sviluppatore, che esercitano una funzionalità atomica del programma.

		\end{itemize}

	\subsection{Test di Accettazione}
		Il test di accettazione serve a confermare che i requisiti derivati dai casi d'uso specificati nel capitolato siano stati soddisfatti. Questo test richiede perciò la presenza del committente e del proponente.  
		Per classificare questo tipo di test verrà associata un codice ad ognuno di essi secondo il seguente modello:
		
		\begin{center}
		\textbf{TS[Priorità]-[Tipologia]-[Identificativo]}
		\end{center}
		dove: 
		
		\textbf{Priorità}: indica la priorità del requisito associato al test e potrà avere i seguenti valori:
		\begin{itemize}
		 	\item \textbf{A}: Obbligatorio, strattamente necessario;
		 	\item \textbf{B}: Desiderabile, non strettamente necessario;
		 	\item \textbf{C}: Relativamente utile o contrattabile in corso d'opera. 
		 \end{itemize} 
		 \textbf{Tipologia}: indica la tipologia del requisito associato al test e potrà avere i seguenti valori:
		 \begin{itemize}
		 	\item \textbf{F}: funzionale;
		 	\item \textbf{P}: prestazionale;
		 	\item \textbf{Q}: qualitativo;
		 	\item \textbf{V}: vincolo.
		 \end{itemize}
		\textbf{Identificativo}: numero progressivo il cui obiettivo sarà di contraddistinguere il singolo componente da testare.

		\begin{center}
		Tabella 1 - Riepilogo Test di Accettazione
			\rowcolors{2}{lightest-grayest}{white}
			\begin{longtable}{|c|p{10cm}|c|}
			\hline
			\rowcolor{lighter-grayer}
			\textbf{Codice} & \textbf{Descrizione} & \textbf{Stato} \\
			\hline
			\endfirsthead

			% ----- Modificare da qui -----
			\hline
			%Esempio
			 TSA-F-1 & L'utente con un account valido deve poter accedere alle sezioni private del sito inserendo la propria mail e password.
			  & NI \\
			 \hline
			 TSA-F-1.1 & L'utente senza un account valido non deve opter accedere ad alcuna sezione prvata del sito. & NI \\
			 \hline
			 TSA-F-1.2 & L'utente con un account valido e che ha effettuato l'accesso al sito, deve poter uscire dalla sessione mediante l'apposito bottone di logout. & NI \\
			 \hline
			 TSA-F-1.3 & L'utente con un account valido e che ha attivato l'opzione di autenticazione a due fattori deve poter accedere alle sezioni private del sito inserendo la propria mail, password ed il codice per l'autenticazione a due fattori ricevuto tramite il Telegram. & NI \\
			 \hline
			 TSA-F-2 & L'utente autenticato deve avere accesso alle sezioni private del sito in base ai suoi permessi e visualizzare di conseguenza un menù per navigare & NI \\
			 \hline
			 TSA-F-2.1 & L'utente autenticato deve poter accedere ad una dashboard che mostra:
			 \begin{itemize}
			 	\item Le statistiche generali del suo ente;
			 	\item Gli ultimi avvisi impostati per il suo ente;
			 	\item I principali contatti di supporto tecnico.
			 \end{itemize} & NI \\
			 \hline
			 TSA-F-2.2 & Il moderatore ente deve poter accedere ad una dashboard che mostra:
			 \begin{itemize}
			 	\item Le statistiche generali del suo ente;
			 	\item Gli ultimi avvisi del suo ente;
			 	\item I principali contatti di supporto tecnico.
			 \end{itemize} & NI \\
			 \hline
			 TSA-F-2.3 & L'amministratore tecnico deve aver accesso ad una dashboard che mostra le statistiche generali del sito & NI \\
			 \hline
			 TSA-F-2.4 & L'utente autorizzato e il moderatore ente devono poter:
			 \begin{itemize}
			 	\item Visualizzare la lista dei dispositivi e dei relativi sensori autorizzati per il proprio ente;
			 	\item Visualizzare le informazioni di uno dei dispositivi autorizzati per il proprio ente;
			 	\item Visualizzare le impostazioni per il proprio account;
			 	\item Visualizzare le informazioni sui sensori a loro autorizzati in forma tabellare o di grafico.
			 \end{itemize} & NI \\
			 \hline
			 TSA-F-2.5 & L'amministratore deve poter:
			 \begin{itemize}
			 	\item Visualizzare la lista completa dei dispositivi e dei relativi sensori;
			 	\item Visualizzare la lista completa delle informazioni di qualunque dispositivo censito;
			 	\item Assegnare un sensore di un qualisasi dispositivo censito ad un ente;
			 	\item Revocare un sensore di un qualunque dispositivo censito ad un ente;
			 	\item Visualizzare tramite grafici le informazioni sui sensori.
			 \end{itemize} & NI \\
			 \hline
			 TSA-F-2.6 & Un qualsiasi utente autentinticato deve poter:
			 \begin{itemize}
			 	\item Accedere alla sezione impostazioni del proprio account;
			 	\item Modificare le proprie impostazioni account.
			 \end{itemize} & NI \\
			 \hline
			 TSA-F-2.10 & Un moderatore ente deve poter:
			 \begin{itemize}
			 	\item Visualizzare i membri appartenenti al proprio ente;
			 	\item Resettare la password di un utente autorizzato appartenente al proprio ente;
			 	\item Modificare le informazioni principali di un utente autorizzato appartenente al suo ente;
			 	\item Rimuovere un utente autorizzato appartenente al suo ente;
			 	\item Creare un nuovo account per un utente autorizzato che apparterrà al suo ente.
			 \end{itemize} & NI \\
			 \hline
			 TSA-F-2.11 & Un amministratore può: 
			 \begin{itemize}
			 	\item Visualizzare la slita completa con tutti gli utenti;
			 	\item Visualizzare le informazioni di uno qualsiasi degli utenti;
			 	\item Disattivare un account di un utente autorizzato o di un moderatore ente;
			 	\item Modificare le informazioni di un utente autorizzato o di un moderatore ente;
			 	\item Resettare la password di un utente autorizzato o di un moderatore ente;
			 	\item Riattivare un utente non autorizzato;
			 	\item Assegnare un utente autorizzato o un moderatore ente ad un ente;
			 	\item Creare un account per un utente autorizzato o un moderatore ente.  
			 \end{itemize} & NI \\
			 \hline
			 TSA-F-2.12 & Un moderatore ente può: 
			 \begin{itemize}
			 	\item Visualizzare la lista degli alert attivi per il proprio ente;
			 	\item Aggiungere un alert di un particolare sensore per il suo ente;
			 	\item Rimuovere uno degli alert attivi per il proprio ente.

			 \end{itemize} & NI \\
			 \hline
			 TSA-F-2.13 & Un amministratore può:
			  \begin{itemize}
			  	\item Visualizzare la lista degli alert attivi per tutti gli enti;
			  	\item Rimuovere un alert di un particolare sensore.
			  \end{itemize} & NI \\
			  \hline
			  TSA-F-2.14 & L'utente autorizzato e il moderatore ente devono poter ricevere notifiche sulla base degli alert impostati per il loro ente al raggiungimento dei valori soglia. & NI \\
			  \hline
			  TSB-F-3 & L'utente autorizzato devono avere a disposizione una sezione view dove:
			  \begin{itemize}
			  	\item Possono creare delle proprie pagine personalizzate;
			  	\item Possono rimuovere le proprie pagine personalizzate;
			  	\item Possono aggiungere grafici in una pagina personalizzata;
			  	\item Personalizzare il grafico (al momnto della sua creazione) per poter visualizzare diversi tipi di correlazione tra i dati;
			  	\item Eliminare un grafico da una delle pagine personalizzate;
			  	\item Spostare un grafico in una delle pagine personalizzate. 
			  \end{itemize} & NI \\
			  \hline
			  TSB-F-4.1 & Il moderatore ente deve poter visualizzre la lista dei logs degli utenti autorizzati del suo ente. & NI \\
			  \hline
			  TSB-F-4.2 & L'amministratore deve poter visualizzare la lista dei logs deigli utenti autorizati e dei moderatori ente; & NI \\
			  \hline
			  TSA-F-5 & L'utente autorizzato e il moderatore ente devono poter:
			  \begin{itemize}
			  	\item Inviare comandi ai singoli dispositivi autorizzati per il loro ente;
			  	\item Visualizzare la lista dei dispositivi autorizzati all'invio dei comandi.
			  \end{itemize} & NI \\
			  \hline
			  TSB-F-5.2 & L'invio dei comandi ai dispositivi deve avvenire tramite un bot Telegram. & NI \\
			  \hline
			  TSA-F-6 & Qualora si ottenga una nuova configurazione del gateway, il sistema deve eseguire le seguenti operazione automatiche di sincronizzazione delle nuove impostazioni:
			  \begin{itemize}
			  	\item Rimozione automatica degli alert attivi di un dispositivo non più presente nella configurazione;
			  	\item Rimozione automatica dei sensori autorizzati agli enti che non sono più esistenti;
			  	\item Censimento automatico dei nuovi dispositivi;
			  	\item Rimozione automatica dei dispositivi non più esistenti.
			  \end{itemize} & NI \\
			  \hline
			\end{longtable}
		\end{center}

