\section{Valutazioni di miglioramento}
In questa sezione il focus è sul miglioramento della produttività dei \glock{processi} coinvolti nella realizzazione del prodotto descritto nel capitolato scelto. Essendo il primo progetto realistico affrontato dai membri del gruppo, problemi di natura organizzativa interna, di adempimento efficace dei ruoli assegnati e di giusto utilizzo degli strumenti scelti sono dietro l'angolo. Per far fronte a queste situazioni e cercare di migliorare continuativamente la produttività del gruppo, di seguito verranno elencati i problemi più grandi rilevati e le relative contromisure, man mano che saranno identificati nel corso della realizzazione del prodotto.

	\subsection{Valutazioni sull'organizzazione}
	\begin{center}
	\rowcolors{2}{lightest-grayest}{white}
	\begin{longtable}{|p{5cm}|p{5cm}|}
	\hline
	\rowcolor{lighter-grayer}
	\textbf{Problema rilevato} & \textbf{Contromisura}\\
	\hline
	\endfirsthead

	% ----- Modificare da qui -----

	% 0.0.2 & Revisione documento & 24-11-2019 & Nome Cognome & Verificatore \\
	
		\hline
	Nei primi mesi di lavoro è stato difficile trovare aule libere che permettessero ai membri di ritrovarsi per discutere delle cose fatte e quelle da fare.
	  & 
 Nel caso in cui ci siano periodi in cui sia difficile avere a disposizione aule libere dell'università per confrontarsi, è il caso di sfruttare \glock{slack} e/o \glock{discord} per avere una comunicazione testuale o vocale se necessario tra alcuni o tutti i membri. \\
	\hline
	
			\end{longtable}
	\end{center}
	
	\subsection{Valutazioni sui ruoli}
	\begin{center}
	\rowcolors{2}{lightest-grayest}{white}
	\begin{longtable}{|p{3cm}|p{6cm}|p{6cm}|}
	\hline
	\rowcolor{lighter-grayer}
	\textbf{Ruolo} & \textbf{Problema rilevato} & \textbf{Contromisura}\\
	\hline
	\endfirsthead

	% ----- Modificare da qui -----

	% 0.0.2 & Revisione documento & 24-11-2019 & Nome Cognome & Verificatore \\
	\hline
	Responsabile
	 &
	Durante la fase relativa alla stesura dei documenti per proporsi ufficialmente come fornitori per il capitolato \textit{C6}, i compiti da svolgere sono stati assegnati dal \textit{responsabile} su base volontaria, portando così a situazioni in cui alcuni membri si sono ritrovati sovraccarichi di lavoro mentre altri erano tenuti a svolgere attività più sbrigative. Questo fatto ha portato alcuni colleghi a chiedere una ridistribuzione dei compiti, cosa che ha rallentato il processo produttivo. 
	 	& 
Dopo ogni assegnazione di compiti da svolgere, il \textit{responsabile} di turno si deve impegnare a ricontrollare se sono stati spartiti equamente tra i membri, cosicchè non si subiscano rallentamenti per via di ridistribuzioni degli oneri di progetto. \\
	\hline
	
		\hline
	Analista
	 &
	Essendo il capitolato chiaramente privo di approfondimenti sui bisogni del committente, è risultato difficile ottenere da essi i requisiti di progetto.
	 	& 
Gli analisti hanno deciso di svolgere quest'attività scambiandosi continuamente i propri risultati e chiedendo chiarimenti agli altri quando necessario. Inoltre, è stato fissato un incontro col committente di modo che sia possibile avere un feedback diretto sulla valenza e coerenza dei requisiti ottenuti, data l'inesperienza dei membri sull'attività di \textit{analisi dei requisiti}. \\
	\hline
	
		\hline
	Verificatori
	 &
Data l'inesperienza dei membri nell'attività di stesura dei documenti, è indispensabile che i verificatori controllino ogni sezione scrupolosamente. Per fare ciò però, quest'ultimi devono avere una certa conoscenza di tutte le tematiche trattate nella documentazione, e ciò richiede lo studio approfondito delle stesse.	 
	 	& 
Per risolvere questi problemi, ai verificatori si è deciso di non assegnare altri compiti durante lo svolgimento del processo di verifica, in modo che avessero il tempo materiale per attuarlo nel modo più efficace possibile. \\

	\hline
	Amministratore
	 &
	Per redarre alcuni sezioni di certi documenti, i membri che hanno ricoperto questo ruolo hanno trovato delle difficoltà relative alla mancata presenza di materiale autorevole e chiaro che le descrivesse.
	 	& 
Si è deciso di prendere spunto dalla documentazione prodotta dai colleghi degli anni passati, confrontarla con quanto appreso durante il corso di \textit{software engineering} e procedere alla loro stesura cercando di trattare ogni sezione in modo approfondito e chiaro. \\
	\hline	
	
		\end{longtable}
	\end{center}

	\newpage	
	
	\subsection{Valutazioni sugli strumenti}

		\begin{center}
	\rowcolors{2}{lightest-grayest}{white}
	\begin{longtable}{|p{3cm}|p{6cm}|p{6cm}|}
	\hline
	\rowcolor{lighter-grayer}
	\textbf{Strumento} & \textbf{Problema rilevato} & \textbf{Contromisura}\\
	\hline
	\endfirsthead

	% ----- Modificare da qui -----

	% 0.0.2 & Revisione documento & 24-11-2019 & Nome Cognome & Verificatore \\
	\hline
	Version Control System
	 &
	Nel \textit{way of working} del particolare \glock{vcs} utilizzato, si sono fissate diverse regole su come condividere il materiale prodotto nel server. Queste regole non sono state attuate da alcuni membri, vista l'inesperienza, e ciò ha portato a rallentamenti nella fase di condivisione del materiale scelto per poter chiudere le \glock{milestone} relative e procedere ad aprire quelle successive.  
	 	& 
E' stato condiviso un documento testuale che spiega nel dettaglio come ottemperare alle regole scelte per l'utilizzo dello strumento, in modo da facilitare i membri più in difficoltà e non rendere il \textit{vcs} un ostacolo all'avanzamento del gruppo verso la realizzazione della documentazione necessaria.  \\
	\hline
	
		\hline
	\LaTeX
	 &
	Questo strumento di scrittura è stato una novità per quasi tutti i membri del gruppo. All'inizio, in molti avevano errori di compilazione nei propri file, per cui non riuscivano a produrre opportuni file pdf. 
	 	& 
Dopo il primo mese e mezzo dalla formazione del gruppo, un membro particolarmente esperto di  \LaTeX ha creato un template da far utilizzare a tutti i membri per la produzione della documentazione, riducendo ai minimi termini il numero di comandi da imparare per utilizzare questo software efficaciemente.\\
	\hline
	
		\end{longtable}
	\end{center}