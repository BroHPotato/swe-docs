\section{Resoconto attività di verifica}

\subsection{Verifica documentazione}
Per la verifica della documentazione si utilizza la tecnica del \glock{walkthrough} revisionando i documenti per intero e la tecnica di \glock{inspection} secondo i punti descritti nel documento \dext{norme di progetto}

\subsubsection{Calcolo leggibilità documenti}
Per verificare quanto sono leggibili i documenti redatti si utilizza \glock{l'indice di Gulpease}, di seguito la tabella contenente i risultati ottenuti:

\begin{center}
	\rowcolors{2}{lightest-grayest}{white}
	\begin{longtable}{|c|c|}
	\hline
	\rowcolor{lighter-grayer}
	\textbf{Documento} & \textbf{Indice di Gulpease} \\
	\hline
	\endfirsthead

	\hline
	Norme di Progetto &  74 \\
	\hline
	\hline
	Studio di fattibilità & 68 \\
	\hline
	\hline
	Verbali & 83.0(media) \\
	\hline
	\hline
	Analisi dei requisiti & 88 \\
	\hline
	\hline
	Glossario & 89.6 \\
	\hline
	\hline
	Piano di Progetto & 84 \\
	\hline
	\hline
	Piano di qualifica & 84 \\
	\hline

	\end{longtable}
\end{center}