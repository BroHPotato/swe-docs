\subsubsection{QC-4 Usabilità}
L'usabilità definisce la capacità del prodotto di essere appreso ed usato dall'utente in determinate situazioni
	\paragraph{Obiettivi}
		\begin{itemize}
			\item \textbf{comprensibilità:} determina la facilità di utilizzo e di comprensione del prodotto e delle sue funzionalità da parte dell'utente;
			\item \textbf{apprendibilità:} definisce il livello di impegno richiesto, da parte dell'utilizzatore, per imparare ad usare il prodotto;
			\item \textbf{operabilità:} stabilisce il grado con cui il software riesce a mettere il suo utilizzatore in condizione di sfruttare il prodotto per i suoi fini;
			\item \textbf{attrattiva:} la proprietà del software di produrre un'esperienza d'uso gradevole per l'utente.
		\end{itemize}
	\paragraph{Metriche}
	L'usabilità del prodotto viene valutata dai seguenti criteri:
	\begin{itemize}
		\item profondità dell'albero delle azioni: quante azioni deve compiere l'utente per arrivare al suo obiettivo;
		\item profondità dell'albero delle pagine: quante pagine deve visitare l'utente per arrivare alla pagina obiettivo.
	\end{itemize}
	\begin{center}
		\rowcolors{2}{lightest-grayest}{white}
		\begin{tabular}{|c|c|c|c|}
			\rowcolor{lighter-grayer}
			\hline
			ID & Nome & Valore ottimale & Valore accettabile \\
			\hline
			QM-PROD-5 & Profondità dell'albero delle azioni (PAA) & NAN & NAN \\
			\hline
			QM-PROD-6 & Profondità dell'albero delle pagine (PAP) & NAN & NAN \\
			\hline
		\end{tabular}
	\end{center}
