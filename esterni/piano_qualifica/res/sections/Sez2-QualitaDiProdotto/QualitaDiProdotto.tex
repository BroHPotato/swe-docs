\section{Qualità di prodotto}

\subsection{Introduzione}
	Per garantire e valutare la qualità del prodotto il gruppo ha deciso di fare riferimento allo standard ISO/IEC 9126, il quale definisce i parametri per produrre un prodotto di buona qualità, questi parametri quantificano il grado di raggiungimento di tale caratteristica.
	\newline
	Oltre alle qualità presenti nello standard sopra citato, il gruppo ha deciso di utilizzare altri parametri per quantificare la qualità della documentazione fornita con il prodotto, oltre al software stesso.
	\newline
	Di seguito sono riportate le qualità che il gruppo ha ritenuto appropriate per quanto riguarda lo stato attuale del progetto.
	
%\subsection{Qualità del software}
	%\subsubsection{QC-1 Funzionabilità}
	La Funzionabilità definisce la capacità del prodotto di fornire le funzioni che soddisfano con le esigenze stabilite nell'Analisi dei Requisiti.
	
	\paragraph{Metriche}
	La Funzionabilità del prodotto viene valutata dal seguente criterio:
	\begin{itemize}
		\item implementazione: misura in percentuale le funzionalità (sia richieste che opzionali) implementate a fronte delle funzionalità proposte.
	\end{itemize}
	\begin{center}
		\rowcolors{2}{lightest-grayest}{white}
		\begin{tabular}{|c|c|c|c|}
			\rowcolor{lighter-grayer}
			\hline
			ID & Nome & Valore ottimale & Valore accettabile \\
			\hline
			QM-PROD-1 & Implementazione (IMP)  & 100\% & 100\% \\
			\hline
		\end{tabular}
	\end{center}

	
\subsection{Monitoraggio dei documenti}
	Le qualità dei documenti, monitorate con delle metriche precise di qualità, sono le seguenti:
	
	\begin{itemize}
		\item QC-1 Comprensione.
	\end{itemize}

	\subsubsection{QC-1 Comprensione}
Tutti i documenti devono essere leggibili e comprensibili, queste qualitá derivano dalla correttezza lessicografica, grammaticale, e semantica.
	\paragraph{Obiettivi}
		\begin{itemize}
			\item \textbf{leggibilità:} per garantire la leggibilità dei documenti si è deciso di utilizzare l'indice di Gulpease come indicatore per questa caratteristica;
			\item \textbf{correttezza:} i documenti presentati non devono contenere errori ortografici di alcun genere.
		\end{itemize}
	\paragraph{Metriche}
	La comprensione dei documenti viene valutata dai seguenti criteri:
	\begin{itemize}
		\item QM-PROD-1 Indice di Gulpease;
    \item QM-PROD-2 Correttezza ortografica.
	\end{itemize}
	\begin{center}
		\rowcolors{2}{lightest-grayest}{white}
		\begin{tabular}{|c|c|c|c|}
			\rowcolor{lighter-grayer}
			\hline
			ID & Nome & Valore ottimale & Valore accettabile \\
			\hline
			QM-PROD-1 & Indice di Gulpease (GULP) & 100 & 40 \\
 		  	\hline
			QM-PROD-2 & Correttezza ortografica (CORT) & 0 & 0 \\
			\hline
		\end{tabular}
	\end{center}

	
\subsection{Monitoraggio del software}
	Le qualità del software, monitorate con delle metriche precise di qualità, sono le seguenti:
	
	\begin{itemize}
		\item QP-2 Sicurezza;
		\item QP-3 Affidabilità;
		\item QP-4 Efficienza;
		\item QP-5 Usabilità;
		\item QC-6 Manutenibilità.
	\end{itemize}

	\subsubsection{QC-2 Sicurezza}
Con il termine sicurezza si intende la proprietà del prodotto di essere privo di vulnerabilità nel codice proprio e nelle librerie esterne utilizzate.

	\paragraph{Metriche}
	Il livello di sicurezza del prodotto è determinato dal valore dei seguenti indici:
	\begin{itemize}
		\item numero di vulnerabilità rilevate(NVUL): numero intero che quantifica il numero di vulnerabilità presenti nel codice;
		\item tempo di risoluzione delle vulnerabilità(TVUL): stima temporale di quanto tempo occuperà la risoluzione di una specifica vulnerabilità.
	\end{itemize}
	\begin{center}
		\rowcolors{2}{lightest-grayest}{white}
		\begin{tabular}{|c|c|c|c|}
			\rowcolor{lighter-grayer}
			\hline
			ID & Nome & Valore ottimale & Valore accettabile \\
			\hline
			QM-PROD-4 & Numero vulnerabilità rilevate (NVUL) & \(\e 0\) & \(\le 5\) \\
			\hline
			QM-PROD-5 & tempo di risoluzione delle vulnerabilità (TVUL) & \(\e 00:00:00\) &  \(\le 01:00:00\) \\
			\hline
		\end{tabular}
	\end{center}

	\subsubsection{QC-2 Affidabilità}
Con il termine affidabilità si intende la capacità del prodotto di mantenere un livello minimo di prestazioni, deciso a priopri, in determinate situazioni ed un dato lasso di tempo.
	\paragraph{Obiettivi}
		\begin{itemize}
			\item \textbf{maturità:} il software deve essere in grado di evitare il verificarsi di errori e/o malfunzionamenti derivanti dalla sua esecuzione;
			\item \textbf{tolleranza degli errori:} il prodotto é in grado di mantenere un livello minimo di prestazioni predeterminate anche in presenza di malfunzionamenti e/o usi impropri di esso;
			\item \textbf{recuperabilità:} il software, in seguito ad un errore e/o malfunzionamento, deve essere in grado di ripristinare uno stato di usabilità in un arco di tempo definito e di recuiperare eventuali dati persi durante il suddetto lasso di tempo;
			\item \textbf{aderenza:} descrive la capacitá del prodotto di aderire alle specifiche relative alláffidabilità.
		\end{itemize}
	\paragraph{Metriche}
	L'affidabilità del prodotto viene valutata dai seguenti criteri:
	\begin{itemize}
		\item densità errori: è una percentuale che indica quanti test sono stati passati a fronte di quelli proposti.
		\item complessità dei test di classe: fornisce il numero di test che coinvolgono una classe;
	\end{itemize}
	\begin{center}	
		\rowcolors{2}{lightest-grayest}{white}
		\begin{tabular}{|c|c|c|c|}
			\rowcolor{lighter-grayer}
			\hline
			ID & Nome & Valore ottimale & Valore accettabile \\
			\hline
			QM-PROD-2 & Densità errori (DE) & 100\% & 100\% \\
			\hline
			QM-PROD-3 & Complessità dei test di classe (CTCLA) & NAN & NAN \\
			\hline
		\end{tabular}
	\end{center}

	\subsubsection{QC-3 Efficienza}
Con efficienza si intende la capacità del prodotto di mantenere un livello adeguato di prestazioni in determinate situazioni.
	\paragraph{Obiettivi}
		\begin{itemize}
			\item \textbf{comportamento nel tempo:} garanzia di tempi di elaborazione accettabili da parte del prodotto;
			\item \textbf{utilizzo di risorse:} utilizzo non eccessivo delle risorse a disposizione.
		\end{itemize}
	\paragraph{Metriche}
	L'efficenza del prodotto viene valutata dal seguente criterio:
	\begin{itemize}
		\item risposta media: è una misurazione in ms che indica il tempo medio di risposta per ogni richiesta.
	\end{itemize}
	\begin{center}
		\rowcolors{2}{lightest-grayest}{white}
		\begin{tabular}{|c|c|c|c|}
			\rowcolor{lighter-grayer}
			\hline
			ID & Nome & Valore ottimale & Valore accettabile \\
			\hline
			QM-PROD-4 & Risposta media (RM) & NAN & NAN \\
			\hline
		\end{tabular}
	\end{center}

	\subsubsection{QC-5 Usabilità}
L'usabilità definisce la capacità del prodotto di essere appreso ed usato dall'utente per raggiungere i suoi scopi.

	\paragraph{Metriche utilizzate}
	\begin{itemize}
		\item QM-PROD-9 Profondità dell'albero delle azioni (PAA);
		\item QM-PROD-10 Profondità dell'albero delle pagine (PAP).
	\end{itemize}

	\paragraph{Indici di qualità}
	\begin{center}
		\rowcolors{2}{lightest-grayest}{white}
		\begin{tabular}{|c|c|c|}
			\rowcolor{lighter-grayer}
			\hline
			\textbf{ID metrica} & \textbf{Valore preferibile} & \textbf{Valore accettabile} \\
			\hline
			QM-PROD-9 (PAA) & \(\le 5\) & \(\le 8\) \\
			\hline
			QM-PROD-10 (PAP) & \(\le 5\) & \(\le 7\) \\
			\hline
		\end{tabular}
	\end{center}

	\subsection{Manutenibilità}
	\subsubsection{Obiettivi}