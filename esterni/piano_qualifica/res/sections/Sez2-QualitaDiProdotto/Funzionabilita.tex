\subsubsection{QC-1 Funzionabilità}
	La Funzionabilità definisce la capacità del prodotto di fornire le funzioni che soddisfano con le esigenze stabilite nell'Analisi dei Requisiti.
	\paragraph{Obiettivi}
		\begin{itemize}
			\item \textbf{appropriatezza:} viene richiesto che il prodotto metta a disposizione tutte le funzionalità richieste dall'utente;
			\item \textbf{accuratezza:} il prodotto deve riuscire a produrre risultati che rispettano l'aspettativa ed il grado di precisione richiesti;
			\item \textbf{interoperabilità:} il prodotto deve essere in grado di interagire ed operaree con tutti i sistemi e vincoli specificati;
			\item \textbf{conformità:} il prodotto deve aderire a standard e regolamenti noti;
			\item \textbf{sicurezza:} i dati sensibili utilizzati e generati dal prodotto devono essere disponibili esclusivamente agli utenti e/o coloro che risultano autorizzati all'uso di tali dati.
		\end{itemize}
	\paragraph{Metriche}
	La Funzionabilità del prodotto viene valutata dal seguente criterio:
	\begin{itemize}
		\item implementazione: misura in percentuale le funzionalità (sia richieste che opzionali) implementate a fronte delle funzionalità proposte.
	\end{itemize}
	\begin{center}
		\rowcolors{2}{lightest-grayest}{white}
		\begin{tabular}{|c|c|c|c|}
			\rowcolor{lighter-grayer}
			\hline
			ID & Nome & Valore ottimale & Valore accettabile \\
			\hline
			QM-PROD-1 & Implementazione (IMP)  & 100\% & 100\% \\
			\hline
		\end{tabular}
	\end{center}
