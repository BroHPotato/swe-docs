\subsubsection{QC-1 Funzionabilità}
	La Funzionabilità definisce la capacità del prodotto di fornire le funzioni che soddisfano con le esigenze stabilite nell'Analisi dei Requisiti.
	
	\paragraph{Metriche}
	La Funzionabilità del prodotto viene valutata dal seguente criterio:
	\begin{itemize}
		\item implementazione: misura in percentuale le funzionalità (sia richieste che opzionali) implementate a fronte delle funzionalità proposte.
	\end{itemize}
	\begin{center}
		\rowcolors{2}{lightest-grayest}{white}
		\begin{tabular}{|c|c|c|c|}
			\rowcolor{lighter-grayer}
			\hline
			ID & Nome & Valore ottimale & Valore accettabile \\
			\hline
			QM-PROD-1 & Implementazione (IMP)  & 100\% & 100\% \\
			\hline
		\end{tabular}
	\end{center}
