\subsubsection{Comprensione}
Tutti i documenti devono essere leggibili e comprensibili, queste qualitá derivano dalla correttezza lessicografico, grammaticale, e semantica.
	\paragraph{Obiettivi}
		\begin{itemize}
			\item \textbf{leggibilità:} per garantire la leggibilità dei documenti si è deciso di utilizzare l'indice di Gulpease come indicatore per questa caratteristica;
			\item \textbf{correttezza:} i doumenti presentati non devono contenere errori ortografici di alcun genere.
		\end{itemize}
	\paragraph{Metriche}
	La comprensione dei documenti viene valutata dai seguenti criteri:
	\begin{itemize}
		\item Indice di Gulpease;
    \item Correttezza ortografica.
	\end{itemize}
	\begin{center}
		\begin{tabular}{|c|c|c|c|c|}
			\hline
			ID & Nome & Misurazione & Valore ottimale & Valore accettabile \\
			\hline
			QM-PROD-10 & Indice di Gulpease (GULP) & \(GULP = 89\plus\frac{300\times\# frasi\minus10\times\#lettere}{\#parole}\) & 100 & 40 \\
      \hline
			QM-PROD-11 & Correttezza ortografica (CORT) & \textit{CORT = \# numero di errori ortografici} & 0 & 0 \\
			\hline
		\end{tabular}
	\end{center}
