\subsubsection{QC-6 Manutenibilità}
Fornisce un indicatore sul livello di semplicità per quanto riguarda la modifica, correzione ed estendibilità del prodotto software.
	\paragraph{Obiettivi}
		
	\paragraph{Metriche}
	La manutenibilità del prodotto viene valutata dai seguenti criteri:
	\begin{itemize}
		\item complessità ciclomatica media: questo indice è dato dal rapporto tra la \glock{complessità ciclomatica} media della componente in esame e il numero di classi che la costituiscono;
		\item complessità della classe: si calcola il numero medio di metodi per classe, così da sapere quanto siano modulari;
		\item numero di \glock{code smells} rilevati: si valuta il numero di violazioni di best practice che potrebbero causare bug nel futuro oppure rallentare lo sviluppo;
		\item Tempo di risoluzione code smells: tempo stimato per risolvere i code smells rilevati;
		\item percentuale di duplicazione del codice: indice che quantifica la percentuale di codice replicata nelle componenti del prodotto;
		\item numero di violazioni dello standard di codifica: si quantificano eventuali violazioni delle regole di uniformità del codice adottate.
	\end{itemize}
	\begin{center}
		\rowcolors{2}{lightest-grayest}{white}
		\begin{tabular}{|c|c|c|c|}
			\rowcolor{lighter-grayer}
			\hline
			ID & Nome & Valore ottimale & Valore accettabile \\
			\hline
			QM-PROD-11 & Complessità del codice (COCIM) & \(\le 10\) & \(\le 20\) \\
			\hline
			QM-PROD-12 & Complessità della classe (CCLA) & \(\le 5\) & \(\le 10\) \\
			\hline
			QM-PROD-13 & Numero di code smells rilevati (NCS) & \(\le 25\) & \(\le 50\) \\
			\hline
			QM-PROD-14 & Tempo di risoluzione code smells (TCS) & \(\le 03:00:00\) & \(\le 06:00:00\) \\
			\hline
			QM-PROD-15 & Percentuale di duplicazione del Codice (DUPC) & \(\le 0\) & \(\le 3\)\% \\
			\hline
			QM-PROD-16 & Numero di violazioni dello standard di codifica (NVSC) & \(\le 0\) & \(\le 10\) \\
			\hline
		\end{tabular}
	\end{center}
