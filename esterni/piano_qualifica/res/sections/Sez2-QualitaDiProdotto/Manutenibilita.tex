\subsubsection{QC-5 Manutenibilità}
Fornisce un indicatore sul livello di semplicità per quanto riguarda la modifica, correzzione ed estendibilità del prodotto software.
	\paragraph{Obiettivi}
		\begin{itemize}
			\item \textbf{analizzabilità:} determina la facilità con cui é possibile analizzare e localizzare un errore all'interno del codice;
			\item \textbf{modificabilità:} definisce la capacitá del prodotto di apportare una modifica o una estensione;
			\item \textbf{stabilità:} il software deve essere un grado di essere usato anche in caso le modifiche apportate siano errate;
			\item \textbf{testabilità:} determina la capacità del software di essere testato facilmente per fornire una validazione delle modifica apportate.
		\end{itemize}
	\paragraph{Metriche}
	La manutenibilità del prodotto viene valutata dai seguenti criteri:
	\begin{itemize}
		\item complessità del codice: consiste nel rapporto tra il numero di linee di commento ed il numero di linee di codice;
		\item complessità della classe: si contano il numero dei metodi di una classe per avere una misura della sua complessità;
		\item complessità del metodo: si valuta la lunghezza del metodo e il numero di chiamate (dirette) ad altri metodoi da parte di quest'ultimo.
	\end{itemize}
	\begin{center}
		\rowcolors{2}{lightest-grayest}{white}
		\begin{tabular}{|c|c|c|c|}
			\rowcolor{lighter-grayer}
			\hline
			ID & Nome & Valore ottimale & Valore accettabile \\
			\hline
			QM-PROD-7 & Complessità del codice (CCOD) & NAN & NAN \\
			\hline
			QM-PROD-8 & Complessità della classe (CCLA) & NAN & NAN \\
			\hline
			QM-PROD-9 & Complessità del metodo (CMET) & NAN & NAN \\
			\hline
		\end{tabular}
	\end{center}
