\subsubsection{QC-3 Affidabilità}
Con il termine affidabilità si intende la capacità del prodotto di mantenere un livello minimo di prestazioni, deciso a priori, in determinate situazioni ed un dato lasso di tempo.

	\paragraph{Metriche}
	L'affidabilità del prodotto viene valutata dai seguenti criteri:
	\begin{itemize}
		\item densità errori: è una percentuale che indica quanti test sono stati passati a fronte di quelli proposti;
		\item complessità dei test di classe: fornisce il numero di test che coinvolgono una classe.
	\end{itemize}
	\begin{center}
		\rowcolors{2}{lightest-grayest}{white}
		\begin{tabular}{|c|c|c|c|}
			\rowcolor{lighter-grayer}
			\hline
			ID & Nome & Valore ottimale & Valore accettabile \\
			\hline
			QM-PROD-2 & Densità errori (DE) & 100\% & 100\% \\
			\hline
			QM-PROD-3 & Complessità dei test di classe (CTCLA) & NAN & NAN \\
			\hline
		\end{tabular}
	\end{center}
