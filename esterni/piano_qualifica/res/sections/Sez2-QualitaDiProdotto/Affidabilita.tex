\subsubsection{QC-3 Affidabilità}
Con il termine affidabilità si intende la capacità del prodotto di mantenere un livello minimo di funzionamento, deciso a priori, in qualsiasi situazione di utilizzo. Quindi, oltre all'affidabilità in senso stretto, vengono considerate anche la correttezza e la tolleranza agli errori.

	\paragraph{Metriche utilizzate}
	\begin{itemize}
		\item QM-PROD-6 Numero di bug rilevati (BUGR);
		\item QM-PROD-7 Tempo stimato risoluzione bug (TBUG).
	\end{itemize}

	\paragraph{Indici di qualità}
	\begin{center}
		\rowcolors{2}{lightest-grayest}{white}
		\begin{tabular}{|c|c|c|}
			\rowcolor{lighter-grayer}
			\hline
			\textbf{ID metrica} & \textbf{Valore preferibile} & \textbf{Valore accettabile} \\
			\hline
			QM-PROD-6 (BUGR) & \(= 0\) &\(\le 3\) \\
			\hline
			QM-PROD-7 (TBUG) & \(= 00:00:00\) & \(\le 01:00:00\) \\
			\hline
		\end{tabular}
	\end{center}
