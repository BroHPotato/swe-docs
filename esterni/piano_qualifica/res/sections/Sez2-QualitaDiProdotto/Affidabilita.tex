\subsection{Affidabilità}
Con il termine affidabilità si intende la capacità del prodotto di mantenere un livello minimo di prestazioni, deciso a priopri, in determinate situazioni ed un dato lasso di tempo.
	\subsubsection{Obiettivi}
		\begin{itemize}
			\item \textbf{maturità:} il software deve essere in grado di evitare il verificarsi di errori e/o malfunzionamenti derivanti dalla sua esecuzione;
			\item \textbf{tolleranza degli errori:} il prodotto é in grado di mantenere un livello minimo di prestazioni predeterminate amche in presenza di malfunzionamenti e/o usi impropri di esso;
			\item \textbf{recuperabilità:} il software, in seguito ad un errore e/o malfunzionamento, deve essere in grado di ripristinare uno stato di usabilità in un arco di tempo definito e di recuiperare eventuali dati persi durante il suddetto lasso di tempo;
			\item \textbf{aderenza:} descrive la capacitá del prodotto di aderire alle specifiche relative alláffidabilità.
		\end{itemize}
	\subsection{Metriche}
