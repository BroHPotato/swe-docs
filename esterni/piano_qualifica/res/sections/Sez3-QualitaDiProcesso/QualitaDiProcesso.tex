\section{Qualità di processo}

\subsection{Introduzione}

Nello svolgimento del progetto, i processi fanno uso di criteri di qualità, attraverso i quali è possibile perseguire un miglioramento continuo che porti alla più completa soddisfazione di questi criteri.
\newline
In questo progetto, si è scelto di fare uso del metodo PDCA e dello standard ISO/IEC 15504 (SPICE). Attraverso \glock{PDCA} e \glock{SPICE}, è possibile garantire uno svolgimento dei processi che tendono, attraverso l'esperienza, a migliorarsi e ad assicurare al cliente l'ottenimento di un prodotto di qualità.
\newline
In questa sezione si espongono i livelli di qualità accettabili e ottimali sulla base delle metriche scelte all'interno del documento \dext{Norme di Progetto v2.3.0}.

\subsection{Monitoraggio dei processi}

I processi monitorati con delle metriche precise di qualità sono i seguenti:

\begin{itemize}
	\item QP-1 Gestione dei processi;
	\item QP-2 Gestione dei rischi;
 	\item QP-3 Sviluppo;
	\item QP-4 Verifica;
	\item QP-5 Garanzia della qualità.
\end{itemize}

	\subsubsection{QP-1 Gestione dei processi}

		Il processo di gestione dei processi si riserva di gestire la copertura delle risorse disponibili e delle attività schedulate all'interno del \dext{Piano di Progetto v2.5.0}.
		\newline
		Di seguito vengono esposte le metriche utilizzate, che possono essere visionate all'interno del documento \dext{Norme di Progetto v2.3.0}.

		\paragraph{Metriche utilizzate}

			\begin{itemize}
				\item QM-PROC-1 Budgeted cost of work scheduled (BCWS);
				\item QM-PROC-2 Actual cost of work performed (ACWP);
				\item QM-PROC-3 Budgeted cost of work performed (BCWP);
				\item QM-PROC-4 Schedule variance (SV);
				\item QM-PROC-5 Cost variance (CV).
			\end{itemize}
		\pagebreak
		\paragraph{Indici di qualità}

			\begin{center}
				\rowcolors{2}{white}{lightest-grayest}
				\begin{longtable}{|c|c|c|}
				\hline
				\rowcolor{lighter-grayer}
				\textbf{ID metrica} & \textbf{Valore preferibile} & \textbf{Valore accettabile}\\
				\hline
				\endfirsthead
				\hline
				QM-PROC-1 (BCWS) & \(\ge 0\) & \(\ge 0\) \\
				\hline
				QM-PROC-2 (ACWP) & \(0 \le ACWP \le BCWS\) & \(0 \le ACWP \le preventivo\) \\
				\hline
				QM-PROC-3 (BCWP) & \(= BCWS\) & \(\ge 0\) \\
				\hline
				QM-PROC-4 (SV) & \(0\%\) & \(\ge -5\%\) \\
				\hline
				QM-PROC-5 (CV) & \(0\%\) & \(\ge -5\%\) \\
				\hline
				\caption{Indici di qualità per le metriche di gestione dei processi}
				\end{longtable}
			\end{center}

	\subsubsection{QP-2 Gestione dei rischi}

		Il processo di gestione dei rischi monitora la comparsa di nuovi rischi che possono verificarsi durante le fasi del progetto.
		\newline
		Per ogni fase del progetto si eseguirà una relativa analisi retrospettiva dei rischi precedentemente segnalati e, in caso di nuovi rischi, si cercherà di risolverli nel minor tempo possibile.
		\newline
		Di seguito vengono esposte le metriche utilizzate, che possono essere visionate all'interno del documento \dext{Norme di Progetto v2.3.0}.

		\paragraph{Metriche utilizzate}

			\begin{itemize}
				\item QM-PROC-6 Unbudgeted risks (UR).
			\end{itemize}

		\paragraph{Indici di qualità}

			\begin{center}
				\rowcolors{2}{white}{lightest-grayest}
				\begin{longtable}{|c|c|c|}
				\hline
				\rowcolor{lighter-grayer}
				\textbf{ID metrica} & \textbf{Valore preferibile} & \textbf{Valore accettabile}\\
				\hline
				\endfirsthead
				\hline
				QM-PROC-6 (UR) & \(0 \text{ rischi}\) & \(\le 5 \text{ rischi}\) \\
				\hline
				\caption{Indici di qualità per le metriche di gestione dei rischi}
				\end{longtable}
			\end{center}
	\newpage
	\subsubsection{QP-3 Sviluppo}

		Il processo di sviluppo si prefigge di realizzare il prodotto commissionato, in modo conforme ai requisiti concordati con il committente.
		\newline
		Di conseguenza è necessario tenere traccia di tutti i requisiti portati a termine fino alla data corrente e dell'eventuale valore aggiunto che possono costituire i requisiti non obbligatori.
		\newline
		Di seguito vengono esposte le metriche utilizzate, che possono essere visionate all'interno del documento \dext{Norme di Progetto v2.3.0}.

		\paragraph{Metriche utilizzate}

			\begin{itemize}
				\item QM-PROC-7 Satisfied mandatory requirements (SMR);
				\item QM-PROC-8 Satisfied desirable requirements (SDR);
				\item QM-PROC-9 Satisfied optional requirements (SOR);
				\item QM-PROC-10 Numero di commit (NCOM).
			\end{itemize}


		\paragraph{Indici di Qualità}

			\begin{center}
				\rowcolors{2}{lightest-grayest}{white}
				\begin{longtable}{|c|c|c|}
				\hline
				\rowcolor{lighter-grayer}
				\textbf{ID metrica} & \textbf{Valore preferibile} & \textbf{Valore accettabile}\\
				\hline
				\endfirsthead
				\hline
				QM-PROC-7 (SMR) & \(= 100\%\) & \(= 100\%\) \\
				\hline
				QM-PROC-8 (SDR) & \(= 100\%\) & \(\geq 50\%\) \\
				\hline
				QM-PROC-9 (SOR) & \(\geq 50\%\) & \(\geq 0\%\) \\
				\hline
				QM-PROC-10 (NCOM) \footnote{Questa metrica non prevede dei valori di soglia con cui effettuare un confronto in quanto l'obiettivo è mostrare la tendenza dei valori rilevati} & \(-\) & \(-\) \\
				\hline
				\caption{Indici di qualità per le metriche di validazione}
				\end{longtable}
			\end{center}
	\newpage
	\subsubsection{QP-4 Verifica}

		Il processo di verifica si pone come obiettivo il controllo dello sviluppo software a livello di codifica.
		\newline
		Di seguito vengono esposte le metriche utilizzate, che possono essere visionate all'interno del documento \dext{Norme di Progetto v2.3.0}.

		\paragraph{Metriche utilizzate}

			\begin{itemize}
				\item QM-TEST-1 Code coverage (COCO);
				\item QM-TEST-2 Condition coverage (CONCO);
				\item QM-TEST-3 Line coverage (LOCO);
				\item QM-TEST-4 Passed test cases percentage (PTCP);
				\item QM-TEST-5 Failed test cases percentage (FTCP);
				\item QM-TEST-6 Bug-fixing percentage (BFP);
				\item QM-TEST-7 Complessità media dei test di classe (CMTC).
			\end{itemize}


		\paragraph{Indici di Qualità}
			\begin{center}
				\rowcolors{2}{white}{lightest-grayest}
				\begin{longtable}{|c|c|c|}
				\hline
				\rowcolor{lighter-grayer}
				\textbf{ID metrica} & \textbf{Valore preferibile} & \textbf{Valore accettabile}\\
				\hline
				\endfirsthead
				\hline
				QM-TEST-1 (COCO) & \(\geq 90\%\) & \(\geq 75\%\) \\ \hline
				QM-TEST-2 (CONCO) & \(\geq 90\%\) & \(\geq 75\%\) \\ \hline
				QM-TEST-3 (LOCO) & \(\geq 90\%\) & \(\geq 75\%\) \\ \hline
				QM-TEST-4 (PTCP) & \(= 100\%\) & \(\geq 90\%\) \\ \hline
				QM-TEST-5 (FTCP) & \(= 0\%\) & \(\le 10\%\) \\ \hline
				QM-TEST-6 (BFP) & \(= 100\%\) & \(\geq 50\%\) \\ \hline
				QM-TEST-7 (CMTC) \footnote{Questa metrica non prevede dei valori di soglia con cui effettuare un confronto in quanto l'obiettivo è mostrare la tendenza dei valori rilevati} & \(-\) & \(-\) \\ \hline
				\hline
				\caption{Indici di qualità per le metriche di verifica}
				\end{longtable}
			\end{center}
	\newpage
	\subsubsection{QP-5 Garanzia della qualità}
	
	Il processo di garanzia della qualità si pone come obiettivo il controllo del livello di qualità mantenuto dal gruppo durante lo svolgimento del progetto.
	\newline
	Di seguito vengono esposte le metriche utilizzate, che possono essere visionate all'interno del documento \dext{Norme di Progetto v2.3.0}.
	
	\paragraph{Metriche utilizzate}
	
	\begin{itemize}
		\item QM-PROC-11 Percentuale delle metriche soddisfatte (PMS).
	\end{itemize}
	
	
	\paragraph{Indici di Qualità}
	
	\begin{center}
		\rowcolors{2}{white}{lightest-grayest}
		\begin{longtable}{|c|c|c|}
			\hline
			\rowcolor{lighter-grayer}
			\textbf{ID metrica} & \textbf{Valore preferibile} & \textbf{Valore accettabile}\\
			\hline
			\endfirsthead
			\hline
			QM-PROC-11 (PMS) & \(= 100\%\) & \(\geq 75\%\) \\ \hline
			\hline
			\caption{Indici di qualità per le metriche di garanzia della qualità}
		\end{longtable}
	\end{center}