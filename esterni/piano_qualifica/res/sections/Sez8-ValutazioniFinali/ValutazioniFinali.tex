% 1- introduzione
% 2- ragionamento sul funzionamento ipotetico del pdca e degli obiettivi di miglioramento
% 3- valutazione critica dei processi che hanno subito miglioramenti
% 4- obiettivi di miglioramento di fatto raggiunti

\section{Valutazioni finali sul progetto}
	In questa sezione vengono riportate alcune valutazioni fatte in seguito ad un'analisi retrospettiva di tutto il lavoro svolto dall'inizio alla fine del progetto. In particolare sono stati fatti dei ragionamenti critici riguardo i miglioramenti sperimentati nei diversi processi istanziati durante lo sviluppo, in modo da poter comprendere meglio qual è stata l'evoluzione del \textit{way of working} e quali benefici ne ha tratto il gruppo.
	\newline\newline
	Il modo migliore per dare un senso ed una struttura al miglioramento è applicare il ciclo PDCA, basandolo su specifici obiettivi di avanzamento quantificabili relativi agli aspetti dei singoli processi di maggior rilevanza ai fini del progetto. Questa prassi permette di attuare dei miglioramenti proattivi che puntano a potenziare le buone prassi già fissate, prima dell'insorgere di problematiche che necessiterebbero correzioni reattive, realizzando quindi un miglioramento continuo.
	\newline
	Durante lo sviluppo, però, questo approccio non è stato seguito a causa dell'inesperienza del gruppo che non ha permesso di capirne pienamente l'utilità; infatti sono state spesso privilegiate azioni correttive, adottate in risposta agli errori riscontrati durante l'avanzamento del progetto. Quest'ultime, sebbene spesso tardive ed inutili per attuare il miglioramento continuo, hanno il vantaggio di avere dei costi inferiori, in termini di risorse, rispetto alle azioni proattive.
	\newline
	Per poter applicare il ciclo PDCA, gli obiettivi di miglioramento devono essere selettivi, ossia devono intervenire su specifici aspetti dei processi, e misurabili, in modo da capire se e quando il loro scopo viene raggiunto. Di conseguenza, essi devono essere scelti oculatamente, valutando bene il rapporto costi/benefici, per fare in modo che il piano di miglioramento continuo sia sostenibile con le risorse disponibili.
	\newline\newline
	
	\newline\newline
	Osservando a ritroso tutto il nostro percorso effettuato all’interno di questo progetto, è visibile l’incrementale acquisizione di competenze, conoscenze e abilità; i tre metri di misura principali per riconoscere le qualifiche ottenute da una persona durante un percorso formativo, come questo progetto.
	\newline
	Grazie ad un auto analisi dei ruoli svolti all’interno del progetto è stato possibile migliorare nel tempo le loro funzioni. All’inizio del progetto, era stato fornito cosa ogni ruolo dovesse svolgere, ma non era subito chiaro come invece dovesse essere svolto. Tali lacune sono state poi riempite grazie alla pratica.
	\newline
	Inserire metriche sempre più granulari è stato utile per poter analizzare aspetti del progetto all’apparenza futili, ma che si sono rivelate importanti per capire dove spendere più risorse nelle fasi di verifica.
	\newline
	In conclusione è stato utile tracciare i risultati delle varie metriche per avere una visione generale, ma anche specifica, dell’andamento del progetto. Ancora di più utile è stata l’autovalutazione dei vari membri del gruppo attraverso suggerimenti reciproci i quali ci hanno aiutato a restare un gruppo unito e con lo stesso livello di conoscenze, evitando disparità.
