% 1- introduzione (da decidere)
% 2- ragionamento sul funzionamento ipotetico del pdca e degli obiettivi di miglioramento
% 3- valutazione critica dei processi che hanno subito miglioramenti
% 4- obiettivi di miglioramento di fatto raggiunti

\section{Valutazioni finali sul progetto}
	In questa sezione vengono riportate le valutazioni fatte in seguito ad un'analisi retrospettiva dei processi istanziati dall'inizio alla fine del progetto.
		
	Osservando a ritroso tutto il nostro percorso effettuato all’interno di questo progetto, è visibile l’incrementale acquisizione di competenze, conoscenze e abilità; i tre metri di misura principali per riconoscere le qualifiche ottenute da una persona durante un percorso formativo, come questo progetto.
	\newline
	Grazie ad un auto analisi dei ruoli svolti all’interno del progetto è stato possibile migliorare nel tempo le loro funzioni. All’inizio del progetto, era stato fornito cosa ogni ruolo dovesse svolgere, ma non era subito chiaro come invece dovesse essere svolto. Tali lacune sono state poi riempite grazie alla pratica.
	\newline
	Inserire metriche sempre più granulari è stato utile per poter analizzare aspetti del progetto all’apparenza futili, ma che si sono rivelate importanti per capire dove spendere più risorse nelle fasi di verifica.
	\newline
	In conclusione è stato utile tracciare i risultati delle varie metriche per avere una visione generale, ma anche specifica, dell’andamento del progetto. Ancora di più utile è stata l’autovalutazione dei vari membri del gruppo attraverso suggerimenti reciproci i quali ci hanno aiutato a restare un gruppo unito e con lo stesso livello di conoscenze, evitando disparità.
