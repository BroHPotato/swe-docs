% 1- introduzione alla sezione
% 2- ragionamento sul funzionamento ipotetico del pdca e degli obiettivi di miglioramento
% 3- introduzione alla valutazione critica (andamento miglioramenti e cambiamenti nelle appendici B e C)
% 4- valutazione critica dei processi che hanno subito miglioramenti (Configurazione, Organizzazione, Pianificazione (comprendente i ruoli))
% 5- obiettivi di miglioramento di fatto raggiunti

\section{Valutazioni finali sul progetto}
	In questa sezione vengono riportate alcune valutazioni fatte in seguito ad un'analisi retrospettiva di tutto il lavoro svolto dall'inizio del progetto. In particolare sono stati fatti dei ragionamenti critici riguardo i miglioramenti sperimentati nei diversi processi istanziati, in modo da comprendere meglio quale sia stata l'evoluzione del \textit{way of working} e quali benefici ne ha tratto il gruppo.
	\newline\newline
	Il modo migliore per garantire l'efficacia a lungo termine dei miglioramenti è applicare il ciclo PDCA, basandolo su specifici obiettivi di avanzamento quantificabili, relativi agli aspetti dei singoli processi di maggior rilevanza ai fini del progetto. Questo approccio permette di attuare dei miglioramenti proattivi, che puntino a potenziare le buone prassi già fissate, prima dell'insorgere di problematiche che necessiterebbero correzioni reattive, realizzando quindi un miglioramento continuo.
	\newline
	Durante i mesi di progetto, però, questo approccio non è stato seguito a causa dell'inesperienza del gruppo che non ha permesso di capirne pienamente l'utilità; invece, sono state spesso privilegiate azioni correttive, adottate in risposta agli errori riscontrati durante l'avanzamento del progetto. Quest'ultime, sebbene spesso tardive ed meno efficaci in ottica di miglioramento continuo, hanno il vantaggio di portare costi inferiori, in termini di risorse, rispetto alle azioni proattive.
	\newline
	Per poter applicare il ciclo PDCA, gli obiettivi di miglioramento devono essere selettivi, ossia intervenire su specifici aspetti dei processi, e misurabili, in modo da capire se e quando il loro scopo viene raggiunto. Di conseguenza, essi vanno scelti oculatamente, valutando opportunamente il rapporto costi/benefici, per fare in modo che il piano di miglioramento continuo sia sostenibile in abse alle risorse disponibili.
	\newline\newline
	Nonostante non siano stati posti specifici obiettivi e siano state applicate principalmente azioni correttive, durante lo sviluppo del progetto è stato comunque possibile apprezzare alcuni miglioramenti a specifiche attività dei processi che compongono il \textit{way of working} del gruppo. Come mostra il grafico HOHO, il numero di valutazioni di miglioramento e cambiamenti era inizialmente molto basso, a causa dell'inesperienza del gruppo e della sua mancanza di capacità autocritica nelle prime settimane, nelle quali, oltretutto, il \textit{way of working} del team era ancora in fase di definizione. Tale numero è poi cresciuto molto nelle fasi successive, sia a causa di una maggior esperienza e familiarità con le attività da svolgere, che ha permesso uno sguardo critico sul modo in cui erano eseguite, che per le osservazioni puntuali ricevute dai committenti. La curva risulta poi sempre discendente, ad indicare un progressivo raffinamento del metodo di lavoro del gruppo, che ha portato ad aver sempre meno bisogno di miglioramenti: il lavoro svolto ha quindi portato dei benefici non solo ai prodotti, ma anche ai processi.
	\newline\newline
	Dei processi a cui sono state apportate più migliorìe nel corso del progetto è sono gli Organizzativi. Nelle prime settimane il gruppo si riuniva spesso e in incontri molto lunghi, perché era necessario conoscersi, oltre che decidere molti dettagli del metodo di lavoro e delle attività da svolgere. Nelle fasi centrali del progetto queste esigenze si sono ridotte, e quando, a causa dell'emergenza sanitaria, il team non ha più potuto riunirsi di presenza, tutti i membri sono stati in grado di adattarsi alla situazione senza che questo causasse rallentamenti allo sviluppo. Il gruppo si è impegnato a sfruttare al massimo le potenzialità degli strumenti di comuncazione in uso; è stata regolarizzata la frequenza degli incontri e la loro durata è stata limitata al necessario per risolvere dubbi e prendere decisioni, aspetto molto più difficile da controllare negli incontri di presenza. Riguardo le attività di sviluppo  vere e proprie, l'uso di un \textit{webhook} ha permesso di collegare la CI ad un bot di Slack, in modo che tutti i membri venissero prontamente informati, tramite notifica, ad ogni fallimento della \textit{continuous integration} nei vari submodules del repository del progetto. 
	\\In generale, tutti gli elementi del gruppo hanno garantito la propria reperibilità: ciò ha permesso di risolvere rapidamente problemi ed incomprensioni e di non rimandare decisioni collettive. In questo modo, nonostante lavorare a distanza rischiasse di danneggiare le comunicazioni interne, il gruppo è riuscito a mantenere costante il ritmo di lavoro e a sfruttare meglio il tempo. In modo analogo sono state gestiti anche i rapporti con il proponente: grazie a videochiamate, chat mediante un apposito canale Slack e scambio di mail, nemmeno il rapporto con l'azienda ha risentito della mancanza di incontri di persona.
	\newline
	Modifiche e miglioramenti sono stati apportati in larga misura anche a tutto ciò che riguardava l'attività di Pianificazione: sia ai ruoli di progetto che alla effettiva pianificazione di progetto del gruppo, contenuta nel \textit{Piano di progetto}.
	\\Nonostante fosse noto sin dall'inizio cosa spettasse ai diversi ruoli, addentrandosi nel progetto è stato necessario rispondere con prontezza alle nuove attività che si presentavano, assegnandole al ruolo giusto; inoltre, l'emergenza sanitaria ha costretto il \textit{Responsabile di progetto} e l'\textit{Amministratore di progetto} a riorganizzare risorse e strumenti in modo da fronteggiare al meglio la situazione. Anche il ruolo di \textit{Verificatore} si è evoluto, soprattutto grazie all'esperienza accumulata nei mesi e al supporto di strumenti di verifica adatti.
	\\Su segnalzione dei committenti, il gruppo ha più volte rivisto la struttura della propria pianificazione di progetto. Il primo cambiamento ha riguardato la divisione degli obiettivi di sviluppo in incrementi, e la riorganizzazione della pianificazione in funzione di questa modifica: ad ogni incremento è stato assegnato un periodo, le attività e il monte ore necessari al raggiungimento degli obiettivi. La divisione degli obiettivi in gruppi piccoli ha aiutato il gruppo a focalizzarsi sullo sviluppo di un ristretto numero di funzionalità alla volta, senza dispersione di energie; la scansione degli incrementi ha anche permesso di regolarizzare le attività di verifica e rilevazione delle metriche su processi e prodotti. Tuttavia, poiché questa divisione troppo minuta rischiava di frammentare lo sviluppo e di far perdere la visione complessiva del prodotto a livello di obiettivi di avanzamento, gli incrementi sono stati poi raggruppati in macrofasi: in questo modo si sono avute sia la visione di dettaglio che quella di più ampio respiro.
	\newline
	Per il processo di Gestione della  configurazione, il gruppo ha lavorato maggiormente su tre aspetti: lo schema di versione, l'automazione dei workflow con \textit{continuous integration} e l'automazione del caricamento in rete e aggiornamento dell'architettura.
	\\Il numero di versione assegnato secondo lo schema iniziale non dava alcuna garanzia rispetto ad aggiunte o modifiche al prodotto non verificate, quindi potenzialmente sbagliate. Questo problema è stato risolto permettendo uno scatto di versione ad ogni modifica o aggiunta \textbf{debitamente verificata e approvata}. Contestualmente, è stata fissata anche una notazione che legasse la versione alla pianificazione per incrementi: questa ulteriore scansione del numero di versione per obiettivi di sviluppo (quelli degli incrementi) permetteva inoltre di unificare la versione di prodotti software e documentali. L'ultimo perfezionamento allo schema di versione ha visto il passaggio ad un sistema ispirato a quello semantico, per cui lo scatto di versione è determinato dall'impatto delle modifiche sul contenuto del prodotto già presente.
	\\Passando agli strumenti per la verifica, l'implementazione della \textit{continuous integration} tramite GitHub Actions ha permesso di innescarne l'attivazione ad ogni push relativo al submodule cui essa riferiva; questa prassi garantiva un monitoraggio costante dello stato del prodotto software. Inoltre, un contorllo automatico impediva l'aggiunta di nuovo codice se quello presente nel submodule non completava con successo la CI.
	\\Roba mr_wolf che non capisco benissimo.
	
	
	\newline\newline
	Osservando a ritroso tutto il nostro percorso effettuato all’interno di questo progetto, è visibile l’incrementale acquisizione di competenze, conoscenze e abilità, i tre metri di misura principali per riconoscere le qualifiche ottenute da una persona durante un percorso formativo, come questo progetto.
	\newline
	Grazie ad un'analisi dei ruoli ricoperti all’interno del progetto è stato possibile migliorare nel tempo le loro funzioni. All’inizio del progetto, era stato fornito cosa ogni ruolo dovesse svolgere, ma non era subito chiaro come invece dovesse essere svolto. Tali lacune sono state poi colmate grazie alla pratica.
	\newline
	Inserire metriche sempre più granulari è stato utile per poter analizzare aspetti del progetto all’apparenza di secondaria importanza, ma che si sono rivelate significativi per capire dove spendere più risorse nelle fasi di verifica.
	\newline
	In conclusione è stato utile tracciare i risultati delle varie metriche per avere una visione generale, ma anche specifica, dell’andamento del progetto. Ancora di più utile è stata l’autovalutazione dei vari membri del gruppo attraverso suggerimenti reciproci, che hanno permesso di restare un gruppo unito e con lo stesso livello di conoscenze, evitando disparità.
