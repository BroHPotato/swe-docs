\section{Introduzione}
	\subsection{Scopo del documento}
		Il documento ha lo scopo di presentare i metodi di verifica e validazione adottate dal gruppo Red Round Robin per garantire la qualità di prodotto e di processo. Per assicurarci di ciò viene applicato un sistema di verifica continua che permettere l'individuazione di eventuali errori in modo da poterli individuare e risolvere efficacemente limitando gli sprechi di tempo.
	\subsection{Scopo del prodotto}
		Lo scopo del prodotto è lo sviluppo di una web-application che, una volta ricevute delle misurazioni da uno o più sensori eterogenei, sia in grado di fornire ad un specifico ente il monitoraggio dei sensori stessi e di offrire un servizio di previsione basato sull'andamento di dati, come ad esempio prevedere un deterioramento complessivo tale da generare una necessaria azione di manutenzione preventiva, notificando il tutto nella web-application stessa.
	\subsection{Glossario}
		Per evitare eventuali disguidi relativi ai termini utilizzati nei documenti viene fornito il glossario, nel quale al suo interno verranno descritti i significati delle parole indicate con una G a pendice. Questo documento verrà aggiornato durante tutto il periodo di svolgimento del progetto.
	\subsection{Riferimenti}
		\subsubsection{Normativi}
		\begin{itemize}
			\item \textbf{Norme di progetto:} Norme di Progetto - v1.0.0;
			\item \textbf{Capitolato d’appalto C6 - ThiReMa - Things Relationship Management:}\\
			\url{https://www.math.unipd.it/~tullio/IS-1/2019/Progetto/C6.pdf}
		\end{itemize}
		\subsubsection{Informativi}
		\begin{itemize}
			\item \textbf{ISO/IEC 9126:}\\
			\url{https://en.wikipedia.org/wiki/ISO/IEC_9126}
			\item \textbf{ISO/IEC 12207:}\\
			\url{https://www.math.unipd.it/~tullio/IS-1/2009/Approfondimenti/ISO_12207-1995.pdf}
			\item \textbf{Indice di Gulpease:}\\
			\url{https://it.wikipedia.org/wiki/Indice_Gulpease}
			\item \textbf{Slide del corso di Ingegneria del Software, qualità del software:}\\
			\url{https://www.math.unipd.it/~tullio/IS-1/2019/Dispense/L12.pdf}
			\item \textbf{Slide del corso di Ingegneria del Software, qualità di processo:}\\
			\url{https://www.math.unipd.it/~tullio/IS-1/2019/Dispense/L13.pdf}
		
		\end{itemize}
	