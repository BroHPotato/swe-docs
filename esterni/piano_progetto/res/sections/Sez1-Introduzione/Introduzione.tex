\section{Introduzione}

	\subsection{Scopo del documento}
		Lo scopo del piano di progetto è organizzare le attività, analizzandone i rischi, suddividendole ed assegnandole ai vari membri del gruppo, in modo tale da raggiungere i risultati nel modo più efficace ed efficiente possibile.
	\subsection{Scopo del prodotto}
		Il capitolato C6 si pone come obiettivo creare una web application che permette di analizzare grosse moli di dati ricevuti da sensori eterogenei tra loro. Tale applicazione mette a disposizione un'interfaccia che permette di visualizzare alcuni dati di interesse od eventuali correlazioni tra i dati stessi. Infine, per ogni tipologia di dato è possibile assegnarne il monitoraggio ad un particolare ente, ruolo o gruppo.
	\subsection{Glossario e documenti esterni}
		Per evitare possibili ambiguità relative alle terminologie (che andranno indicate in \textsc{maiuscoletto}) utilizzate nei vari documenti, verranno utilizzate due simboli:
		\begin{itemize}
			\item una \textit{D} al pedice per indicare il nome di un particolare documento;
			\item una \textit{G} al pedice per indicare un termine che sarà presente nel \dext{Glossario v2.0.0}.
		\end{itemize}
	\subsection{Riferimenti}
		\subsubsection{Normativi}
		\begin{itemize}
			\item \textbf{norme di progetto:} \dext{Norme di Progetto v2.0.0}
		\end{itemize}

	\subsection{Scadenze}
	\label{riferimento_scadenze}
		Dopo una prima pianificazione, il gruppo Red Round Robin ha deciso di sostenere le revisioni con i committenti nelle seguenti date:
		\begin{itemize}
			\item \textbf{Revisione dei Requisiti}: 21 gennaio 2020;
			\item \textbf{Revisione di Progettazione}: 16 marzo 2020;
			\item \textbf{Revisione di Qualifica}: 20 aprile 2020;
			\item \textbf{Revisione di Accettazione}: 18 maggio 2020.
		\end{itemize}
		
