\section{Analisi dei rischi}
	\subsection{Gestione dei rischi}
    Quando si lavora ad un progetto di dimensioni importanti, è facile incorrere in problemi che possono portare allo sforamento dei costi preventivati per realizzarlo, allo sforamento dei tempi prefissati per consegnarlo al committente oppure all'ottenimento di risultati di basso livello. \newline
    Per evitare di incorrere in queste situazioni, si deve attuare un processo di gestione dei rischi, composto da diverse attività descritte nel seguito.

	\subsubsection{Identificazione}
    Questa attività è necessaria per trovare tutti i possibili fattori di rischio che possano causare i problemi descritti nella \textit{gestione dei rischi}. 

	\subsubsection{Analisi}
	Lo scopo dell'\textit{analisi} è di analizzare i rischi trovati, assegnandovi un indice di gravità proporzionale alla probabilità che si verifichino e alle conseguenze che avrebbero sul progetto.

	\subsubsection{Pianificazione}
    La \textit{pianificazione} definisce in modo rigoroso come evitare di incorrere nei rischi individuati e, qualora non fosse possibile evitarne alcuni, come limitarne gli effetti sul proseguo del progetto.

	\subsubsection{Controllo} 
    Per sapere se si sono presentati dei rischi, è indispensabile controllare periodicamente quanto prodotto dai membri del gruppo per poterli rilevare.

	\newpage
		
\subsection{Elenco dei rischi preventivati}

\begin{center}
	\rowcolors{2}{white}{lightest-grayest}
	\begin{longtable}{|p{3cm}|p{4cm}|p{3.5cm}|p{3.5cm}|}
		\hline
		\rowcolor{lighter-grayer}
		\textbf{Codice} & \textbf{Descrizione} & \textbf{Identificazione} & \textbf{Mitigazione} \\
		\hline
		\endfirsthead
	    \hline
	    \multicolumn{4}{|c|}{\textit{Continua nella pagina successiva...}}\\
	    \hline
	    \endfoot
	    \endlastfoot

		\hline
		RSK-TAI-1 \newline Tecnologie previste dal capitolato non conosciute  
		& 
		Nel corso del \textit{processo di sviluppo}
		è necessario usare diverse tecnologie
		che molti componenti del gruppo non conoscono, questo è
		un evidente ostacolo alla realizzazione del prodotto.
		&  
		Il \textit{responsabile} avrà come compito quello di parlare coi membri
		   del gruppo per cercare di capire le lacune di ciascuno.
		& 
		Si cercherà di assegnare compiti che richiedano un alto livello di conoscenza di queste tecnologie a più persone,
		  in modo da alleggerirne il carico favorendo il confronto e la condivisione
		  delle informazioni acquisite da ciascuno in autonomia su di esse. \\
		\hline
		
		RSK-OAI-1 \newline Mancato soddisfacimento dei compiti assegnati 
		& 
		Essendo il primo progetto software impegnativo nel quale il nostro gruppo si cimenta, data l'inesperienza e la presenza di nuove tecnologie da padroneggiare, è altamente probabile che determinati compiti non vengano svolti entro le \glock{milestone} prefissate. 
		&  
		I membri che dovessero incontrare questo tipo di difficoltà, sono tenuti a comunicarlo con anticipo agli altri.
		& 
		Per far fronte a questa problematica, i compiti che rischino di non essere portati a termine entro le \glock{milestone} verranno riassegnati a uno o più membri, al fine di evitare ritardi. \\
		\hline
		
		RSK-OAT-1 \newline  Stime dei costi del progetto
		& 
		Questo è il primo progetto che richiede al team di fare una stima sui costi di realizzazione del prodotto finale, pertanto il rischio di non rientrare nel range preventivato è fondato.
		& 
		Ciascun elemento del team si impegna a rendicontare le ore effettive di lavoro e di sottoporle al \textit{responsabile} di progetto per permettergli di tenere sotto controllo il budget residuo.
		&
		Contando sulla veridicità dei dati relativi alle ore di lavoro spese in corso d'opera dalle figure professionali del team, il \textit{responsabile} sarà in grado di sollecitarli a cercare di essere più produttivi - se necessario - per evitare lo sforamento del budget.	
		  \\
		\hline
		
		RSK-TBI-1 \newline Cancellazione o modifica di documenti o moduli software già approvati & 
		Il team utilizza, per la condivisione di tutto il materiale prodotto, un \glock{vcs}, per cui modifiche non previste sul cloud comportano inconsistenza o possono introdurre \glock{regressioni}.
		&  
		Il sistema \glock{vcs} scelto notifica tutti i membri del team ad ogni modifica sul server tramite email.   
		& 
		Tramite il meccanismo di \glock{pull request}, viene assegnato a un revisore il compito di controllare se queste modifiche siano coerenti e quindi consentirne la scrittura sul server, oppure rigettarle in caso contrario.   \\
		\hline
		
		RSK-OAI-2 \newline Impegni accademici & 
		Ci saranno inevitabilmente momenti in cui uno o più membri del gruppo non possano svolgere compiti a causa di impegni accademici impellenti.
		&  In assoluta trasparenza, ciascun membro è tenuto a sollevare tali questioni con anticipo accettabile tramite i canali di comunicazione utilizzati dal team.
		 
		& Durante queste fasi il \textit{responsabile} cercherà di assegnare compiti ai soli membri del team che non abbiano impegni. \\
		\hline
		
		RSK-OAI-3 \newline Impegni personali & 
		Potrebbero esserci momenti in cui alcuni membri del gruppo non siano in grado di contribuire al progetto per motivi personali.
		&  Ciascun membro è tenuto a sollevare tali questioni con anticipo accettabile tramite i canali di comunicazione utilizzati dal team.
		& Il \textit{responsabile}, se necessario per andare incontro alle necessità di questi colleghi, ridistribuirà i compiti assegnati o prevederà per essi una proroga. \\
		\hline
		
		RSK-PBI-1 \newline Comunicazione col proponente
		& 
		Sarà indispensabile dover comunicare col proponente per chiedere chiarimenti o discutere sulla validità delle scelte prese dal gruppo per la realizzazione del prodotto.
		&  
		Nessuna o tardiva risposta alle mail o ai messaggi inviati sul canale \glock{Slack} predisposto per la comunicazione tra gruppo e azienda proponente.
		& 
		Si provvederà, se necessario, a comunicare il problema al committente.  \\
		\hline

		RSK-PBI-2 \newline Comunicazione tra i membri del team
		& 
		Può succedere che qualche membro non sia reperibile per un lasso di tempo significativo, senza motivazione preventivamente fornita ai colleghi.
		&   
		Nessuna risposta alle mail o ai messaggi inviati sui canali predisposti alle comunicazioni interne del gruppo.
		& 
		In maniera costruttiva, il \textit{responsabile}
		   cercherà di parlare con i colleghi che siano stati irreperibili per un lasso di tempo significativo, in modo di cercare di evitare si ripresentino queste spiacevoli situazioni in futuro. \\
		\hline
		
		RSK-TBT-1 \newline Malfunzionamenti hardware o software dei pc dei membri del gruppo
		& 
		Può accadere che qualcuno abbia il PC non funzionante o in assistenza e non possa contribuire attivamente alla realizzazione del prodotto per un certo periodo.
		&   
		Segnalazione da parte di chi ha questo tipo di problema sui canali di comunicazione del team.
		& 
		Chi si ritrovi con questo tipo di problema, deve poter riprendere a svolgere i propri compiti su un'altra macchina, ad esempio sfruttando quelle a disposizione degli studenti dell'università nei laboratori informatici. \\
		\hline

		RSK-OMI-1 \newline Concorrenza sul capitolato
		& 
		Il capitolato di interesse può essere scelto da altri team, ed poiché il proponente accetta un numero limitato di gruppi a cui assegnare la sua realizzazione, subentra il rischio di non essere scelti come fornitori.
		&   
		Comunicazione preventiva sul tema con i responsabili degli altri team.
		& 
		In caso ci siano più fornitori di quanti sia disposto ad accettarne il proponente, il gruppo punterà a consegnare la propria documentazione di candidatura al committente entro la prima data disponibile. \\
		\hline


		RSK-PMT-1 \newline Influenza e malattie 
		& 
		Durante lo svolgimento del progetto, può accadere che uno o più membri del gruppo siano esposti a malattie o virus influenzali che abbassano conseguentemente il livello di attenzione e la produttività del singolo. 
		&   
		Segnalazione sui canali di comunicazione da parte di chi presenta influenza o malattie particolari, tali da non permettere di adempire alle attività assegnate.
		& 
		Il \textit{responsabile}, una volta informato, provvederà a riassegnare i compiti ad altri membri del gruppo o rivaluterà l'assegnazione di compiti più leggeri per chi presenta malattie o sintomi influenzali. \\
		\hline

		RSK-OBT-1 \newline Riunioni col proponente 
		& 
		Nel corso del progetto è possibile che il proponente non sia disponibile a effettuare riunioni di persona o virtuali col fine di discutere il prodotto realizzato fino a quel momento.
		&   
		Il proponente emana un avviso nei canali di comunicazioni ufficiali, quali \glock{Slack} e via email.
		& 
		Il \textit{responsabile} contatterà il proponente per concordare in modo tempestivo nuove conferenze con maggiore anticipo. Verranno poi concordate nuove modalità di interazione con il proponente per metterlo al corrente degli incrementi realizzati e per richiedere dei feedback nel modo meno invasivo possibile (es. video dimostrativi). \\
		\hline
		RSK-TMI-1 \newline Configurazione corretta dell'ambiente di lavoro  
		& 
		Nel corso del \textit{processo di sviluppo}
		sarà necessario configurare un diverso numero di IDE con
		i relativi plugin per eseguire controlli di verifica della codifica.
		Questo si rende particolarmente necessario a causa dei diversi linguaggi di programmazione 
		usati nelle componenti software di progetto.
		&  
		Notifica sui canali di comunicazione del fallimento dei processi automatici di controllo (tramite \glock{Github Actions}).
		& 
		L'\textit{amministratore} viene messo al corrente dell'accaduto e contatterà direttamente il programmatore che presenta questo genere
		di problema, accertando insieme a lui che tutto sia stato configurato correttamente, come riportato nella normativa.  \\
		\hline
		RSK-TMI-2 \newline Conoscenza superficiale dei design pattern e degli stili architetturali  
		& 
		Nel corso della \textit{progettazione} del software
		sarà necessario studiare a fondo le scelte architetturali da fare, per garantire una buona manutenibilità del codice nel corso del tempo e per fare fronte alle best pratices di programmazione.
		&  
		Confronto collettivo sulle scelte architetturali duranti le riunioni interne all'inizio di ogni incremento.
		& 
		Il \textit{responsabile} dovrà far dialogare i progettisti per trovare una soluzione comune sui design pattern da adottare e sugli stili architetturali da scegliere. Se opportuno, dovrà segnalare il materiale didattico da studiare per approfondire tali argomenti. \\
		\hline
	
		\caption{Tabella contenente l'elenco dei rischi preventivati}
	\end{longtable}

\end{center}
