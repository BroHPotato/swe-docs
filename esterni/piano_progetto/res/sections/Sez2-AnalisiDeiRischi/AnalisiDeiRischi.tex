\section{Analisi dei rischi}
	\subsection{Gestione dei rischi}
Quando si lavora ad un progetto di dimensioni importanti, è facile incorrere in problemi che possono portare allo sforamento dei costi preventivati per realizzarlo, allo sforamento dei tempi prefissati per consegnarlo al committente oppure all'ottenimento di risultati di basso livello. \newline
Per evitare di incorrere in queste situazioni, si deve attuare un processo di gestione dei rischi, composto da diverse attività descritte nel seguito.

	\subsubsection{Identificazione}
Questa attività è necessaria per trovare tutti i possibili fattori di rischio che possano causare i problemi descritti nella \textit{gestione dei rischi}.

	\subsubsection{Analisi}
	 Lo scopo dell'\textit{analisi} è di analizzare i rischi trovati, assegnandovi un indice di gravità proporzionale alla probabilità che si verifichino e alle conseguenze che avrebbero sul progetto.

	\subsubsection{Pianificazione}
La \textit{pianificazione} definisce in modo rigoroso come evitare di incorrere nei rischi individuati e, qualora non fosse possibile evitarne alcuni, come limitarne gli effetti sul proseguo del progetto.

	\subsubsection{Controllo} 
 Per sapere se si sono presentati dei rischi, è indispensabile controllare periodicamente quanto prodotto dai membri del gruppo per poterli rilevare.

	\newpage
		
\subsubsection{Elenco dei rischi rilevati}

\begin{center}
	\rowcolors{2}{lightest-grayest}{white}
	\begin{longtable}{|p{3cm}|p{4cm}|p{3.5cm}|p{3.5cm}|}
	\hline
	\rowcolor{lighter-grayer}
	\textbf{Codice} & \textbf{Descrizione} & \textbf{Identificazione} & 
														\textbf{Mitigazione} \\
	\hline
	\endfirsthead

	% ----- Modificare da qui -----

	% 0.0.2 & Revisione documento & 24-11-2019 & Nome Cognome & Verificatore \\
	\hline
	RSK-TAI-1 \newline Tecnologie previste dal capitolato non conosciute  & 
	Nel corso del \textit{processo di sviluppo}
	è necessario usare diverse tecnologie
	che molti componenti del gruppo non conoscono, questo è
	un evidente ostacolo alla realizzazione del prodotto.
	&  Il \textit{responsabile} avrà come compito quello di parlare coi membri
	   del gruppo per cercare di capire le lacune di ciascuno.
	 
	& Si cercherà di assegnare compiti che richiedano un alto livello di conoscenza di queste tecnologie a più persone,
	  in modo da alleggerirne il carico favorendo il confronto e la condivisione
	  delle informazioni acquisite da ciascuno in autonomia su di esse. \\
	\hline
	
		\hline
	RSK-OAI-1 \newline Mancato soddisfacimento dei compiti assegnati & 
	Essendo il primo progetto software impegnativo nel quale il nostro gruppo si cimenta, data l'inesperienza e la presenza di nuove tecnologie da padroneggiare, è altamente probabile che determinati compiti non vengano svolti entro le \glock{milestone} prefissate. 
	&  I membri che dovessero incontrare questo tipo di difficoltà, sono tenuti a comunicarlo con anticipo agli altri.
	 
	& Per far fronte a questa problematica, i compiti che rischino di non essere portati a termine entro le \glock{milestone} verranno riassegnati ad uno o più altri membri, al fine di evitare o ridurre ritardi. \\
	\hline
	
		\hline

	RSK-OAT-1 \newline  Stime dei costi del progetto
	 & 
	Questo è il primo progetto che richiede al team di fare una stima sui costi di realizzazione del prodotto finale, pertanto il rischio di non rientrare nel range preventivato è fondato.
	& 
	Ciascun elemento del team si impegna a rendicontare le ore effettive di lavoro e di sottoporle al \textit{responsabile} di progetto per permettergli di tenere sotto controllo il budget residuo.
	&
	Contando sulla veridicità dei dati relativi alle ore di lavoro spese in corso d'opera dalle figure professionali del team, il \textit{responsabile} sarà in grado di sollecitarli a cercare di essere più produttivi - se necessario - per evitare lo sforamento del budget.	
	  \\
	\hline
	
		\hline
	RSK-TBI-1 \newline Cancellazione o modifica di documenti o moduli software già approvati & 
	Il team utilizza, per la condivisione di tutto il materiale prodotto, un \glock{vcs}, per cui modifiche non previste sul cloud comportano inconsistenza o possono introdurre \glock{regressioni}.
	&  Il sistema \glock{vcs} scelto notifica tutti i membri del team ad ogni modifica sul server tramite email.   
	 
	& Tramite il meccanismo di \glock{pull request}, viene assegnato a un revisore il compito di controllare se queste modifiche siano coerenti e quindi consentirne la scrittura sul server, oppure rigettarle in caso contrario.   \\
	\hline
	
		\hline
	RSK-OAI-2 \newline Impegni accademici & 
	Ci saranno inevitabilmente momenti in cui uno o più membri del gruppo non possano svolgere compiti a causa di impegni accademici impellenti.
	&  In assoluta trasparenza, ciascun membro è tenuto a sollevare tali questioni con anticipo accettabile tramite i canali di comunicazione utilizzati dal team.
	 
	& Durante queste fasi il \textit{responsabile} cercherà di assegnare compiti ai soli membri del team che non abbiano impegni. \\
	\hline
	
		\hline
	RSK-OAI-3 \newline Impegni personali & 
	Potrebbero esserci momenti in cui alcuni membri del gruppo non siano in grado di contribuire al progetto per motivi personali.
	&  Ciascun membro è tenuto a sollevare tali questioni con anticipo accettabile tramite i canali di comunicazione utilizzati dal team.
	 
	& Il \textit{responsabile}, se necessario per andare incontro alle necessità di questi colleghi, ridistribuirà i compiti preassegnati o prevederà per essi una proroga. \\
	\hline
	
		\hline
	RSK-PBI-1 \newline Comunicazione col proponente
	 & 
	Sarà indispensabile dover comunicare col proponente per chiedere chiarimenti o discutere sulla validità delle scelte prese dal gruppo per la realizzazione del prodotto.
	&  Nessuna o tardiva risposta alle mail o ai messaggi inviati sul canale \glock{Slack} predisposto per la comunicazione tra gruppo e azienda proponente.
	 
	& Si provvederà, se necessario, a comunicare il problema al committente.  \\
	\hline
	
		\hline
	RSK-PBI-2 \newline Comunicazione tra i membri del team
	 & 
	Può succedere che qualche membro non sia reperibile per un lasso di tempo significativo, senza motivazione preventivamente fornita ai colleghi.
	&   

	Nessuna risposta alle mail o ai messaggi inviati sui canali predisposti alle comunicazioni interne del gruppo.
	 
	& In maniera costruttiva, il \textit{responsabile}
	   cercherà di parlare con i colleghi che siano stati irreperibili per un lasso di tempo significativo, in modo di cercare di evitare si ripresentino queste spiacevoli situazioni in futuro. \\
	\hline
	
			\hline
	RSK-TBT-1 \newline Malfunzionamenti hardware o software dei pc dei membri del gruppo
	 & 
	Può succedere che qualcuno abbia il pc malfunzionante e non possa contribuire attivamente alla realizzazione del prodotto.
	&   

	Segnalazione da parte di chi ha questo tipo di problema sui canali di comunicazione del team.
	 
	& Chi si ritrovi con questo tipo di problema, deve poter riprendere a svolgere i propri compiti su un'altra macchina, ad esempio sfruttando quelle a disposizione degli studenti dell'università nei laboratori informatici. \\
	\hline
	
	
				\hline
	RSK-OMI-1 \newline Concorrenza sul capitolato
	 & 
	Il capitolato di interesse può essere scelto da altri team, ed poiché il proponente accetta un numero limitato di gruppi a cui assegnare la sua realizzazione, subentra il rischio di non essere scelti come fornitori.
	&   

	Comunicazione preventiva sul tema con i responsabili degli altri team.
	 
	& In caso ci siano più fornitori di quanti sia disposto ad accettarne il proponente, il gruppo punterà a consegnare la propria documentazione di candidatura al committente entro la prima data disponibile. \\
	\hline
	
		\caption{Tabella contenente l'elenco dei rischi preventivati}
	\end{longtable}

\end{center}
