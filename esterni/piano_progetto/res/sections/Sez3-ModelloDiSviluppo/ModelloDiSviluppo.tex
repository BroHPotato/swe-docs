\section{Modello di sviluppo}
	Per poter svolgere correttamente il progetto ed effettuare la pianificazione, è necessario adottare un modello di sviluppo che, in base alle sue caratteristiche, imporrà dei vincoli alla pianificazione stessa. Una prerogativa del gruppo è qualità, la quale deve riflettersi anche nel modello di sviluppo, in modo da poter raggiungere gli obiettivi posti dal modello stesso e quindi portare avanti lo sviluppo in modo corretto e coerente con esso.
	\newline
	Partendo da queste considerazioni e valutando la natura del progetto è stato adottato il \textbf{modello di sviluppo incrementale}, il quale prevede lo sviluppo del prodotto tramite incrementi multipli e successivi, ossia dei rilasci che realizzano ciascuno una nuova funzionalità che viene integrata nel sistema.
	\newline
	Nel modello di sviluppo incrementale i requisiti vengono classificati in base alla loro importanza strategica a livello di sistema. I requisiti più importanti sono trattati dai primi incrementi, in modo da renderli chiari e stabili nel minor tempo possibile per poterli poi soddisfare con maggiore facilità. Gli incrementi successivi coprono, quindi, requisiti meno importanti che hanno quindi più tempo per integrarsi col sistema.
	\newline
	Sebbene il modello di sviluppo non lo preveda, considerando il numero di componenti e di funzionalità che realizzano il sistema, sono consentite modifiche, aggiunte e rimozioni di requisiti. Tali operazioni sono possibili solamente previa valutazione ed approvazione da parte del proponente. Queste modifiche non possono essere discusse durante lo sviluppo di un incremento, è necessario prima effettuare il rilascio e poi valutare il cambiamento dei requisiti.
	\newline
	I vantaggi del modello di sviluppo incrementale sono i seguenti:
	\begin{itemize}
		\item ogni incremento produce valore aggiunto, rendendo disponibili delle nuove funzionalità ed chiarendo meglio i requisiti per gli incrementi successivi;
		\item ad ogni incremento è possibile ricevere in tempi brevi un feedback da parte del proponente sull'insieme delle funzionalità sviluppate;
		\item le funzionalità principali vengono sviluppate all'inizio con i primi incrementi, in quanto relative ai requisiti più importanti;
		\item ad ogni incremento vengono svolte attività di verifica rivolte specialmente aggiunte/modifiche, rendendo l'intera verifica più semplice ed economica, in quanto il resto del prodotto era già stato testato con gli incrementi precedenti e gli errori sono limitati all'incremento attuale;
		\item gli errori in un singolo incremento sono più facili da individuare e correggere, in quanto relativi solo alle modifiche apportate dall'incremento;
		\item ogni incremento riduce il rischio di fallimento.
	\end{itemize}

\subsection{Incrementi pianificati}

Si prevede di andare a svolgere \textbf{12 incrementi} col fine di poter integrare tutte le funzionalità richieste dal progetto. Di seguito si riporta una tabella riassuntiva degli incrementi con una breve descrizione sul relativo svolgimento.

\newpage
\begin{center}
	\rowcolors{2}{lightest-grayest}{white}
	\begin{longtable}{|c|p{10cm}|p{2cm}|}
	\hline
	\rowcolor{lighter-grayer}
	\textbf{Nome} & \textbf{Breve descrizione} & \textbf{Requisiti} \\
	\hline
	\endfirsthead

	Incremento I	& Configurazione di \glock{Apache Kafka}. &  \\	\hline
	Incremento II & Creazione dell'interfaccia del \glock{gateway} e prima implementazione base. Configurazione della struttura dati con JSON. Implementazione del protocollo di comunicazione con un dispositivo. & RA-F-69 RA-F-70 \\	\hline
	Incremento III	& Creazione interfaccia \glock{API} e implementazione operazioni di lettura dei dati con le \glock{API}. & RA-F-65 RA-F-67 \\	\hline
	Incremento IV & Configurazione base web app e reperimento dati dalle \glock{API}. & RA-F-65.1 RA-F-71 RB-F-71.2 \\	\hline
	Incremento V	& Implementazione della configurazione dinamica per un \glock{gateway}. & RA-F-66 RA-F-66.1 \\	\hline
	Incremento VI & Implementazione dei database (relazionale e non relazionale). Configurazione della comunicazione dei database con \glock{Kafka}. & RA-F-64 RA-F-64.1 RA-F-64.2 RA-F-68 \\	\hline
	Incremento VII	& Implementazione base \glock{bot Telegram} per la ricezione di avvisi e configurazione con le \glock{API}. Implementazione del login per la web app. & RA-F-1 RA-F-1.1 RA-F-1.2 RA-F-2 RA-F-3 RA-F-3.1 RA-F-3.2 RA-F-7 RA-F-8 RA-F-8.1 RA-F-8.2 RA-F-8.3 RB-F-8.4 RB-F-8.5 RA-F-34 RA-F-60 RA-F-61 RA-F-62 \\	\hline
	Incremento VIII	& Implementazione dei grafici per la web app. &  RB-F-3.3 RA-F-35 RA-F-36 RA-F-37 RA-F-38 RA-F-39 RA-F-39.1 RB-F-39.2 RA-F-40 RA-F-71.1 \\	\hline
	Incremento IX	& Implementazione della parte utente per la web app. & RA-F-4 RB-F-8.6 RA-F-9 RA-F-11 RB-F-11.1 RA-F-33 RA-F-63 \\	\hline
	Incremento X	& Implementazione della moderazione per gli enti nella web app. & RA-F-5 RA-F-16 RA-F-17 RB-F-18 RB-F-18.1 RB-F-18.2 RB-F-18.3 RA-F-19 RA-F-20 RA-F-28 RA-F-29 RA-F-30  RC-F-41 RA-F-43 RA-F-44 \\	\hline
	Incremento XI   & Configurazione aggiuntiva del \glock{gateway} con \glock{Kafka}. Implementazione completa delle \glock{API}. &  \\ \hline
	Incremento XII	& Implementazione invio input con il \glock{bot Telegram} e implementazione amministrazione web app. & RA-F-6 RA-F-10 RA-F-12 RB-F-12.1 RA-F-13 RA-F-14 RA-F-15 RA-F-21 RA-F-22 RA-F-23 RB-F-24 RB-F-24.1 RB-F-24.2 RB-F-24.3 RB-F-24.4 RB-F-24.5 RB-F-24.6 RA-F-25 RA-F-26 RA-F-27 RA-F-31 RA-F-32 RC-F-42 RB-F-44.1 RA-F-45 RA-F-46 RA-F-47 RA-F-48 RA-F-49 RA-F-50 RA-F-51 RA-F-52 RA-F-53 RA-F-54 RA-F-55 RA-F-56 RA-F-57 RA-F-58 RA-F-59 \\	\hline

	\caption{Riassunto degli incrementi pianificati}
	\end{longtable}
\end{center}
