\section{Introduzione}

	\subsection{Scopo del documento}
	\subsection{Scopo del prodotto}
		Il capitolato C6 si pone come obiettivo quello di creare una web-application che permette di analizzare grosse moli di dati ricevuti da sensori eterogenei tra loro. Tale applicazione mette a disposizione un'interfaccia che permette di visualizzare alcuni dati di interesse od eventuali correlazioni tra i dati stessi. Infine, per ogni tipologia di dato è possibile assegnarne il monitoraggio ad un particolare ente, ruolo o gruppo.
	\subsection{Glossario e Documenti esterni}
		Per evitare possibili ambiguità relative alle terminologie (che andranno indicate in \textsc{maiuscoletto})utilizzate nei vari documenti, verranno utilizzate due simboli:
		\begin{itemize}
			\item Una \textit{D} al pedice per indicare il nome di un particolare documento.
			\item Una \textit{G} al pedice per indicare un termine che sarà presente nel \dext{Glossario v0.0.1}.
		\end{itemize}
	\subsection{Riferimenti}
		\subsubsection{Normativi}
			
		\subsubsection{Informativi}
			

	\subsection{Scadenze}

