\section{Pianificazione}
	Per gestire lo sviluppo del progetto e soddisfare le scadenze riportate nella sottosezione \hyperref[riferimento_scadenze]{§1.5} di questo stesso documento, il gruppo ha deciso di suddividere il periodo che va dalla sua formazione, fino alla Revisione di Accettazione, nelle seguenti quattro attività:
	\begin{itemize}
		\item Analisi dei requisiti
		\item Progettazione della \textit{technology baseline}
		\item Progettazione di dettaglio e Codifica
		\item Validazione e Collaudo
	\end{itemize}
	Ogni attività sarà poi suddivisa in più periodi, ognuno dei quali avrà una \glock{milestone} di riferimento per il completamento dei singoli compiti previsti al suo interno. All'interno di ogni singolo periodo sono, quindi, individuati più compiti, i quali possono essere svolti sia sequenzialmente, sia in parallelo, in base alle dipendenze che possono sussistere tra di loro.
	
	\subsection{Analisi dei requisiti}
		L'attività di analisi dei requisiti inizia il giorno 2019-11-15, dopo la formazione dei gruppi, e termina il giorno 2020-01-20, ossia il giorno prima della Revisione dei Requisiti.
		\subsubsection{Ruoli attivi}
			Durante questa attività è necessaria la presenza dei seguenti ruoli:
			\begin{itemize}
				\item responsabile;
				\item amministratore;
				\item analista;
				\item verificatore.
			\end{itemize}
		\subsubsection{Periodi}
			L'attività di analisi dei requisiti è stata suddivisa nei seguenti periodi:
			\paragraph{Primo periodo (dal 2019-11-15 al 2019-12-06):}
				\begin{itemize}
					\item \textbf{Analisi dei capitolati:} analisi e discussione interna al gruppo per individuare le preferenze dei componenti in modo da indirizzare l'intero gruppo su certi capitolati piuttosto che altri;
					\item \textbf{Impianto normativo:} discussione interna al gruppo circa le regole da adottare durante lo sviluppo del progetto. Inizio della stesura del documento \dext{Norme di Progetto}, in particola le sezioni riguardanti la documentazione, la verifica e la gestione della configurazione;
					\item \textbf{Metriche di qualità:} analisi preliminare riguardante le metriche da utilizzare per controllare la qualità della documentazione;
					\item \textbf{Pianificazione:} organizzazione interna al gruppo riguardo i ruoli da assegnare e i compiti da svolgere durante il periodo attuale;
					\item \textbf{Ricerca:} studio e ricerca autonoma riguardo gli strumenti e le tecnologie da utilizzare per la gestione del progetto;
					\item \textbf{Studio di fattibilità:} inizio della stesura del documento \dext{Studio di Fattibilità}, in base all'analisi dei capitolati fatta in precedenza;
					\item \textbf{Verifica:} controllo della qualità di tutti i prodotti sviluppati durante il periodo attuale.
				\end{itemize}
			\paragraph{Secondo periodo (dal 2019-12-07 al 2019-12-16):}
				\begin{itemize}
					\item \textbf{Impianto normativo:} discussione interna al gruppo circa le regole da adottare durante lo sviluppo del progetto. Continuazione della stesura del documento \dext{Norme di Progetto}, in particola le sezioni riguardanti i processi primari e processi organizzativi;
					\item \textbf{Metriche di qualità:} analisi preliminare riguardante le metriche da utilizzare per controllare la qualità dei processi;
					\item \textbf{Pianificazione:} organizzazione interna al gruppo riguardo i ruoli da assegnare e i compiti da svolgere durante il periodo attuale;
					\item \textbf{Ricerca:} studio e ricerca autonoma riguardo gli strumenti e le tecnologie implicate dal capitolato scelto, da utilizzare durante lo svolgimento del progetto;
					\item \textbf{Scelta del capitolato:} decisione definitiva riguardo il capitolato scelto;
					\item \textbf{Studio di fattibilità:} fine della stesura del documento \dext{Studio di Fattibilità}, in base al capitolato scelto;
					\item \textbf{Verifica:} controllo della qualità di tutti i prodotti sviluppati durante il periodo attuale.
				\end{itemize}
			\paragraph{Terzo periodo (dal 2019-12-17 al 2019-12-31):}
				\begin{itemize}
					\item \textbf{Analisi:} inizio stesura del documento \dext{Analisi dei Requisiti}, in particolare le sezioni riguardanti l'analisi del prodotto ed i casi d'uso;
					\item \textbf{Impianto normativo:} conclusione della stesura del documento \dext{Norme di Progetto}, in particolare le sezioni rimaste incomplete o ancora da trattare;
					\item \textbf{Pianificazione:} organizzazione interna al gruppo riguardo i ruoli da assegnare e i compiti da svolgere durante il periodo attuale;
					\item \textbf{Progetto:} inizio della stesura del documento \dext{Piano di Progetto}, in particolare le sezioni riguardanti l'analisi dei rischi, il modello di sviluppo e la pianificazione;
					\item \textbf{Qualifica:} inizio stesura del documento \dext{Piano di Qualifica}, in particolare le sezioni riguardanti le metriche di qualità normate;
					\item \textbf{Verifica:} controllo della qualità di tutti i prodotti sviluppati durante il periodo attuale.
				\end{itemize}
			\paragraph{Quarto periodo (dal 2020-01-01 al 2020-01-14):}
				\begin{itemize}
					\item \textbf{Analisi:} conclusione della stesura del documento \dext{Analisi dei Requisiti}, in particolare le sezioni riguardanti l'analisi dei requisiti ed il loro tracciamento;
					\item \textbf{Pianificazione:} organizzazione interna al gruppo riguardo i ruoli da assegnare e i compiti da svolgere durante il periodo attuale;
					\item \textbf{Progetto:} conclusione della stesura del documento \dext{Piano di Progetto}, in particolare le sezioni riguardanti il preventivo, il consuntivo ed il riscontro dei rischi individuati;
					\item \textbf{Qualifica:} conclusione della stesura del documento \dext{Piano di Qualifica}, in particolare le sezioni riguardanti i test, il resoconto delle attività di verifica e le valutazioni per il miglioramento;
					\item \textbf{Verifica:} controllo della qualità di tutti i prodotti sviluppati durante il periodo attuale.
				\end{itemize}
			\paragraph{Quinto periodo (dal 2020-01-15 al 2020-01-20):}
				\begin{itemize}
					\item \textbf{Preparazione presentazione:} redazione della presentazione da portare in sede di revisione;
				\end{itemize}

	\subsection{Progettazione della technology baseline}
	Questa attività è stata suddivisa in tre periodi a partire dal 22 Gennaio 2020 al 16 Marzo 2020:

		\subsubsection{Primo periodo (dal 2020-01-22 al 2020-02-02):}
			\begin{itemize}
			 	\item Studio approfondito ed esaustivo sulle tecnologie da utilizzare e sugli strumenti da utilizzare;
			 	\item Aggiornamento del Piano di Progetto;
			 	\item Aggiornamento Norme di Progetto.
			\end{itemize} 	
		
		\subsubsection{Secondo periodo (dal 2020-02-03 al 2020-03-08):}
			\begin{itemize}
				\item Contatto con l'azienda;
				\item Progettazione: architettura logica;
				\item Progettazione: framework, tecnologie e librerie da impiegare per la realizzazione della Proof of Concept;
				\item Realizzazione della \glock{Proof of Concept};
				\item Aggiornamento del Piano di Progetto;
				\item Redazione della lettera di presentazione.
				\item Verifica della documentazione redatta.
			\end{itemize}

		\subsubsection{Terzo periodo (dal 2020-03-09 al 2020-03-15):}
			\begin{itemize}
				\item \textbf{Preparazione presentazione:} redazione della presentazione da portare in sede di revisione;
			\end{itemize}

	\subsection{Progettazione di dettaglio e Codifica}
	Questa attività è stata suddivisa in tre periodi a partire dal 16 Marzo 2020 al 19 Aprile 2020:


		\subsubsection{Primo periodo (dal 2020-03-16 al 2020-03-20):}
			\begin{itemize}
				\item Studio sugli strumenti e metriche da utilizzare;
			 	\item Aggiornamento del Piano di Progetto;
			 	\item Aggiornamento del Piano di Qualifica;
			 	\item Aggiornamento delle Norme di Progetto.
			\end{itemize} 	
		
		\subsubsection{Secondo periodo (dal 2020-03-21 al 2020-04-11):}
			\begin{itemize}
				\item Scrittura dei diagrammi delle classi e di sequenza;
				\item Scelta dei design pattern;
				\item Implementazione dei test;
				\item Codifica e primo rilascio della \glock{Product baseline};
				\item Aggiornamento del Piano di Progetto con consuntivo di periodo e preventivo a finire;
				\item Redazione iniziale del manuale utente;
				\item Redazione della lettera di presentazione.
			\end{itemize}

		\subsubsection{Terzo periodo (dal 2020-04-12 al 2020-04-19):}
			\begin{itemize}
				\item \textbf{Preparazione presentazione:} redazione della presentazione da portare in sede di revisione;
			\end{itemize}
	
	\subsection{Validazione e Collaudo}		
	Questa attività è stata suddivisa in tre periodi a partire dal 20 Aprile 2020 al 18 Maggio 2020:

		\subsubsection{Primo periodo (dal 2020-04-20 al 2020-04-25):}
			\begin{itemize}
			 	\item Aggiornamento del Piano di Progetto.
			\end{itemize} 	
		
		\subsubsection{Secondo periodo (dal 2020-04-26 al 2020-05-10):}
			\begin{itemize}
				\item Completamento della codifica;
				\item Completamento del manuale utente;
				\item Redazione del manuale per il manutentore;
				\item Validazione preventiva del prodotto;
				\item Aggiornamento del Piano di Qualifica con gli esiti finali dei test;
				\item Aggiornamento del Piano di Progetto con consuntivo finale;
				\item Rilascio finale del prodotto.
			\end{itemize}

		\subsubsection{Terzo periodo (dal 2020-05-13 al 2020-05-18):}
			\begin{itemize}
				\item \textbf{Preparazione presentazione:} redazione della presentazione da portare in sede di revisione;
			\end{itemize}

