
\newpage \section{A}
\subsection{Account}  Un account, indica quell'insieme di funzionalità, strumenti e contenuti attribuiti ad un nome utente in cui il sistema mette a disposizione dell'utente un ambiente con contenuti e funzionalità personalizzabili, oltre ad un conveniente grado di isolamento dalle altre utenze parallele.
\subsection{AJAX}  AJAX, acronimo di Asynchronous JavaScript and XML, è una tecnica di sviluppo software per la realizzazione di applicazioni web interattive. Lo sviluppo di applicazioni HTML con AJAX si basa su uno scambio di dati in background fra web browser e server, che consente l'aggiornamento dinamico di una pagina web senza esplicito ricaricamento da parte dell'utente.
\subsection{Alert}  Segnale generato per comunicare un messaggio.
\subsection{Amazon Web Services (AWS)}  Amazon Web Services, Inc. (nota con la sigla AWS) è un'azienda statunitense di proprietà del gruppo Amazon, che fornisce servizi di cloud computing
\subsection{Amministratore}  Vedere sezione 4.1.4 del documento \dext{Norme di Progetto v1.0.0}.
\subsection{API}  Le API (acronimo di Application Programming Interface, ovvero Interfaccia di programmazione delle applicazioni) sono set di definizioni e protocolli con i quali vengono realizzati e integrati software applicativi.
\subsection{API Connect}  L'API Kafka Connect è un'interfaccia che semplifica ed automatizza l'integrazione di nuove fonti di dati.
\subsection{API Consumer}  L'API Kafka Consumer consente alle applicazioni di leggere flussi di dati dal cluster Kafka.
\subsection{API Gateway}  Amazon API Gateway è un servizio completamente gestito che semplifica per gli sviluppatori la creazione, la pubblicazione, la manutenzione, il monitoraggio e la protezione delle API su qualsiasi scala.
\subsection{API Producer}  L'API Kafka Producer consente alle applicazioni di inviare flussi di dati al cluster Kafka.
\subsection{API Restful}  Delle RESTful API sono un set di API che rispettano il modello architetturale REST.
\subsection{API Streams}  L'API Kafka Streams permette di costruire applicazioni e microservizi, dove i dati in input ed output sono memorizzati in cluster Kafka.
\subsection{App mobile-server, comunicazioni}  Con comunicazioni App Mobile - Server, si intendono comunicazioni di varia natura tra un applicativo mobile ed un server.
\subsection{Apprendimento approfondito (deep learning)}  Per apprendimento profondo o approfondito si intende un insieme di tecniche basate su reti neurali artificiali organizzate in diversi strati, dove ogni strato calcola i valori per quello successivo affinché l'informazione venga elaborata in maniera sempre più completa.
\subsection{Arduino Uno}  Arduino Uno è una board di sviluppo che utilizza un microcontrollore e permette di collegare e gestire sensori, servomotori, potenziometri e molte altri componenti elettronici. Con questa board è possibile sviluppare un piccolo firmware in un linguaggio derivato dal C++ che va a eseguire diverse operazioni in base alle necessità. Il collegamento può avvenire tramite seriale o tramite ethernet (con un apposito modulo).
\subsection{AWS Lambda}  AWS Lambda è una piattaforma di computing serverless ad eventi, fornita da Amazon come parte di AWS.
\subsection{AWS Rekognition Video}  Amazon Rekognition Video è un servizio di analisi video basato su apprendimento approfondito.
\subsection{AWS S3}  Amazon Simple Storage Service (Amazon S3) è un servizio di storage di oggetti che offre scalabilità, disponibilità dei dati, sicurezza e prestazioni all'avanguardia nel settore.
\subsection{AWS Transcode}  AWS Transcode è un servizio per la transcodifica di contenuti multimediali nel cloud.

\newpage \section{B}
\subsection{Back-end}  Con il termine back-end si indica generalmente l'interfaccia amministrativa di un applicativo o sito web.
\subsection{Back-end-as-a-Service (BaaS)}  Back-end come un servizio (BaaS), è un modello per fornire a sviluppatori di app web o mobili un modo per collegare le loro applicazioni a un back-end cloud storage e API esposte da applicazioni back-end, fornendo allo stesso tempo funzioni quali la gestione degli utenti, le notifiche push, e l'integrazione con servizi di rete sociale.
\subsection{Baseline}  Nell'ingegneria del software, una baseline costituisce un avanzamento che farà parte del prodotto finale, in cui le modifiche saranno incrementali.
\subsection{Big data}  Il termine big data, indica genericamente una raccolta di dati così estesa in termini di volume, velocità e varietà da richiedere tecnologie e metodi analitici specifici per l'estrazione di valore o conoscenza.
\subsection{Blockchain}  Una blockchain è una struttura dati condivisa e immutabile. È definita come un registro digitale le cui voci sono raggruppate in blocchi, concatenati in ordine cronologico, e la cui integrità è garantita dall'uso della crittografia.
\subsection{Bootstrap}  Bootstrap è una raccolta di strumenti liberi per la creazione di siti e applicazioni per il Web. Essa contiene modelli di progettazione basati su HTML e CSS, sia per la tipografia, che per le varie componenti dell'interfaccia, come moduli, pulsanti e navigazione, così come alcune estensioni opzionali di JavaScript.
\subsection{Bot}  I bot sono applicazioni di terze parti sviluppati da programmatori esterni per interagire con gli utenti tramite messaggi, comandi e richieste in linea tramite il servizio di messaggistica Telegram.
\subsection{Brainstorming}  è una tecnica creativa di gruppo per far emergere idee volte alla risoluzione di un problema. Consiste nell'organizzare una riunione in cui ogni partecipante propone liberamente soluzioni di ogni tipo al problema, senza che nessuna di esse venga minimamente censurata.
\subsection{Branch}  Con il termine branch si indica un ramo di sviluppo in un Version Control System.

\newpage \section{C}
\subsection{CamelCase}  CamelCase o notazione a cammello deriva dall'utilizzo del carattere maiuscolo per la lettera iniziale di una parola, quando si usano più parole distinte per comporre un sostantivo; questo utilizzo del maiuscolo fa venire in mente le gobbe di un cammello. La prima lettera può essere sia maiuscola (es. CamelCase), come il nome delle classi in Java, che minuscola.
\subsection{Capitolato}  Un capitolato d'appalto è un documento del committente che specifica cosa richiede che sia presente nel prodotto e i suoi vincoli.
\subsection{Cell ID}  Con cell ID si intende il codice idenificativo di una cella di una stazione.
\subsection{Chrome}  Browser web sviluppato da Google.
\subsection{Client-side}  Con client-side si intende che l'azione viene compiuta nel lato utente (ovvero nel suo computer) ed è contrapposta ad un'azione server-side.
\subsection{Cloud}  Con il termine inglese cloud (computing) si indica un paradigma di erogazione di servizi offerti on demand da un fornitore ad un cliente finale attraverso la rete Internet. 
\subsection{Cloud formation}  CloudFormation è un prodotto di AWS che consente di usare linguaggi di programmazione o un semplice file di testo per modellare ed effettuare il provisioning, in modo automatizzato e sicuro, di tutte le risorse di un'infrastruttura nell'ambiente cloud AWS e non solo.
\subsection{Computation-as-a-Service (Caas)}  Computation-as-a-Service è uno dei modelli di cloud computing in cui viene permesso di eseguire del codice su di una piattaforma cloud come servizio.
\subsection{Container, Docker}  Un container \textit{Docker} è una istanza di una immagine \textit{Docker} in cui viene eseguito un sistema operativo e, in genere, un solo applicativo, con le proprie dipendenze e configurazioni. Un container è completamente indipendente dagli altri container e può comunicare con essi tramite una interfaccia di rete apposita.
\subsection{Containerizzazione}  Il processo di containerizzazione consiste nell'isolare processi in modo da poterli eseguire in maniera indipendente.
\subsection{Continuous delivery}  Nell'ingegneria del software, la continuous delivery, è una pratica nella quale un team produce software in brevi cicli, assicurandosi che il software sia rilasciabile in ogni momento.
\subsection{Continuous integration}  Nell'ingegneria del software, l'integrazione continua o continuous integration, è una pratica che consiste nell'allineamento frequente degli ambienti di lavoro degli sviluppatori verso l'ambiente condiviso, attraverso un sistema di versioning.
%\subsection{Continuous testing}  Nell'ingegneria del software, la continuous testing, è una pratica completamente %automatizzata nella quale un team produce software in brevi cicli assicurandosi che il prodotto sia rilasciabile %in qualsiasi momento.
\subsection{Contratti smart}  Uno smart contract è la traduzione o trasposizione in codice di un contratto in modo da verificare in automatico l’avverarsi di determinate condizioni e di eseguire in automatico azioni (o dare disposizione affinché si possano eseguire determinate azioni) nel momento in cui le condizioni determinate tra le parti sono raggiunte e verificate.
\subsection{CSS3}  CSS (Cascading Style Sheets), è un linguaggio usato per definire la formattazione di documenti HTML, XHTML e XML, ad esempio i siti web e relative pagine web. CSS3 rappresenta la quarta release del linguaggio.


\newpage \section{D}
\subsection{Dashboard}  Con il termine dashboard si intende una schermata, di per se vuota, che può contenere informazioni di varia natura.
\subsection{Data scientist}  La professione del data scientist nasce esigenza di trattare i dati, non solo per acquisirli, conservarli e assolvere modesti compiti operativi, ma principalmente per analizzarli e interpretarli opportunatamente.
\subsection{Dataset}  Un dataset costituisce un insieme di dati strutturati in forma relazionale, cioè corrisponde al contenuto di una singola tabella di base di dati, oppure ad una singola matrice di dati statistici, in cui ogni colonna della tabella rappresenta una particolare variabile, e ogni riga corrisponde ad un determinato membro del dataset in questione.
\subsection{Dead Rekoning}  Nella navigazione, il dead rekoning è il processo per calcolare una posizione usando una posizione determinata precendentemente, ed avanzando quella posizione basandosi su una velocità rilevata o stimata, una direzione ed il tempo trascorso.
\subsection{Deploy}  Il significato più comune del termine deploy o deployment in informatica è la consegna o rilascio al cliente, con relativa installazione e messa in funzione o esercizio, di una applicazione o di un sistema software tipicamente all'interno di un sistema informatico aziendale.
\subsection{Design}  Progettazione di un sistema software, intesa come individuazione delle modalità di implementazione del concetto astratto individuato tramite l'analisi.
\subsection{DevOps}  DevOps è un metodo di sviluppo del software che punta alla comunicazione, collaborazione e integrazione tra sviluppatori e addetti alle operations della information technology (IT).
\subsection{Difetto}  Con difetto si intende un errore nella scrittura del codice sorgente di un programma software o nel testo di un documento.
\subsection{Discord}  Discord è un'applicazione VoIP progettata per le comunità di videogiocatori. In questa applicazione è possibile creare dei canali testuali e vocali suddividendoli per tematica. 
\subsection{Dispatcher}  Nei server web il dispatcher è il thread che legge dalla rete le richieste da elaborare in arrivo.
\subsection{Dispatching}  Spedire qualcosa, solitamente un messaggio o un bene, da qualche parte per uno scopo particolare.
\subsection{Docker}  Il software Docker è una tecnologia di containerizzazione che consente la creazione e l'utilizzo dei container Linux.
\subsection{Dockerfile}  Un \textit{Dockerfile} è un file di configurazione che illustra i passaggi che devono essere realizzati per compilare un'immagine di un sistema operativo come base per eseguire un certo applicativo.
\subsection{Docker-compose}  Il docker-compose è un comando che può essere utilizzato per comporre, letteralmente, tutta l'infrastruttura in base ai servizi che si è scelto di integrare. Fa uso dei \textit{Dockerfile} per avviare l'eventuale build delle \textit{immagini Docker} e permette con un solo comando di attivare (\verb!docker-compose up -d!) o disattivare (\verb!docker-compose down!) tutti i servizi. L'intera configurazione è salvata su un file denominato per convenzione \verb!docker-compose.yml!.
\subsection{DynamoDB}  Amazon DynamoDB è un database che supporta i modelli di dati di tipo documento e di tipo chiave-valore che offre prestazioni di pochi millisecondi a qualsiasi scala.

\newpage \section{E}
\subsection{Edge}  Browser open source realizzato da Microsoft installato di default in tutti i sistemi operativi Windows 10.
\subsection{Efficacia}  Con efficacia si intende la misura della capacità di raggiungere un obiettivo; è strettamente legato a quanto ciò che viene fatto rispetta i requisiti.
\subsection{Efficienza}  Con efficienza di intende la misura per cui si impiegano il minimo numero di risorse per raggiungere un obiettivo.
\subsection{Elastic Container Service}  Elastic Container Service (ECS) è un servizio di gestione dei container altamente scalabile e ad alte prestazioni che supporta la containerizzazione e consente di eseguire facilmente applicazioni su un cluster gestito.
\subsection{Elastic Kubernetes Service}  Elastic Kubernetes Service è un servizio di orchestrazione Kubernetes interamente gestito.
\subsection{Ether (ETH)}  Ether è un asset digital (un token). Metaforicamente può essere considerato la benzina per tutte le applicazioni in Ethereum.
\subsection{Ethereum}  Ethereum è una piattaforma decentralizzata per la creazione e pubblicazione peer-to-peer di contratti intelligenti (smart contracts).
\subsection{Ethereum Virtual Machine (EVM)}  La Ethereum Virtual Machine, spesso abbreviata attraverso l’acronimo EVM, è il centro di calcolo che permette l’esecuzione di codici complessi (smart contracts) al di sopra della piattaforma Ethereum.
\subsection{ESP8266}  Board di sviluppo industriale, molto simile all'\textit{arduino uno}, con cui è possibile collegare e controllare diversi componenti elettronici, come sensori di temperatura o potenziometri. A differenza dell'\textit{arduino uno}, questa board consuma meno, ha meno ingressi e utilizza il collegamento WI-FI, oltre che seriale. Lo sviluppo del firmware viene eseguito sempre tramite C++, nelle stesse modalità di \textit{Arduino uno}. Si integra con maggiore facilità nel contesto del \textit{Internet of Things} per il suo costo molto basso e per la sua versatilità.


\newpage \section{F} % to pay respect
\subsection{Firefox}  Browser web libero e gratuito realizzato da un'organizzazione no-profit, Mozilla Foundation.
\subsection{Framework}  Un framework, è un'architettura logica di supporto (spesso un'implementazione logica di un particolare design pattern) su cui un software può essere progettato e realizzato.
\subsection{Front-end}  Con il termine front-end si indica generalmente l'interfaccia utente di un applicativo o sito web.

\newpage \section{G}
\subsection{Gateway}  Software o dispositivo fisico che serve a gestire le comunicazioni che avvengono tra i dispositivi e la piattaforma Kafka.
\subsection{Gherkin}  Gherkin è il formato per le specifiche di Cucumber. È un linguaggio specifico del dominio che aiuta a descrivere il comportamento dello strato business senza la necessità di entrare nei dettagli dell'implementazione.
\subsection{Git}  Sistema di versionamento distribuito multi-piattaforma. Permette di versionare i sorgenti software e di collaborare nella loro realizzazione.
\subsection{GitHub}  GitHub è un servizio di hosting per progetti software, basati sul sistema di versionamento Git.
\subsection{GitHub Actions}  GitHub Actions è uno strumento fornito da GitHub che permette l'automazione di compiti di varia natura.
\subsection{Google Drive}  Google Drive è un servizio web, in ambiente cloud computing, di memorizzazione e sincronizzazione online che permette di realizzare documenti, fogli di calcolo e presentazioni.
\subsection{GPS}  GPS è un sistema di posizionamento e navigazione satellitare militare statunitense. Attraverso una rete dedicata di satelliti artificiali in orbita, fornisce a un terminale mobile o ricevitore GPS informazioni sulle sue coordinate geografiche.
\subsection{Grafana}  Grafana è una soluzione open-source che consente di visualizzare graficamente dati da diverse sorgenti.
\subsection{Gulpease, indice di}  L'indice Gulpease è un indice di leggibilità di un testo tarato sulla lingua italiana.


\newpage \section{H}
\subsection{Hypertable}  Le Hypertable sono delle tabelle implementate in un TimescaleDB che permettono di memorizzare e accedere dati con altissima velocità. Sono molto utili per le rilevazioni di sensori molto frequenti e che non richiedono aspetti relazionali.
\subsection{HTML5}  HTML5 è un linguaggio di markup per la strutturazione delle pagine web.

\newpage \section{I}
\subsection{IDE}   Un IDE o ambiente di sviluppo integrato è un software che, in fase di programmazione, supporta i programmatori nello sviluppo del codice sorgente di un programma.
\subsection{Immagine, Docker} Un'immagine Docker è un eseguibile che è può essere istanziato in un container. L'immagine Docker viene prodotta a seguito della compilazione tramite il comando \verb!docker build! di un \textit{Dockerfile} o di un \textit{docker-compose}.
\subsection{Inspection}  Tecnica di analisi statica che consiste nell'analizzare il prodotto solo nelle parti in cui si prevede che ci possano essere dei difetti.
\subsection{IoT }  Internet delle cose ( IoT dall'inglese Internet of Things), nelle telecomunicazioni è un neologismo riferito all'estensione di Internet al mondo degli oggetti e dei luoghi concreti, rappresentando di fatto un'evoluzione dell'uso della rete Internet.
\subsection{Issue}  Con il termine issue si intende una qualsiasi richiesta o azione da svolgere, legata generalmente al concetto di issue in un ITS (Issue Tracking System).
\subsection{Issue Tracking System}  Un Issue Tracking System o ITS, indica un sistema informatico il cui scopo è tracciare richieste di assistenza o problemi, ma non solo.


\newpage \section{J}
\subsection{Java}  Java è un linguaggio di programmazione ad alto livello, orientato agli oggetti e a tipizzazione statica, che si appoggia sull'omonima piattaforma software di esecuzione, specificamente progettato per essere il più possibile indipendente dalla piattaforma hardware di esecuzione tramite l'utilizzo di macchina virtuale.
\subsection{JavaScript}  JavaScript è un linguaggio di scripting orientato agli oggetti e agli eventi, comunemente utilizzato nella programmazione web lato client per la creazione, in siti web e applicazioni web, di effetti dinamici interattivi tramite funzioni di script invocate da eventi innescati a loro volta in vari modi dall'utente sulla pagina web in uso. 
\subsection{JMeter}  Apache JMeter è un software che permette di testare performance su risorse statiche o dinamiche di un sito web.
\subsection{JMX}  Java Management Extension (JMX) è un insieme di specifiche, pattern che permettono di inserire all'interno di una applicazione sviluppata in Java dei componenti per il monitoraggio della stessa.
\subsection{jQuery}  jQuery è una libreria JavaScript per applicazioni web. Nasce con l'obiettivo di semplificare la selezione, la manipolazione, la gestione degli eventi e l'animazione di elementi DOM in pagine HTML, nonché implementare funzionalità AJAX.
\subsection{JSON }  JSON, acronimo di JavaScript Object Notation, è un formato adatto all'interscambio di dati fra applicazioni client/server.

\newpage \section{K}
\subsection{Kafka, Apache}  Apache Kafka è una piattaforma open source di stream processing scritta in Java e Scala e sviluppata dall'Apache Software Foundation.  Questo progetto viene usato principalmente per tutte le applicazioni di elaborazioni di stream di dati in tempo reale.
\subsection{Keyword}  Con il termine keyword, o parola chiave, si indica una parola di particolare importanza, generalmente utilizzato per effettuare una ricerca.
\subsection{Kubernetes}  Kubernetes (abbreviato K8s) è un sistema open-source di orchestrazione e gestione di container.

\newpage \section{L}
\subsection{Laravel}  Laravel è un framework open-source molto potente e usato per sviluppare applicazioni in PHP, facendo uso nativamente di Bootstrap e Vue.js. Permette di realizzare la parte back-end di un sito web seguendo il modello model-view-controller. 
\subsection{Layers logici}  Con il termine layers logico si intende uno strato software, non necessariamente legato ad uno specifico strato hardware.
\subsection{Logs}  Il termine logs, derivato dal termine logbook (registro di bordo), indica degli eventi, di varia natura, registrati in ordine cronologico.


\newpage \section{M}
\subsection{Macchina Virtuale Decentralizzata}  Una macchina virtuale decentralizzata è innanzitutto una macchina virtuale, ossia un software in grado di emulare in tutto e per tutto una macchina fisica, attraverso un processo di virtualizzazione in cui vengono assegnate le risorse fisiche alle applicazioni che vengono eseguite al di sopra della macchina virtuale. Funge inoltre da garante per i nodi della rete, che “offrono” la propria infrastruttura fisica per la memorizzazione ed il processing di smart contracts potenzialmente dannosi.
\subsection{Machine learning}  L’apprendimento automatico (noto anche come machine learning) è una branca dell'intelligenza artificiale che raccoglie un insieme di metodi statistici per migliorare progressivamente la performance di un algoritmo nell'identificare pattern nei dati. 
\subsection{Major release}  Con major release si intende una versione di un prodotto che si discosta sostanzialmente dalla versione precedente.
\subsection{Manutenzione predittiva}  La manutenzione predittiva è un tipo di manutenzione preventiva che viene effettuata a seguito dell'individuazione di uno o più parametri che vengono misurati ed elaborati utilizzando appropriati modelli matematici allo scopo di individuare il tempo residuo prima del guasto.
\subsection{MariaDB}  MariaDB è un sistema di gestione di un database relazionale(DBMS) retrocompatibile con la tecnologia di MySQL da cui è derivato.
\subsection{Milestone}  Con il termine milestone si indica una tappa nel tempo che garantisce il raggiungimento di un obiettivo nei tempi preventivati.
\subsection{Minor release}   Con minor release, si intende una specifica versione di un prodotto nella quale le differenze con la versione precedente riguardano principalmente correzioni.
\subsection{Mobile-application}  Un'applicazione mobile o mobile-application, è un'applicazione software dedicata ai dispositivi di tipo mobile.
\subsection{Modello a V}  Modello di sviluppo software in cui per ogni fase di sviluppo corrisponde una fase di testing per controllare che la fase di sviluppo corrispondente sia corretta.
\subsection{Modello di machine learning}  Un modello di apprendimento automatico è un file di cui è stato eseguito il training per riconoscere determinati tipi di modelli.

\newpage \section{N}
\subsection{Node Package Manager (NPM)}  NPM (abbreviazione di Node Package Manager) è un gestore di pacchetti per il linguaggio di programmazione JavaScript. È il gestore di pacchetti predefinito per l'ambiente di runtime JavaScript Node.js.
\subsection{Node.js}  Node.js è una runtime di JavaScript Open source multipiattaforma orientato agli eventi per l'esecuzione di codice JavaScript.

\newpage \section{O}
\subsection{Open-source}  Il termine open-source  viene utilizzato per riferirsi ad un tipo di software o al suo modello di sviluppo o distribuzione. Un software open-source è reso tale per mezzo di una licenza attraverso cui i detentori dei diritti ne favoriscono la modifica, lo studio, l'utilizzo e la redistribuzione. Caratteristica principale dunque delle licenze open-source è la pubblicazione del codice sorgente.
\subsection{Openshift}  OpenShift è un Platform-as-a-Service (PaaS) prodotto da Red Hat ed è una piattaforma per applicazioni cloud che rende semplice lo sviluppo, il deploy e la scalabilità di applicazioni cloud. 


\newpage \section{P}
\subsection{Package manager}  Con package manager, si intende una collezione di strumenti software, presenti in un sistema operativo, che automatizzano il processo di installazione, configurazione, aggiornamento e rimozione dei pacchetti software in un computer, venendo dunque utilizzato per installare, aggiornare, verificare e rimuovere software del sistema operativo in maniera semplice ed intuitiva, aiutando spesso a risolvere anche le dipendenze tra i pacchetti.
\subsection{Patch}  Patch indica una porzione di software progettata per aggiornare o migliorare un programma. Ciò include la risoluzione di vulnerabilità di sicurezza e altri bug generici.
\subsection{Pattern, design}  Un design pattern nell'ambito dell'ingegneria del software, è un concetto che può essere definito come una soluzione progettuale generale ad un problema ricorrente.
\subsection{PDCA}  Acronimo di Plan–Do–Check–Act è un metodo di gestione iterativo in quattro fasi utilizzato per il controllo e il miglioramento continuo dei processi e dei prodotti.
\subsection{Piattaforma web}  Con piattaforma web, si intende la sua struttura hardware, operativa e software.
\subsection{Plug-in}  Un plug-in è un modulo aggiuntivo di un programma, utilizzato per aumentarne le funzioni.
\subsection{PostgreSQL}  PostgreSQL è un completo DBMS ad oggetti rilasciato con licenza libera.
\subsection{Processi}  Insieme di attività che interagiscono per produrre un risultato.
\subsection{Producer-Consumer, modello}  Modello di comunicazione utilizzato da kafka per inviare e ricevere dei topic.
\subsection{Programmatore}  Vedere sezione 4.1.4 del documento \dext{Norme di Progetto v1.0.0}.
\subsection{Proof-of-concept }  Con il termine proof-of-concept si intende una realizzazione incompleta o abbozzata di un determinato progetto o metodo, allo scopo di provarne la fattibilità o dimostrare la fondatezza di alcuni principi o concetti costituenti.
\subsection{Proximity Sensing}  Proximity Sensing è una tecnica per ricavare la posizione di un punto mobile è ricavandola dalle coordinate di determinate stazioni che tracciano il segnale che viene trasmesso da esse (cell ID). Ogni stazione ha un suo pattern di segnale.
\subsection{Publisher}  Con il termine Publisher si intende il mittente di un messaggio.
\subsection{Pull-Request}  Richiesta di salvataggio delle modifiche in un branch del Version Control System.
\subsection{Python}  Python è un linguaggio di programmazione ad alto livello, orientato agli oggetti, adatto, tra gli altri usi, a sviluppare applicazioni distribuite, scripting, computazione numerica e system testing.

\newpage \section{R}
\subsection{Rancher}  Rancher è una piattaforma di orchestrazione open-source che permette il rilascio e la gestione di Kubernetes.
\subsection{Regressione}  Azione per cui sono state effettuate delle modifiche che causano il mal funzionamento di altre unità del software.
\subsection{Regressione Lineare (RL)}  La regressione lineare rappresenta un metodo di stima del valore atteso condizionato di una variabile dipendente Y dati i valori di altre variabili indipendenti, X.
\subsection{Repository}  Un repository è un archivio web nel quale vengono raggruppati contenuti di varie tipologie, generalmente pacchetti software.
\subsection{Responsabile}  Vedere sezione 4.1.4 del documento \dext{Norme di Progetto v2.0.0}.
\subsection{Rest, web }  Vedi API REST.
\subsection{Runtime}  Con il termine runtime si intende il periodo di vita di un programma che intercorre durante la sua esecuzione.

\newpage \section{S}
\subsection{Safari}  Browser web sviluppato dalla Apple Inc. inizialmente per i sistemi operativi iOS e macOS, e successivamente sviluppato anche per Windows.
\subsection{Sage Maker}  Amazon Sage Maker è un servizio completamente gestito che consente a data scientist e sviluppatori di creare, formare e distribuire in modo rapido modelli di machine learning (ML). 
\subsection{Scripting, linguaggio di}  Con il termine scripting si intende un linguaggio destinato in genre a compiti di automazione del sistema operativo e delle applicazioni.
\subsection{Server farm}  Il termine Server farm è utilizzato per indicare una serie di server collocati in un unico ambiente in modo da poterne centralizzare la gestione, la manutenzione e la sicurezza.
\subsection{Server side}  Con il termine server side si intende un'azione compiuta lato server e contrapposta ad un'azione client side.
\subsection{Serverless}  Con il termine serverless si intende un network la cui gestione non viene incentrata su dei server, ma viene dislocata fra i vari utenti che utilizzano il network stesso, quindi il lavoro necessario di gestione del network viene eseguito dagli stessi utilizzatori.
\subsection{Slack}  Slack è un software che rientra nella categoria degli strumenti di collaborazione aziendale utilizzato per inviare messaggi in modo istantaneo ai membri del team.
\subsection{Solidity}  Solidity è un linguaggio orientato ad oggetti per la scrittura di contratti smart, soprattutto sulla piattaforma Ethereum.
\subsection{SPICE}  Acronimo di Software Process Improvement and Capability dEtermination è un set di standard tecnici per il processo di sviluppo del software.
\subsection{Spring, framework}  Spring è un framework Java open source che viene utilizzato per lo sviluppo di applicativi web e di servizi API.
\subsection{Stakeholder}  Con il termine stakeholder si intendono le parti interessate nella realizzazione di un progetto o più in generale di un endeavor (o impresa).
\subsection{Subscriber}  Con il termine subscriber si intende il destinatario di un messaggio.
\subsection{Suite}  Insieme di applicazioni software appartenenti allo stesso produttore.
\subsection{Support Vector Machine (SVM)}  Le macchine a vettori di supporto (SVM, Support-Vector Machines) sono dei modelli di apprendimento supervisionato associati ad algoritmi di apprendimento per la regressione e la classificazione.
\subsection{Sviluppo, processo di}  Processo che fa parte del ciclo di vita del software in cui viene sviluppata la parte software del progetto.
\subsection{System testing}  Test che verifica il funzionamento del prodotto completamente integrato con tutte le sue componenti.

\newpage \section{T}
\subsection{Telegram}  Telegram è un servizio di messaggistica istantanea e broadcasting basato su cloud.
\subsection{Test}  Con test si indica un'operazione effettuata durante il periodo di sviluppo in cui il prodotto viene controllato per individuare eventuali problemi o malfunzionamenti.
\subsection{TimeScaleDB}  TimeScaleDB è un database open-source costruito per analizzare dati di serie storiche.
\subsection{Timeseries DB}  Con il termine timeseries DB si intende un database il cui scopo è la memorizzazione di serie storiche di dati.
\subsection{Toolkit}  Con il termine toolkit si intende una raccolta di strumenti atti a qualche scopo specifico.
\subsection{Topic}  Messaggio utilizzato per le comunicazioni della piattaforma Kafka in cui sono salvati array di messaggi che possono contenere degli oggetti.
\subsection{TypeScript}  TypeScript è un linguaggio di programmazione open source sviluppato da Microsoft. Si tratta di un super-set di JavaScript.

\newpage \section{U}
\subsection{UML, diagrammi} UML è un linguaggio di modellazione e di specifica basato sul paradigma orientato agli oggetti. La notazione UML è semi-grafica e semi-formale; un modello UML è costituito da una collezione organizzata di diagrammi correlati, costruiti componendo elementi grafici (con significato formalmente definito), elementi testuali formali, ed elementi di testo libero. Ha una semantica molto precisa e un grande potere descrittivo. 
\subsection{Unix}  Unix è un sistema operativo portabile per computer inizialmente sviluppato da un gruppo di ricerca dei laboratori AT\&T e Bell Laboratories. Storicamente è stato il sistema operativo maggiormente utilizzato su sistemi mainframe a partire dagli anni settanta.
\subsection{User interface}  Con il termine user interface (interfaccia utente) si intende la parte di un programma o sito web che verrà visualizzata dagli utenti dello stesso.
\subsection{UTF-8}  UTF-8 (Unicode Transformation Format, 8 bit) è una codifica di caratteri Unicode in sequenze di lunghezza variabile di byte. UTF-8 usa gruppi di byte per rappresentare i caratteri Unicode, ed è particolarmente utile per il trasferimento tramite sistemi di posta elettronica a 8-bit.

\newpage \section{V}
\subsection{Version Control System (VCS)} In italiano sistema di versionamento, è un sistema per la gestione delle versioni di un prodotto.
\subsection{Versionamento} Il controllo versione o versionamento, in informatica, è la gestione di versioni multiple di un insieme di informazioni. 
\subsection{View} Nel contesto della web app, la view è una pagina personalizzata che viene utilizzata per creare più grafici e visualizzarli in un unico luogo. Diversamente, nell'ambito di un database, la view di un database rappresenta le vedute delle tabelle all'interno di un database relazionale.
\subsection{Vue.js}  Vue.js è un framework open-source di Javascript che permette di realizzare la parte front-end dinamica in unione con Laravel. Il framework è molto flessibile e altamente scalabile, nonché compatibile anche con Bootstrap.

\newpage \section{W}
\subsection{W3C} Acronimo di World Wide Web Consortium, è un'organizzazione non governativa internazionale il cui scopo è sviluppare tutte le potenzialità del World Wide Web.
\subsection{Walkthrough} Tecnica di analisi statica che consiste nell'analizzare il prodotto per intero in ogni sua parte.
\subsection{Web-Application} Con Web-Application o Web-App si intendono tutte le applicazioni distribuite web-based ovvero applicazioni accessibili/fruibili via web per mezzo di un network. 
\subsection{Wiki}  Wiki è un'applicazione web che permette la creazione, la modifica e l'illustrazione collaborativa di pagine all'interno di un sito web.
\subsection{Wireframe} Un wireframe o wire frame model indica un tipo di rappresentazione in computer grafica di oggetti tridimensionali.
\subsection{Workflow} Workflow è l’automazione totale o parziale di un processo aziendale, in cui documenti, informazioni o compiti passano da un partecipante a un altro per svolge attività, secondo un insieme di regole definite.
\subsection{Workspace} Con il termine workspace si indica l'ambiente di lavoro, comprendendo anche la strumentazione utilizzata.

\newpage \section{Y}
\subsection{YAML}  YAML è un formato per la serializzazione di dati utilizzabile da esseri umani. Il linguaggio sfrutta concetti di altri linguaggi come il C, il Perl e il Python e idee dal formato XML.