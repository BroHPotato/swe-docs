\section{Requisiti}

\definecolor{red_requisiti}{HTML}{bf5550} % funzionali
\definecolor{gold_requisiti}{HTML}{bf9950} % qualita
\definecolor{blue_requisiti}{HTML}{50abbf} % vincolo
\definecolor{green_requisiti}{HTML}{9cbf50} % prestazionali

\newcommand{\req}[3]{\textbf{R#1-#2-#3}}
\newcommand{\sreq}[3]{{\color{gray} R#1-#2-}#3}
\newcommand{\autism}{ \\ \hline}


In questa parte, vengono riportati i requisiti del progetto, classificati per tipologia . Ciascun requisito possiede un codice identificativo, il cui formalismo viene riportato all'interno del documento \dext{Norme di Progetto v1.0.0}.


	% Salvataggio dati nei database OK
	% Admin mette nuova config. OK
	% Il gateway deve mandare o ricevere dati sui o dai topic messi a disposizione di kafka OK
	% Le web app e telegram devono interfacciarsi con delle API OK
	% Le API devono inviare o ricevere dati sui o dai topic di kafka. OK


	\subsection{Requisiti Funzionali}

	\begin{center}
		\rowcolors{2}{white}{lightest-grayest}
		\begin{longtable}{|p{3cm}|p{9.85cm}|p{2cm}|}
		\hline
		\rowcolor{red_requisiti}
		{\color{white} \textbf{ID Requisito} } & {\color{white} \textbf{Descrizione} } & {\color{white} \textbf{Fonti} } \\
		\hline
		\endhead
		\hline
        \multicolumn{3}{|c|}{\textit{Continua nella pagina successiva...}}\\
        \hline
        \endfoot
        \endlastfoot
		% autenticazione
		\req{A}{F}{1} 		& L'utente deve potersi autenticare per accedere alla web app & UC1 \autism
		\sreq{A}{F}{1.1} 	& L'utente deve poter usufruire dell'autenticazione a due fattori & UC1.3, UC14, Interna \autism
		\sreq{A}{F}{1.2} 	& L'utente deve poter ricevere un codice di autenticazione a due fattori tramite \glock{Telegram} & UC1.3.3, Interna, Verbale #8 VE-8.3 \autism
		\req{A}{F}{2} 		& L'utente autenticato deve avere accesso alla web app e deve poter navigare al suo interno & UC1.4 \autism
		% dashboard
		\req{A}{F}{3} 		& L'utente autenticato deve avere accesso a una Dashboard & UC2 \autism
		\sreq{A}{F}{3.1}  	& La dashboard deve mettere a disposizione le informazioni dell'utente autenticato & UC2 \autism
		\sreq{A}{F}{3.2}  	& La dashboard deve mettere a disposizione i contatti di supporto tecnico & UC2 \autism
		\sreq{B}{F}{3.3}  	& La dashboard deve mettere a disposizione le statistiche generali del sistema & UC2 \autism
		% lista dispositivi
		\req{A}{F}{4} 		& Il membro deve poter visualizzare i dispositivi autorizzati per il suo ente & UC5.1 \autism
		\req{A}{F}{5} 		& Il moderatore ente deve poter visualizzare i dispositivi autorizzati per il suo ente & UC5.1 \autism
		\req{A}{F}{6} 		& L'amministratore deve poter visualizzare i dispositivi censiti nel sistema & UC5.1 \autism
		% impostazioni
		\req{A}{F}{7}  		& L'utente autenticato deve poter visualizzare le proprie impostazioni account & UC4 \autism
		\req{A}{F}{8}  		& L'utente autenticato deve poter modificare le proprie impostazioni account & UC4.1, UC4.2 \autism
		\sreq{A}{F}{8.1}  	& L'utente autenticato deve poter modificare la propria password & UC4.1 \autism
		\sreq{A}{F}{8.2}  	& L'utente autenticato deve poter modificare la propria email & UC4.2.1 \autism
		\sreq{A}{F}{8.3}  	& L'utente autenticato deve poter modificare il proprio username \glock{Telegram} & UC4.2.2 \autism
		\sreq{B}{F}{8.4}  	& L'utente autenticato deve poter attivare l'autenticazione a due fattori tramite \glock{Telegram} & UC4.2.3 \autism
		\sreq{B}{F}{8.5}  	& L'utente autenticato deve poter disattivare l'autenticazione a due fattori tramite \glock{Telegram} & UC4.2.3 \autism
		\sreq{B}{F}{8.6}  	& L'utente autenticato deve poter modificare la preferenza di notifica di uno specifico alert, in base a quelli disponibili & UC4.3, Verbale #9 VE-9.2 \autism
		% info dispositivi
		\req{A}{F}{9} 		& Il membro e il moderatore ente devono poter visualizzare la lista dei sensori di un dispositivo abilitati al loro ente & UC5.2 \autism
		\req{A}{F}{10} 		& L'amministratore deve poter visualizzare la lista completa dei sensori di un qualunque dispositivo censito nel sistema & UC5.8 \autism
		\req{A}{F}{11}  	& Il membro e il moderatore ente devono poter visualizzare i dati in tempo reale di un sensore abilitato al loro ente & UC5.2 \autism
		\sreq{B}{F}{11.1}   & Il membro e il moderatore ente possono visualizzare i dati di un sensore tramite un grafico & UC6.2 \autism
		\req{A}{F}{12} 		& L'amministratore deve poter visualizzare i dati in tempo reale di tutti i sensori appartenenti a un dispositivo censito nel sistema & UC5.2, Verbale #9 VE-9.1 \autism
		\sreq{B}{F}{12.1}   & L'amministratore deve poter visualizzare i dati di un sensore tramite un grafico & UC6.2, Verbale #9 VE-9.1 \autism
		\req{A}{F}{13} 		& L'amministratore deve poter visualizzare a quali enti è stata autorizzata la lettura di un sensore & UC5.2 \autism
		\req{A}{F}{14} 		& L'amministratore deve poter assegnare a un ente la lettura di un sensore di un qualunque dispositivo censito nel sistema & UC5.6 \autism
		\req{A}{F}{15} 		& L'amministratore deve poter rimuovere da un ente la lettura di un sensore di un qualunque dispositivo censito nel sistema & UC5.7 \autism
		% lista membri mod
		\req{A}{F}{16}  	& Un moderatore ente deve poter visualizzare la lista dei membri appartenenti al suo ente & UC7.1 \autism
		\req{A}{F}{17}  	& Un moderatore ente deve poter visualizzare le informazioni di un membro appartenente al suo ente & UC7.3 \autism
		\req{B}{F}{18} 		& Un moderatore ente deve poter modificare l'account di un membro appartenente al suo ente & UC7.4 \autism
		\sreq{B}{F}{18.1}  	& Un moderatore ente deve poter modificare la email di un membro appartenente al suo ente & UC7.4.1 \autism
		\sreq{B}{F}{18.2}  	& Un moderatore ente deve poter modificare il nome di un membro appartenente al suo ente & UC7.4.2 \autism
		\sreq{B}{F}{18.3}  	& Un moderatore ente deve poter modificare il cognome di un membro appartenente al suo ente & UC7.4.3 \autism
		\req{A}{F}{19} 		& Un moderatore ente deve poter rimuovere un membro appartenente al suo ente & UC7.5 \autism
		\req{A}{F}{20} 		& Un moderatore ente deve poter creare un nuovo account per un nuovo membro che apparterrà solo al suo ente & UC7.2 \autism
		% lista membri admin
		\req{A}{F}{21} 		& Un amministratore deve poter visualizzare la lista degli utenti registrati nel sistema & UC11.1 \autism
		\req{A}{F}{22} 		& Un amministratore deve poter visualizzare le informazioni di un utente qualunque & UC11.3 \autism
		\req{A}{F}{23} 		& Un amministratore deve poter disattivare un account di un utente qualunque & UC11.6 \autism
		\req{B}{F}{24} 		& Un amministratore deve poter modificare le impostazioni di un utente qualunque & UC11.4 \autism
		\sreq{B}{F}{24.1}	& Un amministratore deve poter modificare la email di un utente qualunque  & UC11.4.1 \autism
		\sreq{B}{F}{24.2}  	& Un amministratore deve poter modificare il nome di un utente qualunque  & UC11.4.2 \autism
		\sreq{B}{F}{24.3}  	& Un amministratore deve poter modificare il cognome di un utente qualunque  & UC11.4.3 \autism
		\sreq{B}{F}{24.4}  	& Un amministratore deve poter modificare il lo username \glock{Telegram} di un utente qualunque & UC11.4.4 \autism
		\sreq{B}{F}{24.5}  	& Un amministratore deve poter attivare l'autenticazione a due fattori tramite \glock{Telegram} di un utente qualunque & UC11.4.7 \autism
		\sreq{B}{F}{24.6}  	& Un amministratore deve poter disattivare l'autenticazione a due fattori tramite \glock{Telegram} di un utente qualunque & UC11.4.7 \autism
		\req{A}{F}{25}  	& Un amministratore deve poter riassegnare un membro o un moderatore ente a ente differente & UC11.4.5 \autism
		\req{A}{F}{26} 		& Un amministratore deve poter resettare la password a un membro o a un moderatore ente & UC11.5 \autism
		\req{A}{F}{27} 		& Un amministratore deve poter creare un account per un nuovo membro & UC11.2 \autism
		% lista alert
		\req{A}{F}{28} 		& Un moderatore ente deve poter visualizzare la lista degli alert attivi per il suo ente & UC8.1 \autism
		\req{A}{F}{29} 		& Un moderatore ente deve poter aggiungere un alert di un particolare sensore per il suo ente & UC8.2 \autism
		\req{A}{F}{30} 		& Un moderatore ente deve poter rimuovere un alert di un particolare sensore per il suo ente & UC8.3 \autism
		\req{A}{F}{31} 		& Un amministratore deve poter visualizzare la lista degli alert attivi per tutti gli enti & UC12.1 \autism
		\req{A}{F}{32} 		& Un amministratore deve poter rimuovere un alert di un particolare sensore & UC12.2 \autism
		\req{A}{F}{33} 		& I membri e i moderatori ente devono poter ricevere notifiche \glock{Telegram} sulla base delle soglie impostate negli alert attivi per il loro ente & UC15.1 \autism
		% logout
		\req{A}{F}{34} 		& L'utente autenticato deve poter eseguire il logout dalla web app & UC3 \autism
		% view page
		\req{A}{F}{35} 		& L'utente autenticato deve poter visualizzare la lista delle proprie pagine \textit{View} & UC6.1 \autism
		\req{A}{F}{36} 		& L'utente autenticato deve poter creare delle proprie pagine \textit{View} & UC6.3 \autism
		\req{A}{F}{37} 		& L'utente autenticato deve poter cancellare le proprie pagine \textit{View} & UC6.4 \autism
		\req{A}{F}{38} 		& L'utente autenticato deve poter aggiungere grafici in una propria pagina \textit{View} & UC6.5 \autism
		\req{A}{F}{39} 		& L'utente autenticato deve poter visualizzare due dati in un grafico inserito in una propria pagina \textit{View} & UC6.5, UC6.2 \autism
		\sreq{A}{F}{39.1} 	& I grafici nelle pagine \textit{View} devono permettere di visualizzare almeno una correlazione tra due dati & UC6.5 \autism
		\sreq{B}{F}{39.2} 	& I grafici nelle pagine \textit{View} devono permettere di visualizzare almeno tre correlazioni tra due dati & UC6.5 \autism
		\req{A}{F}{40} 		& L'utente autenticato deve poter cancellare grafici da una propria pagina \textit{View} & UC6.6 \autism
		% logs
		\req{C}{F}{41}  	& Il moderatore ente deve poter visualizzare la lista logs degli utenti del suo ente & UC9 \autism
		\req{C}{F}{42}  	& L'amministratore deve poter visualizzare la lista logs degli utenti di sistema & UC13 \autism
		% input e comandi
		\req{A}{F}{43}  	& Il moderatore ente deve poter visualizzare la lista dispositivi autorizzati all'invio dei comandi & UC16.4 \autism
		\req{A}{F}{44}  	& Il moderatore ente deve poter inviare comandi ai singoli dispositivi autorizzati per il loro ente & UC16.5.5 \autism
		\sreq{B}{F}{44.1} 	& L'invio dei comandi deve avvenire tramite un bot \glock{Telegram} & UC16.5.5 \autism
		% configurazione
		\req{A}{F}{45} 		& Un amministratore deve poter visualizzare la configurazione dei dispositivi censiti nel sistema & UC5.2 \autism
		\req{A}{F}{46}  	& Un amministratore deve poter censire un nuovo dispositivo & UC5.3 \autism
		\req{A}{F}{47}  	& Un amministratore deve poter decidere quali dati ricevere da un dispositivo & UC5.3.4 \autism
		\req{A}{F}{48}  	& Un amministratore deve poter decidere con quale frequenza ricevere i dati di un dispositivo & UC5.3.3, Verbale #9 VE-9.3 \autism
		\req{A}{F}{49}  	& Un amministratore deve poter rimuovere un dispositivo censito & UC5.5 \autism
		\req{A}{F}{50}  	& Un amministratore deve poter modificare le configurazione di un dispositivo già censito & UC5.4 \autism
		\hline
		% sistema
		\req{A}{F}{51} 		& Il sistema deve rimuovere automaticamente gli alert attivi di un dispositivo non più censito  & Interna, Verbale #9 VE-9.4 \autism
		\req{A}{F}{52} 		& Il sistema deve rimuovere automaticamente i sensori dei dispositivi autorizzati agli enti che non sono più esistenti & Interna, Verbale #9 VE-9.4 \autism
		\req{A}{F}{53} 		& Il sistema deve rimuovere automaticamente i grafici creati dagli utenti nella pagina \textit{View} se viene rimosso un sensore dal sistema & Interna \autism
		\req{A}{F}{54} 		& Il sistema deve disattivare automaticamente gli utenti che fanno parte di un ente, qualora l'ente venga disabilitato da un amministratore & Interna, Verbale #9 VE-9.4 \autism
		\req{A}{F}{55} 		& Un amministratore deve poter creare un nuovo ente & UC10.3 \autism
		\req{A}{F}{56} 		& Un amministratore deve poter modificare le informazioni di un ente esistente & UC10.4 \autism
		\req{A}{F}{57} 		& Un amministratore deve poter disattivare un ente & UC10.5, Verbale #9 VE-9.4 \autism
		\req{A}{F}{58} 		& Un amministratore deve poter visualizzare gli enti attivi nel sistema & UC10.1 \autism
		\req{A}{F}{59} 		& Il sistema deve disattivare automaticamente gli alert attivi per un utente, qualora l'utente venga disattivato & Interna, Verbale #9 VE-9.4 \autism
		\req{A}{F}{60} 		& Il sistema non deve permettere l'accesso a utenti non amministratori che non fanno parte di un ente & UC18, UC1.4 \autism
		\req{A}{F}{61} 		& Il sistema non deve permettere l'accesso a utenti disattivati & UC18, UC1.4 \autism
		\req{A}{F}{62} 		& Il sistema non deve permettere notifica degli alert agli utenti disattivati & Interna, Verbale #9 VE-9.4 \autism
		\req{A}{F}{63} 		& Il sistema deve permettere notifica degli alert in base alle preferenze indicate nelle impostazioni di un utente & UC15.1, UC4.3 \autism
		\req{A}{F}{64}  	& I dati usati nel sistema devono essere salvati all'interno di una base di dati & Capitolato \autism
		\sreq{A}{F}{64.1} 	& I dati ricevuti dai gateway e usati per la creazione dei grafici vanno salvati all'interno di un time-series database (non relazionale) & Capitolato \autism
		\sreq{A}{F}{64.2} 	& I dati utilizzati per gli account utenti devono essere salvati in un database relazionale. & Capitolato \autism
		% API
		\req{A}{F}{65} 		& Le applicazioni che vogliono utilizzare il sistema devono interfacciarsi tramite delle \glock{API} che si mettono in comunicazione con \glock{Kafka} & Capitolato \autism
		\sreq{A}{F}{65.1} 	& La web app deve interfacciarsi con le \glock{API} & Capitolato \autism
		\sreq{A}{F}{65.2} 	& Il bot di \glock{Telegram} deve interfacciarsi con le \glock{API} & Capitolato \autism
		\hline
		% gateway
		\req{A}{F}{66} 		& Le configurazioni dei \glock{Gateway} devono essere inviate dagli amministratori attraverso le \glock{API} & Capitolato \autism
		\sreq{A}{F}{66.1} 	& Le configurazioni dei \glock{Gateway} devono poter essere sovrascritte & Capitolato \autism
		\req{A}{F}{67} 		& Le \glock{API} devono interfacciarsi necessariamente con \glock{Kafka} per la ricezione e invio di dati & Capitolato \autism
		\req{A}{F}{68} 		& La base di dati si deve interfacciare con \glock{Kafka} per la scrittura e lettura dei dati & Capitolato \autism
		\req{A}{F}{69} 		& I \glock{Gateway} comunicano con \glock{Kafka} tramite \glock{topic} con il modello \glock{Producer-Consumer} & Capitolato \autism
		\req{A}{F}{70} 		& I dispositivi comunicano con i \glock{Gateway} i dati da inviare & Capitolato \autism
		\req{A}{F}{71}  	& La web app deve essere accessibile da browser a tutte le sue funzioni nelle modalità Desktop e Tablet & Capitolato \autism
		\sreq{A}{F}{71.1} 	& La web app deve permettere di vedere grafici e dati da browser in modalità Mobile & Capitolato \autism
		\sreq{B}{F}{71.2} 	& La web app deve permettere di compilare moduli interni da browser in modalità Mobile & Capitolato \autism

        \caption{Requisiti funzionali con le relative descrizioni e le relative fonti}

		\end{longtable}
	\end{center}


	\subsection{Requisiti Prestazionali}

	\begin{center}
		\rowcolors{2}{white}{lightest-grayest}
		\begin{longtable}{|p{3cm}|p{9.85cm}|p{2cm}|}
		\hline
		\rowcolor{green_requisiti}
		{\color{white} \textbf{ID Requisito} } & {\color{white} \textbf{Descrizione} } & {\color{white} \textbf{Fonti} } \\
		\hline
		\endhead
		\hline
        \multicolumn{3}{|c|}{\textit{Continua nella pagina successiva...}}\\
        \hline
        \endfoot
        \endlastfoot

		
		\req{A}{P}{1} 		& I tempi di risposta della web app per disegnare grafici devono essere inferiori a 7.5 secondi & Interna \autism
		\req{A}{P}{2} 		& Il sistema deve poter gestire un carico di almeno 30 utenti connessi contemporaneamente nella web app & Capitolato \autism

        \caption{Requisiti prestazionali con le relative descrizioni e le relative fonti}

		\end{longtable}
	\end{center}

	\subsection{Requisiti di Qualità}

	\begin{center}
		\rowcolors{2}{white}{lightest-grayest}
		\begin{longtable}{|p{3cm}|p{9.85cm}|p{2cm}|}
		\hline
		\rowcolor{gold_requisiti}
		{\color{white} \textbf{ID Requisito} } & {\color{white} \textbf{Descrizione} } & {\color{white} \textbf{Fonti} } \\
		\hline
		\endhead
		\hline
        \multicolumn{3}{|c|}{\textit{Continua nella pagina successiva...}}\\
        \hline
        \endfoot
        \endlastfoot


		\req{A}{Q}{1} 		& Deve essere realizzato e consegnato un documento con i casi d'uso in formato UML & Capitolato \autism
		\req{A}{Q}{2} 		& Deve essere realizzato e consegnato uno schema della base di basi relazionale (diagramma ER)  & Capitolato \autism
		\req{A}{Q}{3} 		& Deve essere realizzato e consegnato uno schema della base di basi non relazionale & Capitolato \autism
		\req{A}{Q}{4}  		& Deve essere realizzata e consegnata una documentazione delle \glock{API} realizzate per l'interazione con \glock{Kafka} e le applicazioni & Capitolato  \autism
		\sreq{A}{Q}{4.1} 	& La documentazione delle API va scritta con la stessa lingua utilizzata per denominare le funzioni messe a disposizione & Capitolato \autism
		\req{A}{Q}{5}  		& Deve essere realizzata e consegnata una lista dei bug risolti durante la fase di sviluppo & Capitolato  \autism
		\req{A}{Q}{6} 		& Deve essere realizzato e consegnato un \glock{Docker} file contenente la componente applicativa & Capitolato  \autism
		\req{A}{Q}{7} 		& Il codice sorgente deve essere realizzato con un sistema di versionamento & Capitolato  \autism
		\req{A}{Q}{8} 		& Devono essere realizzati test di unità e di integrazione per verificare le singole componenti del prodotto & Interna  \autism
		\req{A}{Q}{9} 		& La parte di progettazione e di codifica deve essere conforme con quanto riportato nel \dext{Piano di Qualifica v1.0.0} & Interna  \autism
		\req{B}{Q}{10} 		& La web app dovrà rispettare la validazione \glock{W3C} & Interna  \autism

		\caption{Requisiti di qualità con le relative descrizioni e le relative fonti}

		\end{longtable}
	\end{center}


	\subsection{Requisiti di Vincolo}

	\begin{center}
		\rowcolors{2}{lightest-grayest}{white}
		\begin{longtable}{|p{3cm}|p{9.85cm}|p{2cm}|}
		\hline
		\rowcolor{blue_requisiti}
		{\color{white} \textbf{ID Requisito} } & {\color{white} \textbf{Descrizione} } & {\color{white} \textbf{Fonti} } \\
		\hline
		\endhead
		\hline
        \multicolumn{3}{|c|}{\textit{Continua nella pagina successiva...}}\\
        \hline
        \endfoot
        \endlastfoot

		\req{A}{V}{1} 		& Le istanze del sistema dovranno essere gestite tramite \glock{Docker} & Interna \autism
		\req{A}{V}{2} 		& La ricezione degli alert deve avvenire attraverso un bot \glock{Telegram} & Capitolato \autism
		\req{A}{V}{3} 		& Il sistema deve fare uso dell'ecosistema \glock{Kafka} & Capitolato \autism
		\req{A}{V}{4} 		& Il sistema deve fare uso di un time-series database (TimescaleDB) & Capitolato \autism
		\req{A}{V}{5} 		& Il sistema deve fare uso di un protocollo per comunicare con il \glock{Gatweay} & Capitolato \autism
		\req{A}{V}{6} 		& Il sistema deve fare uso di \glock{API} per la comunicazione con le applicazioni & Capitolato \autism
		\req{A}{V}{7} 		& Il sistema deve mostrare almeno una correlazione tra due dati & Capitolato \autism
		\req{A}{V}{8} 		& Il sistema deve permettere di visualizzare dati inviati da un dispositivo & Capitolato \autism
		\req{A}{V}{9} 		& Il sistema deve permettere di inviare un input a un dispositivo & Capitolato \autism
		\req{A}{V}{10} 		& Il sistema deve permettere di censire un dispositivo & Capitolato \autism

		% SPOSTAMENTO DA REQUISITI PRESTAZIONALI
		\req{A}{V}{11} 		& La web app deve essere compatibile il browser \glock{Firefox} dalla versione 69.0 {\small \color{white}{nice}} & Capitolato \autism
		\req{A}{V}{12} 		& La web app deve essere compatibile con il browser \glock{Chrome} dalla versione 75.0 & Capitolato \autism
		\req{A}{V}{13} 		& La web app deve essere compatibile con il browser \glock{Safari} dalla versione 13.0 & Capitolato \autism
		\req{B}{V}{14} 		& La web app deve essere compatibile con il browser \glock{Edge} dalla versione 42.0 & Capitolato \autism
		\req{A}{V}{15} 		& La web app dovrà essere sviluppata utilizzando il framework \glock{Bootstrap} & Interna, Capitolato  \autism
		\req{A}{V}{16}		& Tutta la documentazione relativa al software deve essere scritta in lingua italiana, in modo da essere consistente con la lingua utilizzata per la codifica & Capitolato \autism
		\sreq{A}{V}{16.1}	& Il software deve essere accompagnato da un manuale amministratore, contenente tutte le informazioni utili per la distribuzione e l'installazione del prodotto & Capitolato \autism
		\sreq{A}{V}{16.2}	& Il software deve essere accompagnato da un manuale utente, contenente tutte le informazioni utili per l'utilizzo del prodotto da parte degli utenti e dei moderatori degli enti & Capitolato \autism


		\caption{Requisiti di vincolo con le relative descrizioni e le relative fonti}

		\end{longtable}
	\end{center}
