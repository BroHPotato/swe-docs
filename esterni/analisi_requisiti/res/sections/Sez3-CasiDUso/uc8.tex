\subsubsection{UC 8 - Gestione alert ente}
		
		
		\begin{itemize}
			\item \textbf{Attori Primari}: [ME]
			\item \textbf{Descrizione}: Il moderatore ente gestisce gli alert del proprio ente.
			\item \textbf{Precondizione}: Il moderatore ente seleziona la voce "Gestione alert ente"
			\item \textbf{Postcondizione}: Il moderatore ente ha visualizzato/gestito gli alert del proprio ente.
			\item \textbf{Scenario Principale}:
			\begin{enumerate}
				\item{Il moderatore ente seleziona la voce "Gestione alert ente"}
				\item{Il moderatore ente visualizza la schermata per la gestione degli alert ente}
				\item{Il moderatore ente visualizza/gestisce gli alert del proprio ente}
				\item{Il moderatore ente ha visualizzato/gestito gli alert del proprio ente}
			\end{enumerate}	
		\end{itemize}
			
			\paragraph{UC 8.1 - Visualizzazione alert ente}
			\begin{itemize}
				\item \textbf{Attori Primari}: [ME]
				\item \textbf{Descrizione}: Il moderatore ente visualizza gli alert appartenenti al proprio ente.
				\item \textbf{Precondizione}: Il moderatore ente visualizza la schermata per la gestione degli alert del proprio ente
				\item \textbf{Postcondizione}: Il moderatore ente ha visualizzato la lista degli alert del proprio ente.
				\item \textbf{Scenario Principale}:
				\begin{enumerate}
					\item{Il moderatore ente visualizza la schermata per la gestione degli alert del proprio ente}
					\item{L'utente seleziona la voce "Visualizza alert ente"}
					\item{L'utente visualizza la lista degli alert del proprio ente}
					\item{L'utente ha visualizzato la lista degli alert del proprio ente}
				\end{enumerate}	
			\end{itemize}
			
			\paragraph{UC 8.2 - Inserimento alert ente}
			\begin{itemize}
				\item \textbf{Attori Primari}: [ME]
				\item \textbf{Descrizione}: Il moderatore ente inserisce un nuovo alert al proprio ente.
				\item \textbf{Precondizione}: Il moderatore ente visualizza la schermata per la gestione degli alert del proprio ente
				\item \textbf{Postcondizione}: Il moderatore ente ha inserito un nuovo alert per gli utenti appartenenti al proprio ente 
				\item \textbf{Scenario Principale}:
				\begin{enumerate}
					\item{Il moderatore ente inserisce un nuovo alert al proprio ente}
					\item{Il moderatore ente seleziona la voce "Inserisci alert ente"}
					\item{Il moderatore ente compila il form alert con i dati dell'alert da inserire}
					\item{Il moderatore ente preme sul bottone di salvataggio per inserire il nuovo alert}
					\item{Il moderatore ente ha inserito un nuovo alert per gli utenti appartenenti al proprio ente }
				\end{enumerate}
				\item \textbf{Inclusioni}:
				\begin{itemize}
					\item Il moderatore ente compila il form alert con i dati dell'alert da inserire (UC 8.2.1)
				\end{itemize}
				\item \textbf{Estensioni}:
				\begin{itemize}
					\item Vengono inseriti dei valori soglia non validi (UC 8.2.2)
				\end{itemize}		
			\end{itemize}
			
			\paragraph{UC 8.2.1 - Compilazione form alert}
			\begin{itemize}
				\item \textbf{Attori Primari}: [ME]
				\item \textbf{Descrizione}: Il moderatore ente compila i campi per l'inserimento dell'alert: il campo "dato", corrispondente al dato da monitorare dell'alert, e il campo "soglia", corrispondente al valore soglia dopo il quale si riceverà un alert.
				\item \textbf{Precondizione}: Il moderatore ente ha selezionato la voce "Inserisci alert ente".
				\item \textbf{Postcondizione}: Il moderatore ente ha compilato il form alert.
				\item \textbf{Scenario Principale}:
				\begin{enumerate}
					\item{Il moderatore ente seleziona la voce "Inserisci alert ente"}
					\item{Il moderatore ente compila il campo "dato"}
					\item{Il moderatore ente compila il campo "soglia"}
					\item{Il moderatore ente ha compilato il form alert}
				\end{enumerate}	
			\end{itemize}

			\paragraph{UC 8.2.2 - Errore valori soglia}
			\begin{itemize}
				\item \textbf{Attori Primari}: [ME]
				\item \textbf{Descrizione}: Dopo aver premuto il bottone per salvare i dati inseriti nel form alert viene visualizzato il messaggio di errore "Valore soglia non valido" perchè il valore inserito come valore, oltre il quale viene ricevuto un alert, non è valido.
				\item \textbf{Precondizione}: Il moderatore ente ha premuto sul bottone di salvataggio del form alert
				\item \textbf{Postcondizione}: Visualizzazione messaggio di errore "Valore soglia non valido" 
				\item \textbf{Scenario Principale}:
				\begin{enumerate}
					\item{Il moderatore ente ha premuto sul bottone di salvataggio del form alert}
					\item{Il valore soglia inserito nel campo "soglia" non è valido}
					\item{Viene visualizzato il messaggio di errore "Valore soglia non valido" }
				\end{enumerate}
			\end{itemize}
			
			\paragraph{UC 8.3 - Rimozione alert ente}
			\begin{itemize}
				\item \textbf{Attori Primari}: [ME]
				\item \textbf{Descrizione}: Il moderatore ente rimuove l'alert selezionato dalla lista degli alert del proprio ente.
				\item \textbf{Precondizione}: Il moderatore ente seleziona un alert dalla lista degli alert del proprio ente.
				\item \textbf{Postcondizione}: Il moderatore ente ha rimosso l'alert selezionato del proprio ente dal sistema.
				\item \textbf{Scenario Principale}:
				\begin{enumerate}
					\item{Il moderatore ente seleziona un alert dalla lista degli alert del proprio ente}
					\item{Il moderatore ente seleziona la voce "Rimuovi alert ente"}
					\item{Il moderatore ente ha rimosso l'alert selezionato del proprio ente dal sistema.}
				\end{enumerate}	
			\end{itemize}