	\subsection{UC 7 - Ente - Gestione utenti}
		
		\begin{itemize}
			\item \textbf{Attori Primari}: Moderatore ente.
			\item \textbf{Descrizione}: Il moderatore ente gestisce gli utenti del proprio ente.
			\item \textbf{Precondizione}: Il moderatore ente seleziona la voce "Gestione utenti ente".
			\item \textbf{Postcondizione}: Il moderatore ente ha visualizzato/gestito gli utenti appartenenti al proprio ente.
			\item \textbf{Scenario Principale}:
			\begin{enumerate}
				\item{Il moderatore ente seleziona la voce "Gestione utenti ente"}
				\item{Il moderatore ente visualizza la schermata per la gestione degli utenti ente}
				\item{Il moderatore ente visualizza/gestisce gli utenti appartenenti al proprio ente}
				\item{Il moderatore ente ha visualizzato/gestito gli utente del proprio ente}
			\end{enumerate}	
		\end{itemize}
			
			\subsubsection{UC 7.1 - Visualizzazione utenti ente}
			\begin{itemize}
				\item \textbf{Attori Primari}: Moderatore ente.
				\item \textbf{Descrizione}: Il moderatore ente visualizza gli utenti del proprio ente.
				\item \textbf{Precondizione}: Il moderatore ente visualizza la schermata per la gestione degli utenti ente.
				\item \textbf{Postcondizione}: Il moderatore ente ha visualizzato la lista degli utenti appartenenti al proprio ente.
				\item \textbf{Scenario Principale}:
				\begin{enumerate}
					\item{Il moderatore ente visualizza la schermata per la gestione degli utenti ente}
					\item{Il moderatore ente visualizza la lista degli utenti appartenenti al proprio ente}
					\item{Il moderatore ente ha visualizzato gli utenti del proprio ente}
				\end{enumerate}	
			\end{itemize}
			
			\subsubsection{UC 7.2 - Creazione account membro ente}
			\begin{itemize}
				\item \textbf{Attori Primari}: Moderatore ente.
				\item \textbf{Descrizione}: Il moderatore ente crea un nuovo membro al proprio ente.
				\item \textbf{Precondizione}: Il moderatore ente visualizza la schermata per la gestione dei membri dell'ente.
				\item \textbf{Postcondizione}: Il moderatore ente ha creato un nuovo account associato al proprio ente.
				\item \textbf{Scenario Principale}:
				\begin{enumerate}
					\item{Il moderatore ente visualizza la schermata per la gestione degli utenti ente}
					\item{Il moderatore ente seleziona la voce "aggiungi utente ente"}
					\item{Il moderatore ente compila il form utente con i dati dell'utente da aggiungere}
					\item{Il moderatore ente preme sul bottone di salvataggio per aggiungere il nuovo utente}
					\item{Il moderatore ente ha aggiunto un nuovo utente appartenente al proprio ente}
				\end{enumerate}	
				\item \textbf{Inclusioni}:
					\item Il moderatore ente compila il form utente con i dati dell'utente da aggiungere (UC 7.2.1)
				\item \textbf{Estensioni}:
				\begin{itemize}
					\item Il moderatore ente inserisce un email non valida (UC 7.4)
					\item Il moderatore ente inserisce un nome e/o cognome non validi (UC 7.5)
				\end{itemize}
			\end{itemize}
			
			\subsubsection{UC 7.2.1 - Compilazione form nuovo utente}
			\begin{itemize}
				\item \textbf{Attori Primari}: Moderatore ente.
				\item \textbf{Descrizione}: Il moderatore ente compila i campi per l'aggiunta dell'utente: il campo "email", corrispondente all'email dell'utente, e i campi "nome" e "cognome", corrispondenti al nominativo dell'utente.
				\item \textbf{Precondizione}: Il moderatore ente ha selezionato la voce "crea nuovo membro".
				\item \textbf{Postcondizione}: Il moderatore ente ha compilato il form per la creazione di un nuovo membro.
				\item \textbf{Scenario Principale}:
				\begin{enumerate}
					\item{Il moderatore ente seleziona la voce per la creazione di un nuovo membro}
					\item{Il moderatore ente compila il campo "email"}
					\item{Il moderatore ente compila il campo "nome"}
					\item{Il moderatore ente compila il campo "cognome"}
					\item{Il moderatore ente ha compilato il form utente}
				\end{enumerate}	
			\end{itemize}

			\subsubsection{UC 7.3 - Visualizzazione profilo di un membro dell'ente}
			\begin{itemize}
				\item \textbf{Attori Primari}: Moderatore ente.
				\item \textbf{Descrizione}: Il moderatore ente visualizza i dati dell'utente selezionato dalla lista degli utenti del proprio ente.
				\item \textbf{Precondizione}: Il moderatore ente seleziona un utente dalla lista degli utenti del proprio ente.
				\item \textbf{Postcondizione}: Il moderatore ente ha visualizzato i dati dell'utente selezionato appartenente al proprio ente.
				\item \textbf{Scenario Principale}:
				\begin{enumerate}
					\item{Il moderatore ente seleziona un utente dalla lista degli utenti del proprio ente}
					\item{Il moderatore ente seleziona la voce per visualizzare i dati dell'utente}
					\item{Il moderatore ente visualizza i dati dell'utente selezionato}
				\end{enumerate}
			\end{itemize}


			\subsubsection{UC 7.4 - Modifica profilo membro dell'ente}
			\begin{itemize}
				\item \textbf{Attori Primari}: Moderatore ente.
				\item \textbf{Descrizione}: Il moderatore ente modifica l'utente selezionato dalla lista degli utenti del proprio ente.
				\item \textbf{Precondizione}: Il moderatore ente seleziona un utente dalla lista degli utenti del proprio ente.
				\item \textbf{Postcondizione}: Il moderatore ente ha modificato l'utente selezionato appartenente al proprio ente.
				\item \textbf{Scenario Principale}:
				\begin{enumerate}
					\item{Il moderatore ente seleziona un utente dalla lista degli utenti del proprio ente}
					\item{Il moderatore seleziona la voce "modifica utente ente"}
					\item{Il moderatore compila il form utente contenente i dati da modificare dell'utente}
					\item{Il moderatore ente preme sul bottone di salvataggio per modificare l'utente selezionato}
					\item{Il moderatore ha modificato l'utente selezionato}
				\end{enumerate}	
				\item \textbf{Inclusioni}:
					\item Il moderatore compila il form utente con i dati dell'utente da modificare (UC 7.4.1)
				\item \textbf{Estensioni}:
				\begin{itemize}
					\item Il moderatore inserisce un email non valida (UC 7.6)
					\item Il moderatore inserisce un nome e/o cognome non validi (UC 7.7)
				\end{itemize}
			\end{itemize}

			\subsubsection{UC 7.4.1 - Compilazione form modifica utente}
			\begin{itemize}
				\item \textbf{Attori Primari}: Moderatore ente.
				\item \textbf{Descrizione}: Il moderatore ente compila i campi da modificare a un utente: il campo "email", corrispondente all'email dell'utente, e i campi "nome" e "cognome", corrispondenti al nominativo dell'utente.
				\item \textbf{Precondizione}: Il moderatore ente ha selezionato la voce per modificare un membro dell'ente.
				\item \textbf{Postcondizione}: Il moderatore ente ha compilato il form per la modifica profilo di un membro dell'ente.
				\item \textbf{Scenario Principale}:
				\begin{enumerate}
					\item{Il moderatore ente seleziona la voce per la modifica di un nuovo membro}
					\item{Il moderatore ente compila il campo "email"}
					\item{Il moderatore ente compila il campo "nome"}
					\item{Il moderatore ente compila il campo "cognome"}
					\item{Il moderatore ente ha compilato il form utente}
				\end{enumerate}	
			\end{itemize}
			
			\subsubsection{UC 7.5 - Rimozione utente dall'ente}
			\begin{itemize}
				\item \textbf{Attori Primari}: Moderatore ente.
				\item \textbf{Descrizione}: Il moderatore ente rimuove l'utente selezionato dalla lista degli utenti appartenenti al suo ente.
				\item \textbf{Precondizione}: Il moderatore ente seleziona un utente dalla lista membri del proprio ente.
				\item \textbf{Postcondizione}: Il moderatore ente ha rimosso l'utente selezionato appartenente al proprio ente.
				\item \textbf{Scenario Principale}:
				\begin{enumerate}
					\item{Il moderatore ente seleziona un utente appartenente al proprio ente da rimuovere}
					\item{Il moderatore ente seleziona la voce "Rimuovi utente ente"}
					\item{Il moderatore ente ha rimosso l'utente selezionato appartenente al proprio ente dal sistema.}
				\end{enumerate}		
			\end{itemize}

			\subsubsection{UC 7.6 - Errore: email non valida}
			\begin{itemize}
				\item \textbf{Attori Primari}: Moderatore ente.
				\item \textbf{Descrizione}: Dopo aver premuto il bottone per salvare i dati inseriti nel form utente viene visualizzato il messaggio di errore "Email non valida" perchè l'email inserita non è valida. 
				\item \textbf{Precondizione}: Il moderatore ente ha premuto sul bottone di salvataggio del form utente.
				\item \textbf{Postcondizione}: Visualizzazione messaggio di errore "Email non valida".
				\item \textbf{Scenario Principale}:
				\begin{enumerate}
					\item{Il moderatore ente ha premuto sul bottone di salvataggio del form utente}
					\item{La email inserita nel campo "email" non è valida}
					\item{Viene visualizzato il messaggio di errore "Email non valida"}
				\end{enumerate}	
			\end{itemize}
			
			\subsubsection{UC 7.7 - Errore: nome e/o cognome non validi}
			\begin{itemize}
				\item \textbf{Attori Primari}: Moderatore ente.
				\item \textbf{Descrizione}: Dopo aver premuto il bottone per salvare i dati inseriti nel form utente viene visualizzato il messaggio di errore "Nome e/o cognome non validi" perchè il nome e/o il cognome inseriti non sono validi. 
				\item \textbf{Precondizione}: Il moderatore ente ha premuto sul bottone di salvataggio del form utente.
				\item \textbf{Postcondizione}: Visualizzazione messaggio di errore "Nome e/o cognome non validi".
				\item \textbf{Scenario Principale}:
				\begin{enumerate}
					\item{Il moderatore ente ha premuto sul bottone di salvataggio del form utente}
					\item{Il nome nel campo "nome" non è valido e/o il cognome nel campo "cognome" non è valido}
					\item{Viene visualizzato il messaggio di errore "Nome e/o cognome non validi"}
				\end{enumerate}	
			\end{itemize}