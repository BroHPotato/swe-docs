	\subsection{UC 4 - Impostazioni Account}
		
		%\begin{figure}[H]
		%	\centering
		%	\includegraphics[height=30em]{res/images/UC4 - Impostazioni.jpg}
		%\end{figure}
		
		\begin{itemize}
			\item \textbf{Attori Primari}: Utente autenticato.
			\item \textbf{Descrizione}: L'utente ha la possibilità di gestire le proprie impostazioni account, tra cui la modifica della password, le preferenze di notifica e le informazioni a lui associate.
			\item \textbf{Precondizione}: L'utente risulta autenticato all'interno della web app.
			\item \textbf{Postcondizione}: L'utente ha aggiornato le proprie impostazioni.
			\item \textbf{Scenario Principale}:
			\begin{enumerate}
				\item{L'utente naviga all'interno delle impostazioni del proprio account}
				\item{L'utente modifica le proprie impostazioni}
				\item{L'utente ha aggiornato le proprie impostazioni}
			\end{enumerate}	
		\end{itemize}
			

			\subsubsection{UC 4.1 - Modifica password account}
			\begin{itemize}
				\item \textbf{Attori Primari}: Utente autenticato.
				\item \textbf{Descrizione}: L'utente può cambiare la password associata al proprio account.
				\item \textbf{Precondizione}: L'utente naviga all'interno delle sue impostazioni.
				\item \textbf{Postcondizione}: L'utente ha cambiato la propria password.
				\item \textbf{Scenario Principale}:
				\begin{enumerate}
					\item L'utente deve inserire dei campi obbligatori per proseguire;
					\item L'utente inserisce il campo per la password attuale (4.1.1);
					\item L'utente inserisce il campo per la nuova password (4.1.2);
					\item L'utente inserisce il campo per la conferma della nuova password (4.1.3);
					\item L'utente ha cambiato la propria password.
				\end{enumerate}	
				\item \textbf{Estensioni}:
					\begin{itemize}
						\item Errore modifica password: password attuale errata (UC 4.4)
						\item Errore modifica password: nuova password non valida (UC 4.5)
						\item Errore modifica password: la password da confermare non coincide (UC 4.6)
					\end{itemize}
			\end{itemize}

				\paragraph{UC 4.1.1 - Inserimento password attuale}
				\begin{itemize}
					\item \textbf{Attori Primari}: Utente autenticato.
					\item \textbf{Descrizione}: Per proseguire nella modifica password, l'utente deve inserire la sua password attuale associata all'account. Il campo è obbligatorio.
					\item \textbf{Precondizione}: L'utente è all'interno delle sue impostazioni account.
					\item \textbf{Postcondizione}: L'utente ha compilato il campo richiesto.
					\item \textbf{Scenario Principale}:
					\begin{enumerate}
						\item L'utente compila il campo per la password attuale.
					\end{enumerate}
				\end{itemize}

				\paragraph{UC 4.1.2 - Inserimento nuova password}
				\begin{itemize}
					\item \textbf{Attori Primari}: Utente autenticato.
					\item \textbf{Descrizione}: Per proseguire nella modifica password, l'utente deve scegliere e inserire una nuova password. Il campo è obbligatorio.
					\item \textbf{Precondizione}: L'utente è all'interno delle sue impostazioni account.
					\item \textbf{Postcondizione}: L'utente ha compilato il campo richiesto.
					\item \textbf{Scenario Principale}:
					\begin{enumerate}
						\item L'utente compila il campo per la nuova password.
					\end{enumerate}
				\end{itemize}

				\paragraph{UC 4.1.3 - Inserimento conferma nuova password}
				\begin{itemize}
					\item \textbf{Attori Primari}: Utente autenticato.
					\item \textbf{Descrizione}: Per proseguire nella modifica password, l'utente deve ripetere la nuova password scelta. Il campo è obbligatorio.
					\item \textbf{Precondizione}: L'utente è all'interno delle sue impostazioni account.
					\item \textbf{Postcondizione}: L'utente ha compilato il campo richiesto.
					\item \textbf{Scenario Principale}:
					\begin{enumerate}
						\item L'utente compila il campo per la conferma della nuova password.
					\end{enumerate}
				\end{itemize}

			\subsubsection{UC 4.2 - Modifica informazioni account}
			\begin{itemize}
				\item \textbf{Attori Primari}: Utente autenticato.
				\item \textbf{Descrizione}: L'utente può modificare le proprie informazioni associate all'account.
				\item \textbf{Precondizione}: L'utente naviga all'interno delle sue impostazioni.
				\item \textbf{Postcondizione}: L'utente ha cambiato le proprie informazioni account.
				\item \textbf{Scenario Principale}:
				\begin{enumerate}
					\item L'utente deve inserire dei campi per proseguire;
					\item L'utente modifica il campo relativo alla propria email (4.2.1);
					\item L'utente modifica il campo relativo allo username di \glock{Telegram} (4.2.2);
					\item L'utente seleziona la preferenza per l'abilitazione dell'autenticazione a due fattori (4.2.3);
					\item L'utente ha cambiato le proprie informazioni associate al suo account.
				\end{enumerate}	
				\item \textbf{Estensioni}:
					\begin{itemize}
						\item L'utente inserisce un email non valida (UC 21);
						\item L'utente inserisce uno username \glock{Telegram} non valido (UC 22).
					\end{itemize}
			\end{itemize}

				\paragraph{UC 4.2.1 - Modifica della propria email}
				\begin{itemize}
					\item \textbf{Attori Primari}: Utente autenticato.
					\item \textbf{Descrizione}: Per proseguire nella modifica delle informazioni, l'utente deve modificare il campo della propria email. Il campo è obbligatorio.
					\item \textbf{Precondizione}: L'utente è all'interno delle sue impostazioni account.
					\item \textbf{Postcondizione}: L'utente ha compilato il campo richiesto.
					\item \textbf{Scenario Principale}:
					\begin{enumerate}
						\item L'utente compila il campo per la email.
					\end{enumerate}
				\end{itemize}

				\paragraph{UC 4.2.2 - Modifica dello username \glock{Telegram}}
				\begin{itemize}
					\item \textbf{Attori Primari}: Utente autenticato.
					\item \textbf{Descrizione}: Per proseguire nella modifica delle informazioni, l'utente deve modificare il proprio username \glock{Telegram}. Il campo è obbligatorio.
					\item \textbf{Precondizione}: L'utente è all'interno delle sue impostazioni account.
					\item \textbf{Postcondizione}: L'utente ha compilato il campo richiesto.
					\item \textbf{Scenario Principale}:
					\begin{enumerate}
						\item L'utente compila il campo dello username \glock{Telegram}.
					\end{enumerate}
				\end{itemize}

				\paragraph{UC 4.2.3 - Modifica delle preferenze per l'autenticazione a due fattori}
				\begin{itemize}
					\item \textbf{Attori Primari}: Utente autenticato.
					\item \textbf{Descrizione}: Per proseguire nella modifica delle informazioni, l'utente deve selezionare le preferenze per l'abilitazione o meno dell'autenticazione a due fattori.
					\item \textbf{Precondizione}: L'utente è all'interno delle sue impostazioni account.
					\item \textbf{Postcondizione}: L'utente ha compilato il campo richiesto.
					\item \textbf{Scenario Principale}:
					\begin{enumerate}
						\item L'utente seleziona la preferenza per l'autenticazione a due fattori (abilitata o disabilitata).
					\end{enumerate}
				\end{itemize}


			\subsubsection{UC 4.3 - Modifica preferenze notifiche alert}
			\begin{itemize}
				\item \textbf{Attori Primari}: Utente autenticato.
				\item \textbf{Descrizione}: L'utente può modificare le preferenze dei singoli alert a lui attivi, in base a quelli a lui disponibili.
				\item \textbf{Precondizione}: L'utente naviga all'interno delle sue impostazioni.
				\item \textbf{Postcondizione}: L'utente ha cambiato le proprie preferenze per gli alert.
				\item \textbf{Scenario Principale}:
				\begin{enumerate}
					\item{L'utente seleziona la preferenza di uno o più alert, in base a quelli disponibili;}
					\item{Le preferenze alert dell'utente vengono aggiornate.}
				\end{enumerate}
			\end{itemize}	

			\subsubsection{UC 4.4 Errore modifica password: password attuale errata}
			\begin{itemize}
				\item \textbf{Attori Primari}: Utente autenticato.
				\item \textbf{Descrizione}: Durante la modifica della password, il sistema rileva che la password attualmente associata all'account non è valida.
				\item \textbf{Precondizione}: L'utente ha compilato i campi richiesti e il sistema elabora la richiesta.
				\item \textbf{Postcondizione}: Viene visualizzato un messaggio di errore specifico.
				\item \textbf{Scenario Principale}:
				\begin{enumerate}
					\item Il sistema sta elaborando la richiesta;
					\item Viene visualizzato un messaggio di errore che segnala che la password attuale è errata.
				\end{enumerate}
			\end{itemize}

			\subsubsection{UC 4.5 Errore modifica password: nuova password non valida}
			\begin{itemize}
				\item \textbf{Attori Primari}: Utente autenticato.
				\item \textbf{Descrizione}: Durante la modifica della password, il sistema rileva che la nuova password scelta non è valida, dal momento che potrebbe essere uguale a quella attuale o troppo corta.
				\item \textbf{Precondizione}: L'utente ha compilato i campi richiesti e il sistema elabora la richiesta.
				\item \textbf{Postcondizione}: Viene visualizzato un messaggio di errore specifico.
				\item \textbf{Scenario Principale}:
				\begin{enumerate}
					\item Il sistema sta elaborando la richiesta;
					\item Viene visualizzato un messaggio di errore che segnala che la nuova password non è valida per uno dei seguenti motivi:
					\begin{itemize}
						\item è troppo corta (almeno 6 caratteri);
						\item è uguale alla password attuale.
					\end{itemize}
				\end{enumerate}
			\end{itemize}

			\subsubsection{UC 4.6 - Errore modifica password: la password da confermare non coincide}
			\begin{itemize}
				\item \textbf{Attori Primari}: Utente autenticato.
				\item \textbf{Descrizione}: Durante la modifica della password, il sistema rileva che la nuova password scelta non coincide con la password riportata in conferma password.
				\item \textbf{Precondizione}: L'utente ha compilato i campi richiesti e il sistema elabora la richiesta.
				\item \textbf{Postcondizione}: Viene visualizzato un messaggio di errore specifico.
				\item \textbf{Scenario Principale}:
				\begin{enumerate}
					\item Il sistema sta elaborando la richiesta;
					\item Viene visualizzato un messaggio di errore che segnala che la password attuale è errata.
				\end{enumerate}
			\end{itemize}
			
