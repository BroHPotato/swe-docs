\subsection{UC 10 - Amministrazione - Gestione enti}

		\begin{itemize}
			\item \textbf{Attori Primari}: Amministratore.
			\item \textbf{Descrizione}: L'utente, che sta visualizzando il menù, seleziona la voce "Gestione enti" che permette la gestione degli enti all'interno del sistema.
			\item \textbf{Precondizione}: L'utente visualizza il menù di navigazione.
			\item \textbf{Postcondizione}: L'utente ha visualizzato/gestito gli enti all'interno del sistema. 
			\item \textbf{Scenario Principale}:
			\begin{enumerate}
				\item{L'utente seleziona la voce "Gestione enti"}
				\item{L'utente può svolgere diverse azioni allo scopo di gestire gli enti all'interno del sistema}
			\end{enumerate}	
		\end{itemize}

			\subsubsection{UC 10.1 - Visualizzazione lista enti }
			\begin{itemize}
				\item \textbf{Attori Primari}: Amministratore.
				\item \textbf{Descrizione}: L'utente, che sta visualizzando la schermata per la gestione enti, visualizza la lista degli enti presenti nel sistema.
				\item \textbf{Precondizione}: L'utente visualizza la schermata per la gestione degli enti.
				\item \textbf{Postcondizione}: L'utente visualizza la lista degli enti presenti nel sistema.
				\item \textbf{Scenario Principale}:
				\begin{enumerate}
					\item{L'utente seleziona la voce "visualizza enti"}
					\item{L'utente visualizza la lista enti}
				\end{enumerate}	
			\end{itemize}

			\subsubsection{UC 10.2 - Visualizzazione informazioni ente}
			\begin{itemize}
				\item \textbf{Attori Primari}: Amministratore.
				\item \textbf{Descrizione}: L'amministratore, che sta visualizzando la lista degli enti, seleziona un ente e ne visualizza le informazioni riguardanti.
				\item \textbf{Precondizione}: L'utente visualizza la la schermata gestione enti e la lista degli enti.
				\item \textbf{Postcondizione}: L'utente visualizza le informazioni di un ente selezionato.
				\item \textbf{Scenario Principale}:
				\begin{enumerate}
					\item{L'utente seleziona dalla lista un ente}
					\item{L'utente visualizza le informazioni riguardanti l'utente selezionato}
				\end{enumerate}	
			\end{itemize}

			\subsubsection{UC 10.3 - Aggiunta nuovo ente}
			\begin{itemize}
				\item \textbf{Attori Primari}: Amministratore.
				\item \textbf{Descrizione}: L'amministratore, che sta visualizzando la schermata per la gestione degli enti, aggiunge un nuovo ente al sistema.
				\item \textbf{Precondizione}: L'utente visualizza la schermata per la gestione degli enti.
				\item \textbf{Postcondizione}: L'utente ha creato un nuovo ente.
				\item \textbf{Scenario Principale}:
				\begin{enumerate}
					\item{L'utente inserisce i dati nei campi}
					\item{L'ente viene creato dall'utente con le informazioni fornite}
				\end{enumerate}	
				\item \textbf{Inclusioni}:
					\begin{itemize}
						\item Compilazione form aggiunta nuovo ente (UC 10.3.1)
					\end{itemize}
				\item \textbf{Estensioni}:
					\begin{itemize}
						\item Il nome ente che si sta tentando di inserire è già presente (UC 10.6)
					\end{itemize}
			\end{itemize}	

			\subsubsection{UC 10.3.1 - Compilazione form aggiunta nuovo ente}
			\begin{itemize}
				\item \textbf{Attori Primari}: Amministratore.
				\item \textbf{Descrizione}: L'amministratore ha deciso di aggiungere un nuovo ente al sistema e deve compilare un form.
				\item \textbf{Precondizione}: L'utente visualizza la schermata per la gestione degli enti e clicca sul bottone di aggiunta nuovo ente.
				\item \textbf{Postcondizione}: L'utente ha creato un nuovo ente.
				\item \textbf{Scenario Principale}:
				\begin{enumerate}
					\item{L'utente visualizza un form da compilare per inserire un nuovo ente}
					\item{L'utente inserisce i seguenti campi obbligatori e invia il form:}
					\begin{itemize}
						\item nome ente;
						\item nome ente per esteso;
						\item sede o luogo in cui risiede l'ente;
					\end{itemize}
					\item{L'ente viene creato dall'utente con le informazioni fornite}
				\end{enumerate}	
			\end{itemize}			

			\subsubsection{UC 10.4 - Modifica ente}
			\begin{itemize}
				\item \textbf{Attori Primari}: Amministratore.
				\item \textbf{Descrizione}: L'amministratore, che sta visualizzando la lista con tutti gli enti creati fino a quel momento, seleziona l'ente di cui desidera modificare le informazioni. 
				\item \textbf{Precondizione}: L'utente visualizza la lista degli enti disponibili nel sistema.
				\item \textbf{Postcondizione}: L'utente ha modificato l'ente selezionato.
				\item \textbf{Scenario Principale}:
				\begin{enumerate}
					\item{L'utente seleziona l'ente da modificare}
					\item{L'utente modifica i dati dell'ente selezionato}
				\end{enumerate}
				\item \textbf{Inclusioni}:
					\begin{itemize}
						\item Compilazione form modifica nuovo ente (UC 10.4.1)
					\end{itemize}
				\item \textbf{Estensioni}:
					\begin{itemize}
						\item Il nome ente che si sta tentando di inserire è già presente (UC 10.6)
					\end{itemize}
			\end{itemize}	

			\subsubsection{UC 10.4.1 - Compilazione form modifica ente}
			\begin{itemize}
				\item \textbf{Attori Primari}: Amministratore.
				\item \textbf{Descrizione}: L'amministratore ha deciso di modificare ente presente sistema e deve compilare un form.
				\item \textbf{Precondizione}: L'utente visualizza la schermata per la gestione degli enti e clicca sul bottone di modifica di un ente esistente.
				\item \textbf{Postcondizione}: L'utente ha creato un nuovo ente.
				\item \textbf{Scenario Principale}:
				\begin{enumerate}
					\item{L'utente visualizza un form da compilare per inserire un nuovo ente}
					\item{L'utente modifica i seguenti campi obbligatori, a sua discrezione, e invia il form:}
					\begin{itemize}
						\item nome ente;
						\item nome ente per esteso;
						\item sede o luogo in cui risiede l'ente;
					\end{itemize}
					\item{L'ente viene aggiornato dall'utente in base alle informazioni fornite}
				\end{enumerate}	
			\end{itemize}		

			\subsubsection{UC 10.5 - Disabilitazione ente}
			\begin{itemize}
				\item \textbf{Attori Primari}: Amministratore.
				\item \textbf{Descrizione}: L'amministratore, che sta visualizzando la lista degli enti, seleziona un ente e lo disabilita dal sistema.
				\item \textbf{Precondizione}: L'utente visualizza la lista degli enti appartenenti al sistema.
				\item \textbf{Postcondizione}: L'utente he rimosso un ente.
				\item \textbf{Scenario Principale}:
				\begin{enumerate}
					\item{L'utente seleziona l'ente da rimuovere.}
					\item{L'ente selezionato viene disabilitato dal sistema.}
				\end{enumerate}	
			\end{itemize}		