\section{Introduzione}
	\subsection{Scopo del documento}
		Lo scopo di questo documento è la candidatura del gruppo Red Round Robin allo svolgimento del progetto relativo al capitolato C6 - ThiReMa.
		All'interno di questa analisi è  possibile seguire la classificazione, il tracciamento e la descrizione dettagliata dei requisiti individuati dall'analisi del capitolato scelto.
	\subsection{Glossario e documenti esterni}
		Per evitare possibili ambiguità relative alle terminologie (che andranno indicate in \textsc{maiuscoletto}) utilizzate nei vari documenti, verranno utilizzate due simboli:
		\begin{itemize}
			\item una \textit{D} al pedice per indicare il nome di un particolare documento;
			\item una \textit{G} al pedice per indicare un termine che sarà presente nel \dext{Glossario v2.0.0}.
		\end{itemize}
	\subsection{Riferimenti}
		\subsubsection{Normativi}
			\begin{itemize}
				\item \textbf{norme di progetto: }\dext{Norme di Progetto v2.0.0} 
				\item \textbf{capitolato C6 - ThiReMa: }\url{https://www.math.unipd.it/~tullio/IS-1/2019/Progetto/C6.pdf}
			\end{itemize}
		\subsubsection{Informativi}
			\begin{itemize}
				\item \textbf{presentazione seminario capitolato C6 - ThiReMa: }\url{https://www.math.unipd.it/~tullio/IS-1/2019/Progetto/C6a.pdf};
				\item \textbf{slide Ingegneria del Software - Analisi dei requisiti: }\url{https://www.math.unipd.it/~tullio/IS-1/2019/Dispense/L08.pdf};
				\item \textbf{slide Ingegneria del Software - Diagrammi dei casi d'uso: }\url{https://www.math.unipd.it/~tullio/IS-1/2019/Dispense/E03.pdf}.
			\end{itemize}
