\section*{Tracciamento delle Decisioni}

\begin{center}
	\rowcolors{2}{lightest-grayest}{white}
	\begin{longtable}{|c|p{12.25cm}|}
	\hline
	\rowcolor{lighter-grayer}
	\textbf{Codice} & \textbf{Descrizione} \\
	\hline
	\endfirsthead
	
	\hline
	VE\_2020-01-10\_9.1 & Le funzionalità delle view devono essere disponibili anche per l'amministratore. \\
	\hline
	VE\_2020-01-10\_9.2 & Deve essere possibile, da parte degli utenti, poter selezionare quali alert visualizzare. \\
	\hline
	VE\_2020-01-10\_9.3 & Gli alert devono essere inviati con una frequenza appropriata. \\
	\hline
	VE\_2020-01-10\_9.4 & Gli alert e i device devono essere eliminati dal db fisicamente, mentre gli enti e gli utenti logicamente (flag disabilitato 1 o 0) \\
	\hline
	VE\_2020-01-10\_9.5 & Utilizzo di swagger come strumento per la doumentazione delle API fornite in formato JSON. \\
	\hline
	VE\_2020-01-10\_9.6 & I tempi di risposta delle pagine devono essere accettabili. \\
	\hline
	VE\_2020-01-10\_9.7 & Deve esserci un buon tempo di calcolo delle correlazioni. \\
	\hline
	VE\_2020-01-10\_9.8 & È richiesto che il sistema gestisca n utenti contemporaneamente. \\
	\hline
	\caption{Tabella contenente il tracciamento delle decisioni prese durante lo svolgimento della riunione}
	\end{longtable}
\end{center}