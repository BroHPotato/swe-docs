\section*{Introduzione}

\subsection*{Luogo e data dell'incontro}
	\begin{itemize}
		\item \textbf{Luogo:} Sede aziendale San Marco Informatica (Via Alcide de Gasperi 132, Grisignano di Zocco(VI) )
		\item \textbf{Data:} 2020-01-10
		\item \textbf{Ora di inizio:} 14:30
		\item \textbf{Ora di fine:} 16:30
	\end{itemize}

\subsection*{Ordine del giorno}
	\begin{enumerate}
		\item Discussione sui casi d'uso individuati
		\item Discussione sui requisiti individuati
		\item Varie ed eventuali
	\end{enumerate}

\subsection*{Presenze}
	\begin{itemize}
		\item \textbf{Totale presenti:} 6 su 7
		\item \textbf{Presenti: }
			\begin{itemize}			
				\item Giuseppe Vito Bitetti
				\item Lorenzo Dei Negri
				\item Nicolò Frison
				\item Fouad Mouad
				\item Mariano Sciacco
				\item Alessandro Tommasin
			\end{itemize}
		\item \textbf{Assenti: } 
			\begin{itemize}	
				\item Giovanni Vidotto
			\end{itemize}
		\item \textbf{Partecipanti esterni:}
			\begin{itemize}
				\item Alex Beggiato (Referente, San Marco Informatica)
			\end{itemize}
	\end{itemize}


\newpage
\section*{Svolgimento}

	La riunione si è svolta presso la sede aziendale di San Marco Informatica a cui hanno partecipato i membri del gruppo e il referente aziendale Alex Beggiato. Tutto quello che è stato detto a livello tecnico è stato annotato e condiviso con i membri del gruppo tramite \glock{Google Drive} e \glock{Github}.

	\subsection*{1. Discussione sui casi d'uso individuati}

	I casi d'uso fin'ora analizzati hanno avuto un riscontro positivo, in linea generale, con quanto richiesto dal committente. Si sono modificati alcuni casi:
	\begin{itemize}
		\item Le funzionalità delle view sono disponibili anche per l'amministratore;
		\item Deve essere possibile, da parte degli utenti, poter selezionare quali alert visualizzare.
	\end{itemize}

	\subsection*{2. Discussione sui requisiti individuati}

	I requisiti individuati precedentemente alla riunione sono stati sufficienti con necessità di essere modificati leggermente. Di seguito ciò che è stato rilevato in più a ciò che era già stato trascritto:
	\begin{itemize}
		\item Gli alert devono essere inviati con una frequenza appropriata
		\item alert e device eliminazione da db fisica, eliminazione enti e utenti logica (flag disabilitato 1 o 0)
		\item Utilizzo di swagger come strumento per la doumentazione delle API fornite in formato JSON
		\item tempi di risposta pagine accettabili
		\item buon tempo di calcolo delle correlazioni
		\item gestione di n utenti contemporaneamente
	\end{itemize}
	
	\subsection*{4. Varie ed eventuali}
	
	Prefissate le ultime modifiche da fare prima di iniziare le revisioni e successivamente consegnare la documentazione.

	