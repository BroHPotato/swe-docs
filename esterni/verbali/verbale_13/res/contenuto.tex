\section*{Introduzione}

\subsection*{Luogo e data dell'incontro}
	\begin{itemize}
		\item \textbf{luogo:} Dipartimento di Matematica, Padova;
		\item \textbf{data:} 2020-02-11;
		\item \textbf{ora di inizio:} 12:30;
		\item \textbf{ora di fine:} 13:30.
	\end{itemize}

\subsection*{Ordine del giorno}
	\begin{enumerate}
			\item discussione col professor Cardin su alcune correzioni dell'analisi dei requisiti;
  			\item chieste delucidazioni sugli obiettivi della \glock{technology baseline};
  			\item fatta prima bozza degli incrementi da attuare durante il processo di sviluppo;
  			\item varie ed eventuali.
	\end{enumerate}

\subsection*{Presenze}
	\begin{itemize}
		\item \textbf{totale presenti:} 6 su 7
		\item \textbf{presenti: }
			\begin{itemize}			
				\item Lorenzo Dei Negri;
				\item Fouad Mouad (segretario);
				\item Mariano Sciacco;
				\item Alessandro Tommasin;
				\item Giuseppe Vito Bitetti;
				\item Giovanni Vidotto.
			\end{itemize}
		\item \textbf{assenti: } 
			\begin{itemize}	
				\item Nicolò Frison (giustificato).
			\end{itemize}
	\end{itemize}


\newpage
\section*{Svolgimento}

	\subsection*{Discussione col professor Cardin su alcune correzioni dell'analisi dei requisiti}
		Si è discusso in videoconferenza col professor Cardin - dalle 12.30 alle 13.00 - su alcuni casi d'uso scorretti o incompleti presenti nell'analisi dei requisiti. In particolare, si è capito come sia indispensabile non focalizzarsi sulle relazioni causa-effetto, dato che si possono esprimere solo tramite i diagrammi di attività.
		Oltre a ciò, è stato chiarito come i vincoli prestazionali vadano necessariamente testati, e che le relative precondizioni devono tenere conto del carico di sistema.

	\subsection*{Chieste delucidazioni sugli obiettivi della Technology Baseline}
		Si è capito che non è richiesta progettazione alla \glock{RP}, ma l'obiettivo è quello di realizzare un \glock{POC}, implementando un sottoinsieme dei casi d'uso individuati nell'analisi dei requisiti a scelta, e mostrare come le tecnologie implicate dal capitolato si integrano. Partendo dal suddetto \glock{POC}, si partirà poi a fare la progettazione completa del prodotto.

	\subsection*{Fatta prima bozza degli incrementi da attuare durante il processo di sviluppo}
		Sono stati fissati degli incrementi indicativi e plausibili da realizzare per poter
		avere un \glock{POC} funzionante da presentare alla \glock{RP} e riuscire a implementare il prodotto finito entro la prima revisione d'accettazione disponibile (vedi \dext{piano di progetto}).

	\subsection*{Varie ed eventuali}
		Nulla da rilevare.