\section*{Introduzione}

\subsection*{Luogo e data dell'incontro}
	\begin{itemize}
		\item \textbf{luogo:} \glock{Slack};
		\item \textbf{data:} 2020-05-14;
		\item \textbf{ora di inizio:} 17:15;
		\item \textbf{ora di fine:} 18:45.
	\end{itemize}

\subsection*{Ordine del giorno}
	\begin{enumerate}
		\item demo e riscontro generale sul prodotto;
  		\item varie ed eventuali.
	\end{enumerate}

\subsection*{Presenze}
	\begin{itemize}
		\item \textbf{totale presenti:} 7 su 7;
		\item \textbf{presenti: }
			\begin{itemize}
				\item Lorenzo Dei Negri;
				\item Fouad Mouad;
				\item Mariano Sciacco;
				\item Alessandro Tommasin;
				\item Giuseppe Vito Bitetti (segretario);
				\item Giovanni Vidotto;
				\item Nicolò Frison;
			\end{itemize}
		\item \textbf{assenti: }
			\begin{itemize}
				\item nessuno;
			\end{itemize}
		\item  \textbf{partecipanti esterni:}
			\begin{itemize}
				\item Alex Beggiato (referente, SanMarco Informatica).
			\end{itemize}
	\end{itemize}


\newpage
\section*{Svolgimento}

	\subsection*{Demo e riscontro generale sul prodotto}
		Il gruppo ha mostrato una demo del prodotto al proponente con successivo feedback sulla stessa. In particolare il nome scelto per il prodotto, la sua struttura e le funzionalità generali vanno più che bene.
		\newline
		Per quanto riguarda la visualizzazione dei grafici e delle correlazioni, il proponente ha proposto possibili alternative che sarebbero risultate più intuitive da parte dell'utente, in particolare per le correlazioni, in quanto esse vengono mostrate a schermo come un unico numero e non come grafico a dispersione XY.

	\subsection*{Varie ed eventuali}
		Si è discusso come ottimizzare i tempi di presentazione, in previsione della revisione di accettazione, assieme al referente.
