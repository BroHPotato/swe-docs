\section*{Introduzione}

\subsection*{Luogo e data dell'incontro}
	\begin{itemize}
		\item \textbf{luogo:} sede aziendale SanMarco Informatica (Via Alcide de Gasperi 132, Grisignano di Zocco(VI));
		\item \textbf{data:} 2020-01-10;
		\item \textbf{ora di inizio:} 14:30;
		\item \textbf{ora di fine:} 16:30.
	\end{itemize}

\subsection*{Ordine del giorno}
	\begin{enumerate}
		\item discussione sui casi d'uso individuati;
		\item discussione sui requisiti individuati;
		\item varie ed eventuali.
	\end{enumerate}

\subsection*{Presenze}
	\begin{itemize}
		\item \textbf{totale presenti:} 6 su 7
		\item \textbf{presenti: }
			\begin{itemize}			
				\item Giuseppe Vito Bitetti;
				\item Lorenzo Dei Negri;
				\item Nicolò Frison (segretario);
				\item Fouad Mouad;
				\item Mariano Sciacco;
				\item Alessandro Tommasin.
			\end{itemize}
		\item \textbf{assenti: } 
			\begin{itemize}	
				\item Giovanni Vidotto.
			\end{itemize}
		\item \textbf{partecipanti esterni:}
			\begin{itemize}
				\item Alex Beggiato (referente, SanMarco Informatica).
			\end{itemize}
	\end{itemize}


\newpage
\section*{Svolgimento}

	La riunione si è svolta presso la sede aziendale di SanMarco Informatica a cui hanno partecipato i membri del gruppo e il referente aziendale Alex Beggiato. Tutto quello che è stato detto a livello tecnico è stato annotato e condiviso con i membri del gruppo tramite \glock{Google Drive} e \glock{Github}.

	\subsection*{1. Discussione sui casi d'uso individuati}

	I casi d'uso fin'ora analizzati hanno avuto un riscontro positivo, in linea generale, con quanto richiesto dal committente. Si sono modificati alcuni casi:
	\begin{itemize}
		\item le funzionalità delle view sono disponibili anche per l'amministratore;
		\item deve essere possibile, da parte degli utenti, poter selezionare quali alert visualizzare.
	\end{itemize}

	\subsection*{2. Discussione sui requisiti individuati}

	I requisiti individuati precedentemente alla riunione sono stati sufficienti con necessità di essere modificati leggermente. Di seguito ciò che è stato rilevato in più a ciò che era già stato trascritto:
	\begin{itemize}
		\item gli alert devono essere inviati con una frequenza appropriata;
		\item alert e device devono essere eliminati fisicamente dal database, enti e utenti devono essere eliminati logicamente dal database (flag disabilitato 1 o 0);
		\item utilizzo di swagger come strumento per la documentazione delle API fornite in formato JSON;
		\item tempi di risposta pagine accettabili;
		\item buon tempo di calcolo delle correlazioni;
		\item gestione di n utenti contemporaneamente.
	\end{itemize}
	
	\subsection*{4. Varie ed eventuali}
	
	Prefissate le ultime modifiche da fare prima di iniziare le revisioni e successivamente consegnare la documentazione.

	