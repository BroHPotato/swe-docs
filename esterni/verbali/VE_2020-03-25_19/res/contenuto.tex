\section*{Introduzione}

\subsection*{Luogo e data dell'incontro}
	\begin{itemize}
		\item \textbf{luogo:} videoconferenza sulla piattaforma \glock{zoom};
		\item \textbf{data:} 2020-03-25;
		\item \textbf{ora di inizio:} 17:00;
		\item \textbf{ora di fine:} 17:45.
	\end{itemize}

\subsection*{Ordine del giorno}
	\begin{enumerate}
			\item chiarimenti sulla correzione dell'RP;
  		\item varie ed eventuali.
	\end{enumerate}

\subsection*{Presenze}
	\begin{itemize}
		\item \textbf{totale presenti:} 7 su 7
		\item \textbf{presenti: }
			\begin{itemize}
				\item Lorenzo Dei Negri;
				\item Fouad Mouad;
				\item Mariano Sciacco;
				\item Alessandro Tommasin;
				\item Giuseppe Vito Bitetti (segretario);
				\item Giovanni Vidotto;
				\item Nicolò Frison.
			\end{itemize}
		\item \textbf{assenti: }
			\begin{itemize}
				\item nessuno.
			\end{itemize}
		\item  \textbf{partecipanti esterni:}
			\begin{itemize}
				\item Professor Tullio Vardanega.
			\end{itemize}
	\end{itemize}


\newpage
\section*{Svolgimento}

	\subsection*{Chiarimenti sulla correzione dell'RP}
		Si è discusso in videoconferenza col professor Vardanega sull'esito della consegna dell'RP. In particolare, si è capito come ristrutturare il codice di versionamento dei documenti per  renderlo piú significativo.
		Inoltre, é stato fatto notare come gli incrementi non facciano trasparire una visione di insieme.

	\subsection*{Varie ed eventuali}
		Delucidazioni sulla consegna per la \glock{product baseline}.
