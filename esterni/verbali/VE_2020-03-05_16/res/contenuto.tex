\section*{Introduzione}

\subsection*{Luogo e data dell'incontro}
	\begin{itemize}
		\item \textbf{luogo:} SanMarco Informatica, Grisignano di Zocco, Vicenza;
		\item \textbf{data:} 2020-03-05;
		\item \textbf{ora di inizio:} 15:15;
		\item \textbf{ora di fine:} 16:30.
	\end{itemize}

\subsection*{Ordine del giorno}
	\begin{enumerate}
			\item feedback con il proponente del video realizzato alla fine del IV incremento;
  			\item confronto con il proponente degli incrementi che si vogliono portare;
  			\item confronto sulle tecnologie che si vogliono utilizzare e sui linguaggi di programmazione;
  			\item visualizzazione dei diagrammi ER dei database con il proponente e della lista delle API;
  			\item varie ed eventuali.
	\end{enumerate}

\subsection*{Presenze}
	\begin{itemize}
		\item \textbf{totale presenti:} 7 su 7
		\item \textbf{presenti: }
			\begin{itemize}			
				\item Lorenzo Dei Negri;
				\item Fouad Mouad;
				\item Mariano Sciacco (segretario);
				\item Alessandro Tommasin;
				\item Giuseppe Vito Bitetti;
				\item Giovanni Vidotto;
				\item Nicolò Frison.
			\end{itemize}
		\item \textbf{assenti: } 
			\begin{itemize}	
				\item Nessuno.
			\end{itemize}
		\item  \textbf{partecipanti esterni:}
			\begin{itemize}
				\item Alex Beggiato (referente, SanMarco Informatica).
			\end{itemize}
	\end{itemize}


\newpage
\section*{Svolgimento}

	\subsection*{Feedback con il proponente del video realizzato alla fine del IV incremento}
		Il gruppo ha richiesto un feedback da parte del proponente, Alex Beggiato, sulla qualità e sulla bontà di quanto dimostrato come \textit{proof of concept} nel video consegnato la settimana precedente. Il proponente si è dichiarato molto contento del lavoro che il gruppo ha ottenuto fino a ora e ha fatto notare che quanto realizzato è in linea con quanto richiesto dal capitolato.

	\subsection*{Confronto con il proponente degli incrementi che si vogliono portare}
		Il proponente è stato messo a corrente con gli incrementi che intendiamo portare nel corso delle prossime settimane e con gli incrementi che sono stati svolti a oggi. E' stato dato un feedback per ogni incremento che riguarda la complessità dello sviluppo del componente software, ma in linea generale non sono stati rilevati particolari cambiamenti da effettuare.

	\subsection*{Confronto sulle tecnologie che si vogliono utilizzare e sui linguaggi di programmazione}
		Il gruppo ha messo al corrente il proponente sulle tecnologie che si è deciso di utilizzare per sviluppare tutte le componenti. Successivamente, sono state fatte domande tecniche sulle tecnologie così da poter appuntarsi eventuali idee sulle scelte progettuali che potranno tornare utili durante per lo sviluppo della componente software.

	\subsection*{Visualizzazione dei diagrammi ER dei database con il proponente e della lista delle API}
		Il gruppo ha mostrato e spiegato una bozza del diagramma ER del database relazionale che si intende implementare per i prossimi incrementi. In aggiunta, sono state mostrate anche le API con le relative funzionalità che verranno implementate.

	\subsection*{Varie ed eventuali}
		Nulla da rilevare.