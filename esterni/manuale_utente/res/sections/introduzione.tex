\section{Introduzione}
	\subsection{Scopo del documento}
		Lo scopo del documento è raggruppare i video informativi creati dal gruppo Red Round Robin. Questi video verranno utilizzati per integrare ed in alcuni casi sostituire parti scritte del \dext{manuale utente}. Il documento è suddiviso, oltre all'introduzione, in tre sezioni: una per l'utente, una per il moderatore ente ed infine una per l'amministratore.	
	\subsection{Glossario e documenti esterni}
		Per evitare possibili ambiguità relative alle terminologie (che andranno indicate in \textsc{maiuscoletto}) utilizzate nei vari documenti, verranno utilizzate due simboli:
		\begin{itemize}
			\item una \textit{D} al pedice per indicare il nome di un particolare documento;
			\item una \textit{G} al pedice per indicare un termine che sarà presente nel \dext{Glossario v2.0.0}.
		\end{itemize}
	\subsection{Requisiti di sistema}
		L'interfaccia utente del progetto ThireMa è accessibile tramite i seguenti \glock{browser}:
		\begin{itemize}
		 	\item \glock{Firefox} a partire dalla versione 69.0;
		 	\item \glock{Chrome} a partire dalla versione 75.0;
		 	\item \glock{Safari} a partire dalla versione 13.0;
		 	\item \glock{Edge} a partire dalla versione 42.0.
		 	%Controllare se ci sono altri requisiti
		\end{itemize} 
		Per inviare comandi e ricevere degli alert tramite un bot Telegram, è necessario inoltre disporre di un \textit{Username}, configurabile all'interno della sezione impostazioni dell'applicazione Telegram.
	\subsection{Riferimenti}
		\subsubsection{Normativi}
			\begin{itemize}
				\item \textbf{norme di progetto: }\dext{Norme di Progetto v2.0.0} 
				\item \textbf{capitolato C6 - ThiReMa: }\url{https://www.math.unipd.it/~tullio/IS-1/2019/Progetto/C6.pdf}
			\end{itemize}
		\subsubsection{Informativi}
			\begin{itemize}
				\item \textbf{presentazione seminario capitolato C6 - ThiReMa: }\url{https://www.math.unipd.it/~tullio/IS-1/2019/Progetto/C6a.pdf};
			\end{itemize}