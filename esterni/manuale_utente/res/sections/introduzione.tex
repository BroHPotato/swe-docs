\section{Introduzione}
	\subsection{Premessa}
		\textbf{Il documento non è da intendersi come concluso e definitivo.} Alcune funzionalità devono essere ancora implementate e pertanto la struttura e i contenuti del documento potrebbe cambiare.
	\subsection{Scopo del documento}
		Lo scopo del documento è raggruppare i video informativi creati dal gruppo Red Round Robin. Il documento è suddiviso, oltre all'introduzione, in tre sezioni: una con le informazioni di carattere generale, una per il moderatore ente ed infine una per l'amministratore.	
	\subsection{Scopo del prodotto}
		Per grandi e medie aziende, ma non solo, la gestione e l'analisi di grosse moli di dati sta diventando sempre di più una realtà concreta.
	 	\newline
		Il progetto ThiReMa si prefigge come obiettivo la creazione di una web application, la quale permetta di analizzare ingenti moli di dati, ricevuti da più sensori eterogenei tra loro. Tale applicazione metterà a disposizione un'interfaccia intuitiva che permetterà di visualizzare più dati di interesse od eventuali correlazioni tra gli stessi. Infine, per ogni tipologia di dato sarà possibile assegnarne il monitoraggio ad un particolare ente.	
	\subsection{Glossario e documenti esterni}
		Per evitare possibili ambiguità relative alle terminologie (che andranno indicate in \textsc{maiuscoletto}) utilizzate nei vari documenti, verranno utilizzate due simboli:
		\begin{itemize}
			\item una \textit{D} al pedice per indicare il nome di un particolare documento;
			\item una \textit{G} al pedice per indicare un termine che sarà presente nel \textit{Glossario} all'interno dell'appendice del documento.
		\end{itemize}
	