\section{Introduzione}
	\subsection{Premessa}
		Il nome \textit{ThiReMa project} è il nome in codice interno usato dai fornitori per identificare il prodotto \textit{RIoT}. Qualunque riferimento a \textit{ThiReMa project} è un riferimento diretto a \textit{RIoT}. 
	\subsection{Scopo del documento}
		Lo scopo del documento è illustrare le modalità di utilizzo e le funzionalità del prodotto \textit{RIoT}. Il documento è suddiviso, oltre all'introduzione, in tre sezioni: una con le informazioni per l'utente base, una per il moderatore ente ed infine una per l'amministratore.	
	\subsection{Scopo del prodotto}
		Per grandi e medie aziende, ma non solo, la gestione e l'analisi di grosse moli di dati sta diventando sempre di più una realtà concreta.
	 	\newline
		Il progetto RIoT si prefigge come obiettivo la creazione di una web application, che permetta di analizzare ingenti moli di dati, ricevuti da più sensori eterogenei tra loro. Tale applicazione metterà a disposizione un'interfaccia intuitiva che permetterà di visualizzare più dati di interesse o eventuali correlazioni tra gli stessi. Infine, per ogni tipologia di dato sarà possibile assegnarne il monitoraggio ad un particolare ente.	
	\subsection{Glossario}
		Per evitare possibili ambiguità relative ad alcuni termini usati nel documento, questi verranno indicati in \textsc{maiuscoletto} con una G al pedice e saranno riportati nel glossario presente nell'appendice \S A.
