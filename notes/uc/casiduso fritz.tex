\section{Casi d'uso}
	Per i casi d'uso verranno utilizzate le seguenti sigle:
	\begin{itemize}
		\item \textbf{[UNA]}: Utente non autenticato;
		\item \textbf{[UA]}: Utente autenticato;
		\item \textbf{[ME]}: Moderatore ente;
		\item \textbf{[AM]}: Amministratore.
	\end{itemize}
	\subsection{Elenco dei casi d'uso}

		\subsubsection{UC1 - Login}
		
		\begin{figure}[t!]
			\centering
			\includegraphics[height=10em]{res/images/UC1 - Login.jpg}
		\end{figure}
		
		\begin{itemize}
			\item \textbf{Attori Primari}: [UNA]
			\item \textbf{Descrizione}: L'utente tenta di autenticarsi nella web-application mediante un form.
			\item \textbf{Precondizione}: L'utente visualizza la schermata di login con il relativo form.
			\item \textbf{Postcondizione}: L'utente effettua la login e in base alle credenziali inserite viene identificato come un altro attore(Utente autenticato, Moderatore ente o Amministratore) con i relativi privilegi.
			\item \textbf{Scenario Principale}:
			\begin{enumerate}
				\item{L'utente compila i campi del form}
				\item{L'utente preme il pulsante di login}
			\end{enumerate}
			\item \textbf{Inclusioni}:
				\begin{itemize}
					\item Compilazione email e password (UC 1.1)
				\end{itemize}
			\item \textbf{Estensioni}:
				\begin{itemize}
					\item Viene inserita una mail o una password sbagliata (UC 1.2)
					\item L'utente non ha i permessi necessari  (UC 1.3)
				\end{itemize}

		\end{itemize}

			\paragraph{UC 1.1 - Compilazione email e password}
			\begin{itemize}
				\item \textbf{Attori Primari}: [UNA]
				\item \textbf{Descrizione}: L'utente compila i campi del form: email e password.
				\item \textbf{Precondizione}: L'utente visualizza la schermata di login con il relativo form
				\item \textbf{Postcondizione}: L'utente ha compilato i campi email e password.
				\item \textbf{Scenario Principale}:
				\begin{enumerate}
					\item{L'utente compila il campo con la mail}
					\item{L'utente compila il campo con la password}
				\end{enumerate}	
			\end{itemize}

			\paragraph{UC 1.2 - Errore mail o password}
			\begin{itemize}
				\item \textbf{Attori Primari}: [UNA]
				\item \textbf{Descrizione}: L'utente prova ad autenticarsi ma inserisce una mail o una password che non coincidono con quelle salvate nel database, quindi visualizza un errore.
				\item \textbf{Precondizione}: L'utente ha sbagliato ad inserire la mail o la password
				\item \textbf{Postcondizione}: Viene visualizzato un errore relativo alla mail o la password
				\item \textbf{Scenario Principale}:
				\begin{enumerate}
					\item{L'utente prova ad accedere usando una mail o una password errati}
				\end{enumerate}	
			\end{itemize}

			\paragraph{UC 1.3 - Errore utente non autorizzato}
			\begin{itemize}
				\item \textbf{Attori Primari}: [UNA]
				\item \textbf{Descrizione}: L'utente prova ad autenticarsi ma non ha i privilegi per visualizzare nulla, quindi visualizza un errore.
				\item \textbf{Precondizione}: L'utente prova ad accedere non avendo i permessi necessari
				\item \textbf{Postcondizione}: L'utente visualizza un errore riguardo la mancanza di permessi
				\item \textbf{Scenario Principale}:
				\begin{enumerate}
					\item{L'utente riceve un messaggio di errore indicante la mancanza di permessi}
				\end{enumerate}	
			\end{itemize}
		
		\subsubsection{UC 2 - Visualizzazione dashboard e menù}
		
		\begin{figure}[t!]
			\centering
			\includegraphics[height=10em]{res/images/Dashboard e Menù.jpg}
		\end{figure}
		
		\begin{itemize}
			\item \textbf{Attori Primari}: [UA] [ME] [AM]
			\item \textbf{Descrizione}: L'utente ha eseguito il login e visualizza la dashboard con le funzionalità di cui dispone i permessi.
			\item \textbf{Precondizione}: L'utente ha eseguito il login.
			\item \textbf{Postcondizione}: L'utente visualizza la dashboard con il contenuto basato sull'attore autenticato.
			\item \textbf{Scenario Principale}:
			\begin{enumerate}
				\item{L'utente ha eseguito il login}
				\item{L'utente visualizza la dashboard e il menù}
			\end{enumerate}	
		\end{itemize}

		\subsubsection{UC 3 - Logout}
		\begin{itemize}
			\item \textbf{Attori Primari}: [UA] [ME] [AM]
			\item \textbf{Descrizione}: L'utente effettua l'uscita dalla web-application diventando l'attore Utente non autenticato.
			\item \textbf{Precondizione}: L'utente seleziona la voce "logout".
			\item \textbf{Postcondizione}: L'utente visualizza la schermata di login.
			\item \textbf{Scenario Principale}:
			\begin{enumerate}
				\item{l'utente seleziona la voce "logout"}
				\item{L'utente effettua il logout}
				\item{L'utente visualizza la schermata di login}
			\end{enumerate}	
		\end{itemize}

		\subsubsection{UC 4 - Gestione Impostazioni}
		
		\begin{figure}[H]
			\centering
			\includegraphics[height=30em]{res/images/UC4 - Impostazioni.jpg}
		\end{figure}
		
		\begin{itemize}
			\item \textbf{Attori Primari}: [UA] [ME] [AM]
			\item \textbf{Descrizione}: L'utente gestisce le proprie impostazioni utente.
			\item \textbf{Precondizione}: L'utente seleziona la voce "impostazioni".
			\item \textbf{Postcondizione}: L'utente ha cambiato le proprie impostazioni.
			\item \textbf{Scenario Principale}:
			\begin{enumerate}
				\item{L'utente seleziona la voce "impostazioni"}
				\item{L'utente visualizza la schermata per la modifica delle proprie impostazioni}
				\item{L'utente modifica le proprie impostazioni}
				\item{L'utente ha cambiato le proprie impostazioni}
			\end{enumerate}	
		\end{itemize}
			

			\paragraph{UC 4.1 - Salva dati form password}
			\begin{itemize}
				\item \textbf{Attori Primari}: [UA] [ME] [AM]
				\item \textbf{Descrizione}: L'utente cambia la propria password, e per farlo compila il form password e salva
				\item \textbf{Precondizione}: L'utente visualizza la schermata per la modifica delle proprie impostazioni
				\item \textbf{Postcondizione}: L'utente ha cambiato la propria password
				\item \textbf{Scenario Principale}:
				\begin{enumerate}
					\item{L'utente visualizza la form della password nelle impostazioni}
					\item{L'utente compila i campi presenti nel form password}
					\item{L'utente preme sul bottone di salvataggio del form password}
					\item{L'utente ha cambiato la propria password}
				\end{enumerate}	
				\item \textbf{Inclusioni}:
					\begin{itemize}
						\item L'utente compila i dati del form password (UC 4.1.1)
					\end{itemize}
				\item \textbf{Estensioni}:
					\begin{itemize}
						\item L'utente inserisce la vecchia password errata (UC 4.1.2)
						\item L'utente inserisce una nuova password identica a quella precendentemente salvata (UC 4.1.3)
						\item L'utente inserisce la conferma password che non coincide con la prima password immessa (UC 4.1.4)
					\end{itemize}
			\end{itemize}

			\paragraph{UC 4.1.1 - Compila form password}
			\begin{itemize}
				\item \textbf{Attori Primari}: [UA] [ME] [AM]
				\item \textbf{Descrizione}: L'utente compila il form per la modifica della propria password, compila i campi password, che dovrà contenere la password dell'utente attuale, il campo nuova password, ovvero la nuova password che si vuole utilizzare, e la conferma password che dovrà essere uguale alla nuova password.
				\item \textbf{Precondizione}: L'utente visualizza il form della password.
				\item \textbf{Postcondizione}: L'utente ha compilato il form password.
				\item \textbf{Scenario Principale}:
				\begin{enumerate}
					\item{L'utente visualizza il form della password}
					\item{L'utente compila il campo vecchia password}
					\item{L'utente compila il campo nuova password}
					\item{L'utente compila il campo conferma password}
					\item{L'utente ha compilato il form password}
				\end{enumerate}
			\end{itemize}

			\paragraph{UC 4.1.2 - Errore modifica password}
			\begin{itemize}
				\item \textbf{Attori Primari}: [UA] [ME] [AM]
				\item \textbf{Descrizione}: Dopo aver premuto il bottone per salvare la nuova password viene visualizzato il messaggio di errore "vecchia password non corretta" perchè password inserita nel campo "vecchia password" dall'utente non coincide con la password attuale.
				\item \textbf{Precondizione}: L'utente ha premuto sul bottone di salvataggio del form password.
				\item \textbf{Postcondizione}: Visualizzazione messaggio errore "vecchia password non corretta".
				\item \textbf{Scenario Principale}:
				\begin{enumerate}
					\item{L'utente ha premuto sul bottone di salvataggio del form password}
					\item{La password inserita nel campo "vecchia password" dall'utente non coincide con la password attuale}
					\item{Viene visualizzato il messaggio di errore "vecchia password non corretta"}
				\end{enumerate}
			\end{itemize}

			\paragraph{UC 4.1.3 - Errore password inserita non valida}
			\begin{itemize}
				\item \textbf{Attori Primari}: [UA] [ME] [AM]
				\item \textbf{Descrizione}: Dopo aver premuto il bottone per salvare la nuova password viene visualizzato il messaggio di errore "nuova password non valida" perchè la password inserita nel campo "nuova password" dall'utente coincide con la password attuale.
				\item \textbf{Precondizione}: L'utente ha premuto sul bottone di salvataggio del form password.
				\item \textbf{Postcondizione}: Visualizzazione messaggio errore "nuova password non valida".
				\item \textbf{Scenario Principale}:
				\begin{enumerate}
					\item{L'utente ha premuto sul bottone di salvataggio del form password}
					\item{La password inserita nel campo "nuova password" dall'utente coincide con la password attuale}
					\item{Viene visualizzato il messaggio di errore "nuova password non valida"}
				\end{enumerate}
			\end{itemize}

			\paragraph{UC 4.1.4 - Errore conferma password}
			\begin{itemize}
				\item \textbf{Attori Primari}: [UA] [ME] [AM]
				\item \textbf{Descrizione}: Dopo aver premuto il bottone per salvare la nuova password viene visualizzato il messaggio di errore "conferma password errata" perchè la password inserita nel campo "conferma password" dall'utente non coincide con la password inserita nel campo "nuova password".
				\item \textbf{Precondizione}: L'utente ha premuto sul bottone di salvataggio del form password.
				\item \textbf{Postcondizione}: Visualizzazione messaggio errore "conferma password errata".
				\item \textbf{Scenario Principale}:
				\begin{enumerate}
					\item{L'utente ha premuto sul bottone di salvataggio del form password}
					\item{La password inserita nel campo "nuova password" dall'utente non coincide con la password inserita nel campo "conferma password"}
					\item{Viene visualizzato il messaggio di errore "conferma password errata"}
				\end{enumerate}
			\end{itemize}
			
			\paragraph{UC 4.2 - Salva dati form informazioni}
			\begin{itemize}
				\item \textbf{Attori Primari}: [UA] [ME] [AM]
				\item \textbf{Descrizione}: L'utente cambia le proprie informazioni, e per farlo compila il form informazioni e salva.
				\item \textbf{Precondizione}: L'utente visualizza la schermata per la modifica delle proprie impostazioni
				\item \textbf{Postcondizione}: L'utente ha cambiato la propria email e/o il proprio username telegram.
				\item \textbf{Scenario Principale}:
				\begin{enumerate}
					\item{L'utente visualizza la form delle informazioni nelle impostazioni}
					\item{L'utente compila i campi presenti nel form informazioni}
					\item{L'utente preme sul bottone di salvataggio del form informazioni}
					\item{L'utente ha cambiato la propria email e/o il proprio username telegram}
				\end{enumerate}	
				\item \textbf{Inclusioni}:
					\begin{itemize}
						\item L'utente compila i dati del form delle informazioni (UC 4.2.1)
					\end{itemize}
				\item \textbf{Estensioni}:
					\begin{itemize}
						\item L'utente inserisce un email non valida (UC 4.2.2)
						\item L'utente inserisce uno username telegram non valido (UC 4.2.3)
					\end{itemize}
			\end{itemize}

			\paragraph{UC 4.2.1 - Compila form delle informazioni}
			\begin{itemize}
				\item \textbf{Attori Primari}: [UA] [ME] [AM]
				\item \textbf{Descrizione}: L'utente compila il form per la modifica delle proprie informazioni, compila i campi email, che dovrà contenere la nuova email che l'utente intende salvare, e il campo username Telegram, ovvero lo username del proprio account Telegram per accedere alle funzionalità offerte dal bot.
				\item \textbf{Precondizione}: L'utente visualizza il form delle informazioni.
				\item \textbf{Postcondizione}: L'utente ha compilato il form delle informazioni.
				\item \textbf{Scenario Principale}:
				\begin{enumerate}
					\item{L'utente visualizza il form delle informazioni}
					\item{L'utente compila il campo email e/o il campo username Telegram}
					\item{L'utente ha compilato il form delle informazioni}
				\end{enumerate}
			\end{itemize}

			\paragraph{UC 4.2.2 - Errore email non valida}
			\begin{itemize}
				\item \textbf{Attori Primari}: [UA] [ME] [AM]
				\item \textbf{Descrizione}: Dopo aver premuto il bottone per salvare le nuove informazioni del proprio utente viene visualizzato il messaggio di errore "Email non valida" perchè l'email inserita non è valida. 
				\item \textbf{Precondizione}: L'utente ha premuto sul bottone di salvataggio del form delle informazioni.
				\item \textbf{Postcondizione}: Visualizzazione messaggio di errore "Email non valida"
				\item \textbf{Scenario Principale}:
				\begin{enumerate}
					\item{L'utente ha premuto sul bottone di salvataggio del form delle informazioni}
					\item{La email inserita nel campo "email" non è valida}
					\item{Viene visualizzato il messaggio di errore "Email non valida"}
				\end{enumerate}	
			\end{itemize}

			\paragraph{UC 4.2.3 - Errore username Telegram non valido}
			\begin{itemize}
				\item \textbf{Attori Primari}: [UA] [ME] [AM]
				\item \textbf{Descrizione}: Dopo aver premuto il bottone per salvare le nuove informazioni del proprio utente viene visualizzato il messaggio di errore "Username Telegram non valido" perchè l'username Telegram inserito non è valido. 
				\item \textbf{Precondizione}: L'utente ha premuto sul botton di salvataggio del form delle informazioni.
				\item \textbf{Postcondizione}: Visualizzazione messaggio di errore "Username Telegram non valido".
				\item \textbf{Scenario Principale}:
				\begin{enumerate}
					\item{L'utente ha premuto sul bottone di salvataggio del form delle informazioni}
					\item{L'username Telegram inserito nel campo "username Telegram" non è valido}
					\item{Viene visualizzato il messaggio di errore "Username Telegram non valido"}
				\end{enumerate}	
			\end{itemize}

		\subsubsection{UC 5 - Gestione dispositivi}
		
		\begin{figure}[t!]
			\centering
			\includegraphics[height=10em]{res/images/UC5 - Gestione dispositivi.jpg}
		\end{figure}
		
		\begin{itemize}
			\item \textbf{Attori Primari}: [UA] [ME] [AM]
			\item \textbf{Descrizione}: L'utente gestisce i dispositivi a cui ha accesso.
			\item \textbf{Precondizione}: L'utente seleziona la voce "Gestione dispositivi"
			\item \textbf{Postcondizione}: L'utente ha visualizzato/gestito i dispositivi.
			\item \textbf{Scenario Principale}:
			\begin{enumerate}
				\item{L'utente seleziona la voce "Gestione dispositivi"}
				\item{L'utente visualizza la schermata per la gestione dei dispositivi}
				\item{L'utente gestisce i dispositivi a cui ha accesso}
				\item{L'utente ha visualizzato/gestito i dispositivi}
			\end{enumerate}
		\end{itemize}
			
			\subsubsection{UC 5.1 - Visualizzazione lista dispositivi}
			\begin{itemize}
				\item \textbf{Attori Primari}: [UA] [ME] [AM]
				\item \textbf{Descrizione}: L'utente visualizza i dispositivi.
				\item \textbf{Precondizione}: L'utente seleziona la voce "Visualizza dispositivi"
				\item \textbf{Postcondizione}: L'utente ha visualizzato i dispositivi.
				\item \textbf{Specializzazioni}:
				\begin{itemize}
					\item Visualizzazione lista dispositivi ente (UC 5.1.1)
					\item Visualizzazione lista dispositivi completa (UC 5.1.2)
				\end{itemize}
				\item \textbf{Scenario Principale}:
				\begin{enumerate}
					\item{L'utente seleziona la voce "Visualizza dispositivi"}
					\item{L'utente visualizza i dispositivi in base a che tipo di utente è}
					\item{L'utente ha visualizzato i dispositivi}
				\end{enumerate}
			\end{itemize}
			
			\subsubsection{UC 5.1.1 - Visualizzazione lista dispositivi ente}
			\begin{itemize}
				\item \textbf{Attori Primari}: [UA] [ME]
				\item \textbf{Descrizione}: L'utente visualizza i dispositivi visualizzabili dal proprio ente.
				\item \textbf{Precondizione}: L'utente seleziona la voce "Visualizza dispositivi ente"
				\item \textbf{Postcondizione}: L'utente ha visualizzato i dispositivi.
				\item \textbf{Scenario Principale}:
				\begin{enumerate}
					\item{L'utente seleziona la voce "Visualizza dispositivi ente"}
					\item{L'utente visualizza i dispositivi visualizzabili dal proprio ente}
					\item{L'utente ha visualizzato i dispositivi}
				\end{enumerate}
			\end{itemize}
			
			\subsubsection{UC 5.1.2 - Visualizzazione lista dispositivi completa}
			\begin{itemize}
				\item \textbf{Attori Primari}: [AM]
				\item \textbf{Descrizione}: L'amministratore visualizza la lista dispositivi completa.
				\item \textbf{Precondizione}: L'amministratore seleziona la voce "Visualizza tutti i dispositivi"
				\item \textbf{Postcondizione}: L'amministratore ha visualizzato la lista di tutti i dispositivi.
				\item \textbf{Scenario Principale}:
				\begin{enumerate}
					\item{L'amministratore seleziona la voce "Visualizza tutti i dispositivi"}
					\item{L'amministratore visualizza tutti i dispositivi presenti nel sistema}
					\item{L'amministratore ha visualizzato la lista di tutti i dispositivi}
				\end{enumerate}
			\end{itemize}
			
			\subsubsection{UC 5.2 - Visualizza info dispositivo}
			\begin{itemize}
				\item \textbf{Attori Primari}: [UA] [ME] [AM]
				\item \textbf{Descrizione}: L'utente visualizza le informazioni riguardanti il dispositivo selezionato.
				\item \textbf{Precondizione}: L'utente seleziona il dispositivo dalla lista dispositivi
				\item \textbf{Postcondizione}: L'utente ha visualizzato le informazioni del dispositivo selezionato
				\item \textbf{Scenario Principale}:
				\begin{enumerate}
					\item{L'utente seleziona il dispositivo dalla lista dispositivi}
					\item{L'utente seleziona la voce "visualizza informazioni dispositivo"}
					\item{L'utente visualizza le informazioni del dispositivo}
					\item{L'utente ha visualizzato le informazioni del dispositivo}
				\end{enumerate}
			\end{itemize}
			
			\subsubsection{UC 5.3 - Aggiunta dispositivo}
			\begin{itemize}
				\item \textbf{Attori Primari}: [AM]
				\item \textbf{Descrizione}: L'amministratore aggiunge un nuovo dispositivo al sistema.
				\item \textbf{Precondizione}: L'amministratore seleziona la voce "aggiungi dispositivo"
				\item \textbf{Postcondizione}: L'amministratore ha aggiunto un nuovo dispositivo al sistema
				\item \textbf{Scenario Principale}:
				\begin{enumerate}
					\item{L'amministratore seleziona la voce "aggiungi dispositivo"}
					\item{L'amministratore compila il form dispositivo con i dati relativa la dispositivo da aggiungere}
					\item{L'amministratore preme il bottone di salvataggio del form dispositivo}
					\item{L'amministratore ha aggiunto un nuovo dispositivo al sistema}
				\end{enumerate}
				\item \textbf{Inclusioni}:
				\begin{itemize}
					\item Compilazione form dispositivo (UC 5.3.1)
				\end{itemize}
				\item \textbf{Estensioni}:
				\begin{itemize}
					\item Dispositivo non valido (UC 5.3.2)
				\end{itemize}
			\end{itemize}
			
			\subsubsection{UC 5.3.1 - Compilazione form dispositivo}
			\begin{itemize}
				\item \textbf{Attori Primari}: [AM]
				\item \textbf{Descrizione}: L'amministratore compila il form dispositivo contenente i seguenti campi:
				\begin{itemize}
					\item identificativo dispositivo
					\item nome dispositivo
					\item tempo di frequenza ricezione dati, ovvero ogni quanto tempo devono essere letti i valori dal dispositivo
					\item dati sensori, ovvero quali dati si desiderano ricevere dai sensori del dispositivo
				\end{itemize}.
				\item \textbf{Precondizione}: L'amministratore visualizza il form dispositivi
				\item \textbf{Postcondizione}: L'amministratore preme sul buttone di salvataggio del form dispositivi
				\item \textbf{Scenario Principale}:
				\begin{enumerate}
					\item{L'amministratore visualizza il form dei dispositivi}
					\item{L'amministratore compila il campo \textit{identificativo dispositivo}}
					\item{L'amministratore compila il campo \textit{nome dispositivo}}
					\item{L'amministratore compila il campo \textit{tempo di frequenza ricezione dati}}
					\item{L'amministratore compila il campo \textit{dati sensori}}
					\item{L'amministratore preme sul buttone di salvataggio del form dispositivi}
				\end{enumerate}
			\end{itemize}
			
			\subsubsection{UC 5.3.2 - Dispositivo non valido}
			\begin{itemize}
				\item \textbf{Attori Primari}: [AM]
				\item \textbf{Descrizione}: L'amministratore visualizza l'errore "Il dispositivo richiesto non è valido" perchè il campo identificativo dispositivo non corrisponde ad alcun dispositivo
				\item \textbf{Precondizione}: L'amministratore ha premuto il bottone di salvataggio del form dispositivo
				\item \textbf{Postcondizione}: Visualizzazione dell'errore "Il dispositivo richiesto non è valido"
				\item \textbf{Scenario Principale}:
				\begin{enumerate}
					\item{L'amministratore ha premuto il bottone di salvataggio del form dispositivo}
					\item{Il campo identificativo dispositivo non è valido}
					\item{L'amministratore compila il campo \textit{nome dispositivo}}
					\item{L'amministratore compila il campo \textit{tempo di frequenza ricezione dati}}
					\item{L'amministratore compila il campo \textit{dati sensori}}
					\item{L'amministratore preme sul buttone di salvataggio del form dispositivi}
				\end{enumerate}
			\end{itemize}
			
			\subsubsection{UC 5.4 - Modifica dispositivo}
			\begin{itemize}
				\item \textbf{Attori Primari}: [AM]
				\item \textbf{Descrizione}: L'amministratore modifica il dispositivo selezionato.
				\item \textbf{Precondizione}: L'amministratore seleziona un dispositivo dalla lista dispositivi
				\item \textbf{Postcondizione}: L'amministratore ha modificato il dispositivo selezionato
				\item \textbf{Scenario Principale}:
				\begin{enumerate}
					\item{L'amministratore seleziona il dispositivo dalla lista dei dispositivi}
					\item{L'amministratore compila il form modifica dispositivo con i dati del dispositivo che intende modificare}
					\item{L'amministratore preme il bottone di salvataggio del form modifica dispositivo}
					\item{L'amministratore ha modificato il dispositivo selezionato}
				\end{enumerate}
				\item \textbf{Inclusioni}:
				\begin{itemize}
					\item Compilazione form modifica dispositivo (UC 5.4.1)
				\end{itemize}
			\end{itemize}
			
			\subsubsection{UC 5.4.1 - Compilazione form modifica dispositivo}
			\begin{itemize}
				\item \textbf{Attori Primari}: [AM]
				\item \textbf{Descrizione}: L'amministratore cambia i valori dei campi che intende modificare del form modifica dispositivo:
				\begin{itemize}
					\item nome dispositivo
					\item tempo di frequenza ricezione dati, ovvero ogni quanto tempo devono essere letti i valori dal dispositivo
					\item dati sensori, ovvero quali dati si desiderano ricevere dai sensori del dispositivo
				\end{itemize}.
				\item \textbf{Precondizione}: L'amministratore visualizza il form modifica dispositivo
				\item \textbf{Postcondizione}: L'amministratore preme sul buttone di salvataggio del form modifica dispositivo
				\item \textbf{Scenario Principale}:
				\begin{enumerate}
					\item{L'amministratore visualizza il form modifica dispositivo}
					\item{L'amministratore modifica il campo \textit{nome dispositivo} se intende cambiarne il valore}
					\item{L'amministratore modifica il campo \textit{tempo di frequenza ricezione dati} se intende cambiarne il valore}
					\item{L'amministratore modifica i \textit{dati sensori} se intende cambiare i valori}
					\item{L'amministratore preme sul buttone di salvataggio del form modifica dispositivi}
				\end{enumerate}
			\end{itemize}
			
			\subsubsection{UC 5.5 - Rimozione dispositivo}
			\begin{itemize}
				\item \textbf{Attori Primari}: [AM]
				\item \textbf{Descrizione}: L'amministratore rimuove dal sistema il dispositivo selezionato dalla lista dispositivi
				\item \textbf{Precondizione}: L'amministratore seleziona il dispositivo dalla lista dispositivi
				\item \textbf{Postcondizione}: L'amministratore ha rimosso il dispositivo selezionato dal sistema
				\item \textbf{Scenario Principale}:
				\begin{enumerate}
					\item{L'amministratore seleziona il dispositivo dalla lista dispositivi}
					\item{L'amministratore seleziona la voce \textit{rimuovi dispositivo}}
					\item{L'amministratore ha rimosso il dispositivo selezionato dal sistema}
				\end{enumerate}
			\end{itemize}
			
			\subsubsection{UC 5.6 - Aggiunta sensori ad ente}
			\begin{itemize}
				\item \textbf{Attori Primari}: [AM]
				\item \textbf{Descrizione}: L'amministratore aggiunge i permessi di monitoraggio dei dati del dispositivo selezionato all'ente selezionato
				\item \textbf{Precondizione}: L'amministratore visualizza le informazioni del dispositivo
				\item \textbf{Postcondizione}: L'amministratore ha permesso il monitoraggio del dato del dispositivo all'ente
				\item \textbf{Scenario Principale}:
				\begin{enumerate}
					\item{L'amministratore visualizza le informazioni del dispositivo}
					\item{L'amministratore seleziona il dato del dispositivo}
					\item{L'amministratore seleziona l'ente a cui permettere il monitoraggio del dato}
					\item{L'amministratore ha permesso il monitoraggio del dato del dispositivo all'ente selezionato}
				\end{enumerate}
			\end{itemize}
			
			\subsubsection{UC 5.7 - Rimozione sensori ad ente}
			\begin{itemize}
				\item \textbf{Attori Primari}: [AM]
				\item \textbf{Descrizione}: L'amministratore rimuove i permessi di monitoraggio dei dati del dispositivo selezionato all'ente selezionato
				\item \textbf{Precondizione}: L'amministratore visualizza le informazioni del dispositivo
				\item \textbf{Postcondizione}: L'amministratore ha rimosso il permesso di monitoraggio del dato del dispositivo all'ente selezionato
				\item \textbf{Scenario Principale}:
				\begin{enumerate}
					\item{L'amministratore visualizza le informazioni del dispositivo}
					\item{L'amministratore seleziona il dato del dispositivo}
					\item{L'amministratore seleziona l'ente a cui rimuovere i permessi di monitoraggio del dato}
					\item{L'amministratore ha rimosso il permesso di monitoraggio del dato del dispositivo all'ente selezionato}
				\end{enumerate}
			\end{itemize}

		\subsubsection{UC 6 - Gestione view}
		\begin{itemize}
			\item \textbf{Attori Primari}: [UA] [ME]
			\item \textbf{Descrizione}: L'utente gestisce le proprie view.
			\item \textbf{Precondizione}: L'utente seleziona la voce "Gestione view"
			\item \textbf{Postcondizione}: L'utente ha visualizzato/gestito le proprie view.
			\item \textbf{Scenario Principale}:
			\begin{enumerate}
				\item{L'utente seleziona la voce "Gestione view"}
				\item{L'utente visualizza la schermata per la gestione delle view}
				\item{L'utente visualizza/gestisce le proprie view}
				\item{L'utente ha visualizzato/gestito le proprie view}
			\end{enumerate}	
		\end{itemize}

			\paragraph{UC 6.1 - Visualizzazione view}
			\begin{itemize}
				\item \textbf{Attori Primari}: [UA] [ME]
				\item \textbf{Descrizione}: L'utente visualizza le proprie view.
				\item \textbf{Precondizione}: L'utente visualizza la schermata di gestione delle view.
				\item \textbf{Postcondizione}: L'utente ha visualizzato le proprie view.
				\item \textbf{Scenario Principale}:
				\begin{enumerate}
					\item{L'utente visualizza la schermata di gestione delle view}
					\item{L'utente visualizza le proprie view}
					\item{L'utente ha visualizzato le proprie view}
				\end{enumerate}	
			\end{itemize}

			\paragraph{UC 6.2 - Aggiunta view}
			\begin{itemize}
				\item \textbf{Attori Primari}: [UA] [ME]
				\item \textbf{Descrizione}: L'utente, che sta visualizzando la sezione view ed il bottone di creazione di un nuovo grafico, clicca quest'ultimo e visualizza un nuovo grafico vuoto.
				\item \textbf{Precondizione}: L'utente visualizza la sezione view ed il bottone di aggiunta grafico
				\item \textbf{Postcondizione}: L'utente ha aggiunto un grafico vuoto alla view
				\item \textbf{Scenario Principale}:
				\begin{enumerate}
					\item{L'utente clicca sul bottone di aggiunta grafico}
				\end{enumerate}	
			\end{itemize}

			\paragraph{UC 6.3 - Creazione grafici view}
			\begin{itemize}
				\item \textbf{Attori Primari}: [UA] [ME]
				\item \textbf{Descrizione}: L'utente, che sta visualizzando nella pagina view un grafico vuoto, compila i campi con i dispositivi e i relativi sensori che intende graficare, seleziona il tipo di correlazione che vule visualizzare tra i dati, clicca sul bottone di creazione grafico e visualizza un grafico con i dati del/dei sensore selezionato/i.
				\item \textbf{Precondizione}: L'utente visualizza nella pagina un grafico vuoto con i campi per crearlo e un altro bottone di aggiunta
				\item \textbf{Postcondizione}: L'utente visualizza il grafico di uno o due sensori, i dispositivi/sensori relativi e il tipo di correlazione 
				\item \textbf{Scenario Principale}:
				\begin{enumerate}
					\item{L'utente seleziona uno o due dispositivi}
					\item{L'utente seleziona uno o due sensori relativi ai dispositivi precedentemente selezionati}
					\item{L'utente seleziona il tipo di correlazione tra i dati che intende visualizzare}
					\item{L'utente clicca sul bottone di creazione del grafico}
				\end{enumerate}	
			\end{itemize}

			\paragraph{UC 6.4 - Eliminazione grafico}
			\begin{itemize}
				\item \textbf{Attori Primari}: [UA] [ME]
				\item \textbf{Descrizione}: L'utente, che visualizza nella pagina uno dei grafici che ha creato, clicca sul bottone di eliminazione di uno dei grafici e il grafico viene eliminato.
				\item \textbf{Precondizione}: L'utente visualizza almeno un grafico nella pagina
				\item \textbf{Postcondizione}: l'utente ha eliminato un grafico dalla pagina
				\item \textbf{Scenario Principale}:
				\begin{enumerate}
					\item{L'utente clicca sul bottone di eliminazine del grafico}
				\end{enumerate}	
			\end{itemize}

		\subsubsection{UC 7 - Gestione utenti ente}
		
		\begin{figure}[t!]
			\centering
			\includegraphics[height=10em]{res/images/UC7 - Gestione utenti ente.jpg}
		\end{figure}
		
		\begin{itemize}
			\item \textbf{Attori Primari}: [ME]
			\item \textbf{Descrizione}: Il moderatore ente gestisce gli utenti del proprio ente.
			\item \textbf{Precondizione}: Il moderatore ente seleziona la voce "Gestione utenti ente".
			\item \textbf{Postcondizione}: Il moderatore ente ha visualizzato/gestito gli utenti appartenenti al proprio ente.
			\item \textbf{Scenario Principale}:
			\begin{enumerate}
				\item{Il moderatore ente seleziona la voce "Gestione utenti ente"}
				\item{Il moderatore ente visualizza la schermata per la gestione degli utenti ente}
				\item{Il moderatore ente visualizza/gestisce gli utenti appartenenti al proprio ente}
				\item{Il moderatore ente ha visualizzato/gestito gli utente del proprio ente}
			\end{enumerate}	
		\end{itemize}
			
			\paragraph{UC 7.1 - Visualizzazione utenti ente}
			\begin{itemize}
				\item \textbf{Attori Primari}: [ME]
				\item \textbf{Descrizione}: Il moderatore ente visualizza gli utenti del proprio ente.
				\item \textbf{Precondizione}: Il moderatore ente visualizza la schermata per la gestione degli utenti ente.
				\item \textbf{Postcondizione}: Il moderatore ente ha visualizzato la lista degli utenti appartenenti al proprio ente.
				\item \textbf{Scenario Principale}:
				\begin{enumerate}
					\item{Il moderatore ente visualizza la schermata per la gestione degli utenti ente}
					\item{Il moderatore ente visualizza la lista degli utenti appartenenti al proprio ente}
					\item{Il moderatore ente ha visualizzato gli utenti del proprio ente}
				\end{enumerate}	
			\end{itemize}
			
			\paragraph{UC 7.2 - Creazione utente ente}
			\begin{itemize}
				\item \textbf{Attori Primari}: [ME]
				\item \textbf{Descrizione}: Il moderatore ente aggiunge un nuovo utente al proprio ente.
				\item \textbf{Precondizione}: Il moderatore ente visualizza la schermata per la gestione degli utenti ente.
				\item \textbf{Postcondizione}: Il moderatore ente ha aggiunto un utente al proprio ente.
				\item \textbf{Scenario Principale}:
				\begin{enumerate}
					\item{Il moderatore ente visualizza la schermata per la gestione degli utenti ente}
					\item{Il moderatore ente seleziona la voce "aggiungi utente ente"}
					\item{Il moderatore ente compila il form utente con i dati dell'utente da aggiungere}
					\item{Il moderatore ente preme sul bottone di salvataggio per aggiungere il nuovo utente}
					\item{Il moderatore ente ha aggiunto un nuovo utente appartenente al proprio ente}
				\end{enumerate}	
				\item \textbf{Inclusioni}:
					\item Il moderatore ente compila il form utente con i dati dell'utente da aggiungere (UC 7.3)
				\item \textbf{Estensioni}:
				\begin{itemize}
					\item Il moderatore ente inserisce un email non valida (UC 7.4)
					\item Il moderatore ente inserisce un nome e/o cognome non validi (UC 7.5)
				\end{itemize}
			\end{itemize}
			
			\paragraph{UC 7.3 - Compilazione form utente}
			\begin{itemize}
				\item \textbf{Attori Primari}: [ME]
				\item \textbf{Descrizione}: Il moderatore ente compila i campi per l'aggiunta/modifica dell'utente: il campo "email", corrispondente all'email dell'utente, e i campi "nome" e "cognome", corrispondenti al nominativo dell'utente.
				\item \textbf{Precondizione}: Il moderatore ente ha selezionato la voce "aggiungi utente ente" o "modifica utente ente".
				\item \textbf{Postcondizione}: Il moderatore ente ha compilato il form utente.
				\item \textbf{Scenario Principale}:
				\begin{enumerate}
					\item{Il moderatore ente seleziona la voce "aggiungi utente ente"}
					\item{Il moderatore ente compila il campo "email"}
					\item{Il moderatore ente compila il campo "nome"}
					\item{Il moderatore ente compila il campo "cognome"}
					\item{Il moderatore ente ha compilato il form utente}
				\end{enumerate}	
			\end{itemize}
			
			\paragraph{UC 7.4 - Email non valida}
			\begin{itemize}
				\item \textbf{Attori Primari}: [ME]
				\item \textbf{Descrizione}: Dopo aver premuto il bottone per salvare i dati inseriti nel form utente viene visualizzato il messaggio di errore "Email non valida" perchè l'email inserita non è valida. 
				\item \textbf{Precondizione}: Il moderatore ente ha premuto sul bottone di salvataggio del form utente.
				\item \textbf{Postcondizione}: Visualizzazione messaggio di errore "Email non valida"
				\item \textbf{Scenario Principale}:
				\begin{enumerate}
					\item{Il moderatore ente ha premuto sul bottone di salvataggio del form utente}
					\item{La email inserita nel campo "email" non è valida}
					\item{Viene visualizzato il messaggio di errore "Email non valida"}
				\end{enumerate}	
			\end{itemize}
			
			\paragraph{UC 7.5 - Nome e/o cognome non validi}
			\begin{itemize}
				\item \textbf{Attori Primari}: [ME]
				\item \textbf{Descrizione}: Dopo aver premuto il bottone per salvare i dati inseriti nel form utente viene visualizzato il messaggio di errore "Nome e/o cognome non validi" perchè il nome e/o il cognome inseriti non sono validi. 
				\item \textbf{Precondizione}: Il moderatore ente ha premuto sul bottone di salvataggio del form utente.
				\item \textbf{Postcondizione}: Visualizzazione messaggio di errore "Nome e/o cognome non validi"
				\item \textbf{Scenario Principale}:
				\begin{enumerate}
					\item{Il moderatore ente ha premuto sul bottone di salvataggio del form utente}
					\item{Il nome nel campo "nome" non è valido e/o il cognome nel campo "cognome" non è valido}
					\item{Viene visualizzato il messaggio di errore "Nome e/o cognome non validi"}
				\end{enumerate}	
			\end{itemize}
			
			\paragraph{UC 7.6 - Visualizza dati utenti ente}
			\begin{itemize}
				\item \textbf{Attori Primari}: [ME]
				\item \textbf{Descrizione}: Il moderatore ente visualizza i dati dell'utente selezionato dalla lista degli utenti del proprio ente.
				\item \textbf{Precondizione}: Il moderatore ente seleziona un utente dalla lista degli utenti del proprio ente.
				\item \textbf{Postcondizione}: Il moderatore ente ha visualizzato i dati dell'utente selezionato appartenente al proprio ente.
				\item \textbf{Scenario Principale}:
				\begin{enumerate}
					\item{Il moderatore ente seleziona un utente dalla lista degli utenti del proprio ente}
					\item{Il moderatore ente seleziona la voce "visualizza dati utente ente"}
					\item{Il moderatore ente visualizza i dati dell'utente selezionato}
					\item{Il moderatore ha visualizzato i dati dell'utente selezionato}
				\end{enumerate}
			\end{itemize}
			
			\paragraph{UC 7.7 - Modifica dati utenti ente}
			\begin{itemize}
				\item \textbf{Attori Primari}: [ME]
				\item \textbf{Descrizione}: Il moderatore ente modifica l'utente selezionato dalla lista degli utenti del proprio ente.
				\item \textbf{Precondizione}: Il moderatore ente seleziona un utente dalla lista degli utenti del proprio ente.
				\item \textbf{Postcondizione}: Il moderatore ente ha modificato l'utente selezionato appartenente al proprio ente.
				\item \textbf{Scenario Principale}:
				\begin{enumerate}
					\item{Il moderatore ente seleziona un utente dalla lista degli utenti del proprio ente}
					\item{Il moderatore seleziona la voce "modifica utente ente"}
					\item{Il moderatore compila il form utente contenente i dati da modificare dell'utente}
					\item{Il moderatore ente preme sul bottone di salvataggio per modificare l'utente selezionato}
					\item{Il moderatore ha modificato l'utente selezionato}
				\end{enumerate}	
				\item \textbf{Inclusioni}:
					\item Il moderatore compila il form utente con i dati dell'utente da modificare (UC 7.3)
				\item \textbf{Estensioni}:
				\begin{itemize}
					\item Il moderatore inserisce un email non valida (UC 7.4)
					\item Il moderatore inserisce un nome e/o cognome non validi (UC 7.5)
				\end{itemize}
			\end{itemize}
			
			\paragraph{UC 7.8 - Rimozione utenti ente}
			\begin{itemize}
				\item \textbf{Attori Primari}: [ME]
				\item \textbf{Descrizione}: Il moderatore ente rimuove dal sistema l'utente selezionato dalla lista degli utenti del proprio ente.
				\item \textbf{Precondizione}: Il moderatore ente seleziona un utente dalla lista degli utenti del proprio ente.
				\item \textbf{Postcondizione}: Il moderatore ente ha rimosso l'utente selezionato appartenente al proprio ente dal sistema.
				\item \textbf{Scenario Principale}:
				\begin{enumerate}
					\item{Il moderatore ente seleziona un utente appartenente al proprio ente da rimuovere}
					\item{Il moderatore ente seleziona la voce "Rimuovi utente ente"}
					\item{Il moderatore ente ha rimosso l'utente selezionato appartenente al proprio ente dal sistema.}
				\end{enumerate}		
			\end{itemize}

		\subsubsection{UC 8 - Gestione alert ente}
		
		\begin{figure}[t!]
			\centering
			\includegraphics[height=10em]{res/images/UC8 - Gestione alert ente.jpg}
		\end{figure}
		
		\begin{itemize}
			\item \textbf{Attori Primari}: [ME]
			\item \textbf{Descrizione}: Il moderatore ente gestisce gli alert del proprio ente.
			\item \textbf{Precondizione}: Il moderatore ente seleziona la voce "Gestione alert ente"
			\item \textbf{Postcondizione}: Il moderatore ente ha visualizzato/gestito gli alert del proprio ente.
			\item \textbf{Scenario Principale}:
			\begin{enumerate}
				\item{Il moderatore ente seleziona la voce "Gestione alert ente"}
				\item{Il moderatore ente visualizza la schermata per la gestione degli alert ente}
				\item{Il moderatore ente visualizza/gestisce gli alert del proprio ente}
				\item{Il moderatore ente ha visualizzato/gestito gli alert del proprio ente}
			\end{enumerate}	
		\end{itemize}
			
			\paragraph{UC 8.1 - Visualizzazione alert ente}
			\begin{itemize}
				\item \textbf{Attori Primari}: [ME]
				\item \textbf{Descrizione}: Il moderatore ente visualizza gli alert appartenenti al proprio ente.
				\item \textbf{Precondizione}: Il moderatore ente visualizza la schermata per la gestione degli alert del proprio ente
				\item \textbf{Postcondizione}: Il moderatore ente ha visualizzato la lista degli alert del proprio ente.
				\item \textbf{Scenario Principale}:
				\begin{enumerate}
					\item{Il moderatore ente visualizza la schermata per la gestione degli alert del proprio ente}
					\item{L'utente seleziona la voce "Visualizza alert ente"}
					\item{L'utente visualizza la lista degli alert del proprio ente}
					\item{L'utente ha visualizzato la lista degli alert del proprio ente}
				\end{enumerate}	
			\end{itemize}
			
			\paragraph{UC 8.2 - Inserimento alert ente}
			\begin{itemize}
				\item \textbf{Attori Primari}: [ME]
				\item \textbf{Descrizione}: Il moderatore ente inserisce un nuovo alert al proprio ente.
				\item \textbf{Precondizione}: Il moderatore ente visualizza la schermata per la gestione degli alert del proprio ente
				\item \textbf{Postcondizione}: Il moderatore ente ha inserito un nuovo alert per gli utenti appartenenti al proprio ente 
				\item \textbf{Scenario Principale}:
				\begin{enumerate}
					\item{Il moderatore ente inserisce un nuovo alert al proprio ente}
					\item{Il moderatore ente seleziona la voce "Inserisci alert ente"}
					\item{Il moderatore ente compila il form alert con i dati dell'alert da inserire}
					\item{Il moderatore ente preme sul bottone di salvataggio per inserire il nuovo alert}
					\item{Il moderatore ente ha inserito un nuovo alert per gli utenti appartenenti al proprio ente }
				\end{enumerate}
				\item \textbf{Inclusioni}:
				\begin{itemize}
					\item Il moderatore ente compila il form alert con i dati dell'alert da inserire (UC 8.2.1)
				\end{itemize}
				\item \textbf{Estensioni}:
				\begin{itemize}
					\item Vengono inseriti dei valori soglia non validi (UC 8.2.2)
				\end{itemize}		
			\end{itemize}
			
			\paragraph{UC 8.2.1 - Compilazione form alert}
			\begin{itemize}
				\item \textbf{Attori Primari}: [ME]
				\item \textbf{Descrizione}: Il moderatore ente compila i campi per l'inserimento dell'alert: il campo "dato", corrispondente al dato da monitorare dell'alert, e il campo "soglia", corrispondente al valore soglia dopo il quale si riceverà un alert.
				\item \textbf{Precondizione}: Il moderatore ente ha selezionato la voce "Inserisci alert ente".
				\item \textbf{Postcondizione}: Il moderatore ente ha compilato il form alert.
				\item \textbf{Scenario Principale}:
				\begin{enumerate}
					\item{Il moderatore ente seleziona la voce "Inserisci alert ente"}
					\item{Il moderatore ente compila il campo "dato"}
					\item{Il moderatore ente compila il campo "soglia"}
					\item{Il moderatore ente ha compilato il form alert}
				\end{enumerate}	
			\end{itemize}

			\paragraph{UC 8.2.2 - Errore valori soglia}
			\begin{itemize}
				\item \textbf{Attori Primari}: [ME]
				\item \textbf{Descrizione}: Dopo aver premuto il bottone per salvare i dati inseriti nel form alert viene visualizzato il messaggio di errore "Valore soglia non valido" perchè il valore inserito come valore, oltre il quale viene ricevuto un alert, non è valido.
				\item \textbf{Precondizione}: Il moderatore ente ha premuto sul bottone di salvataggio del form alert
				\item \textbf{Postcondizione}: Visualizzazione messaggio di errore "Valore soglia non valido" 
				\item \textbf{Scenario Principale}:
				\begin{enumerate}
					\item{Il moderatore ente ha premuto sul bottone di salvataggio del form alert}
					\item{Il valore soglia inserito nel campo "soglia" non è valido}
					\item{Viene visualizzato il messaggio di errore "Valore soglia non valido" }
				\end{enumerate}
			\end{itemize}
			
			\paragraph{UC 8.3 - Rimozione alert ente}
			\begin{itemize}
				\item \textbf{Attori Primari}: [ME]
				\item \textbf{Descrizione}: Il moderatore ente rimuove l'alert selezionato dalla lista degli alert del proprio ente.
				\item \textbf{Precondizione}: Il moderatore ente seleziona un alert dalla lista degli alert del proprio ente.
				\item \textbf{Postcondizione}: Il moderatore ente ha rimosso l'alert selezionato del proprio ente dal sistema.
				\item \textbf{Scenario Principale}:
				\begin{enumerate}
					\item{Il moderatore ente seleziona un alert dalla lista degli alert del proprio ente}
					\item{Il moderatore ente seleziona la voce "Rimuovi alert ente"}
					\item{Il moderatore ente ha rimosso l'alert selezionato del proprio ente dal sistema.}
				\end{enumerate}	
			\end{itemize}

		\subsubsection{UC 9 - Visualizzazione log utenti ente}
		\begin{itemize}
			\item \textbf{Attori Primari}: [ME]
			\item \textbf{Descrizione}: Il moderatore ente visualizza i log degli utente appartenenti al proprio ente.
			\item \textbf{Precondizione}: Il moderatore seleziona la voce "Visualizza log utenti ente"
			\item \textbf{Postcondizione}: L'utente ha visualizzato i log relativi agli utenti appartenenti al proprio ente
			\item \textbf{Scenario Principale}:
			\begin{enumerate}
				\item{L'utente seleziona la voce "Visualizzazione log utenti ente"}
				\item{L'utente visualizza la lista dei log relativa agli utenti appartenenti al proprio ente}
				\item{L'utente ha visualizzato i log relativi agli utenti appartenenti al proprio ente}
			\end{enumerate}	
		\end{itemize}

		\subsubsection{UC 10 - Gestione enti}
		\begin{itemize}
			\item \textbf{Attori Primari}: [AM]
			\item \textbf{Descrizione}: L'utente, che sta visualizzando il menù, seleziona la voce "Gestione enti" che permette la gestione degli enti all'interno del sistema.
			\item \textbf{Precondizione}: L'utente visualizza il menu
			\item \textbf{Postcondizione}: L'utente ha visualizzato/gestito gli enti all'interno del sistema. 
			\item \textbf{Scenario Principale}:
			\begin{enumerate}
				\item{L'utente seleziona la voce "Gestione enti"}
				\item{L'utente può svolgere diverse azioni allo scopo di gestire gli enti all'interno del sistema}
			\end{enumerate}	
		\end{itemize}

			\paragraph{UC 10.1 - Visualizzazione lista enti }
			\begin{itemize}
				\item \textbf{Attori Primari}: [AM]
				\item \textbf{Descrizione}: L'utente, che sta visualizzando la schermata per la gestione enti, visualizza la lista degli enti presenti nel sistema.
				\item \textbf{Precondizione}: L'utente visualizza la schermata per la gestione degli enti
				\item \textbf{Postcondizione}: L'utente visualizza la lista degli enti presenti nel sistema
				\item \textbf{Scenario Principale}:
				\begin{enumerate}
					\item{L'utente seleziona la voce "visualizza enti"}
					\item{L'utente visualizza la lista enti}
				\end{enumerate}	
			\end{itemize}

			\paragraph{UC 10.2 - Visualizzazione informazioni ente}
			\begin{itemize}
				\item \textbf{Attori Primari}: [AM]
				\item \textbf{Descrizione}: L'amministratore, che sta visualizzando la lista degli enti, seleziona un ente e ne visualizza le informazioni riguardanti.
				\item \textbf{Precondizione}: L'utente visualizza la la schermata gestione enti e la lista degli enti
				\item \textbf{Postcondizione}: L'utente visualizza le informazioni di un ente selezionato
				\item \textbf{Scenario Principale}:
				\begin{enumerate}
					\item{L'utente seleziona dalla lista un ente}
					\item{L'utente visualizza le informazioni riguardanti l'utente selezionato}
				\end{enumerate}	
			\end{itemize}

			\paragraph{UC 10.3 - Aggiunta ente}
			\begin{itemize}
				\item \textbf{Attori Primari}: [AM]
				\item \textbf{Descrizione}: L'amministratore, che sta visualizzando la schermata per la gestione degli enti, aggiunge un nuovo ente al sistema.
				\item \textbf{Precondizione}: L'utente visualizza la schermata per la gestione degli enti
				\item \textbf{Postcondizione}: L'utente ha creato un nuovo ente
				\item \textbf{Scenario Principale}:
				\begin{enumerate}
					\item{L'utente inserisce i dati nei campi}
					\item{L'ente viene creato dall'utente con le informazioni fornite}
				\end{enumerate}	
				\item \textbf{Estensioni}:
					\begin{itemize}
						\item Il nome dell'ente viene inserito in un formato non valido
					\end{itemize}
			\end{itemize}		

			\paragraph{UC 10.4 - Modifica ente}
			\begin{itemize}
				\item \textbf{Attori Primari}: [AM]
				\item \textbf{Descrizione}: L'amministratore, che sta visualizzando la lista con tutti gli enti creati fino a quel momento, seleziona l'ente di cui desidera modificare le informazioni. 
				\item \textbf{Precondizione}: L'utente visualizza la lista degli enti
				\item \textbf{Postcondizione}: L'utente ha modificato l'ente selezionato
				\item \textbf{Scenario Principale}:
				\begin{enumerate}
					\item{L'utente seleziona l'ente da modificare}
					\item{L'utente modifica i dati dell'ente selezionato}
				\end{enumerate}
			\end{itemize}	

			\paragraph{UC 10.5 - Rimozione ente}
			\begin{itemize}
				\item \textbf{Attori Primari}: [AM]
				\item \textbf{Descrizione}: L'amministratore, che sta visualizzando la lista degli enti, seleziona un ente e lo rimuove dal sistema.
				\item \textbf{Precondizione}: L'utente visualizza la lista degli enti appartenenti al sistema
				\item \textbf{Postcondizione}: L'utente he rimosso un ente
				\item \textbf{Scenario Principale}:
				\begin{enumerate}
					\item{L'utente seleziona l'ente da rimuovere}
					\item{L'ente selezionato viene rimosso dal sistema}
				\end{enumerate}	
			\end{itemize}		

		\subsubsection{UC 11 - Gestione utenti amministratore}
		\begin{itemize}
			\item \textbf{Attori Primari}: [AM]
			\item \textbf{Descrizione}: L'amministratore, che sta visualizzando il menù, seleziona la voce "Gestione utenti amministratore" che permette la gestione degli utenti all'interno del sistema.
			\item \textbf{Precondizione}: L'utente visualizza il menu
			\item \textbf{Postcondizione}: L'utente ha visualizzato/gestito gli utenti all'interno del sistema. 
			\item \textbf{Scenario Principale}:
			\begin{enumerate}
				\item{L'utente seleziona la voce "Gestione utenti amministratore"}
				\item{L'utente può svolgere diverse azioni allo scopo di gestire gli utenti all'interno del sistema}
			\end{enumerate}	
		\end{itemize}

			\paragraph{UC 7.1 - Visualizzazione utenti ente}
			\begin{itemize}
				\item \textbf{Attori Primari}: [ME]
				\item \textbf{Descrizione}: L'utente, che sta visualizzando la schermata di gestione utenti ente, visualizza la lista degli utenti appartenenti al proprio ente.
				\item \textbf{Precondizione}: L'utente sta visualizzando la schermata per la gestione degli utenti del proprio ente.
				\item \textbf{Postcondizione}: l'utente visualizza la lista degli utenti appartenenti al proprio ente.
				\item \textbf{Scenario Principale}:
				\begin{enumerate}
					\item{L'utente visualizza la lista degli utenti appartenenti al proprio ente}
				\end{enumerate}	
			\end{itemize}
			
			\paragraph{UC 7.2- Creazione utenti ente}
			\begin{itemize}
				\item \textbf{Attori Primari}: [ME]
				\item \textbf{Descrizione}: L'utente, che sta visualizzando la schermata di gestione utenti ente e la lista degli utenti appartenenti al proprio ente, crea un nuovo utente che apparterrà al proprio ente.
				\item \textbf{Precondizione}: L'utente sta visualizzando la schermata per la gestione degli utenti del proprio ente e la lista degli utenti appartenenti al proprio ente.
				\item \textbf{Postcondizione}: L'utente ha aggiunto un utente al proprio ente.
				\item \textbf{Scenario Principale}:
				\begin{enumerate}
					\item{L'utente seleziona la voce "aggiungi utente ente"}
					\item{L'utente compila i dati necessari per la creazione dell'utente}
					\item{L'utente è aggiunto nel sistema come utente appartenente al proprio ente}
				\end{enumerate}	
				\item \textbf{Estensioni}:
				\begin{itemize}
					\item L'utente inserisce un email non valida (UC 7.3)
					\item L'utente inserisce un nome e/o cognome non validi (UC 7.4)
				\end{itemize}
			\end{itemize}
			
			\paragraph{UC 7.4 - Email non valida}
			\begin{itemize}
				\item \textbf{Attori Primari}: [ME]
				\item \textbf{Descrizione}: L'utente ha compilato i dati per l'aggiunta del nuovo utente ed ha inserito un email non valida.
				\item \textbf{Precondizione}: L'utente ha compilato i dati per l'aggiunta del nuovo utente.
				\item \textbf{Postcondizione}: l'utente visualizza un messaggio di errore riguardante l'email non valida.
				\item \textbf{Scenario Principale}:
				\begin{enumerate}
					\item{L'utente seleziona la voce "aggiungi utente ente"}
					\item{L'utente compila i dati necessari per la creazione dell'utente}
					\item{L'utente visualizza un messaggio di errore riguardante l'email non valida}
				\end{enumerate}	
			\end{itemize}
			
			\paragraph{UC 7.5 - Nome e/o cognome non validi}
			\begin{itemize}
				\item \textbf{Attori Primari}: [ME]
				\item \textbf{Descrizione}: L'utente ha compilato i dati per l'aggiunta del nuovo utente ed ha inserito un nome e/o cognome non validi.
				\item \textbf{Precondizione}: L'utente ha compilato i dati per l'aggiunta del nuovo utente.
				\item \textbf{Postcondizione}: l'utente visualizza un messaggio di errore riguardante il nome e/o cognome non validi.
				\item \textbf{Scenario Principale}:
				\begin{enumerate}
					\item{L'utente seleziona la voce "aggiungi utente ente"}
					\item{L'utente compila i dati necessari per la creazione dell'utente}
					\item{L'utente visualizza un messaggio di errore riguardante il nome e/o cognome non validi}
				\end{enumerate}	
			\end{itemize}
			
			\paragraph{UC 7.6 - Modifica dati utenti ente}
			\begin{itemize}
				\item \textbf{Attori Primari}: [ME]
				\item \textbf{Descrizione}: L'utente,  che sta visualizzando la schermata di gestione utenti ente e la lista degli utenti appartenenti al proprio ente, seleziona un utente e ne modifica le informazioni.
				\item \textbf{Precondizione}: L'utente, che sta visualizzando la schermata di gestione utenti ente e la lista degli utenti appartenenti al proprio ente, seleziona un utente da modificare.
				\item \textbf{Postcondizione}: l'utente ha modificato un utente appartenente al proprio ente.
				\item \textbf{Scenario Principale}:
				\begin{enumerate}
					\item{L'utente seleziona un utente appartenente al proprio ente}
					\item{L'utente modifica le impostazioni dell'utente}
				\end{enumerate}	
				\item \textbf{Estensioni}:
				\begin{itemize}
					\item L'utente inserisce un email non valida (UC 7.3)
					\item L'utente inserisce un nome e/o cognome non validi (UC 7.4)
				\end{itemize}
			\end{itemize}
			
			\paragraph{UC 7.7 - Rimozione utenti ente}
			\begin{itemize}
				\item \textbf{Attori Primari}: [ME]
				\item \textbf{Descrizione}: L'utente,  che sta visualizzando la schermata di gestione utenti ente e la lista degli utenti appartenenti al proprio ente, seleziona un utente e rimuove tale utente dal sistema.
				\item \textbf{Precondizione}: L'utente, che sta visualizzando la schermata di gestione utenti ente e la lista degli utenti appartenenti al proprio ente, seleziona un utente da rimuovere.
				\item \textbf{Postcondizione}: l'utente ha rimosso un utente appartenente al proprio ente dal sistema.
				\item \textbf{Scenario Principale}:
				\begin{enumerate}
					\item{L'utente seleziona un utente appartenente al proprio ente}
					\item{L'utente rimuove l'utente selezionato}
				\end{enumerate}		
			\end{itemize}

		\subsubsection{UC 12 - Gestione alert amministratore}
		\begin{itemize}
			\item \textbf{Attori Primari}: [AM]
			\item \textbf{Descrizione}: L'aministratore, che sta visualizzando il menù, seleziona la voce "Gestione alert amministratore" che permette la gestione degli alert impostati dagli enti all'interno del sistema.
			\item \textbf{Precondizione}: L'utente visualizza il menu.
			\item \textbf{Postcondizione}: L'utente ha gestito gli alert all'interno del sistema.
			\item \textbf{Scenario Principale}:
			\begin{enumerate}
				\item{L'utente seleziona la voce "Gestione alert amministratore"}
				\item{L'utente può svolgere diverse azioni allo scopo di gestire gli alert impostati dagli enti all'interno del sistema}
			\end{enumerate}	
		\end{itemize}

		\subsubsection{UC 13 - Visualizzazione log amministratore}
		\begin{itemize}
			\item \textbf{Attori Primari}: [AM]
			\item \textbf{Descrizione}: seleziona la voce "Visualizzazione log utenti ente" che mostra all'utente la lista dei log degli utenti presenti all'interno del sistema.
			\item \textbf{Precondizione}: L'utente visualizza il menu
			\item \textbf{Postcondizione}: L'utente visualizza la lista degli alert assegnati
			\item \textbf{Scenario Principale}:
			\begin{enumerate}
				\item{L'utente seleziona la voce "Visualizzazione log amministratore"}
				\item{L'utente visualizza la lista dei log degli utenti presenti all'interno del sistema}
			\end{enumerate}	
		\end{itemize}
		