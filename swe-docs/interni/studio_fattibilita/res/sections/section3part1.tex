

\section{Valutazione capitolati rimanenti}
    \subsection{Capitolato C1 - Autonomous Highlights Platform}
       \subsubsection{Informazioni generali}
       \begin{itemize}
           \item \textbf{Proponente: }Zero12;
           \item \textbf{Committente: }Prof. Tullio Vardanega e Prof. Riccardo Cardin.
       \end{itemize}

    \subsubsection{Descrizione}
        L'obiettivo di questo capitolato è creare una piattaforma web che è capace di ricevere in input dei video di eventi sportivi, come una partita di calcio o di tennis, e che riesca a creare autonomamente un video di massimo 5 minuti contenente soltanto i suoi momenti chiave (highlights). 
    
    \subsubsection{Finalità del progetto}
        Il prodotto finale, ovvero la piattaforma web, dovrà essere dotata di un modello di machine learning in grado di identificare quelli che possono essere considerati come i momenti più importanti dell'evento sportivo che le è stato inviato. Un fattore importante è che andrà scelto uno sport sul quale focalizzare la propria attenzione e sul quale verrà addestrato il modello di apprendimento atto all'identificazione automatica dei momenti salienti di un evento del suddetto sport.  
        Il flusso di generazione del suddetto highlight dovrà avere la seguente struttura:
        \begin{itemize}
            \item Caricamento del video; 
            \item Identificazione dei momenti salienti;
            \item Estrazione delle corrispondenti parti di video;
            \item Generazione del video di sintesi.
        \end{itemize}
    \subsubsection{Tecnologie}
    Le tecnologie consigliate dall'azienda riguardano la tecnologia di Amazon Web Services ed in particolare:
    \begin{itemize}
        \item \textbf{Elastic Container Service o Elastic Kubernetes Service: }è un servizio che permette la gestione di contenitori, altamente dimensionabile e ad elevate prestazioni;
        \item \textbf{DynamoDB: }è un database non relazionale per applicazioni che necessitano di prestazioni elevate su qualsiasi scala.
        \item \textbf{AWS Transcode: }è un servizio di transcodifica di contenuti multimediali nel cloud;
        \item \textbf{Sage Maker: } è un servizio completamente gestito che permette a sviluppatori e data scientist di creare, addestrare e distribuire modelli di apprendimento automatico;
        \item \textbf{AWS Rekognition video: } è un servizio di analisi video basato su apprendimento approfondito; è in grado di riconoscere i movimenti delle persone in un fotogramma e di riconoscere soggetti, volti, oggetti, celebrità e contenuti inappropriati.
    \end{itemize}
        \subsubsection{Linguaggi di programmazione}
        \begin{itemize}
            \item \textbf{NodeJS: }linguaggio ideale per sviluppare API Restful JSON a supporto dell'applicativo;
            \item \textbf{Python: }linguaggio ideale per lo sviluppo delle componenti di machine learning;
            \item \textbf{HTML5}, \textbf{CSS3}, \textbf{Javascript: }linguaggi per la realizzazione dell'interfaccia web di gestione del flusso di lavoro, utilizzando un framework responsive come Twitter.
        \end{itemize}
    Vincoli del progetto:
    \begin{itemize}
        \item Utilizzo di Sage Maker;
        \item L'architettura dovrà essere basata su micro-servizi, suddividendo il progetto in tante funzioni di base denominate servizi. Questi ultimi dovranno essere indipendenti fra loro;
        \item Caricamento dei video da elaborare tramite riga di comando;
        \item Console web di analisi e controllo degli stati di elaborazione dei video.
      \end{itemize}
    
    \subsubsection{Aspetti positivi}
    \begin{itemize}
    		\item Per quanto riguarda le tecnologie interessate è presente molta documentazione a riguardo;	
    		\item Il tema principale del progetto, ovvero machine learning, è stato accolto con molto interesse dal gruppo, che si è dimostrato interessato ad approfondire l'argomento (che potrebbe essere un'aggiunta interessante al proprio CV);
    		 \item Il proponente fornisce attività di formazione sulle principali tecnologie AWS e wireframe dell'interfaccia della console web di analisi e controllo dello stato di elaborazione dei video.
    \end{itemize}
    \subsubsection{Criticità}
    \begin{itemize}
    		\item Il proponente non fornisce nessun data-set per effettuare il training dell'algoritmo;
    		\item  Le tecnologie nonostante siano state accolte con interesse non sono conosciute dal gruppo e  richiedono un apprendimento individuale avanzato sul machine learning.
    \end{itemize}
    
    \subsubsection{Conclusione}
	DA FARE 