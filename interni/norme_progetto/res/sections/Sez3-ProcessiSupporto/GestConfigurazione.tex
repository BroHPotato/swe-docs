\subsection{Gestione della configurazione}

% Inizio gestione della configurazione

\subsubsection{Scopo}

La gestione della configurazione definisce i principi normativi utili a predisporre il \glock{workspace} per tutto il gruppo, semplificando e automatizzando la conservazione dei documenti e delle componenti software, che andranno a formare il prodotto finale.

\subsubsection{Aspettative}

La gestione della configurazione si predispone a rendere usabile l'ambiente di lavoro, controllando attivamente e autonomamente tutte le attività in modo metodico e organizzato. Ciò è importante perché riduce la complessità di coordinamento nel corso dello sviluppo del prodotto, sia che riguardi la documentazione, sia che riguardi le componenti software.

\subsubsection{Descrizione}

Il processo di gestione della configurazione si compone di una serie di attività e strumenti atte a costituire un ambiente di lavoro semplice e ordinato. 
In particolare, si fa uso dei seguenti strumenti:
\begin{itemize}
	\item repository;
	\item tecnologie impiegate.
\end{itemize}

Gli strumenti vengono utilizzati per svolgere le seguenti attività:
\begin{itemize}
	\item versionamento e rilascio del prodotto;
	\item versionamento e rilascio dei documenti;
	\item versionamento e rilascio dei componenti software;
	\item Continuous Integration;
	\item notification \& monitoring;
\end{itemize}


\subsubsection{Attività}

	\paragraph{Versionamento e rilascio del prodotto}

	Il prodotto viene inteso come l'insieme di componenti che dovranno essere progettate, sviluppate, verificate e validate prima di essere consegnate al cliente. Il prodotto si compone principalmente di:
	\begin{itemize}
		\item documentazione;
		\item componenti software.
	\end{itemize}

	Ciascuna componente viene versionata seguendo la propria evoluzione, mentre nel caso del prodotto si segue un versionamento più macroscopico e basato sulle \glock{baseline}.
	Nell'ambito del progetto, il gruppo ha deciso di integrare una prassi di versionamento che permetta di avere un riferimento del prodotto dalle singole componenti sviluppate.

		\subparagraph{Sintassi codice di versionamento}

		Per una versione del prodotto associato a una baseline si associa il seguente identificativo:

		\[%
			\text{+b}[\alpha].[\beta]
		\]

		\begin{itemize}
			\item \((\alpha)\): numero identificativo del rilascio del prodotto
			\begin{itemize}
				\item parte da 0 e non si resetta mai;
				\item viene incrementato quando tutte le componenti sono state concluse e pronte per la consegna al cliente;
			\end{itemize}
			\item \((\beta)\): numero identificativo che viene associato a una baseline del prodotto:
			\begin{itemize}
	  			\item parte da 0 e si resetta quando \(\alpha\) viene incrementato;
				\item viene incrementato quando si raggiunge una nuova baseline di prodotto. 
			\end{itemize}
		\end{itemize}

		\subparagraph{Esempi di versionamento del prodotto}

		\begin{itemize}
			\item \textbf{v0.5.1+b0.1}: indica la versione 0.5.1 di una componente proveniente dalla baseline di prodotto in versione 0.1;
			\item \textbf{v1.13.1+b0.2}:  indica la versione 1.12.1 di una componente proveniente dalla baseline di prodotto in versione 0.2;
			\item \textbf{v3.0.2+b1.3}: indica la versione 3.0.2 di una componente proveniente dalla baseline di prodotto in versione 1.3.
		\end{itemize}

		\subparagraph{Rilascio}

		Il rilascio del prodotto viene effettuato nel momento in cui la prima cifra della versione viene incrementata. Ad essa si devono allineare tutte le altre componenti con delle versione approvate ed autorizzate per il rilascio. 

	\paragraph{Versionamento e rilascio dei documenti}

	Tutti i file che riguardano la documentazione vengono conservati in una repository e ogni documento viene versionato per mezzo di un identificativo, in base alla loro fase di avanzamento. Questo permette di poter fare riferimento alle nuove versioni del documento durante tutto il ciclo di vita del software.

		\subparagraph{Sintassi codice di versionamento}

		Ciascun documento possiede un identificativo di versionamento su ogni pagina che ha il seguente formalismo:

		\[%
			\text{v}[\alpha].[\beta].[\gamma]
		\]

		\begin{itemize}
			\item \((\alpha)\): numero identificativo del rilascio esterno del documento al proponente e/o al committente
				\begin{itemize}
					\item parte da 0 e non si resetta mai;
					\item viene incrementato solo dal responsabile a seguito di una approvazione di rilascio del documento verso un destinatario esterno al team;
				\end{itemize}
			\item \((\beta)\): numero identificativo che rappresenta un'approvazione del documento
				\begin{itemize}
					\item parte da 0 e si resetta solo a incrementi di \(\alpha\);
					\item viene incrementato solo dal responsabile a seguito del raggiungimento di una baseline se il prodotto è conforme;
				\end{itemize}
			\item \((\gamma)\): numero rappresentativo di una aggiunta o modifica verificata fatta al documento
				\begin{itemize}
					\item parte da 0 e si resetta solo a incrementi di \(\beta\) o di \(\alpha\);
					\item viene incrementato da un revisore dopo aver verificato che una sezione sia conforme a quanto deciso.
				\end{itemize}
		\end{itemize}

		Generalmente ogni documento prima di poter essere rilasciato verso l'esterno dovrà essere stato approvato e rilasciato internamente.

		\subparagraph{Esempi codice di versionamento dei documenti}

		\begin{itemize}
			\item \verb!v0.2.1!;
			\item \verb!v1.12.1!;
			\item \verb!v3.0.2!.
		\end{itemize}

		\subparagraph{Gestione delle modifiche}

		Le modifiche ai documenti vengono gestite seguendo il \textbf{registro delle modifiche} presente in tutti i documenti nella pagina successiva al frontespizio. All'interno di questo documento si tiene traccia di ogni aggiunta, modifica, eliminazione, revisione o approvazione da parte di tutti i membri del team. La normativa sul formalismo può essere trovata al paragrafo \ref{par: Registro modifiche} a pagina \pageref{par: Registro modifiche}.

		\subparagraph{Rilascio}

		Un documento viene rilasciato alle parti proponenti solamente quando vi è un incremento del primo numero (\(\alpha\)), che ne implica una approvazione da parte del responsabile. \\
		Per quanto concerne la distribuzione interna, tutti gli \glock{artefatti} dei documenti realizzati durante la fase di sviluppo sono resi disponibili in modo rapido e automatizzato a tutti i membri del gruppo, a cui vengono notificate tutte le modifiche tramite il \glock{workspace} di \textit{Slack}.

		\subparagraph{Integrazione con il versionamento di prodotto}

		La documentazione viene interpretata come componente, e in quanto tale riceve come aggiunta al proprio codice di versionamento anche l'identificativo di prodotto. Questo identificativo aggiuntivo viene esplicitato solamente all'interno del documento e non nella denominazione dei file.

	\paragraph{Versionamento e rilascio dei componenti software}

	I sorgenti del software che riguardano la codifica e la configurazione del prodotto da realizzare sono mantenuti nella repository, insieme alla documentazione. Ogni file viene versionato con un apposito storico delle modifiche, mentre ogni componente software viene versionata come \glock{baseline} di prodotto in relazione alle funzionalità presenti e dei requisiti obbligatori implementati.

		\subparagraph{Sintassi codice di versionamento}

		Il versionamento del software viene eseguito sulla base delle implementazioni effettuate a livello di codifica.
		In aggiunta, si definisce una sintassi di stato nella versione per definirne l'usabilità.

		\[%
			\text{v}[\delta].[\epsilon].[\mu]\text{-}[\lambda]
		\]

		\begin{itemize}
			\item \((\delta)\): numero identificativo del rilascio software per una \glock{major release}
			\begin{itemize}
				\item parte da 0 e non si resetta mai;
				\item viene incrementato solo dopo aver implementato tutti i requisiti obbligatori;
			\end{itemize}
			\item \((\epsilon)\): numero identificativo per una \glock{minor release}
			\begin{itemize}
				\item parte da 0 e si resetta solo a incrementi di \(\delta\);
				\item viene incrementato solo dopo l'implementazione di uno o più requisiti;
			\end{itemize}
			\item \((\mu)\): numero rappresentativo per una \glock{patch}
			\begin{itemize}
				\item parte da 0 e si resetta solo a incrementi di \(\epsilon\) o di \(\delta\);
				\item viene incrementato a ogni modifica di uno o più requisiti e ad ogni cambio di configurazione del software;
			\end{itemize}
			\item \((\lambda)\): sigla che identifica uno stato del software; può avere i seguenti significati in ordine crescente di stato
			\begin{itemize}
				\item \verb!dev!: derivante da \textbf{dev}elopment, versione ancora in sviluppo
					\begin{itemize}
						\item software non completo, che ha ricevuto aggiunte, modifiche o eliminazioni recenti;
						\item passa tutti i test sviluppati fino a quel momento;
						\item \textbf{non} si presta all'uso di un utente finale, poiché incompleto nelle funzionalità sviluppate;
					\end{itemize}
				\item \verb!rc!: \textbf{r}elease \textbf{c}andidate, versione candidata al rilascio
				\begin{itemize}
					\item software che contiene tutti o una parte dei requisiti obbligatori richiesti;
					\item passa tutte le tipologie di test;
					\item si presta all'uso di un utente finale;
				\end{itemize}
				\item \verb!stable!: versione \textbf{stabile}, pronta al rilascio pubblico
				\begin{itemize}
					\item software completo che implementa tutti i requisiti obbligatori;
					\item passa tutte le tipologie di test ed è validato;
					\item può essere collaudato con il proponente e/o pubblicato.
				\end{itemize}
			\end{itemize}
		\end{itemize}

		Una versione del software subisce un incremento di stato in base alla sua completezza in termini di funzionalità e requisiti soddisfatti. In particolare:
		\begin{itemize}
			\item una versione può ricevere lo stato di \verb!rc! solo se il primo numero (\(\delta\)) o il secondo numero (\(\epsilon\)) è diverso da 0;
			\item una versione può ricevere lo stato di \verb!stable! solo se il primo numero (\(\delta\)) è diverso da 0.
		\end{itemize}

		\subparagraph{Esempi codice di versionamento}

		\begin{itemize}
			\item \verb!v0.3.5-dev! : versione non completa, non usabile e in parte funzionante;
			\item \verb!v0.7.0-rc! : versione non completa, usabile, funzionante e con alcuni requisiti implementati;
			\item \verb!v1.0.0-rc! : versione completa, usabile, funzionante, verificata e validata;
			\item \verb!v1.0.0-stable! : versione completa, usabile, funzionante, verificata e validata, pronta al rilascio o al collaudo.
		\end{itemize}

		\subparagraph{Rilascio}

		Le release del software vengono eseguite internamente nella \glock{repository} in base alle funzionalità sviluppate. Inoltre, i rilasci interni saranno normati come segue:
		\begin{itemize}
			\item le versioni \verb!stable! e/o \glock{Major Release} devono essere approvate dall'\glock{amministratore} e dal \glock{responsabile};
			\item le versioni \verb!rc! e/o \glock{Minor Release} devono essere approvate, previa notifica di conferma di tutti i \glock{test}, dall'\glock{amministratore} e dai \glock{verificatori};
			\item le versioni \verb!dev! e/o \glock{Patch} non richiedono approvazione e possono essere rilasciate autonomamente dal programmatore.
		\end{itemize}

		\subparagraph{Integrazione con il versionamento di prodotto}

		Il software, che è formato da diverse parti con le proprie evoluzioni, viene considerato come componente del prodotto. Pertanto, nel codice di versionamento viene aggiunto in coda l'identificativo della baseline di prodotto su cui si basa. Questa prassi non si applica alla nomenclatura dei nomi dei file che riportano la versione.

	\paragraph{Continuous Integration}

	L'attività di \textit{Continuous Integration} viene impiegata per eseguire degli incrementi a livello di funzionalità delle componenti del prodotto. In particolare, permette di controllare l'aggiunta e la rimozione di parti di codice con i test che vengono appositamente creati.

	La \textit{Continuous Integration} permette quindi di lavorare in modo sufficientemente sicuro, così da avere a disposizione un prodotto sempre funzionante che viene costruito a piccoli pezzi, i cui errori possono essere facilmente identificati e revisionati da parte dei programmatori e dei verificatori.

		\subparagraph{Github Actions}

		Nell'ambito del progetto, si fa uso delle \glock{Github Actions} per eseguire l'attività di integrazione continua. Una \textit{action} è un insieme di uno o più operazioni (\textit{job}) che devono essere eseguiti su una istanza virtuale di un sistema operativo. È importante fare riferimento a quanto segue:

		\begin{itemize}
			\item una singola operazione, per considerarsi corretta, deve avere un codice di uscita (\textit{exit code}) positivo (0);
			\item il fallimento di una \glock{Github Actions} può avvenire nel momento in cui il processo venga manualmente annullato, oppure una operazione ha avuto un codice di uscita negativo (1);
			\item una operazione può essere eseguita sulla istanza corrente del sistema operativo selezionato dalla configurazione corrente;
			\item gran parte delle operazioni vengono eseguite su \glock{Docker} in contenitori indipendenti e su nuove istanze appositamente generate;
			\item una \textit{action} può essere avviata sulla repository ospitante.
		\end{itemize}

		La struttura sequenziale delle operazioni è la seguente:

		\begin{itemize}
			\item configurazione delle operazioni da eseguire;
			\item esecuzione delle operazioni con riscontro dei codici di uscita;
			\item esito finale e notifica sui canali telematici agli attori interessati.
		\end{itemize}

		\subparagraph{Documentazione}

		La repository è stata configurata per garantire una buona accessibilità e correttezza nello sviluppo dei documenti di progetto da parte dei membri del team. A tal proposito, è stato messo in atto un processo di integrazione continua in ambiente di sviluppo nel momento in cui si vuole eseguire un incremento di versione di un documento. 
		Seguendo la configurazione del workflow adottata per il progetto, a partire da una modifica di un documento che viene eseguito su un branch \verb!feature/!, il documento deve essere sempre compilabile in \LaTeX{}. 
		Viene quindi eseguito un controllo preventivo prima di permettere l'integrazione finale delle modifiche tali da essere convalidate.

		\begin{enumerate}
			\item ad ogni \textit{pull request} da un branch \verb!feature! al \verb!develop!, viene eseguita una \glock{build} di controllo di tutti i documenti modificati;
			\item la \glock{build} crea un \glock{artefatto} con tutti i documenti PDF, che viene salvato nella repository e reso disponibile per essere visionato da remoto:
			\begin{itemize}
				\item \href{https://artifacts.redroundrobin.site}{artifacts.redroundrobin.site}
			\end{itemize}
			\item tramite gli appositi canali di comunicazione, vengono notificati tutti i membri del team delle nuove modifiche ai documenti.
		\end{enumerate}

		Nel caso di \textbf{errori} nella compilazione dei file, viene inviato un avviso all'ultima persona che ha eseguito il \glock{commit} e vengono mantenuti gli ultimi file correttamente compilati.

		\subparagraph{Componenti software}

		I singoli componenti fanno uso delle \glock{Github Actions} per l'attività di integrazione continua. In particolare, in base alla tipologia di componente, vengono create delle operazioni di controllo basate sui \glock{Test} per realizzare i requisiti del progetto.
	

	\paragraph{Notification \& monitoring}

	L'attività di \textit{notification \& monitoring} permette di notificare i membri del team al fine di avere sempre sotto controllo le modifiche effettuate in un componente del prodotto. Questo, ad esempio, permette di venire a conoscenza dei fallimenti o successi per l'attività di \textit{Continuous Integration}. In questo caso, notifiche più specifiche possono essere inviate solamente ai membri interessati che sono stati coinvolti nel processo di modifica della componente. 
	\newline
	Generalmente, si usano i seguenti canali telematici per le notifiche:
	\begin{itemize}
		\item pagina delle GitHub Actions della repository;
		\item email automatiche;
		\item canali Slack.
	\end{itemize}

	Il \textbf{monitoraggio} viene eseguito principalmente per mezzo di notifiche. Tuttavia, poiché il coordinamento risulta essere di essenziale importanza, si è deciso di integrare uno strumento che traccia in tempo reale gli ultimi avvenimenti all'interno di una repository da parte dei singoli membri del gruppo.
	Questo strumento, denominato \textit{Workflow Tracker}, permette di individuare l'ultimo branch in cui un utente ha lavorato, così da ridurre i conflitti di modifiche in modo preventivo e sapere le ultime attività all'interno della repository.
	\newline
	Lo strumento è pubblicamente visibile e reperibile al seguente indirizzo per tutti i membri del gruppo:

	\begin{itemize}
		\item \href{https://workflow.redroundrobin.site}{workflow.redroundrobin.site}
	\end{itemize}


\subsubsection{Strumenti}

	\paragraph{Repository}

	La repository è un luogo remoto in cui vengono mantenuti, salvati e versionati tutti i file che riguardano il progetto per tutto il ciclo di vita del prodotto. Questa risiede in modo condiviso all'interno di \glock{GitHub} ed è accessibile solamente ai membri del team.

		\subparagraph{Integrazione di VCS e ITS con GitHub}

		La repository fa uso di un \textit{Version Control System} (VCS) di tipo distribuito sotto il motore \textit{Git}, che permette la condivisione dal locale al remoto del proprio spazio di lavoro su un luogo comune. Attraverso l'utilizzo di un web browser, è possibile collegarsi a \textit{GitHub} e controllare i file contenuti nella repository, usando come indirizzo web:

		\href{http://project.redroundrobin.site}{http://project.redroundrobin.site}

		Inoltre, \textit{GitHub} integra un \textit{Issue Tracking System} (ITS) che permette di tracciare lo sviluppo tramite ticketing, assegnando a ciascun membro del team una o più \glock{issue} in base alle necessità.

		\subparagraph{Configurazione del workflow}

		Usando \textit{Git}, è possibile clonare e scaricare da remoto tutto il contenuto della repository per averne una copia in locale su cui poter lavorare, visionando la cronologia dei file modificati ad ogni \glock{commit} da parte di un membro del gruppo.
		\textit{Git}, inoltre, include la possibilità di creare dei \glock{branch} (locali e remoti) in cui poter sviluppare in maniera indipendente una funzionalità, che potrà essere integrata successivamente, senza bisogno di stare al passo con gli aggiornamenti della repository.

		Pertanto, si è deciso di stabilire il seguente canone di \glock{workflow} per quanto concerne la documentazione:
		\begin{itemize}
			\item \verb!master! branch: branch principale su cui vengono fatti i rilasci ad ogni \glock{milestone};
			\item \verb!develop! branch: branch di sviluppo su cui viene fatta integrazione di nuove funzionalità concluse;
			\item \verb!feature/xxxx! branch: branch indipendente usato da uno o più membri del gruppo per sviluppare una sezione o fare una revisione di un documento;
			\begin{itemize}
				\item \textbf{xxxx:} si riferisce al numero della \glock{issue} o al titolo della funzionalità da integrare.
				\item il \textit{feature branch} riferisce generalmente una singola \glock{issue} ed è rimosso alla sua chiusura.
			\end{itemize}
		\end{itemize}

		Ad ogni milestone viene associata una release interna alla repository, così da tenere traccia di una \glock{baseline}.

		\subparagraph{Formati dei file per la configurazione}

		In generale, vengono utilizzati alcuni file con formati speciali per la configurazione della repository. Questi file vanno modificati solamente dall'\glock{amministratore}, o su richiesta, anche da parte di altri membri del gruppo, in base alle necessità.
		\begin{itemize}
			\item \textbf{.gitignore} contiene tutte le regole per evitare di caricare nella repository dei formati non autorizzati (es: file eseguibili);
			\item \textbf{.yml} contiene la configurazione di una \glock{GitHub Action} per dirigere il \glock{Workflow};
			\item \textbf{file senza formati} contengono configurazioni aggiuntive per \glock{GitHub} (es: template delle \glock{Issue});
		\end{itemize}

		\subparagraph{Formati dei file per i documenti}

		Per la documentazione si usano principalmente i seguenti tipi di file:
		\begin{itemize}
			\item \textbf{file .tex:} contiene il codice sorgente \LaTeX{} del documento;
			\item \textbf{file .pdf:} è il documento compilato;
			\item \textbf{file .png:} identifica una immagine;
			\item \textbf{file .md:} identifica un file scritto in \glock{Markdown}, generalmente usato per gli appunti.
		\end{itemize}

		\subparagraph{Struttura della repository}

		%ndr: da aggiornare con i git submodules

		La repository si compone di diverse tipologie di cartelle. Oltre alle cartelle che riguardano la documentazione e il software, è presente una cartella che viene usata ai fini della configurazione di \glock{GitHub}.
		\begin{itemize}
			\item \verb!.github/!: cartella per la repository di \glock{Github} con i file di configurazione per le \glock{Github Actions} e l'\glock{Issue Tracking System}.
		\end{itemize}

		La documentazione nella repository si compone di 4 cartelle principali:
		\begin{itemize}
			\item \verb!interni/!: contiene tutta la documentazione interna;
			\item \verb!esterni/!: contiene tutta la documentazione esterna;
			\item \verb!template/!: contiene tutti i template delle varie tipologie di documento;
			\item \verb!notes/!: contiene tutte le note aggiuntive (\textit{non}-\LaTeX{}) sui seminari, le guide e l'organizzazione.
		\end{itemize}


	\paragraph{Tecnologie}

	Le tecnologie coinvolte per la configurazione del workflow del progetto sono essenzialmente di 2 tipologie:
	\begin{itemize}
		\item tecnologie per le comunicazioni e lo scheduling delle riunioni;
		\item tecnologie per la repository, con integrazione di soluzioni DevOps per lo sviluppo del software e dei documenti.
	\end{itemize}

		\subparagraph{Repository remota e locale}

		La repository viene gestita principalmente in un server remoto, accessibile tramite il portale di \href{https://github.com}{Github.com} da tutti i membri del team.

		Il processo di sviluppo, tuttavia, avviene in locale, eseguendo una clonazione della repository con tutti i branch attivi tramite uno dei seguenti programmi:

		\begin{itemize}
			\item \textbf{Github Desktop:} applicazione ufficiale di Github, molto semplice e leggera, disponibile per Linux, Mac OS e Windows;
			\item \textbf{GitKraken:} applicazione compatibile e più avanzata di Github desktop, disponibile per tutte le piattaforme;
			\item \textbf{Git CLI:} applicazione da terminale, disponibile per tutte le piattaforme con tutti i comandi utili per Git.
		\end{itemize}

		Per ciascun programma viene richiesta autenticazione, che può avvenire tramite semplice login (username e password) o tramite generazione di chiavi SSH, successivamente importate in maniera guidata su Github.

		\subparagraph{Automazione del Workflow}

		La repository, che fa uso di \glock{Github}, integra nativamente uno strumento denominato \glock{Github Actions} che permette di configurare le seguenti attività:
		\begin{itemize}
			\item \textbf{Continuous Integration:} integrazione continua del software con controllo sui \glock{Test};
			\item \textbf{Continuous Delivery:} consegna continua del software sfruttando gli incrementi minori per eseguire test più spesso;
			\item \textbf{Continuous Deployment:} distribuzione continua e messa in funzione del software;
			\item \textbf{notification \& monitoring:} invio di notifiche ai membri del team e monitoraggio delle attività di \glock{DevOps}.
		\end{itemize}

		Nel nostro caso, le \glock{Github Actions} permettono di fare in un solo file di configurazione tutti i passaggi in base al \glock{workflow} che ci interessa integrare. 
		\newline
		Delle operazioni descritte precedentemente, si fa uso solamente di:
		\begin{itemize}
			\item Countinuous Integration;
			\item notification \& monitoring.
		\end{itemize}
		I rimanenti processi verranno eventualmente impiegati per la manutenzione del prodotto.

		%ndr: Da inserire le tre operazioni come attività


% ndr: RIMOSSO A SEGUITO DI COLLOQUIO CON TULLIO
% ndr2: fix con uso delle cartelle e fix dei verbali, da approvare

% \subsubsection{Processi di DevOps} % Da verificarne la posizione e il contenuto

%	\paragraph{Countinuous Deployment dei documenti}
%	\label{sec:cd_docs}

%	La repository è stata configurata per garantire una buona accessibilità e correttezza nello sviluppo dei documenti di progetto da parte dei membri del team. A tal proposito, è stato messo in atto un processo di \glock{Continuous Deployment} in ambiente di sviluppo, per avere sempre disponibili gli ultimi documenti modificati.

%	\begin{enumerate}
%		\item Ad ogni \glock{commit} del branch \verb!develop!, viene fatta una \glock{build} di tutti i documenti modificati;
%		\item la \glock{build} crea un \glock{artefatto} con tutti i documenti PDF, che viene salvato online, oltre che nella repository, e reso disponibile per essere visionato da remoto:
%		\begin{itemize}
%			\item \href{https://artifacts.redroundrobin.site}{artifacts.redroundrobin.site}
%		\end{itemize}
%		\item Tramite gli appositi canali di comunicazione, vengono notificati tutti i membri del team delle nuove modifiche.
%	\end{enumerate}

%	Nel caso di \textbf{errori} nella compilazione dei file, viene inviato un avviso all'ultima persona che ha eseguito il \glock{commit} e vengono mantenuti gli ultimi file correttamente compilati.
