\subsection{Gestione della Configurazione}

	\subsubsection{Scopo}	

	La gestione della configurazione definisce i principi normativi utili a predisporre il \glock{workspace} per tutto il gruppo, semplificando e automatizzando la conservazione dei documenti e del software.

	\subsubsection{Repository}

	La repository è un luogo remoto in cui vengono mantenuti, salvati e versionati tutti i file che riguardano il progetto per tutto il ciclo di vita del prodotto. Questa risiede in modo condiviso all'interno di \glock{Github} ed è accessibile solamente ai membri del team.

		\paragraph{Integrazione di Git}

		La respository fa uso di un \textit{Version Control System} di tipo distribuito, che permette la condivisione dal locale al remoto del proprio spazio di lavoro. Attraverso l'utilizzo di software di 
			\begin{itemize}
				\item Utilizzo di git
				\item Tipi di file e .gitignore
			\end{itemize}

	\subsubsection{Versionamento e Rilascio}

	\paragraph{Documentazione}

	Tutti i file che riguardano la documentazione vengono conservati in una repository e vengono versionati per mezzo di un identificativo, in base alla loro fase di avanzamento. Questo permette di poter fare riferimento alle nuove versioni del documento durante tutto il ciclo di vita del software.\\

		\subparagraph{Codice di Versionamento}

		Ciascun documento possiede un identificativo di versionamento su ogni pagina che ha il seguente formalismo:

		\[%
			\text{v}[\alpha].[\beta].[\gamma]
		\]

		\begin{itemize}
			\item \((\alpha)\): numero identificativo del rilascio del documento;
			\begin{itemize}
				\item parte da 0 e non si resetta mai;
				\item viene incrementato solo dal responsabile a seguito di una approvazione.
			\end{itemize}
			\item \((\beta)\): numero identificativo che rappresenta un avanzamento sostanziale dopo le revisioni del documento;
			\begin{itemize}
				\item parte da 0 e si resetta solo a incrementi di \(\alpha\);
				\item viene incrementato solo dai revisori.
			\end{itemize}
			\item \((\gamma)\): numero rappresentativo di una aggiunta, modifica o eliminazione fatta al documento
			\begin{itemize}
				\item parte da 0 e si resetta solo a incrementi di \(\beta\) o di \(\alpha\);
				\item viene incrementato da un redattore o da un revisore in base alle necessità. 
			\end{itemize}
		\end{itemize}

		\subparagraph{Esempi Codice di Versionamento} 
		
		\begin{itemize}
			\item \verb!v0.2.1!
			\item \verb!v1.12.1!
			\item \verb!v3.0.2!
		\end{itemize}

		\subparagraph{Rilascio}

		Un documento viene rilasciato alle parti proponenti solamente quando vi è un incremento del primo numero (\(\alpha\)), che ne deduce una approvazione da parte del responsabile. \\ 
		Per quanto concerne la distribuzione interna, tutti gli \glock{artefatti} dei documenti realizzati durante la fase di sviluppo sono resi disponibili in modo rapido e automatizzato a tutti i membri del gruppo, a cui vengono notificate tutte le modifiche tramite il \glock{workspace} di \textit{Slack}. 

	\paragraph{Software}

	I sorgenti del software che riguardano la codifica e la configurazione del prodotto da realizzare sono mantenuti nella repository insieme alla documentazione. Ogni file viene versionato con un apposito storico delle modifiche, mentre l'intero software viene versionato in quanto \glock{baseline} sulla base delle funzionalità presenti e dei requisiti obbligatori implementati. 


		\subparagraph{Codice di Versionamento}

		Il versionamento del software viene eseguito sulla base delle implementazioni effettuate a livello di codifica. 
		In aggiunta, si definisce una sintassi di stato nella versione per definirne l'usabilità.

		\[%
			\text{v}[\delta].[\epsilon].[\mu]\text{-}[\lambda]
		\]

		\begin{itemize}
			\item \((\delta)\): numero identificativo del rilascio software per una \glock{major release};
			\begin{itemize}
				\item parte da 0 e non si resetta mai;
				\item viene incrementato solo dopo aver implementato tutti i requisiti obbligatori.
			\end{itemize}
			\item \((\epsilon)\): numero identificativo per una \glock{minor release};
			\begin{itemize}
				\item parte da 0 e si resetta solo a incrementi di \(\delta\);
				\item viene incrementato solo dopo l'implementazione di uno o più requisiti.
			\end{itemize}
			\item \((\mu)\): numero rappresentativo per una \glock{patch};
			\begin{itemize}
				\item parte da 0 e si resetta solo a incrementi di \(\epsilon\) o di \(\delta\);
				\item viene incrementato a ogni modifica di uno o più requisiti e ad ogni cambio di configurazione del software. 
			\end{itemize}
			\item \((\lambda)\): sigla che identifica uno stato del software; può avere i seguenti significati in ordine crescente di stato:
			\begin{itemize}
				\item \verb!dev!: derivante da \textbf{dev}elopment, versione ancora in sviluppo.
					\begin{itemize}
						\item software non completo, che ha ricevuto aggiunte, modifiche o eliminazioni recenti;
						\item può contenere errori che fanno fallire i test;
						\item \textbf{non} si presta all'uso di un utente finale.
					\end{itemize}
				\item \verb!rc!: \textbf{r}elease \textbf{c}andidate, versione candidata al rilascio;
				\begin{itemize}
					\item software che contiene tutti o una parte dei requisiti obbligatori richiesti;
					\item passa tutte le tipologie di test;
					\item si presta all'uso di un utente finale.
				\end{itemize}
				\item \verb!stable!: versione \textbf{stabile}, pronta al rilascio pubblico;
				\begin{itemize}
					\item software completo che implenta tutti i requisiti obbligatori;
					\item passa tutte le tipologie di test ed è validato;
					\item può essere collaudato con il proponente e/o pubblicato.
				\end{itemize}
			\end{itemize}
		\end{itemize}

		Una versione del software subisce un incremento di stato in base alla verifica da parte del meccanismo automatico di \glock{test}. In particolare:
		\begin{itemize}
			\item una versione può ricevere lo stato di \verb!rc! solo se il primo numero (\(\delta\)) o il secondo numero (\(\epsilon\)) è diverso da 0;
			\item una versione può ricevere lo stato di \verb!stable! solo se il primo numero (\(\delta\)) è diverso da 0;
		\end{itemize}
		
		\subparagraph{Esempi Codice di Versionamento} 
		
		\begin{itemize}
			\item \verb!v0.3.5-dev! : versione non completa, non usabile e forse funzionante.
			\item \verb!v0.7.0-rc! : versione non completa, usabile, funzionante e con alcuni requisiti implementati. 
			\item \verb!v1.0.0-rc! : versione forse completa, usabile, funzionante e in attesa di validazione.
			\item \verb!v1.0.0-stable! : versione completa, usabile, funzionante, verificata e validata.
		\end{itemize}

		\subparagraph{Rilascio}

		Le release del software vengono eseguite internamente nella \glock{repository} in base alle funzionalità sviluppate. Inoltre, i rilasci interni saranno normati come segue:
		\begin{itemize}
			\item le versioni \verb!stable! e/o \glock{Major Release} devono essere approvate dall'\glock{amministratore} e dal \glock{responsabile}.
			\item le versioni \verb!rc! e/o \glock{Minor Release} devono essere approvate, previa notifica di conferma di tutti i \glock{test}, dall'\glock{amministratore} e dai \glock{verificatori}.
			\item le versioni \verb!dev! e/o \glock{Patch} non richiedono approvazione e possono essere rilasciate autonomamente dal \glock{programmatore}.
		\end{itemize}


		\paragraph{Tecnologie}

		Le tecnologie coinvolte riguardano principalmente \glock{Github}

		\paragraph{Repository}
			
		\paragraph{Gestione delle modifiche}

	