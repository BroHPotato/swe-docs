\subsection{Gestione dei cambiamenti}

	\subsubsection{Scopo}
		L'obiettivo del processo di gestione dei cambiamenti è fornire un mezzo tempestivo, efficace e documentato per assicurarsi che tutti i problemi vengano analizzati, documentati, risolti e, in un'ottica di miglioramento continuo, evitati.

	\subsubsection{Aspettative}
		Avviando questo processo si intende ottenere:
		\begin{itemize}
		 	\item una qualità della documentazione e del codice più elevata;
		 	\item un modo efficiente ed efficace per trovare e correggere errori evitando che si propaghino;
		 	\item un tracciamento continuo degli errori più comuni e delle loro fonti, in modo tale da risolverli all'origine.
		 \end{itemize}
	\subsubsection{Descrizione}
		Il processo di gestione dei cambiamenti definisce una procedura da seguire per analizzare e rimuovere dei problemi, qualsiasi sia la loro origine, scoperti durante l'esecuzione del processo di sviluppo, di manutenzione o di altri processi.

	\subsubsection{Attività}
		\paragraph{Implementazione del processo}
			Il processo di gestione dei cambiamenti dovrà essere eseguito ogni volta che sarà necessario gestire un problema e dovrà soddisfare i seguenti requisiti:
			\begin{enumerate}
				\item il processo dovrà essere un ciclo chiuso, assicurandosi che:
					\begin{itemize}
				 		\item tutti i problemi rilevati dovranno essere prontamente riportati ed immessi nel processo di gestione dei cambiamenti;
					 	\item dovranno essere avvisate le parti interessate;
				 		\item le cause dovranno essere identificate, analizzate e nel limite del possibile rimosse;
					 \end{itemize}

				\item il processo utilizzerà uno schema per categorizzare e dare la giusta priorità ai problemi scovati;

					\begin{center}
						\rowcolors{2}{lightest-grayest}{white}
						\begin{longtable}{|c|c|c|c|}
							\hline
							\rowcolor{lighter-grayer}
							
							\textbf{Identificativo} & 
							\textbf{Processo} & 
							\textbf{Osservazione} & 
							\textbf{Soluzione}\\
							\hline
							\endfirsthead
							\hline

						\end{longtable}
					\end{center}

				 l'identificativo è composto da:
				 	\begin{center}
						\textbf{CMB-[tipologia][priorità]-[numero]}
					\end{center}

				 dove \textbf{CMB} identifica un cambiamento che è stato attuato;

				 la \textbf{tipologia} potrà avere i seguenti valori:
					\begin{itemize}
						\item \textbf{O:} per i problemi ortografici;
						\item \textbf{C:} per problemi legati al contenuto;
					\end{itemize}
				 la \textbf{priorità} avrà invece i seguenti valori:
					\begin{itemize}
					 	\item \textbf{A:} alta, richiede una tempestiva risoluzione;
					 	\item \textbf{M:} media, per problemi la cui risoluzione potrebbe aggravarsi nel tempo;
					 	\item \textbf{B:} bassa, se il problema ha una possibilità molto ridotta di aggravarsi o provocare altri problemi;
					 \end{itemize}
				 il \textbf{numero} alla fine sarà un numero progressivo che parte da 1;

				\item verranno effettuate delle analisi per verificare la presenza di tendenze nei problemi riportati;

				\item il procedimento di risoluzione dei problemi verrà valutato: si valuterà che i problemi siano stati effettivamente risolti, che le eventuali tendenze siano state annullate ed infine che non siano stati introdotti altri errori.
			\end{enumerate}
		\paragraph{Risoluzione del problema}
			Quando uno o più problemi saranno rilevati, nel prodotto software o in un'attività, dovrà essere preparato un report aprendo una issue nell'ITS (Issue tracking System) utilizzato per ogni problema individuato. Lo stesso report dovrà essere parte integrante del procedimento che è stato descritto nell'attività di istanziazione del processo.
	\subsubsection{Metriche}
		La gestione dei cambiamenti non fa uso di particolari metriche per misurare la qualità.
	\subsubsection{Strumenti}
		\paragraph{GitHub}
			Per creare report si è deciso di utilizzare l'ITS fornito da \glock{GitHub}, in quanto fornisce anche un sistema di versionamento ed è conosciuto da tutti i membri del gruppo.
			Questo ITS fornisce inoltre la possibilità di collegare i commit del cambiamento con la issue legata al report, in modo tale da poter tracciare tutto il percorso dalla modifica alla risoluzione del problema.
