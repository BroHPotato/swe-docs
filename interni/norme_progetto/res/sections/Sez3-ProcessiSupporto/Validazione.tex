\subsection{Validazione}

	\subsubsection{Scopo}
		Lo scopo del processo di validazione consiste nel garantire che il prodotto rispetti le richieste del committente, e quindi che esegua correttamente.
	\subsubsection{Aspettative}
		Per garantire che venga raggiunto lo scopo, le attività vengono testate dal processo di validazione, mandandole in esecuzione e confrontando l'output ottenuto con quello atteso.
	\subsubsection{Descrizione}
		Il processo di validazione esegue il test completo sul sistema affinché sia chiaro se sono state rispettate le necessità del committente, il che porta a definire se il prodotto esegue in modo corretto. Per poter effettuare questo processo è necessario che venga eseguito dopo il processo di verifica, in modo che tutte le unità del sistema permettano il test completo su di esso. 
	\subsubsection{Attività}
		\paragraph{Test}\mbox{}
			In questo processo viene eseguito il test di accettazione, effettuato insieme al committente per verificare se il prodotto rispetta le richieste.
			Per elencare le specifiche del test si è scelta una rappresentazione tabellare contenente il codice del componente da testare, la descrizione dei test, il suo stato di avanzamento e il risultato del test stesso secondo la seguente nomenclatura: \\
                Stato:
                \begin{itemize}
                    \item \textbf{I:} il test è stato implementato;
                    \item \textbf{NI:} il test non è ancora stato implementato.     
                \end{itemize}
                Risultato:
                \begin{itemize}
                    \item \textbf{S:} il test ha successo;
                    \item \textbf{F:} il test ha fallito.
                \end{itemize}
                Tuttavia si è scelto di omettere il risultato dei test in quanto non significativo allo stato attuale del prodotto.

        \subparagraph*{Test di Accettazione}
            	I test di accettazione (o collaudo) accertano il soddisfacimento degli use case e dei requisiti concordati con il cliente.
                Più in particolare i test di accettazione servono a confermare che i requisiti derivati dai casi d'uso specificati nel capitolato siano stati soddisfatti. Questi test richiedono perciò la presenza del committente e del proponente.  
                Per classificare questi tipi di test verrà associata un codice ad ognuno di essi secondo il seguente modello:
                
                \begin{center}
                	\textbf{TA[Priorità]-[Tipologia]-[Identificativo]}
                \end{center}
                Dove: 
                
                \begin{itemize}
	                \item \textbf{Priorità:} indica la priorità del requisito associato al test e potrà avere i seguenti valori:
		                \begin{itemize}
		                    \item \textbf{A:} obbligatorio, strettamente necessario;
		                    \item \textbf{B:} desiderabile, non strettamente necessario;
		                    \item \textbf{C:} relativamente utile o contrattabile in corso d'opera. 
		                \end{itemize} 
	                \item \textbf{Tipologia:} indica la tipologia del requisito associato al test e potrà avere i seguenti valori:
		                \begin{itemize}
		                    \item \textbf{F:} funzionale;
		                    \item \textbf{P:} prestazionale;
		                    \item \textbf{Q:} qualitativo;
		                    \item \textbf{V:} vincolo.
		                \end{itemize}
	                \item \textbf{Identificativo:} numero progressivo il cui obiettivo sarà di contraddistinguere il singolo componente da testare; parte da 1.
                \end{itemize}

    \subsubsection{Metriche}

    	\paragraph{QP-3 Validazione}

			\subparagraph{Scopo}
				Si vuole monitorare l'avanzamento dello sviluppo dei requisiti illustrati nel documento di \dext{Analisi dei Requisiti v1.0.0}. Questo può essere utile al cliente, per comprendere la percentuale di completamento del progetto nel corso del tempo.

			\subparagraph{Introduzione alle Metriche}

			Per la validazione si farà uso delle seguenti metriche:

			\begin{itemize}
				\item QM-PROC-7. Satisfied Mandatory Requirements (SMR);
				\item QM-PROC-8. Satisfied Desirable Requirements (SDR);
				\item QM-PROC-9. Satisfied Optional Requirements (SOR).
			\end{itemize}

			\subparagraph{QM-PROC-7. Satisfied Mandatory Requirements (SMR)}

			\begin{itemize}
			 
				\item \textbf{descrizione: }
				La metrica SMR indica il quantitativo di requisiti obbligatori soddisfatti (progettati, sviluppati, verificati e validati) fino alla data corrente. Questa metrica permette sia al gruppo, che al cliente, di comprendere la percentuale di completamento del progetto;

				\item \textbf{unità di misura: }
				La metrica viene espressa in percentuale;

				\item \textbf{formula: }
				La formula della metrica è la seguente:
				\[
					\text{SMR} = \frac{\text{requisiti obbligatori soddisfatti}}{\text{requisiti obbligatori totali}} \times 100
				\]

				\item \textbf{risultato: }
				\begin{itemize}
					\item Un risultato pari a 0\% indica indica che non è stato soddisfatto ancora alcun requisito obbligatorio;
					\item Un risultato pari a 100\% indica che sono stati soddisfatti tutti i requisiti obbligatori.
				\end{itemize}

			\end{itemize}

			\subparagraph{QM-PROC-8. Satisfied Desirable Requirements (SDR)}

			\begin{itemize}
      			\item \textbf{descrizione: }
				La metrica SDR indica il quantitativo di requisiti desiderabili soddisfatti (progettati, sviluppati, verificati e validati) fino alla data corrente;

				\item \textbf{unità di misura: }
				La metrica viene espressa in percentuale;

				\item \textbf{formula: }
				La formula della metrica è la seguente:
				\[
					\text{SDR} = \frac{\text{requisiti desiderabili soddisfatti}}{\text{requisiti desiderabili totali}} \times 100
				\]

				\item \textbf{risultato: }
				\begin{itemize}
					\item Un risultato pari a 0\% indica indica che non è stato soddisfatto ancora alcun requisito desiderabile;
					\item Un risultato pari a 100\% indica che sono stati soddisfatti tutti i requisiti desiderabili.
				\end{itemize}
			\end{itemize}

			\subparagraph{QM-PROC-9. Satisfied Optional Requirements (SOR)}

			\begin{itemize}
      			\item \textbf{descrizione: }
				La metrica SDR indica il quantitativo di requisiti desiderabili soddisfatti (progettati, sviluppati, verificati e validati) fino alla data corrente;

				\item \textbf{unità di misura: }
				La metrica viene espressa in percentuale;

				\item \textbf{formula: }
				La formula della metrica è la seguente:
				\[
					\text{SOR} = \frac{\text{requisiti opzionali soddisfatti}}{\text{requisiti opzionali totali}} \times 100
				\]

				\item \textbf{risultato: }
				\begin{itemize}
					\item Un risultato pari a 0\% indica indica che non è stato soddisfatto ancora alcun requisito opzionale;
					\item Un risultato pari a 100\% indica che sono stati soddisfatti tutti i requisiti opzionali.
				\end{itemize}
			\end{itemize}
                
    \subsubsection{Strumenti}
        Non sono stati individuati degli strumenti per il processo di validazione.