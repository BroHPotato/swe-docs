\subsection{Validazione}

	\subsubsection{Scopo}
		Lo scopo del processo di validazione consiste nel garantire che il prodotto rispetta le richieste del committente, e quindi che esegue correttamente.
	\subsubsection{Aspettative}
		Per garantire che venga raggiunto lo scopo del processo di validazione, si eseguono le attività con dei test documentati e l'output generato corrisponde a quello aspettato.
	\subsubsection{Descrizione}
		Il processo di validazione esegue il test completo sul sistema affinchè sia chiaro se sono state rispettate le necessità del committente, il che porta a definire se il prodotto esegue in modo corretto. Per poter effettuare questo processo è necessario che venga eseguito dopo il processo di verifica, in modo che tutte le unità del sistema permettano il test completo su di esso. 
	\subsubsection{Attività}
		\paragraph{Test}\mbox{}\\
			I test eseguiti in questo processo riguardano il test sul sistema, eseguito internamente, e il test di accettazione, eseguito insieme al committente.
			\subparagraph*{Test di sistema}
				Dopo aver eseguito i test su tutte le unità e sulla loro integrazione, si testa il sistema nella sua interezza. Viene testato se le interazioni tra le varie componenti del sistema ritornano il risultato atteso o meno in concordanza con ciò che è stato definito nel processo di analisi dei requisiti.
			\subparagraph*{Test di accettazione}   
				Anche detto "test di collaudo" è il test eseguito su input definiti dal committente in modo da verificare se l'output atteso da esso è corretto o meno.