\subsection{Validazione}

	\subsubsection{Scopo}
		Lo scopo del processo di validazione consiste nel garantire che il prodotto rispetta le richieste del committente, e quindi che esegue correttamente.
	\subsubsection{Aspettative}
		Per garantire che venga raggiunto lo scopo del processo di validazione, si eseguono le attività con dei test documentati e l'output generato corrisponde a quello aspettato.
	\subsubsection{Descrizione}
		Il processo di validazione esegue il test completo sul sistema affinché sia chiaro se sono state rispettate le necessità del committente, il che porta a definire se il prodotto esegue in modo corretto. Per poter effettuare questo processo è necessario che venga eseguito dopo il processo di verifica, in modo che tutte le unità del sistema permettano il test completo su di esso. 
	\subsubsection{Attività}
		\paragraph{Test}\mbox{}\\
			I test eseguiti in questo processo riguardano il test sul sistema, eseguito internamente, e il test di accettazione, eseguito insieme al committente.
			Per elencare le specifiche dei test si è scelta una rappresentazione tabellare contenente il codice del componente da testare, la descrizione dei test, il suo stato di avanzamento e il risultato del test stesso secondo la seguente nomenclatura: \\
                Stato:
                \begin{itemize}
                    \item \textbf{I}: se il test è stato implementato;
                    \item \textbf{NI}: se il test non è ancora stato implementato.     
                \end{itemize}
                Risultato:
                \begin{itemize}
                    \item \textbf{S}: se il test ha successo;
                    \item \textbf{F}: se il test ha fallito.
                \end{itemize}
                Tuttavia si è scelto di omettere il risultato dei test in quanto non significativo allo stato attuale del prodotto.

        \subparagraph*{Test di Accettazione}
            	I test di accettazione (o collaudo) accertano il soddisfacimento degli use case e dei requisiti concordati con il cliente.
                Più in particolare test di accettazione servono a confermare che i requisiti derivati dai casi d'uso specificati nel capitolato siano stati soddisfatti. Questi test richiedono perciò la presenza del committente e del proponente.  
                Per classificare questi tipi di test verrà associata un codice ad ognuno di essi secondo il seguente modello:
                
                \begin{center}
                \textbf{TA[Priorità]-[Tipologia]-[Identificativo]}
                \end{center}
                dove: 
                
                \textbf{Priorità}: indica la priorità del requisito associato al test e potrà avere i seguenti valori:
                \begin{itemize}
                    \item \textbf{A}: Obbligatorio, strettamente necessario;
                    \item \textbf{B}: Desiderabile, non strettamente necessario;
                    \item \textbf{C}: Relativamente utile o contrattabile in corso d'opera. 
                 \end{itemize} 
                 \textbf{Tipologia}: indica la tipologia del requisito associato al test e potrà avere i seguenti valori:
                 \begin{itemize}
                    \item \textbf{F}: funzionale;
                    \item \textbf{P}: prestazionale;
                    \item \textbf{Q}: qualitativo;
                    \item \textbf{V}: vincolo.
                 \end{itemize}
                \textbf{Identificativo}: numero progressivo il cui obiettivo sarà di contraddistinguere il singolo componente da testare.
