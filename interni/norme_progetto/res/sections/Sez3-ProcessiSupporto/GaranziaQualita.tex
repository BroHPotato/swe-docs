\subsection{Garanzia della qualità}

	\subsubsection{Scopo}

		Si occupa di stabilire una metrica precisa per tutti i servizi nell'ambito della verifica e della validazione, mantenendo un dato livello di qualità che rimanga uniforme e misurabile durante tutto il ciclo di vita del software.

	\subsubsection{Aspettative}

		Il sistema di qualità deve fornire delle metriche di giudizio uniformi volte a quantificare con chiarezza la correttezza dei documenti e del software. Ciò va unito anche all'affidabilità nello svolgimento dei processi di verifica, che vanno monitorati e guidati nell'intera procedura, senza lasciare a interpretazioni. Pertanto, ci si aspetta:
		\begin{itemize}
			\item un prodotto software di qualità;
			\item una documentazione completa e facilmente comprensibile per tutti;
			\item dei processi che seguono dei punti ben specificati per l'analisi di qualità;
			\item una comunicazione chiara e semplice delle problematiche relative alla qualità tra i membri del team;
			\item una registrazione dei risultati ottenuti.
		\end{itemize}

	\subsubsection{Descrizione}

		La garanzia della qualità si compone di diversi controlli che devono essere effettuati per:
		\begin{itemize}
			\item il software;
			\item la documentazione;
			\item tutti i processi che portano alla realizzazione della documentazione e del software.
		\end{itemize}

		Per ogni processo mirato alla qualità si definiscono delle metriche che vengono riportate in ciascuna sezione del presente documento.
		Le registrazioni dei risultati ottenuti dall'analisi della qualità sono salvate con degli appositi report.

		\paragraph{Obiettivi di qualità di prodotto}

			La qualità del prodotto viene garantita attraverso l'attuazione dei processi di verifica e validazione basati su fondamenti normativi. In particolare, definiamo quanto segue:
			\begin{itemize}
				\item \textbf{verifica:} processo di controllo che garantisce qualità dei processi di fornitura del prodotto;
				\item \textbf{validazione:} processo di controllo del prodotto volto a confermare le aspettative, i requisiti e le funzionalità concordate.
			\end{itemize}

			L'insieme di questi processi deve portare a un miglioramento continuo del prodotto, che viene sottoposto agli standard di qualità riportati nel \textit{way of working}.

		\paragraph{Obiettivi qualità di processo}

			La qualità di processo deve essere perseguita nel corso del ciclo di vita del software attraverso i principi di efficacia ed efficienza mirati al prodotto.
			Nello specifico definiamo quanto segue:
			\begin{itemize}
				\item \textbf{efficacia:} si richiede un prodotto che soddisfi le richieste del proponente;
				\item \textbf{efficienza:} i processi devono convergere con costi ridotti in termini di risorse a pari qualità di prodotto.
			\end{itemize}

			Ciascun processo va migliorato durante la sua esecuzione facendo uso di monitoraggi mirati che permettano di acquisire, attraverso l'esperienza, una risposta critica alla qualità stessa del processo.

	\subsubsection{Attività}

		\paragraph{Classificazione dei processi}

		Nell'ambito di qualità, i processi vengono tracciati con il seguente identificativo:

		\[
				\text{QP}-[\lambda]
		\]

		Dove:

		\begin{itemize}
			\item QP: indica letteralmente \textit{Quality Process};
			\item \(\lambda\): numero intero che indica il processo e parte da 1.
		\end{itemize}

		\paragraph{Classificazione delle caratteristiche di prodotto}

		Nell'ambito di qualità, le caratteristiche del prodotto vengono tracciate con il seguente identificativo:

		\[
				\text{QC}-[\lambda]
		\]

		Dove:

		\begin{itemize}
			\item QC: indica letteralmente \textit{Quality Characteristic};
			\item \(\lambda\): numero intero che indica la caratteristica di prodotto e parte da 1.
		\end{itemize}

		\paragraph{Classificazione delle metriche}

		Le metriche sono i criteri che vengono utilizzati per misurare nei processi e nel prodotto i gradi di qualità raggiunti. A ciascuna metrica si associa il seguente identificatore:

		\[
				\text{QM}-[\delta]-[\lambda]
		\]

		Dove:

		\begin{itemize}
			\item QM: indica letteralmente \textit{Quality Metric};
			\item \(\delta\): riguarda la tipologia della metrica e può assumere i valori riportati di seguito:
				\begin{itemize}
					\item \textbf{PROC}: indica che la metrica si associa per un processo;
					\item \textbf{PROD}: indica che la metrica si associa per il prodotto;
					\item \textbf{TEST}: indica che la metrica si associa per i test;
				\end{itemize}
			\item \(\lambda\): numero intero che indica la metrica e parte da 1.
		\end{itemize}

	\subsubsection{Metriche}

		Per ciascuna attività, sia che riguardi la documentazione, il software o il monitoraggio di processo, si riporta nella relativa sezione una classificazione delle metriche di qualità.

	\subsubsection{Strumenti}

	Non sono stati identificati degli strumenti particolari per la garanzia di qualità.

	%ndr: da modificare



% ----------------------------------------------------------------------



		
