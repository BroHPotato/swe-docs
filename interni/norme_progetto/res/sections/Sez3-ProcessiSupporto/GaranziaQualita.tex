\subsection{Garanzia della Qualità}

	\subsubsection{Scopo}

	Si occupa di stabilire una metrica precisa per tutti i servizi nell'ambito della verifica e della validazione, mantenendo un certo grado di qualità che rimanga uniforme e misurabile durante tutto il ciclo di vita del software. 

	\subsubsection{Aspettative}

	Il sistema qualità deve fornire delle metriche di giudizio uniformi volte a quantificare in maniera comprensibile la correttezza dei documenti e del software. Ciò va unito anche all'affidabilità nello svolgimento dei processi di verifica, che vanno monitorati e guidati nell'intera procedura, senza lasciare a interpretazioni. Pertanto, ci si aspetta:
	\begin{itemize}
		\item un prodotto software di qualità;
		\item una documentazione completa e facilmente comprensibile per tutti;
		\item dei processi che seguono dei punti ben specificati per l'analisi di qualità;
		\item una comunicazione chiara e semplice delle problematiche relative alla qualità tra i membri del team;
		\item una registrazione dei risultati ottenuti.
	\end{itemize}

	\subsubsection{Descrizione}

	La garanzia della qualità si compone di diversi controlli che devono essere effettuati per:
	\begin{itemize}
		\item il software;
		\item la documentazione;
		\item tutti i processi che portano alla realizzazione della documentazione e del software.
	\end{itemize}

	Per ogni processo mirato alla qualità si definiscono delle metriche che vengono riportate in ciascuna sezione del presente documento. 
	La registrazione dei risultati ottenuti dall'analisi della qualità sono salvati con degli appositi report.

	\subsubsection{Controllo qualità prodotto}

	La qualità del prodotto viene garantita attraverso l'attuazione dei processi di verifica e validazione basati su fondamenti normativi. In particolare, definiamo quanto segue:
	\begin{itemize}
		\item \textbf{Verifica:} processo di analisi continua che garantisce qualità dei processi di fornitura del prodotto;
		\item \textbf{Validazione:} processo di controllo del prodotto volto a confermare le aspettative, i requisiti e le funzionalità concordate.
	\end{itemize}

	L'insieme di questi processi deve portare a un miglioramento continuo del prodotto, che viene sottoposto agli standard di qualità riportati \glock{way of working}. 

	\subsubsection{Controllo qualità di processo}

	La qualità di processo deve essere perseguita nel corso del ciclo di vita del software attraverso i principi di efficacia ed efficienza mirati al prodotto. 
	Nello specifico definiamo quanto segue:
	\begin{itemize}
		\item \textbf{Efficacia:} si richiede un prodotto valido in relazione alle aspettative;
		\item \textbf{Efficienza:} i processi devono convergere con costi ridotti in termini di risorse a pari qualità di prodotto.
	\end{itemize}

	Ciascun processo va migliorato durante la sua esecuzione facendo uso di monitoraggi mirati che permettano di acquisire, attraverso l'esperienza, una risposta critica alla qualità stessa del processo. \\ 

	
	\paragraph{Standard utilizzati}

	% Da aggiungere - modificare questa parte

	Il miglioramento continuo, inoltre, viene standardizzato da PDCA, SPY e SPICE.  % Quale usare?


	\subsubsection{Classificazioni metriche} % Da mettere nelle singole sezioni delle attività
	
	\subsubsection{Strumenti} % Ce ne sono?
	
