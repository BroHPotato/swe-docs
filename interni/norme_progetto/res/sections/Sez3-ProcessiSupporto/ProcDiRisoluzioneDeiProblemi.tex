\subsection{Processo di risoluzione dei problemi}
	
	Il processo di risoluzione dei problemi definisce una procedura da seguire per analizzare e rimuovere dei problemi, qualsiasi sia la loro origine, scoperti durante l'esecuzione del processo di sviluppo, di manutenzione o di altri processi.
	
	\subsubsection{Scopo}
		L'obiettivo del processo di risoluzione dei problemi è quello di fornire un mezzo tempestivo, efficace e documentato per assicurarsi che tutti i problemi vengano analizzati, documentati, risolti e, in un'ottica di miglioramento continuo evitati.	
	
	\subsubsection{Aspettative}
		Instanziando questo processo si intende ottenere:
		\begin{itemize}
		 	\item Una qualità della documentazione e del codice più elevata;
		 	\item Un modo efficiente ed efficace per trovare e correggere errori evitando che si propaghino;
		 	\item Un tracciamento continuo degli errori più comuni e delle loro fonti, in modo tale da risolverli all'origine.
		 \end{itemize} 
	\subsubsection{Attività}
		\paragraph{Implementazione del processo}
			Il processo di risoluzione dei problemi dovrà essere instanziato ogni volta che sarà necessario gestire un problema e dovrà soddisfare i seguenti requisiti:
			\begin{enumerate}
				\item Il processo dovrà essere un ciclo chiuso, assicurandosi che:
					\begin{itemize}
				 		\item tutti i problemi rilevati dovranno essere prontamente riportati ed immessi nel Processo di risoluzione dei problemi;
					 	\item dovranno essere avvisate le parti interessate;
				 		\item le cause dovranno essere identificate, analizzate e nel limite del possibile rimosse.
					 \end{itemize}  
				
				\item Il processo utilizzerà uno schema per categorizzare e dare la giusta priorità ai problemi scovati.

					\begin{center}
						\rowcolors{2}{lightest-grayest}{white}
						\begin{longtable}{|c|c|c|c|}
							\hline
							\rowcolor{lighter-grayer}
							\textbf{Processo} & \textbf{Priorità} & \textbf{Tipologia} & \textbf{Identificativo}\\
							\hline
							\endfirsthead
							\hline
			
						\end{longtable}
					\end{center}

				 La priorità potrà avere i seguenti valori:
					\begin{itemize}
					 	\item \textbf{A}, richiedendo quindi una tempestiva risoluzione;
					 	\item \textbf{B}, per problemi la cui risoluzione poptrebbe aggravarsi nel tempo;
					 	\item \textbf{C}, se il problema ha una possibilità molto ridotta di aggravarsi o provocare altri problemi. 
					 \end{itemize} 
				 La tipologia avrà invece i seguenti valori:
					\begin{itemize}
						\item \textbf{O}, per i problemi ortografici
						\item \textbf{C}, per problemi legati al contenuto
					\end{itemize}
				 L'identificativo sarà un numero progressivo a partire da 1.	
					
				\item Verranno effettuate delle analisi per verificare la presenza di trend nei problemi riportati.

				\item Il procedimento di risoluzione dei problemi verra valutato: si valuterà che i problemi siano stati effettivamente risolti, che gli eventuali trend siano stati annullati ed infine che non siano stati introdotti altri errori.
			\end{enumerate}	
		\paragraph{Risoluzione del problema}
			Quando uno o più problemi saranno scovati, nel prodotto software o in un'attività, dovràessere preparato un report per ogni problema individuato. Lo stesso report dovrà essere parte integrante del procedimento che è stato descritto nell'attività di instanziazione del processo.
