\section{Processi di Supporto}

	\subsection{Documentazione}

		\subsubsection{Scopo}
			Lo scopo principale di questo capitolo è fornire una guida esaustiva di tutti gli standard e regole per quanto riguarda la stesura ed approvazione dei documenti.
		\subsubsection{Aspettative}
			Si vuole fornire uno strumento per la stesura dei documenti unico per tutto il gruppo in modo da avere una documentazione uniforme e aderente agli standard e regole sotto riportate.
		\subsubsection{Descrizione}
			Questo capitolo fornisce i dettagli su come deve essere redatta e verificata la documentazione. Tutte le norme descritte devono essere rispettate in pieno da tutti i documenti, sia interni che esterni, rilasciati durante il ciclo di vita del software.
		\subsubsection{Ciclo di vita}
			Il ciclo di vita dei documenti è suddiviso in varie attività, eventualmente ripetibili:
			\begin{itemize}
				\item \textbf{Stesura:} è il processo di scrittura del documento in sé; questa attività viene assegnata ad un redattore che, terminato il suo svolgimento, farà riferimento al responsabile, il quale autorizzerà l'avanzamento del documento all'attività successiva;
				\item \textbf{Verifica:} è l'attività eseguita dai verificatori, i quali hanno il compito di controllare che la stesura del documento sia avvenuta in modo corretto, sintatticamente e semanticamente, seguendo le norme di progetto. Ogni problema viene riferito al responsabile che provvederà a notificare il redattore e riporterà il documento all'attività di stesura. Quando questa attività sarà portata a termine con successo il responsabile farà avanzare il documento nell'ultima attività del ciclo di vita;
				\item \textbf{Approvazione:} è l'ultima attività del ciclo di vita del documento, in cui il verificatore ha terminato il suo compito ed ha comunicato l'esito positivo al responsabile. Il responsabile procederà a confermare il documento ed eseguire il rilascio.
			\end{itemize}
		\subsubsection{Template LaTeX}
			Si è deciso di utilizzare una struttura template \LaTeX{} per facilitare il versionamento e la stesura dei documenti. Inoltre, l'utilizzo di tale struttura fornisce uniformità al layout di tutti i documenti.
		\subsubsection{Struttura dei documenti}
			Un file ``main.tex'' provvederà a raccogliere tutte le sezioni, pacchetti e comandi necessari per la sua compilazione. Tutti i documenti hanno una struttura predefinita e determinata.
			\paragraph{Frontespizio}
				Il frontespizio ha la funzione di fornire i dati principali del documento. Esso presenterà il logo e il nome del gruppo, il titolo del documento e la sua appartenenza ad un determinato progetto, le informazioni sul documento quali:
				\begin{itemize}
					\item \textbf{versione:} versione attuale del documento;
					\item \textbf{uso:} destinazione d'uso del documento, che potrà essere ``interno'' o ``esterno'';
					\item \textbf{stato:} attuale stato del documento, che potrà essere ``in redazione'' o ``approvato'';
					\item \textbf{destinatari:} destinatari del documento;
					\item \textbf{redattori:} lista dei membri del gruppo che si sono occupati della stesura dello specifico documento;
					\item \textbf{verificatori:} lista dei membri del gruppo che si sono occupati della verifica dello specifico documento;
					\item \textbf{approvazione:} nominativo del membro del gruppo che ha approvato il documento per il rilascio.
				\end{itemize}
				Tutti gli elementi di questa pagina sono centrati ed incolonnati.
			\paragraph{Registro modifiche}
			\label{par: Registro modifiche}
				Il registro delle modifiche occupa la seconda pagina del documento e consiste in una tabella contenente le informazioni riguardanti il ciclo di vita del documento. Più precisamente, la tabella riporta per ogni modifica:
				\begin{itemize}
					\item \textbf{versione:} versione del documento relativa alla modifica effettuata;
					\item \textbf{descrizione:} breve descrizione della modifica effettuata;
					\item \textbf{data:} data in cui la modifica è stata effettuata;
					\item \textbf{autore:} nominativo della persona che ha effettuato la modifica;
					\item \textbf{ruolo:} ruolo della persona che ha effettuato la modifica;
				\end{itemize}
			\paragraph{Indice}
				L'indice ha lo scopo di riepilogare e dare una visione generale della struttura del documento, mostrando le parti di cui è composto. L'indice ha una struttura standard: numero e titolo del capitolo, con eventuali sottosezioni, e il numero della pagina del contenuto; inoltre, ogni titolo è un link alla pagina del contenuto. L'indice dei contenuti è seguito da un eventuale indice per le tabelle e le figure presenti nel documento.
			\paragraph{Contenuto principale}
				La struttura del contenuto principale di una pagina è strutturato con:
					\begin{itemize}
						\item in alto a sinistra è presente il logo del gruppo;
						\item in alto a destra è riportata la sezione alla quale la pagina appartiene;
						\item il contenuto principale è posto tra l'intestazione e il piè di pagina;
						\item una riga divide il contenuto principale e il piè di pagina;
						\item in basso a sinistra è presente il nome del documenti con relativa versione;
						\item in basso a destra riporta il numero della pagina attuale ed il numero totale delle pagine che compongono il documento; se la pagina non fa parte del contenuto principale la sua numerazione avviene tramite i numeri romani.
					\end{itemize}
		\subsubsection{Classificazione dei documenti}
			\paragraph{Documenti ufficiosi}
				I documenti ufficiosi sono utilizzati all'interno dell'ambiente di lavoro ed sono divisi in due categorie:
				\begin{itemize}
					\item \textbf{informativi:} hanno il solo scopo di passare informazioni meno rilevanti tra i membri del gruppo (es. appunti, richieste, riflessioni);
					\item \textbf{proto-ufficiali:} sono tutti i documenti che in futuro diventeranno ``ufficiali'' ma sono in attesa di revisione e verifica.
				\end{itemize}
			\paragraph{Documenti ufficiali}
				Si dicono ufficiali i documenti che:
				\begin{itemize}
					\item sono stati revisionati, verificati ed approvati dal responsabile di progetto;
					\item sono gli unici rilasciabili all'esterno del gruppo di progetto.
				\end{itemize}
			\paragraph{Verbali}
				I verbali hanno lo scopo di riassumere, in modo coinciso e preciso, tutti gli argomenti che sono stati discussi in una riunione, sia interna che esterna. È prevista un'unica stesura del verbale per ogni riunione in quanto una modifica ad una decisione avrebbe un effetto retroattivo. I verbali seguono la stessa struttura di tutti gli altri documenti ad eccezione fatta della numerazione delle pagine, che usa la notazione romana anziché araba. Inoltre, il verbale è suddiviso in:
				\begin{itemize}
					\item \textbf{introduzione:} essa contiene:
						\begin{itemize}
							\item \textbf{luogo:} luogo o la piattaforma online, in caso di incontri virtuali;
							\item \textbf{data:} data dell'incontro;
							\item \textbf{ora di inizio:} l'ora dell'inizio dell'incontro in formato HH:MM;
							\item \textbf{ora di fine:} l'ora della fine dell'incontro in formato HH:MM;
							\item \textbf{ordine del giorno:} consiste in una lista degli argomenti che il gruppo si è proposto di discutere durante l'incontro;
							\item \textbf{presenze:} contiene il numero totale dei partecipanti, la lista dei presenti e la lista degli assenti con eventuale giustifica;
							\item \textbf{segretario: } scelto tra i componenti del gruppo e incaricato di prendere nota delle discussioni effettuate per poi redigere il verbale della riunione.
						\end{itemize}
					\item \textbf{svolgimento:} per ogni punto presente nell'ordine del giorno, viene riportato un riassunto di ciò che è stato trattato durante l'incontro;
					\item \textbf{tracciamento delle decisioni:} è un riepilogo in formato tabellare delle decisioni prese durante l'incontro; esso è composto da:
						\begin{itemize}
							\item \textbf{codice:} del tipo ``VT\_AAAA-MM-GG\_X.Y'' dove la prima lettera indica che il documento è un verbale(V), la seconda indica la sua tipologia, esterno(E) o interno(I), seguito dalla data dell'incontro a cui fa riferimento e terminato, infine, da un numero che indica il numero del verbale(X) ed un secondo numero che indica il punto all'ordine del giorno a cui si riferisce(Y);
							\item \textbf{descrizione:} breve descrizione riassuntiva della decisione presa riguardante il punto dell'ordine del giorno.
					\end{itemize}
				\end{itemize}
				Ogni verbale dovrà essere denominato seguendo il formato ``Verbale riunione \#X'', dove la ``X'' corrisponde al numero del verbale in ordine temporale.
			\paragraph{Glossario}
				Il glossario ha la funzione di disambiguare alcune parole all'interno di determinati contesti. Al suo interno saranno presenti tutte parole con le seguenti caratteristiche:
				\begin{itemize}
					\item sono presenti in almeno un documento;
					\item trattano argomenti di natura tecnica;
					\item trattano argomenti ambigui e/o poco conosciuti;
					\item rappresentano delle sigle e/o degli acronimi non di uso comune.
				\end{itemize}
				Inoltre, il glossario è strutturato in maniera precisa seguendo due regole:
				\begin{itemize}
					\item i termini seguono l'ordine lessicografico;
					\item ogni termine è spiegato in maniera chiara e in nessun modo ambigua.
				\end{itemize}
				La stesura del glossario deve avvenire in parallelo alla stesura dei documenti al fine di evitare confusione tra i termini. Inoltre, ogni parola dei documenti riportata nel glossario, deve essere caratterizzata dallo stile ``maiuscoletto'' con il pedice ``G''. Non si richiede tale stile se la parola presente nel documento è stata precedentemente caratterizzata con la suddetta notazione.
			\paragraph{Lettere}
				La lettera di presentazione dovrà seguire il classico layout per lettere, il che implica la presenza dei mittenti e destinatari, il logo del gruppo e la lista di tutti i documenti rilasciati, nonché il preventivo per il progetto.
		\subsubsection{Norme tipografiche}
			\paragraph{Convenzioni sui nomi dei file}
				Si è deciso di usare la convenzione ``camel case'' per i nomi di file e cartelle. Le regole seguite saranno le seguenti:
				\begin{itemize}
					\item il nome dei file composti da più parole avranno la prima lettera minuscola ed ogni parola in seguito inizierà con una maiuscola;
					\item tra le parole non sarà presente alcun separatore;
					\item le preposizioni non verranno omesse;
					\item sono escluse da questa sintassi le estensione dei file.
				\end{itemize}
				Alcuni esempi corretti sono:
				\begin{itemize}
					\item convenzioniSuiNomiDeiFile.tex;
					\item documentazione.tex
				\end{itemize}
				Alcuni esempi non corretti sono:
				\begin{itemize}
					\item ConvenzioniSuiNomiDeiFile (la prima lettera è maiuscola);
					\item convenzioni Sui Nomi Dei File (usa un carattere separatore);
					\item convenzioni\_Sui\_Nomi\_Dei\_File (usa un carattere separatore);
					\item convenzioniNomiFile (omette preposizioni).
				\end{itemize}
			\paragraph{Stile del testo}
				I vari stili del testo hanno una specifica funzione semantica.
				\begin{itemize}
					\item \textbf{corsivo:} viene utilizzato per denotare termini tecnici appartenenti ad una particolare tecnologia, esempio \textit{branch};
					\item \textbf{grassetto:} viene utilizzato per evidenziare le parole con la definizione della stessa in seguito, per esempio in un elenco puntato, queste includeranno i due punti in grassetto, per esempio ``\textbf{def.:} abbreviazione per la parola definizione''; oppure per denotare le sezioni e sotto-sezioni dei documenti;
					\item \textbf{maiuscoletto:} viene utilizzato per denotare parole che sono:
						\begin{itemize}
							\item riferimenti a documenti esterni, con pedice una ``D'';
							\item appartenenti al glossario, con pedice una ``G''.
						\end{itemize}
				\end{itemize}
			\paragraph{Elenchi puntati}
				Ogni elemento dell'elenco deve essere seguito da un punto e virgola, fatta eccezione per l'ultimo elemento che sarà seguito da un punto; di conseguenza la prima lettera di ogni voce dell'elenco deve essere minuscola, ad eccezione della prima frase se quest'ultima è posta all'inizio di paragrafo. Gli elenchi avranno punto elenco differente a seconda della loro tipologia:
				\begin{itemize}
					\item per gli elenchi non ordinati si è scelto di usare come punto elenco un cerchietto pieno e come sub-punto elenco il trattino;
					\item per gli elenchi ordinati si è optato per un punto elenco ``flessibile'', ossia possono essere usati sia i numeri che i letterali, purché quest'ultimi siano in minuscolo, seguiti da un punto, esempio ``1.'' o ``a.''.
				\end{itemize}
			\paragraph{Formati comuni}
				\begin{itemize}
					\item \textbf{data:} viene utilizzato lo standard ISO 8601, esempio 2020-01-22, per la rappresentazione di tutte le date presenti della documentazione;
					\item \textbf{ora:} viene utilizzato il formato HH:MM;
					\item \textbf{versione:} viene utilizzato il formato vXX.YY.ZZ+bJJ.KK.
				\end{itemize}
			\paragraph{Sigle}
				Il progetto richiede la redazione di un insieme di documenti; segue l'elenco dei documenti con relative sigle:
				\begin{itemize}
					\item documenti esterni:
						\begin{itemize}
							\item \textbf{analisi dei requisiti - AdR:} descrive le caratteristiche del software;
							\item \textbf{piano di progetto - PdP:} riguarda la gestione del progetto, ossia descrive parametri come fattibilità, costi, vincoli del progetto;
							\item \textbf{piano di qualifica - PdQ:} descrive la qualità del software e dei processi coinvolti, come e con quali strumenti si intende raggiungere tale qualità;
							\item \textbf{manuale utente - MU:} manuale per gli utilizzatori del software;
							\item \textbf{manuale sviluppatore - MS:} manuale per gli sviluppatori e manutentori.
						\end{itemize}
					\item documenti interni:
						\begin{itemize}
							\item \textbf{glossario - G:} raccoglie tutti i termini che necessitano di una disambiguazione e/o una descrizione più approfondita;
							\item \textbf{norme di progetto - NdP:} è una raccolta di tutte le regole e le norme utilizzate durante il ciclo di vita del software;
							\item \textbf{studio di fattibilità - SdF:} descrive i vari capitolati analizzando brevemente i loro pro e contro.
						\end{itemize}
					\item \textbf{verbali - V:} essi possono essere sia interni che esterni e descrivono in maniera concisa tutti gli argomenti discussi e le decisione prese durante un incontro.
				\end{itemize}
				Inoltre, il ciclo di vita del progetto è diviso in quattro fasi:
				\begin{itemize}
					\item \textbf{revisione dei requisiti - RR:} studio iniziale del capitolato;
					\item \textbf{revisione di progettazione - RP:} definizione dell'architettura e della fattibilità del software;
					\item \textbf{revisione di qualifica - RQ:} produzione di codice e descrizione dettagliata delle sue componenti;
					\item \textbf{revisione di accettazione - RA:} approvazione del prodotto da parte del cliente e rilascio del software.
				\end{itemize}
				Altre sigle utilizzate all'interno dei documenti sono:
				\begin{itemize}
					\item AWS: \glock{amazon web service};
					\item ETH: ethereum;
					\item EVM: ethereum virtual machine;
					\item LR: \glock{regressione lineare};
					\item SVM: \glock{support vector machine};
					\item ML: \glock{machine learning};
					\item JS: \glock{JavaScript};
					\item AoA: angolo di arrivo;
					\item UI: \glock{user interface};
					\item PaaS: platform as a service;
					\item IaaS: infrastructure as a service;
					\item FaaS: function as a service;
					\item CaaS: container as a service;
					\item BaaS: back end as a service;
					\item IoT: internet of things;
					\item DB: \glock{database};
					\item VCS: \glock{version control system}.
				\end{itemize}
		\subsubsection{Comprensione (QC-1)}
				\paragraph{Scopo}
					Per fornire una documentazione fruibile e comprensibile si è deciso di monitorare la sua qualità tramite delle metriche significative.
				\paragraph{Introduzione alle Metriche di Qualità}
					Per la funzionabilità si é deciso di utilizzare le seguenti metriche:
					\begin{itemize}
						\item QM-PROD-1 \glock{Indice di Gulpease} (GULP);
						\item QM-PROD-2 Correttezza ortografica (CORT).
					\end{itemize}
				\paragraph{QM-PROD-1 Indice di Gulpease (GULP)}
					\subparagraph{Descrizione}
						La metrica GULP permette di misurare la leggibilità di un documento basandosi sulla formula riportata di seguito.
					\subparagraph{Unità di Misura}
						La metrica è espressa tramite un numero intero.
					\subparagraph{Formula}
						La formula della metrica è la seguente:
						\(
							GULP = 89+\frac{300\times\# frasi-10\times\#lettere}{\#parole}
						\)
					\subparagraph{Risultato}
						Il risultato della formula ha i seguenti significati:
						\begin{itemize}
							\item se il risultato è pari 0 allora il documento non esiste e/o la sua leggibilità è terribile;
							\item se il risultato è maggiore di 40 allora il documento esiste ed è leggibile da chi possiede un diploma superiore;
							\item se il risultato è maggiore di 60 allora il documento esiste ed è leggibile da chi possiede una licenza media;
							\item se il risultato è maggiore di 80 allora il documento esiste ed è leggibile da chi possiede una licenza elementare;
							\item se il risultato è pari a 100 allora il documento esiste ed è molto più che leggibile.
						\end{itemize}
				\paragraph{QM-PROD-2 Correttezza ortografica (CORT)}
					\subparagraph{Descrizione}
					La metrica CORT permette di misurare la correttezza, a livello lessicografico, di un documento.
					\subparagraph{Unità di Misura}
						La metrica è espressa tramite un numero intero.
					\subparagraph{Formula}
						La formula della metrica è la seguente:
						\textit{CORT = \# numero di errori ortografici}
					\subparagraph{Risultato}
						Il risultato della formula ha i seguenti significati:
						\begin{itemize}
							\item se il risultato è pari 0 allora il documento non esiste o non ha errori ortografici;
							\item se il risultato è maggiore di 0 allora il documento esiste e presenta errori ortografici.
						\end{itemize}
		\subsubsection{Elementi grafici}
			\paragraph{Tabelle}
				Le tabelle sono sempre accompagnate da un titolo e dal numero della tabella; sono indicizzate separatamente rispetto al resto del contenuto.
			\paragraph{Immagini}
				Le immagini sono sempre accompagnate da una didascalia descrittiva e dal numero della figura; sono indicizzate separatamente rispetto al resto del contenuto.
			\paragraph{Diagrammi UML}
				I diagrammi UML vengono inseriti all'interno della documentazione sotto forma di immagini.
		\subsubsection{Strumenti}
			\paragraph{LaTeX}
				Per la scrittura dei documenti è stato scelto di usare \LaTeX{}, esso è un linguaggio di markup basato sul programma di tipografia digitale \TeX{}. Questo permette di scrivere documenti in maniera modulare e collaborativa.
			\paragraph{Editor di testo}
				Il gruppo ha deliberato su un utilizzo eterogeneo degli editor di testo, questo poichè ogni membro ha familiarità con editor ed ambienti differenti.
