\subsection{Verifica}

    \subsubsection{Scopo}
    
    Il processo di verifica ha lo scopo di capire se il prodotto è realizzato nel modo corretto.
    
    \subsubsection{Aspettative}
    
    Per il corretto svolgimento del processo di verifica si rispettano dei punti determinati:
    
    \begin{itemize}
        \item 
    \end(itemize)
    
    \subsubsection{Descrizione}
		La verifica consiste nel cercare e risolvere possibili difetti all'interno della documentazione e del codice prodotto.
    \subsubsection{Attività}
        \paragraph{Analisi statica e dinamica}
        
        Esistono 2 tipi di analisi: la statica e la dinamica.
        
        \begin{itemize}
            \item Analisi statica
                L'analisi statica viene effettuata sulla documentazione e sul codice senza necessità di eseguire il prodotto e serve per verificare che non ci siano errori o difetti. I due tipi di analisi statica sono:
                \begin{itemize}
                    \item Walkthrough: consiste nell'analizzare i vari documenti e file in tutto il loro contenuto per trovare eventuali difetti. Il verificatore controlla se sono presenti difetti e, in caso ne trovi, la correzione verrà effettuata dagli sviluppatori; 
                    \item Inspection: in questa tecnica si conosce dove possono trovarsi i possibili difetti, quindi non si analizzano i documenti e file per intero, ma solo le parti in cui di solito sono presenti. Il verificatore compone una lista di controllo (checklist) inserendo i punti in cui si possono rilevare possibili difetti, controlla in quei punti della lista e, se trova delle incorrettezze, la correzione viene poi effettuata dagli sviluppatori.
                \end{itemize}
            \item Analisi dinamica
				L'analisi dinamica è una tecnica per cui è necessaria l'esecuzione dell'oggetto di verifica.
        \end{itemize}
        
        \paragraph{Test}
			I test fanno parte dell'attività di analisi dinamica e servono per individuare possibili errori di funzionamento del codice. Per effettuare i test, essi devono essere automatizzati, tramite strumenti appositi, e ripetibili, ovvero devono essere definiti:
			\begin{itemize}
				\item l'ambiente di sviluppo e lo stato iniziale;
				\item le istruzioni eseguite;
				\item i dati di input e i dati di output attesi.
			\end{itemize}
            \subparagraph*{Test di unità}
				Test eseguiti sul funzionamento di unità di software in modo automatico: viene definito l'input e l'output atteso per verificare il corretto funzionamento dell'unità.
            \subparagraph*{Test di integrazione}
				Test eseguiti su componenti del software per verificare se l'insieme di unità si interfaccia come dovrebbe. Questo test è eseguito in modo ricorrente, ogni volta che un insieme di unità esegue correttamente, esso viene integrato con altri insiemi di unità, fino al test completo sul sistema.
            \subparagraph*{Test di sistema}
				Dopo aver eseguito i test su tutte le unità e sulla loro integrazione, si testa il sistema nella sua interezza. Viene testato se le interazioni tra le varie componenti del sistema ritornano il risultato atteso o meno.
            \subparagraph*{Test di non regressione}
				Test eseguito ogni volta che un'unità viene modificata allo scopo di trovare difetti nelle funzionalità già testate, e per questo non dover retrocedere. Si rieseguono tutti i test necessari affinchè si possa essere certi che la modifica non causa il funzionamento scorretto di altre unità collegate all'unità modificata.
            \subparagraph*{Test di accettazione}   
				Anche detto "test di collaudo" è il test eseguito sull'intero sistema allo scopo di convalidare le specifiche definite dal committente sulla base di casi d'uso definiti da esso.

    \subsubsection{Strumenti}
        \paragraph{Verifica ortografica}
    