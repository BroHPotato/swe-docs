\section{Introduzione}
	\subsection{Scopo del documento}
		Il documento ha lo scopo di definire le regole su cui si basa il \textit{way of working} del gruppo Red Round Robin per lo svolgimento del progetto. Le attività che possono essere trovate all'interno di questo documento sono state prese da processi appartenenti allo standard ISO/IEC 12207:1995. Tutti i membri del gruppo sono quindi tenuti a prendere visione di questo documento così da garantire uniformità e coesione all'interno del progetto.   
	\subsection{Scopo del prodotto}
		Il capitolato C6 si pone l'obiettivo di creare una web-application che consenta l'analisi di grosse moli di dati ricevuti da sensori di natura eterogenea. Tale applicazione mette a disposizione un'interfaccia che permette di visualizzare alcuni dati di interesse od eventuali correlazioni tra i dati stessi. Infine, per ogni tipologia di dato è possibile assegnare il monitoraggio di ogni tipologia di dato ad un particolare ente. 
	\subsection{Glossario e documenti esterni}
		Per evitare possibili ambiguità relative alle terminologie (che andranno indicate in \textsc{maiuscoletto}) utilizzate nei vari documenti, saranno adottati due simboli:
		\begin{itemize}
			\item una \textit{D} a pedice per indicare il nome di un particolare documento.
			\item una \textit{G} a pedice per indicare un termine che sarà 
			presente nel \dext{Glossario v1.0.0}.
		\end{itemize}
	\subsection{Riferimenti}

		\subsubsection{Riferimenti normativi}
			\begin{itemize}
				\item \textbf{capitolato d'appalto C6 - ThiReMa: } \\
				\url{https://www.math.unipd.it/~tullio/IS-1/2019/Progetto/C6.pdf}
			\end{itemize}	
		\subsubsection{Riferimenti informativi}
			\begin{itemize}
				\item \textbf{Standard ISO/IEC 12207:1995: } \\
				\url{https://www.math.unipd.it/~tullio/IS-1/2009/Approfondimenti/ISO_12207-1995.pdf}
				\item \textbf{documentazione git: }\url{https://git-scm.com/docs} 
				\item \textbf{documentazione GitHub: }\url{https://help.github.com/en/github}
				\item \textbf{documentazione LaTeX: }\url{https://www.latex-project.org/help/documentation/}
				\item \textbf{Cost Variance: } \url{http://acqnotes.com/acqnote/tasks/cost-variances}
				\item \textbf{Schedule Variance: } \url{http://acqnotes.com/acqnote/tasks/schedule-variances}
			\end{itemize}
		

