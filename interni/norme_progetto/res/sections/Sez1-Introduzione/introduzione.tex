\section{Introduzione}
	\subsection{Scopo del documento}
		Il documento ha lo scopo di definire quelle che sono le regole su cui si basa il way of working del gruppo Red Round Robin per lo svolgimento del progetto. Le attività che possono essere trovate all'interno di questo documento sono state prese da processi appartenenti allo standard ISO 12207. Tutti i membri del gruppo sono quindi tenuti a prendere visione di questo documento così da garantire uniformità e coesione all'interno del progetto.   
	\subsection{Scopo del prodotto}
		Il capitolato C6 si pone come obiettivo quello di creare una web-application che permette di analizzare grosse moli di dati ricevuti da sensori eterogenei tra loro. Tale applicazione mette a disposizione un'interfaccia che permette di visualizzare alcuni dati di interesse od eventuali correlazioni tra i dati stessi. Infine, per ogni tipologia di dato è possibile assegnarne il monitoraggio ad un particolare ente, ruolo o gruppo. 
	\subsection{Glossario e Documenti esterni}
		Per evitare possibili ambiguità relative alle terminologie (che andranno indicate in \textsc{maiuscoletto})utilizzate nei vari documenti, verranno utilizzate due simboli:
		\begin{itemize}
			\item Una \textit{D} al pedice per indicare il nome di un particolare documento.
			\item Una \textit{G} al pedice per indicare un termine che sarà 
			presente nel \dext{Glossario v0.0.1}.
		\end{itemize}
	\subsection{Riferimenti}

		\subsubsection{Riferimenti normativi}
			\begin{itemize}
				\item \textbf{Standard ISO/IEC 12207:1995: } 
				\url{https://www.math.unipd.it/~tullio/IS-1/2009/Approfondimenti/ISO_12207-1995.pdf}
				\item \textbf{Capitolato d'appalto C6 - ThiReMa: } 
				\url{https://www.math.unipd.it/~tullio/IS-1/2019/Progetto/C6.pdf}
			\end{itemize}	
		\subsubsection{Riferimenti informativi}
			\begin{itemize}
				\item Da aggiungere man mano che si fa riferimento alle slide del prof
				\item Guardare bene gli approfondimenti sul sito:
				\url{https://www.math.unipd.it/~tullio/IS-1/2019/}
				\item \textbf{Documentazione git: }\url{https://git-scm.com/docs} 
				\item \textbf{Documentazione GitHub: }\url{https://help.github.com/en/github}
				\item \textbf{Documentazione LaTeX: }\url{https://www.latex-project.org/help/documentation/}
			\end{itemize}
		

