\section{Processi Organizzativi}

\subsection{Gestione dei Processi}

	\subsubsection{Scopo}

		Il processo di gestione dei processi ha lo scopo di:

		\begin{itemize}
			\item identificare i possibili rischi e definirne la gestione;
			\item definire un modello di sviluppo;
			\item pianificare i task da svolgere in base alle scadenze temporali;
			\item calcolare un preventivo in termini di ore e costi suddiviso per ruoli;
			\item calcolare un preventivo a finire delle risorse richieste, visto il consuntivo di periodo.
		\end{itemize}

	\subsubsection{Aspettative}

		Le principali aspettative del processo di gestione dei processi sono:

		\begin{itemize}
			\item agire preventivamente per evitare i rischi identificati e, qualora si verificassero, limitare le loro ripercussioni sull'efficacia e l'efficienza del lavoro svolto dal gruppo;
			\item effettuare una pianificazione ragionevole dei task da svolgere ed assegnarli in modo equilibrato ai diversi ruoli, tenendo conto dei tempi e delle risorse disponibili;
			\item gestire i componenti del gruppo e i loro task in modo da facilitare la collaborazione e la comunicazione interna tra di loro;
			\item mantenere sotto controllo l'andamento del progetto, monitorando il lavoro svolto dal gruppo in modo tale da non comprometterne l'efficienza.
		\end{itemize}

	\subsubsection{Descrizione}
		Le attività previste dal processo di gestione dei processi sono raccolte nel \dext{Piano di Progetto}, la cui redazione è in carico al responsabile, con la collaborazione dell'amministratore.
		\newline
		Nello specifico, sono trattati:

		\begin{itemize}
			\item introduzione;
			\item analisi dei rischi, classificazione degli stessi;
			\item istanziazione dei processi che realizzano il modello di sviluppo adottato;
			\item pianificazione dei task e assegnazione degli stessi ai ruoli di progetto;
			\item stima dei costi in termini di tempo e risorse;
			\item calcolo delle risorse necessarie per terminare il progetto, visto il bilancio del lavoro svolto nel periodo;
			\item revisione delle attività sulla base dei riscontri dei rischi.
		\end{itemize}

	\subsubsection{Ruoli di Progetto}

		Ogni membro del gruppo deve ricoprire più ruoli, i quali sono cambiati a rotazione con una frequenza che permetta ad ogni componente di assumere almeno una volta ogni ruolo previsto per il progetto e, allo stesso tempo, di garantire la continuità delle attività in corso.
		\newline
		Le attività assegnate ad ogni ruolo sono programmate ed organizzate nel \dext{Piano di Progetto}.
		\newline
		Di seguito vengono descritti tutti i ruoli richiesti dal progetto.

		\paragraph{Responsabile}

			Il responsabile accentra tutte le responsabilità di pianificazione, controllo, gestione e coordinamento di attività e risorse all'interno del progetto. Inoltre svolge la funzione di intermediario verso le persone esterne al gruppo, quali committente e proponente del capitolato, ed è il responsabile ultimo dei risultati del progetto.
			\newline
			In particolare si occupa di:

			\begin{itemize}
				\item elaborare ed emanare piani e scadenze;
				\item approvare l'emissione dei documenti;
				\item coordinare le attività, le risorse e i componenti del gruppo;
				\item gestire le criticità incontrate dal gruppo;
				\item redigere l'\glock{Organigramma} e il Piano di Progetto;
				\item approvare l'Offerta sottoposta al committente.
			\end{itemize}

		\paragraph{Amministratore}

			L'amministratore è incaricato della gestione dell'ambiente di lavoro.
			\newline
			All'interno del gruppo egli:

			\begin{itemize}
				\item è responsabile dell'efficacia e dell'efficienza dell'ambiente di sviluppo e di tutte le installazioni di supporto;
				\item è responsabile della redazione ed attuazione dei piani e delle procedure per la gestione della qualità;
				\item controlla le versioni e le configurazioni del prodotto;
				\item gestisce la documentazione del progetto;
				\item collabora alla redazione del Piano di Progetto;
				\item redige le Norme di Progetto.
			\end{itemize}

		\paragraph{Analista}
			L'analista è il responsabile di tutte le attività di analisi svolte durante l'Analisi dei Requisiti, al cui termine hanno fine anche tutti i sui incarichi all'interno del gruppo. Egli infatti è una figura che non è presente all'interno del gruppo per tutta la durata del progetto.
			\newline
			L'analista ha il compito di:

			\begin{itemize}
				\item studiare il dominio applicativo del progetto;
				\item definire i requisiti del progetto;
				\item redigere lo Studio di Fattibilità e l'Analisi dei Requisiti.
			\end{itemize}

		\paragraph{Progettista}

			Il progettista è il responsabile di tutte le attività di progettazione svolte durante la Progettazione dell'Architettura e la Progettazione di Dettaglio.
			\newline
			Il progettista deve:

			\begin{itemize}
				\item prendere decisioni riguardanti gli aspetti tecnici del progetto, favorendo l'efficacia e l'efficienza;
				\item definire l'architettura del prodotto da sviluppare, perseguendo la sua efficienza, efficacia e manutenibilità, tramite l'utilizzo di apposite tecnologie individuate a partire dai requisiti definiti dall'analista;
				\item redigere la Specifica Tecnica, la Definizione di Prodotto e la parte pragmatica del Piano di Qualifica.
			\end{itemize}

		\paragraph{Programmatore}

			Il programmatore è il responsabile di tutte le attività di codifica effettuate per lo sviluppo del progetto.
			\newline
			In particolare, il programmatore è responsabile:

			\begin{itemize}
				\item dell'implementazione della Specifica Tecnica redatta dal progettista;
				\item della codifica mirata alla realizzazione del prodotto;
				\item della codifica di componenti di ausilio necessarie per l'esecuzione delle prove di verifica e validazione.
			\end{itemize}

		\paragraph{Verificatore}

			Il verificatore è il responsabile di tutte le attività di verifica dei documenti e del codice scritti dagli altri componenti del gruppo. Il suo compito è quello di trovare errori, di qualunque tipo, nei prodotti che controlla e di segnalare tali errori a chi ha la responsabilità diretta sul quel prodotto, in modo che possa apportare le dovute correzioni.
			\newline
			Il verificatore non ha il compito di correggere gli errori rilevati, deve quindi:

			\begin{itemize}
				\item esaminare i prodotti in fase di revisione, utilizzando le tecniche e gli strumenti definiti nelle Norme di Progetto;
				\item indicare eventuali errori riscontrati nel prodotto in esame;
				\item segnalare eventuali errori rilevati al responsabile dell'oggetto in fase di verifica, in modo che possa correggerli.
			\end{itemize}

	\subsubsection{Introduzione ale Procedure}

		Sono state definite delle procedure da adottare all'interno del gruppo durante la realizzazione del progetto, le quali hanno lo scopo di regolamentare tutte le operazioni di gestione e coordinamento del lavoro, con il fine di garantire efficacia ed efficienza.


		\subsubsection{Gestione delle Risorse (QP-1)}

			\paragraph{Scopo}

				Si vuole gestire la copertura di risorse disponibili per la realizzazione del progetto, monitorando i costi aggiuntivi e le tempistiche non rispettate dalla schedulazione pianificata. Questo può essere utile al cliente per capire in fase di sviluppo l'andamento del progetto a livello di gestione delle risorse.

			\paragraph{Introduzione alle Metriche di Qualità}

				Per la gestione delle risorse si farà uso delle seguenti metriche:

				\begin{itemize}
					\item QM-PROC-1. Budgeted Cost of Work Scheduled (BCWS);
					\item QM-PROC-2. Actual Cost of Work Performed (ACWP);
					\item QM-PROC-3. Budgeted Cost of Work Performed (BCWP);
					\item QM-PROC-4. Schedule Variance (SV);
					\item QM-PROC-5. Cost Variance (CV).
				\end{itemize}

			\paragraph{QM-PROC-1. Budgeted Cost of Work Scheduled (BCWS)}

				\subparagraph{Descrizione}
				La metrica BCWS definisce il costo pianificato per realizzare le attività di progetto alla data corrente.

				\subparagraph{Unità di Misura}
				Il costo pianificato è misurato in EURO.

			\paragraph{QM-PROC-2. Actual Cost of Work Performed (ACWP)}

				\subparagraph{Descrizione}
				La metrica ACWP definisce il costo effettivamente sostenuto per realizzare le attività di progetto alla data corrente.

				\subparagraph{Unità di Misura}
				Il costo sostenuto è misurato in EURO.

			\paragraph{QM-PROC-3. Budgeted Cost of Work Performed (BCWP)}

				\subparagraph{Descrizione}
				La metrica BCWP definisce il valore delle attività realizzate alla data corrente. In altre parole, misura il valore del prodotto fino ad ora realizzato.

				\subparagraph{Unità di Misura}
				Il valore del prodotto è misurato in EURO.

			\paragraph{QM-PROC-4. Schedule Variance (SV)}

				\subparagraph{Descrizione}
				La metrica SV indica se si è in anticipo, in ritardo o in linea rispetto alle schedulazioni pianificate per il progetto. Questo può essere utile per il cliente per valutare l'efficacia del gruppo nei confronti della realizzazione del progetto.

				\subparagraph{Unità di Misura}
				La metrica viene espressa in percentuale.

				\subparagraph{Formula}
				La formula per il calcolo della metrica è la seguente:

				\[
					\text{SV} = \frac{\text{BCWP} - \text{BCWS}}{\text{BCWS}} \times 100
				\]

				\subparagraph{Risultato}
				\begin{itemize}
					\item Un risultato \textbf{positivo} (\(> 0\)) indica che il progetto è avanti rispetto alla schedulazione;
					\item Un risultato \textbf{negativo} (\(< 0\)) indica che il progetto è indietro rispetto alla schedulazione;
					\item Un risultato \textbf{pari a zero} indica che il progetto è in linea rispetto alla schedulazione.
				\end{itemize}

			\paragraph{QM-PROC-5. Cost Variance (CV)}

				\subparagraph{Descrizione}
				La metrica CV indica se il valore del costo realmente maturato è maggiore, minore o uguale rispetto al costo effettivo. In altre parole, permette di comprendere con che livello di efficienza il gruppo sta sviluppando il progetto, rispetto a quanto pianificato.

				\subparagraph{Unità di Misura}
				La metrica viene espressa in percentuale.

				\subparagraph{Formula}
				La formula per il calcolo della metrica è la seguente:

				\[
					\text{CV} = \frac{\text{BCWP} - \text{ACWP}}{\text{BCWP}} \times 100
				\]

				\subparagraph{Risultato}
				\begin{itemize}
					\item Un risultato \textbf{positivo} (\(> 0\)) indica che il progetto sta sviluppando con un costo minore rispetto a quanto pianificato (maggiore efficienza);
					\item Un risultato \textbf{negativo} (\(< 0\)) indica che il progetto sta sviluppando con un costo maggiore rispetto a quanto pianificato (minore efficienza);
					\item Un risultato \textbf{pari a zero} indica che il progetto sta sviluppando con un costo in linea rispetto a quello pianificato.
				\end{itemize}


		\subsubsection{Gestione delle Comunicazioni}

      Le comunicazioni vengono gestite sfruttando tutte le potenzialità di \glock{Slack}, il quale permette di creare un \textit{workspace} con degli appositi canali di comunicazione. Si usano i seguenti canali e prefissi:
      \begin{itemize}
        \item \verb!#announcement:! canale di annunci importanti;
        \item \verb!#offtopic:! canale di svago per parlare di argomenti extra-lavorativi;
        \item \verb!#gen_[*]:! canali generali per il software e la documentazione;
        \item \verb!#bot_[*]:! canali dedicati per le notifiche di \glock{Github} e di \glock{Google};
        \item \verb!#int_[*]:! canali specifici per i documenti interni di progetto;
        \item \verb!#ext_[*]:! canali specifici per i documenti esterni di progetto;
        \item \verb!#devops_[*]:! canali di notifica per le operazioni di \glock{DevOps} per i documenti interni di progetto.
      \end{itemize}

      Per organizzare le riunioni si fa uso di \textbf{Google Calendar}, grazie a cui è possibile fissare degli appuntamenti sul calendario comune di tutti, notificando o meno la propria presenza.
      In modo più informale, infine, si fa anche uso di una chat di \textbf{Telegram} per coordinarsi su eventuali cambi di luogo o per discutere di argomenti extra-lavorativi.

			\paragraph{Comunicazioni Interne}

				Le comunicazioni interne riguardano i soli membri del gruppo e sono volte a favorire la collaborazione e la suddivisione del lavoro tra i componenti stessi.
				\newline
				Sono state individuate due categorie principali di comunicazioni:

				\begin{itemize}
					\item \textbf{relative agli incontri:} finalizzate ad organizzare gli incontri interni, in base alle necessità e alle disponibilità dei singoli membri del gruppo.
					\newline
					Come strumento è stato scelto \glock{Telegram}, tramite il quale è stato creato un gruppo in cui ogni componente può interagire con gli altri comunicando le proprie disponibilità;
					\item \textbf{relative al lavoro:} finalizzate alla discussione dei task da svolgere e delle criticità incontrate durante la loro esecuzione.
					\newline
					Come strumento di comunicazione è stato scelto \glock{Slack}, tramite il quale è stato creato un \glock{workspace} del gruppo al cui interno sono stati predisposti diversi canali separati, uno per ogni task in svolgimento. Ogni membro del gruppo può scrivere in ognuno dei canali presenti ed ottenere informazioni riguardanti lo stato di progresso del lavoro e le eventuali problematiche riscontrate, facilitando quindi la collaborazione.
				\end{itemize}

			\paragraph{Comunicazioni Esterne}

				Le comunicazioni esterne riguardano, oltre ai membri del gruppo, anche il committente e l'azienda proponente, la quale si interfaccia con il gruppo tramite un suo referente interno. L'interazione con le figure esterne è di sola competenza del responsabile, il quale provvederà a tenere costantemente aggiornato il gruppo riguardo gli sviluppi del suo operato.
				\newline
				Lo strumento predefinito per le comunicazioni esterne è la posta elettronica, per la quale il gruppo ha previsto una casella unica all'indirizzo \href{mailto:redroundrobin.site@gmail.com}{redroundrobin.site@gmail.com}, accessibile in lettura da parte di tutti i componenti e in scrittura solamente da parte del responsabile.

				Oltre alla posta elettronica, per favorire una riscontro più rapido e diretto con il proponente, quest'ultimo ha predisposto un \glock{workspace} \glock{Slack} con, all'interno, un apposito canale riservato alle comunicazioni con il gruppo. Di conseguenza il responsabile ha provveduto alla creazione di un singolo account \glock{Slack} del gruppo, utilizzato unicamente per comunicare con il proponente all'interno del canale. Per ridurre il carico di lavoro del responsabile è stato scelto di permettere l'accesso all'account sopra citato da parte di ciascun membro del gruppo, in modo da facilitare la fruizione degli argomenti trattati con il proponente. Tuttavia la partecipazione attiva al canale è permessa soltanto al responsabile.

		\subsubsection{Gestione degli Incontri}

			In caso di necessità, il responsabile può organizzare delle riunioni volte a trattare argomenti critici, che richiedono un confronto diretto tra tutti i componenti del gruppo ed, eventualmente, con il proponente.
			\newline
			Ogni incontro dovrà essere fissato in accordo con tutti i partecipanti, in base alle loro disponibilità. Per semplificarne la gestione, il responsabile deve creare un apposito evento su Google Calendar, tramite l'account del gruppo, con tutte le informazioni riguardanti la riunione. L'evento deve poi essere condiviso con tutti i componenti del gruppo, in modo che ne possano prendere visione ed essere immediatamente informati di eventuali modifiche tramite il servizio email di notifica di Google.
			
			\paragraph{Incontri Interni}

				Agli incontri interni partecipano solamente i membri del gruppo, i quali sono tenuti a farsi trovare nel luogo prestabilito entro l'orario indicato.
				\newline
				Sono tollerati ritardi e assenze giustificate se segnalati con sufficiente anticipo da permettere al responsabile di apportare le dovute modifiche al programma della riunione, e a tutti gli altri partecipanti di prenderne visione.
				\newline
				Gli incontri interni possono avvenire in due modalità:
				
				\begin{itemize}
					\item \textbf{fisica:} incontri svolti di persona, nei quali i membri del gruppo si ritrovano per discutere di tematiche critiche e prendere decisioni in merito a questioni importanti che riguardano il lavoro da svolgere;
					\item \textbf{virtuale:} incontri svolti tramite chiamate di gruppo, nei quali i componenti possono discutere tra loro di eventuali dubbi sorti durante lo svolgimento dei task assegnati o comunicare le loro difficoltà nel portare avanti uno specifico compito, in modo che possano essere attuate le opportune misure correttive nel più breve tempo possibile.
					\newline
					Come strumento di comunicazione a chiamate è stato scelto \glock{Discord}, che permette, tramite la predisposizione di un apposito server del gruppo, di realizzare una suddivisione a canali vocali degli argomenti da trattare nelle diverse discussioni. Questi canali sono del tutto analoghi, per funzionamento, a quelli messi a disposizione da \glock{Slack}, con l'unica differenza che la comunicazione al loro interno avviene tramite chiamate vocali.
				\end{itemize}

			\paragraph{Incontri Esterni}

				Negli incontri esterni, assieme ai membri del gruppo, sono coinvolti anche uno o più rappresentanti dell'azienda proponente.
				\newline
				Data la presenza di persone esterne, questi incontri assumono una criticità maggiore, che deve essere opportunamente gestita dal responsabile tramite un'adeguata gestione delle comunicazioni che punti a mediare tra l'interno del gruppo e gli individui esterni.
				\newline
				Data la complessità dell'organizzazione, ritardi e assenze da parte dei componenti del gruppo sono da evitare, a meno di un'opportuna giustificazione non preventivabile.\newline
				Gli incontri esterni, come quelli interni, possono avvenire in due modalità:
				\begin{itemize}
					\item \textbf{fisica:} incontri svolti di persona presso la sede dell'azienda proponente, nei quali i membri del gruppo si ritrovano assieme ai rappresentanti della stessa, per avere un riscontro diretto sulle decisioni prese in merito al lavoro da svolgere e per chiarire eventuali parti del progetto che non sono del tutto chiare al gruppo;
					\item \textbf{virtuale:} incontri svolti tramite chiamate di gruppo, nei quali i componenti possono discutere con il proponente di eventuali dubbi sorti durante l'analisi e/o lo svolgimento dei task assegnati, in modo da poter porre domande dirette ed aver una risposta immediata da parte dell'azienda.
					\newline
					Come strumento di comunicazione a chiamate è stato scelto Skype, che permette di effettuare facilmente videochiamate con una o più persone.
				\end{itemize}

			\paragraph{Verbali degli Incontri}

				Successivamente ad un incontro, sia esso interno od esterno, il segretario incaricato dal rappresentate ne redige il verbale.
				\newline
				La nomenclatura dei documenti dei verbali deve essere tale da permettere l'identificazione univoca di ogni verbale e il loro ordinamento temporale all'interno del \glock{repository}. A questo scopo la forma da adottare è la seguente:
				\begin{center}
					\textbf{V[Tipo]-[Numero]}
				\end{center}
				dove:
				\begin{itemize}
					\item \textbf{V:} indica che si tratta di un verbale;
					\item \textbf{Tipo:} indica la tipologia del verbale, ossia:
					\begin{itemize}
						\item \textbf{I:} indica che il verbale si riferisce ad un incontro interno;
						\item \textbf{E:} indica che il verbale si riferisce ad un incontro esterno;
					\end{itemize}
					\item \textbf{Numero:} numero intero progressivo che fornisce un'indicazione riguardo all'ordine temporale di svolgimento degli incontri.
				\end{itemize}
		
		\subsubsection{Gestione degli Strumenti di Coordinamento}

			In ogni momento, durante lo sviluppo del progetto, tutti i membri del gruppo devono poter sapere quali compiti sono pianificati, quali sono in corso e quali sono già stati completati. Inoltre, ognuno deve essere a conoscenza dei compiti a lui assegnati e della relativa data di scadenza per il loro completamento, in modo da poter gestire il proprio carico di lavoro. Infine, il responsabile deve poter assegnare nuovi compiti ai membri del gruppo e controllarne lo stato di avanzamento per verificarne la coerenza con la pianificazione.
			\newline
			Per soddisfare queste necessità è stato scelto di utilizzare l'\glock{Issue Tracking System} fornito da \glock{GitHub} per il \glock{repository} usato dal gruppo per il \glock{versionamento} del progetto.
			\newline
			Questo strumento permette di creare delle \glock{Issue} che possono rappresentare dei singoli compiti, assegnabili ad una o più persone, attraverso le seguenti informazioni:
			\begin{itemize}
				\item \textbf{titolo:} nome del compito da eseguire;
				\item \textbf{descrizione:} descrizione dettagliata del compito da eseguire;
				\item \textbf{assegnatari:} persone a cui compete lo svolgimento del compito;
				\item \textbf{bacheca:} cruscotto di progetto in cui il compito sarà monitorato;
				\item \textbf{scadenza:} data entro la quale lo svolgimento del compito deve essere completato.
			\end{itemize}
			Ogni \glock{Issue} attraversa degli stati che permettono di monitorare l'avanzamento nello svolgimento del compito che essa rappresenta, questi sono:
			\begin{itemize}
				\item \textbf{to do:} compito da svolgere;
				\item \textbf{in progress:} compito in svolgimento;
				\item \textbf{done:} compito svolto.
			\end{itemize}
			È stato scelto l'\glock{Issue Tracking System} di \glock{GitHub} in quanto è completamente integrato con il \glock{repository} del progetto, è semplice da gestire e si adatta a più casi di utilizzo.

		\subsubsection{Gestione dei Rischi (QP-2)}

			I rischi che si verificano durante lo svolgimento del progetto devono essere prontamente rilevati, classificati e documentati nel \textit{Piano di Progetto}, prevedendo, inoltre, delle strategie per la loro gestione. Una volta fatto ciò, oltre a individuare nuovi rischi, è necessario monitorare i quelli noti, attuando le corrette strategie di mitigazione nel caso in cui essi si verifichino nuovamente.
			\newline
			Per permettere un facile riferimento ad ogni rischio documentato, è stata prevista una codifica che, oltre a consentirne l'identificazione univoca, riesca a mostrarne anche le principali caratteristiche. I codici di riferimento dei rischi hanno, quindi, la seguente struttura:
			\begin{center}
				\textbf{RSK-[Tipo][Probabilità][Gravità]-[ID]}
			\end{center}
			dove:
			\begin{itemize}
				\item \textbf{RSK:} indica che si tratta di un rischio;
				\item \textbf{Tipo:} tipologia del rischio, che può essere:
				\begin{itemize}
					\item \textbf{O:} organizzativo, riguardante l'organizzazione del lavoro all'interno del gruppo e la gestione dei costi e delle risorse, tra cui persone e tempi;
					\item \textbf{P:} personale, riguardante le persone che fanno parte del gruppo, le loro conoscenze e le loro capacità;
					\item \textbf{R:} requisiti, riguardante le modifiche ai requisiti che possono avvenire durante lo sviluppo del progetto;
					\item \textbf{S:} strumentale, riguardante gli strumenti software impiegati per lo sviluppo del progetto, il loro utilizzo e le loro prestazioni;
					\item \textbf{T:} tecnologico, riguardante le tecnologie hardware o software impiegate per lo sviluppo del progetto, il loro utilizzo e le loro funzionalità;
				\end{itemize}
				\item \textbf{Probabilità:} livello di possibilità che il rischio si verifichi, che può essere può essere:
				\begin{itemize}
					\item \textbf{A}: alta;
					\item \textbf{M}: media;
					\item \textbf{B}: bassa;
				\end{itemize}
				\item \textbf{Gravità:} livello di pericolosità del rischio, che può essere:
				\begin{itemize}
					\item \textbf{A:} accettabile;
					\item \textbf{T:} tollerabile;
					\item \textbf{I:} inaccettabile;
				\end{itemize}
				\item \textbf{ID:} numero intero progressivo che fornisce la numerazione dei vari rischi della stessa tipologia, probabilità e gravità.

			\end{itemize}

			\paragraph{Introduzione alle Metriche di Qualità}

			Per la gestione dei rischi si farà uso delle seguenti metriche:

			\begin{itemize}
				\item QM-PROC-6. Unbudgeted Risks (UR)
			\end{itemize}

			\paragraph{QM-PROC-6. Unbudgeted Risks (UR)}

				\subparagraph{Descrizione}
				La metrica UR viene utilizzata per tracciare in modo incrementale tutti i nuovi rischi non precedentemente preventivati che avvengono durante una fase del progetto.

				\subparagraph{Unità di Misura}
				La metrica viene espressa con un valore intero che parte da 0.

				\subparagraph{Formula}
				Per ogni rischio non preventivato e non individuato precedentemente che viene viene rilevato, si incrementa di una unità il numero di rischi rilevati fino alla data corrente, a partire da una fase del progetto.
				La formula della metrica è la seguente:
				\[
					\text{UR} = \text{UR} + 1
				\]

				\subparagraph{Risultato}
				\begin{itemize}
					\item Un valore pari a 0, indica che non sono stati trovati rischi nella fase del progetto;
					\item Un valore superiore a 0, indica che sono stati trovati rischi nella fase del progetto.
				\end{itemize}

	\subsubsection{Strumenti}

		Durante lo svolgimento del progetto, per favorire l'attuazione delle procedure descritte in precedenza, il gruppo ha scelto di utilizzare i seguenti strumenti:
		\begin{itemize}
			\item \textbf{\glock{Telegram}:} strumento di messaggistica utilizzato per le comunicazione relative agli incontri tra i membri del gruppo;
			\item \textbf{\glock{Slack}:} strumento di collaborazione utilizzato dal gruppo sia per le comunicazioni interne relative al lavoro svolto dai singoli componenti, sia per le comunicazioni con il proponente;
			\item \textbf{GMail:} servizio di posta elettronica fornito da Google, tramite l'account del gruppo, attraverso il quale avvengono le comunicazioni con il committente ed il primo contatto con il proponente;
			\item \textbf{Google Calendar:} calendario fornito da Google, tramite l'account del gruppo, sul quale vengono segnati tutti gli incontri programmati in modo che siano visibili a tutti i componenti del gruppo;
			\item \textbf{\glock{Discord}:} strumento che offre un servizio di chiamata online, utilizzato per gli incontri virtuali interni al gruppo;
			\item \textbf{Skype:} strumento che offre un servizi di chiamata e videochiamata online, utilizzato per gli incontri virtuali con il proponente;
			\item \textbf{\glock{GitHub}:} strumento che fornisce un servizio di versionamento remoto, utilizzato per il salvataggio, la storicizzazione e la condivisione di tutti i file del progetto;
			\item \textbf{\glock{Issue Tracking System}:} strumento fornito da \glock{GitHub}, utilizzato per la gestione e il coordinamento dei compiti da svolgere.
		\end{itemize}


