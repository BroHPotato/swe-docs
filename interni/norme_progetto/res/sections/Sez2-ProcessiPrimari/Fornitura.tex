\section{Processi primari} 


\subsection{Fornitura}

	\subsubsection{Scopo}
	Il processo di fornitura è indispensabile per:
	\begin{itemize}
	\item capire quali strumenti e competenze sono necessari per la realizzazione dei prodotti descritti nei capitolati proposti;
	\item redigere un documento che descriva nel dettaglio come il fornitore (il gruppo) intenda organizzare il lavoro che porterà alla realizzazione del prodotto oggetto del capitolato scelto;
	\item stabilire se il materiale prodotto dai membri del fornitore per soddisfare i compiti a loro assegnati faccia quanto atteso e sia di qualità.    
	\end{itemize}
 	\subsubsection{Aspettative}
	Il gruppo si prefigge l'obiettivo di confrontarsi frequentemente col proponente, per capire se il lavoro svolto è in linea con quanto richiesto dal capitolato scelto, e chiarire eventuali requisiti ove necessario.
	\subsubsection{Descrizione}
	Il processo di fornitura si riassume in sei attività principali che vengono descritte sinteticamente di seguito. 
	\subsubsection{Attività}
		
		\paragraph{Inizializzazione}
			Gli analisti del gruppo Red Round Robin devono effettuare una valutazione dei capitolati proposti tramite un documento che contenga uno studio di fattibilità. 
			Il documento dovrà contenere, per ogni capitolato, una breve descrizione del capitolato stesso e delle tecnologie proposte, i suoi aspetti positivi e quelli negativi. Dovrà essere presente infine una conclusione che riassuma la motivazione per la quale si è scelto un capitolato rispetto ad un altro. 
			Il contenuto del documento indicherà, alla fine, il capitolato scelto dal gruppo.
		\paragraph{Preparazione della risposta}
			Il gruppo dovrà preparare una lettera di presentazione in cui si candiderà alla fornitura del prodotto implicato dal capitolato scelto.
		\paragraph{Pianificazione}
			Il gruppo deve stipulare un piano per la gestione del progetto e della qualità; più nello specifico dovrà realizzare:
			\begin{itemize}
				\item un piano di progetto, contenente:
				\begin{itemize}
				 	\item analisi e riscontro dei rischi;
				 	\item modello di sviluppo scelto;
				 	\item pianificazione coesa con il modello di sviluppo;
				 	\item preventivo e consuntivo di periodo;
				 \end{itemize} 
				\item un piano di qualifica, contenente: 
				\begin{itemize}
					\item metriche di qualità di processo;
					\item metriche di qualità di prodotto;
					\item specifiche dei test;
					\item resoconto delle attività di verifica;
					\item valutazioni di miglioramento.
				\end{itemize}
			\end{itemize}
			
		\paragraph{Esecuzione e controllo}
			Il gruppo Red Round Robin dovrà seguire quanto descritto nei piani elencati in precedenza, monitorando i costi, le problematiche riscontrate, l'avanzamento nello sviluppo e la qualità.
		\paragraph{Revisione e valutazione}
			Il gruppo Red Round Robin dovrà coordinare le revisioni delle attività svolte e le comunicazioni con il proponente ed il committente. Il gruppo dovrà inoltre eseguire la verifica e la validazione del prodotto garantendo che il prodotto sia conforme con quanto pattuito con il proponente ed il committente.
		\paragraph{Rilascio e completamento}
			Il gruppo dovrà rilasciare il prodotto in maniera conforme a quanto pattuito con il proponente ed il committente.
	\subsubsection{Metriche}
		Il processo di fornitura non fa uso di metriche qualitative particolari.
	\subsubsection{Strumenti}
		Per il processo di fornitura non sono stati individuati particolari strumenti da impiegare.