\section{Processi Primari} 


\subsection{Fornitura}

	\subsubsection{Scopo}
	\text Il \textit{processo di Fornitura} è indispensabile per:
	\begin{itemize}
	\item capire quali strumenti e competenze sono necessari per la realizzazione dei prodotti descritti nei capitolati proposti;
	\item redigere un documento che descriva nel dettaglio come l'azienda fornitrice (il gruppo) intenda organizzare il lavoro che porterà alla realizzazione del prodotto oggetto del capitolato scelto;
	\item stabilire se il materiale prodotto dai membri dell'azienda per soddisfare i compiti a loro assegnati faccia quanto atteso e che sia di qualità.    
	\end{itemize}
Le tre attività elencate sopra sono formalizzate rispettivamente nello \textit{Studio di Fattibiltà}, nel \textit{Piano di Progetto} e nel \textit{Piano di Qualifica}.
 	\subsubsection{Aspettative}
Il gruppo si prefigge come obiettivo quello di confrontarsi frequentemente col proponente, per capire se il lavoro svolto è in linea con quanto richiesto dal capitolato scelto e chiarire eventuali requisiti ove necessario.
	\subsubsection{Descrizione}
	Di seguito verranno definiti sinteticamente gli argomenti principali di cui trattano i documenti citati in precedenza. Per approfondire, si rimanda alle omonime sezioni. 
	\subsubsection{Attività}

		\paragraph{Studio di fattibilità}
			Nello \textit{Studio di Fattibiltà}, per ogni capitolato proposto, si fornisce una breve descrizione del prodotto desiderato e delle sue caratteristiche, oltre a citare quali tecnologie sono richieste per realizzarlo. Infine, gli analisti del gruppo espongono quali siano gli aspetti positivi, quelli negativi e i fattori di rischio riscontrati. A fronte di questa analisi, nella sezione delle \textit{Conclusioni}, si motiva perchè il capitolato e' stato rifiutato o accettato dal gruppo.  
		\paragraph{Piano di progetto}
			In questo documento, il responsabile di progetto e gli amministratori sono tenuti a fare un'analisi dei rischi completa, partendo da quelli individuati dagli analisti nello \textit{Studio di Fattibiltà}, approfondendoli e aggiungendone altri qualora ve ne fossero. Inoltre, è indispensabile decidere quale \glock{Modello di Sviluppo} tra quelli standardizzati prendere come riferimento per svolgere tutti i compiti che porteranno alla realizzazione del prodotto finito. Una volta scelto il \glock{Modello di Sviluppo}, i redattori procedono a definire nella sezione \textit{Pianificazione} quali compiti eseguire nelle diverse fasi del progetto e definiscono delle deadline entro le quali vanno portati a termine. Infine, viene stimato il costo complessivo per la realizzazione del progetto in un preventivo a finire da proporre al committente, oltre a dei consuntivi di periodo che determinano se i costi del progetto effettivi rilevati durante la realizzazione del prodotto sono in linea con le stime definite nel preventivo.
		\paragraph{Piano di qualifica}
 			I verificatori redigono questo documento allo scopo di garantire \textit{qualità al prodotto}, sia lato cliente che lato fornitore, e \textit{qualità di processo}, in modo da essere certi che i membri lavorino, durante la realizzazione delle varie attività, che siano o di supporto alla realizzazione del prodotto o che lo costituiscano, secondo le metriche scelte. In particolare, vengono definiti i \textit{test} ai quali viene sottoposto il prodotto per assicurarsi che soddisfi i requisiti individuati dagli analisti, e si sottolineano i punti in cui e' possibile apportare miglioramenti ai processi di supporto e al processo di verifica stesso. Dopodichè, per ogni attività significativa sottoposta a verifica, viene redatto un resoconto che mostra i livelli di qualità raggiunti relativamente alle metriche adottate nel progetto.
	\subsubsection{Strumenti}
	DA FARE