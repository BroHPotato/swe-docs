\subsection{Sviluppo}
		\subsubsection{Scopo}
			Il processo di sviluppo definisce i compiti e le attività da intraprendere per ottenere il prodotto finale richiesto dal proponente.
		\subsubsection{Aspettative}
			Per una corretta implementazione di questo processo è necessario fissare:
				\begin{itemize}
					\item obiettivi di sviluppo;
					\item vincoli tecnologici e di design.
				\end{itemize}	
			Il prodotto finale deve rispettare i requisiti e le aspettative del proponente, superando i test definiti dalle norme di qualità.
		\subsubsection{Descrizione}
			Il processo di sviluppo, secondo lo standard ISO/IEC 12207:1995, si articola nelle seguenti attività:
				\begin{itemize}
					\item analisi dei requisiti;
					\item progettazione;
					\item codifica.
				\end{itemize}
			
		\subsubsection{Attività}
			Di seguito verranno analizzate dettagliatamente le attività menzionate nella sezione precedente.
			\paragraph{Analisi dei requisiti}
				\subparagraph{Scopo}
					Gli analisti si occupano di stilare il documento \dext{Analisi dei Requisiti v1.0.0}, il cui scopo è definire ed elencare tutti i requisiti del capitolato. Il documento finale conterrà:
					\begin{itemize}
						\item descrizione generale del prodotto;
						\item argomentazioni precise ed affidabili per i progettisti;
						\item casi d'uso rappresentati tramite diagrammi UML;
						\item funzionalità e requisiti concordi con le richieste del cliente;
						\item tracciamento dei requisiti individuati. 
					\end{itemize}
				\subparagraph{Aspettative}
					Creazione del documento formale contente tutti i requisiti richiesti dal proponente per la realizzazione del capitolato.
				\subparagraph{Classificazione dei Requisiti}
					Al fine di facilitarne la consultazione e comprensione, i requisiti saranno classificati ed identificati univocamente secondo il seguente schema identificativo:
					\begin{center}
						\textbf{R[Priorità]-[Tipologia]-[Identificativo]}
					\end{center}
					Dove:
					\begin{itemize}
						\item \textbf{R:} requisito 
						\item \textbf{Priorità:} ogni requisito assumerà uno dei seguenti valori:
						\begin{itemize}
							\item \textbf{A:} obbligatorio, strettamente necessario;
							\item \textbf{B:} desiderabile, non strettamente necessario;
							\item \textbf{C:} opzionale, relativamente utile o contrattabile in corso d'opera.
						\end{itemize}
						\item \textbf{Tipologia:} ogni requisito assumerà uno dei seguenti valori:
						\begin{itemize}
							\item \textbf{F:} funzionale;
							\item \textbf{P:} prestazionale;
							\item \textbf{Q:} qualitativo;
							\item \textbf{V:} vincolo.
						\end{itemize}
						\item \textbf{Identificativo:} numero progressivo per contraddistinguere il requisito, in forma gerarchica padre-figlio strutturato come segue: 
						\begin{center}
							\textbf{[codicePadre].[codiceFiglio]}
						\end{center}
					\end{itemize}
				\subparagraph{Classificazione dei casi d'uso}
					Gli analisti, dopo la stesura dei requisiti, hanno anche il compito di identificare ed elencare i casi d’uso. Ognuno di essi è identificato, in maniera univoca, secondo il seguente schema identificativo:
					\begin{center}
						\textbf{UC[codiceCaso].[codiceSottoCaso].[codiceSottoSottoCaso]}
					\end{center}
					Ogni caso d'uso oltre al codice di identificazione deve contenere, integralmente o parzialmente, i seguenti campi:
					\begin{itemize}
						\item \textbf{diagrammi UML:} diagrammi realizzati usando la versione 2.0 del linguaggio;
						\item \textbf{attori primari:} attori principali del caso d’uso; 
						\item \textbf{attori secondari:} attori secondari del caso d’uso;
						\item \textbf{descrizione:} breve descrizione del caso d'uso;
						\item \textbf{attori secondari:} attori secondari del caso d’uso;
						\item \textbf{estensioni:} eventuali estensioni coinvolte;
						\item \textbf{inclusioni:} eventuali inclusioni coinvolte;
						\item \textbf{precondizione:} condizioni che devono essere soddisfatte perché si verifichino gli eventi del caso d’uso;
						\item \textbf{postcondizione:} condizioni che devono essere soddisfatte dopo il verificarsi degli eventi del caso d’uso;
						\item \textbf{scenario principale:} flusso degli eventi, in forma di elenco numerato, con eventuale riferimento ad ulteriori casi d’uso.
					\end{itemize}
				
			\paragraph {Progettazione}
				\subparagraph{Scopo}
					L'attività di progettazione precede quella di codifica ed avviene successivamente all'analisi dei requisiti.
					\newline
					In questa attività i progettisti hanno il compito di definire una soluzione del problema che sia soddisfacente per gli \glock{stakeholder}. 
					\newline
					Lo scopo della progettazione è definire l'architettura logica del prodotto da sviluppare e deve permettere di:
					\begin{itemize}
						\item garantire l'efficacia del prodotto, soddisfacendo tutti i requisiti individuati nell'attività di analisi, attraverso un sistema di qualità;
						\item garantire l'efficienza nella realizzazione del prodotto, impiegando parti riusabili con specifiche chiare, organizzate in modo da facilitarne la manutenzione e realizzabili con risorse sostenibili;
						\item gestire la complessità del sistema, suddividendolo fino ad ottenere delle parti di complessità trattabile, che possano essere fornite in ingresso all'attività di codifica come singoli compiti individuali che siano, quindi, fattibili, rapidi e verificabili.
					\end{itemize}
				
				\subparagraph{Aspettative}
					Le aspettative della progettazione è la definizione l'architettura logica del prodotto, la quale dovrà godere delle seguenti qualità:
					\begin{itemize}
						\item \textbf{sufficienza:} capace di soddisfare tutti i requisiti;
						\item \textbf{comprensibilità:} capibile da tutti gli stakeholder;
						\item \textbf{modularità:} suddivisibile in parti chiare e ben distinte, riducendo la dipendenza tra le parti stesse, per ridurre i cambiamenti esterni causati da modifiche interne ad una singola parte;
						\item \textbf{robustezza:} capace di sopportare ingressi diversi da parte dell'utente e dell'ambiente;
						\item \textbf{flessibilità:} modificabile a costo contenuto al variare dei requisiti;
						\item \textbf{efficienza nella gestione delle risorse;}
						\item \textbf{affidabilità:} svolge bene il suo compito, quando utilizzata;
						\item \textbf{disponibilità:} richiede un tempo di indisponibilità limitato o nullo per effettuare la manutenzione;
						\item \textbf{sicurezza:} non presenta malfunzionamenti gravi e non presenta vulnerabilità alle intrusioni;
						\item \textbf{semplicità:} composta di parti contenenti solamente il necessario e nulla di superfluo;
						\item \textbf{incapsulazione:} composta di parti il cui interno non risulta visibile dall'esterno, in modo da ridurre le dipendenze indotte sull'esterno e quindi facilitare la manutenzione;
						\item \textbf{coesione:} composta da parti raggruppate per obiettivi comuni, in modo da ridurre l'interdipendenza tra le componenti e quindi aumentare la manutenibilità e la comprensibilità;
						\item \textbf{basso accoppiamento:} composta da parti distinte che dipendono poco o niente le une dalle altre.
					\end{itemize}
				
				\subparagraph{Descrizione}
					Per perseguire la qualità nella progettazione dell'architettura è necessario seguire le seguenti regole:
					\begin{itemize}
						\item se possibile, prediligere sempre l'utilizzo di opportuni \glock{design pattern};
						\item evitare l'utilizzo dell'ereditarietà tra classi concrete, prediligendo sempre l'uso di classi astratte e/o interfacce;
						\item perseguire sempre il principio dell'incapsulamento e dell'\textit{information hiding}, limitando il più possibile la visibilità dei dettagli implementativi;
						\item assegnare sempre nomi significativi e parlati a package, classi, metodi e variabili;
						\item evitare la definizione di dipendenze circolari tra classi;
						\item se possibile, definire sempre relazioni tra componenti con il minor grado di dipendenza realizzabile.
					\end{itemize}
					L'attività di progettazione si articola nelle seguenti due parti: 
					\begin{itemize}
						\item \textbf{technology baseline:} consiste delle specifiche, alto livello, della progettazione del prodotto, delle sue componenti e dei test di verifica;
						\item \textbf{product baseline:} consiste delle specifiche di dettaglio della progettazione del prodotto, dell'insieme dei diagrammi UML che ne descrivono l'architettura e dei test necessari per la sua verifica, a partire da quanto è stato definito nella technology baseline.
					\end{itemize}
					
					\subparagraph{Technology baseline}
						La technology baseline include:
						\begin{itemize}
							\item \textbf{tecnologie utilizzate:} descrizione dettagliata delle tecnologie impiegate nello sviluppo del progetto, evidenziandone i pregi e i difetti riscontrati;
							\item \textbf{test di integrazione:} definizione dei test eseguiti per verificare che l'interazione tra le varie componenti del sistema funzioni correttamente e in modo conforme ai requisiti;
							\item \textbf{tracciamento delle componenti:} associazione tra requisiti e componenti che li soddisfano.
						\end{itemize}
					
					\subparagraph{Product baseline}
						La product baseline include:
						\begin{itemize}
							\item \textbf{\glock{design pattern}:} descrizione dei \glock{design pattern} utilizzati nella definizione dell'architettura, per la soluzione progettuale ai problemi \textit{ricorrenti} riscontrati; ogni \glock{design pattern} deve essere opportunamente descritto, con una spiegazione del suo significato, ed accompagnato da un diagramma che ne mostri la struttura;
							\item \textbf{\glock{diagrammi UML}:} diagrammi realizzati in linguaggio UML versione 2.0, utilizzati per rendere più chiare le soluzioni progettuali adottate; essi si suddividono in:
							\begin{itemize}
								\item \textbf{diagrammi dei package:} descrivono le dipendenze tra classi raggruppate in diversi package, ossia raggruppamenti di elementi in un'unità di livello più alto;
								\item \textbf{diagrammi delle classi:} rappresentano le classi che compongono il sistema e le relazioni di dipendenza che sussistono tra loro;			
								\item \textbf{diagrammi di sequenza:} descrivono la collaborazione di un gruppo di oggetti che devono realizzare un determinato comportamento;
								\item \textbf{diagrammi delle attività:} descrivono la logica procedurale di un processo, aiutando a descrivere gli aspetti dinamici dei casi d'uso.
							\end{itemize}
							\item \textbf{test di unità:} definizione dei test eseguiti per verificare che il funzionamento delle varie classi e metodi che implementano il sistema software sia corretto e conforme ai requisiti;
							\item \textbf{tracciamento delle classi:} associazione tra requisiti e classi che li soddisfano.
						\end{itemize}
	
			\paragraph{Codifica}
				\subparagraph{Scopo}	
					In questa attività vengono stese le norme alle quali i programmatori devono attenersi durante l’attività di programmazione ed implementazione.
				\subparagraph{Aspettative}
					 L’obiettivo è lo sviluppo del software richiesto dal proponente utilizzando le norme di programmazione stabilite in modo da:
					 	\begin{itemize}
					 	\item ottenere codice leggibile ed uniforme per i programmatori;
					 	\item agevolare le fasi di manutenzione, verifica e validazione.
					 \end{itemize} 
%				 \subparagraph{Stile di codifica}
%				 	Per garantire l'uniformità del codice, ciascun Programmatore dovrà attenersi alle seguenti regole norme di programmazione: 
%					\begin{itemize}
%						\item \textbf{Indentazione:} i blocchi del codice devono essere indentati, per ciascun livello, con tabulazione la cui larghezza sia impostata a quattro (4) spazi. Ogni programmatore dovrà configurare il proprio editor di testo secondo questa regola;
%						\item \textbf{Univocità dei nomi:} classi, metodi e variabili dovono avere nomi univoci e che ne descrivano il più possibile la funzione dove la prima lettera deve essere sempre minuscola e, nel caso in cui la classe/metodo/variabile sia una concatenazione di più parole, i programmatori devo attenersi al \glock{CamelCase}.
%						\item \textbf{Spazi:} prima di ogni apertura di parentesi graffa, tonda e quadra ci deve essere uno (1) spazio. Ogni chiusura di parentesi graffa per metodi, classi e condizioni va fatta andando a capo;
%					\end{itemize}
%					
%				 
%		\subsubsection{Strumenti}
%			Di seguito elencati gli strumenti che verranno utilizzati nella fase di sviluppo.
%				\subparagraph{Chrome}
%					Browser web sviluppato da Google, basato sul motore di rendering Blink.
%				\subparagraph{Visual Studio Code}
%					Visual Studio Code è un editor di codice sorgente sviluppato da Microsoft per Windows, Linux e macOS. Include il supporto per debugging.
%				\subparagraph{Draw.io}
%					Strumento open source semplice ed  intuitivo per la creazione dei diagrammi UML.
%				\subparagraph{Bootstrap}
%					Raccolta di strumenti liberi per la creazione di siti e applicazioni per il Web. Essa contiene modelli di progettazione basati su HTML e CSS per le varie componenti dell'interfaccia.
%				\subparagraph{Apache Kafka}
%					Piattaforma open source di stream processing scritta in Java sviluppata da Apache Software Foundation. Il progetto mira a creare una piattaforma a bassa latenza ed alta velocità per la gestione di feed dati in tempo reale;
%				\subparagraph{Docker}
%					Tecnologia di containerizzazione che consente la creazione e l'utilizzo di container Linux. Nel progetto risulterebbe utile per l'instanziazione di tutti i componenti;   
%				\subparagraph{API Producer, Consumer, Connect e Streams}	
%					API consigliate per la produzione di componenti custom per Kafka;
%
%			
\subsubsection{Strumenti}
		Per il processo di fornitura non sono stati individuati particolari strumenti da impiegare.