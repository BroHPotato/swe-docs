\subsection{Sviluppo}
		\subsubsection{Scopo}
			Il processo di sviluppo definisce i compiti e le attività da intraprendere volte al raggiungimento del prodotto finale richiesto dal proponente.
		\subsubsection{Aspettative}
			Per una corretta implementazione di questo processo è necessario fissare:
				\begin{itemize}
					\item Obbiettivi di sviluppo;
					\item Vincoli tecnologici e di design.
				\end{itemize}	
			Il prodotto finale deve rispettare i requisiti e le aspettative del proponente, superando i test definiti dalle norme di qualità.
		\subsubsection{Descrizione}
			Il processo di sviluppo, secondo lo standard ISO/IEC 12207:1995, si articola nelle seguenti attività:
				\begin{itemize}
					\item Analisi dei Requisiti;
					\item Progettazione;
					\item Codifica.
				\end{itemize}
		\subsubsection{Attività}
			Di seguito verranno analizzate dettagliatamente le attività menzionate nella sezione precedente.
			\subsubsubsection{Analisi dei requisiti}
				\subparagraph{Scopo \\  \\}
					Gli Analisti si occupano di stilare il documento di Analisi dei Requisiti, il cui scopo è appunto quello di definire ed elencare tutti i requisiti del capitolato. Il documento finale conterrà:
					\begin{itemize}
						\item Descrizione generale del prodotto;
						\item Argomentazioni precise ed affidabili per i Progettisti;
						\item Casi d'uso rappresentati tramite diagrammi UML;
						\item Fissare funzionalità e requisiti concordi con le richieste del cliente;
						\item Stima dei costi.
					\end{itemize}
				\subparagraph{Classificazione dei Requisiti \\ \\}
					I requisiti verranno classificati per facilitarne la comprensione e vengono identificati, in maniera univoca, secondo il seguente schema identificativo:
					\begin{center}
						\textbf{R[Priorità]-[Tipologia]-[Identificativo]}
					\end{center}
					Dove:
					\begin{itemize}
						\item \textbf{R:} Requisito 
						\item \textbf{Priorità:} ogni requisito assumerà uno dei seguenti valori:
						\begin{itemize}
							\item \textbf{A:} obbligatorio, strettamente necessario;
							\item \textbf{B:} desiderabile, non strettamente necessario;
							\item \textbf{C:} opzionale, relativamente utile o contrattabile in corso d'opera.
						\end{itemize}
						\item \textbf{Tipologia:} ogni requisito assumerà uno dei seguenti valori:
						\begin{itemize}
							\item \textbf{F:} funzionale;
							\item \textbf{P:} prestazionale;
							\item \textbf{Q:} qualitativo;
							\item \textbf{V:} vincolo.
						\end{itemize}
						\item \textbf{Identificativo:} numero progressivo per contraddistinguere il requisito, in forma gerarchica padre/figlio strutturato come segue: 
						\begin{center}
							\textbf{[codicePadre].[codiceFiglio]}
						\end{center}
					\end{itemize}
				\subparagraph{Classificazione dei Casi d'Uso \\ \\}
					Gli Analisti, dopo la stesura dei requisiti, hanno anche il compito di identificare ed elencare i casi d’uso. Ognuno di essi è identificato, in maniera univoca, secondo il seguente schema identificativo:
					\begin{center}
						\textbf{UC[codicePadre].[codiceFiglio]}
					\end{center}
					Ogni caso d'uso oltre al codice di identificazione deve contenere i seguenti campi:
					\begin{itemize}
						\item \textbf{Diagrammi UML:} diagrammi realizzati usando la versione 2.0 del linguaggio;
						\item \textbf{Attori primari:} attori principali del caso d’uso; 
						\item \textbf{Attori secondari:} attori secondari del caso d’uso;
						\item \textbf{Descrizione:} breve descrizione del caso d'uso;
						\item \textbf{Attori secondari:} attori secondari del caso d’uso;
						\item \textbf{Estensioni:} eventuali estensioni coinvolte;
						\item \textbf{Inclusioni:} eventuali inclusioni coinvolte;
						\item \textbf{Precondizione:} condizioni identificate come vere prima del verificarsi degli eventi del caso d’uso;
						\item \textbf{Postcondizione:} condizioni identificate come vere dopo il verificarsi degli eventi del caso d’uso;
						\item \textbf{Scenario principale:} flusso degli eventi come elenco numerato con eventuale riferimento ad ulteriori casi d’uso.
					\end{itemize}
				
			\subsubsubsection {Progettazione \\ \\}
				L'attività di progettazione avviene una volta concluso il documento di Analisi dei Requisiti, dove i Progettisti hanno il compito di definire una soluzione soddisfacente del problema e di realizzare l'architettura del sistema.
				Questa fase si divide nelle seguenti fasi: 
				\begin{itemize}
					\item \textbf{Tecnology baseline:} specifiche della progettazione del prodotto e delle sue componenti, insieme dei diagrammi UML dell'architettura ed i test di verifica;
					\item \textbf{Product baseline:} specifica maggiormente l'attività di progettazione e definisce i test necessari per la verifica. 
					\item \textbf{Diagrammi UML:} Diagrammi necessari per rendere più chiare le soluzioni progettuali utilizzate e si suddividono in:	
					\begin{itemize}
						\item \textbf{Diagrammi delle attività:} descrivono un processo o un algoritmo;
						\item \textbf{Diagrammi delle classi:} rappresentano gli oggetti del sistema e loro relazioni;
						\item \textbf{Diagrammi dei casi d'uso:} descrivono le funzioni offerte dal sistema;
						\item \textbf{Diagrammi dei package:} descrivono le dipendenze tra classi raggruppate in package;
						\item \textbf{Diagrammi di sequenza:} descrivono la una sequenza di processi o funzioni.
					\end{itemize}
					\item \textbf{Tecnologie utilizzate:} elenco dettagliato delle tecnologie impiegate; 
					\end{itemize}	
				
			\subsubsubsection{Codifica}
				\subparagraph{Scopo \\ \\}	
					In questa attività vengono stese le norme alle quali i Programmatori devono affidarsi durante l’attività di  programmazione ed implementazione.
				\subparagraph{Aspettative}
					 L’obiettivo è quello di sviluppare il software richiesto dal proponente utilizzando le norme di programmazione stabilite in modo da ottenere:
					 	\begin{itemize}
					 	\item codice leggibile ed uniforme per i Programmatori;
					 	\item agevolare le fasi di manutenzione, verifica e validazione.
					 \end{itemize} 
				 \subparagraph{Stile di codifica}
				 DA FARE
				 

		\subsubsection{Strumenti}
		Di seguito verranno elencati gli strumenti che verranno utilizzati nella fase di sviluppo.
			\subparagraph{Chrome \\}
			Browser web sviluppato da Google, basato sul motore di rendering Blink.
			\subparagraph{Visual Studio Code \\}
			Visual Studio Code è un editor di codice sorgente sviluppato da Microsoft per Windows, Linux e macOS. Include il supporto per debugging, un controllo per Git integrato, Syntax highlighting, IntelliSense, Snippet e refactoring del codice.
			\subparagraph{Draw.io \\}
			Strumento open source semplice ed  intuitivo per la creazione dei diagrammi UML.
			\subparagraph{Bootstrap \\}
			Raccolta di strumenti liberi per la creazione di siti e applicazioni per il Web. Essa contiene modelli di progettazione basati su HTML e CSS per le varie componenti dell'interfaccia.
			
			DA RIVEDERE
			