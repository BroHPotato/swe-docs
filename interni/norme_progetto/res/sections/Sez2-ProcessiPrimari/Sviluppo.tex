\subsection{Sviluppo}
		\subsubsection{Scopo}
			Il processo di sviluppo definisce i compiti e le attività da intraprendere per ottenere il prodotto finale richiesto dal proponente.
		\subsubsection{Aspettative}
			Per una corretta implementazione di questo processo è necessario fissare:
				\begin{itemize}
					\item obiettivi di sviluppo;
					\item vincoli tecnologici e di design.
				\end{itemize}
			Il prodotto finale deve rispettare i requisiti e le aspettative del proponente, superando i test definiti dalle norme di qualità.
		\subsubsection{Descrizione}
			Il processo di sviluppo, secondo lo standard ISO/IEC 12207:1995, si articola nelle seguenti attività:
				\begin{itemize}
					\item analisi dei requisiti;
					\item progettazione;
					\item codifica.
				\end{itemize}

		\subsubsection{Attività}
			Di seguito verranno analizzate dettagliatamente le attività menzionate nella sezione precedente.
			\paragraph{Analisi dei requisiti}
				\subparagraph{Scopo}
					Gli analisti si occupano di stilare il documento \dext{Analisi dei Requisiti v1.0.0}, il cui scopo è definire ed elencare tutti i requisiti del capitolato. Il documento finale conterrà:
					\begin{itemize}
						\item descrizione generale del prodotto;
						\item argomentazioni precise ed affidabili per i progettisti;
						\item casi d'uso rappresentati tramite diagrammi UML;
						\item funzionalità e requisiti concordi con le richieste del cliente;
						\item tracciamento dei requisiti individuati.
					\end{itemize}
				\subparagraph{Aspettative}
					Creazione del documento formale contente tutti i requisiti richiesti dal proponente per la realizzazione del capitolato.
				\subparagraph{Classificazione dei Requisiti}
					Al fine di facilitarne la consultazione e comprensione, i requisiti saranno classificati ed identificati univocamente secondo il seguente schema identificativo:
					\begin{center}
						\textbf{R[Priorità]-[Tipologia]-[Identificativo]}
					\end{center}
					Dove:
					\begin{itemize}
						\item \textbf{R:} requisito
						\item \textbf{Priorità:} ogni requisito assumerà uno dei seguenti valori:
						\begin{itemize}
							\item \textbf{A:} obbligatorio, strettamente necessario;
							\item \textbf{B:} desiderabile, non strettamente necessario;
							\item \textbf{C:} opzionale, relativamente utile o contrattabile in corso d'opera.
						\end{itemize}
						\item \textbf{Tipologia:} ogni requisito assumerà uno dei seguenti valori:
						\begin{itemize}
							\item \textbf{F:} funzionale;
							\item \textbf{P:} prestazionale;
							\item \textbf{Q:} qualitativo;
							\item \textbf{V:} vincolo.
						\end{itemize}
						\item \textbf{Identificativo:} numero progressivo per contraddistinguere il requisito, in forma gerarchica padre-figlio strutturato come segue:
						\begin{center}
							\textbf{[codicePadre].[codiceFiglio]}
						\end{center}
					\end{itemize}
				\subparagraph{Classificazione dei casi d'uso}
					Gli analisti, dopo la stesura dei requisiti, hanno anche il compito di identificare ed elencare i casi d’uso. Ognuno di essi è identificato, in maniera univoca, secondo il seguente schema identificativo:
					\begin{center}
						\textbf{UC[codiceCaso].[codiceSottoCaso].[codiceSottoSottoCaso]}
					\end{center}
					Ogni caso d'uso oltre al codice di identificazione deve contenere, integralmente o parzialmente, i seguenti campi:
					\begin{itemize}
						\item \textbf{diagrammi UML:} diagrammi realizzati usando la versione 2.0 del linguaggio;
						\item \textbf{attori primari:} attori principali del caso d’uso;
						\item \textbf{attori secondari:} attori secondari del caso d’uso;
						\item \textbf{descrizione:} breve descrizione del caso d'uso;
						\item \textbf{attori secondari:} attori secondari del caso d’uso;
						\item \textbf{estensioni:} eventuali estensioni coinvolte;
						\item \textbf{inclusioni:} eventuali inclusioni coinvolte;
						\item \textbf{precondizione:} condizioni che devono essere soddisfatte perché si verifichino gli eventi del caso d’uso;
						\item \textbf{postcondizione:} condizioni che devono essere soddisfatte dopo il verificarsi degli eventi del caso d’uso;
						\item \textbf{scenario principale:} flusso degli eventi, in forma di elenco numerato, con eventuale riferimento ad ulteriori casi d’uso.
					\end{itemize}

			\paragraph {Progettazione}
				\subparagraph{Scopo}
					L'attività di progettazione avviene una volta concluso il documento \dext{Analisi dei Requisiti v1.0.0}, in essa i progettisti hanno il compito di definire una soluzione soddisfacente del problema.
				\subparagraph{Aspettative}
					Realizzazione dell'architettura del sistema.
				\subparagraph{Descrizione}
					Questa fase si divide nelle seguenti fasi:
					\begin{itemize}
						\item \textbf{tecnology baseline:} specifiche della progettazione del prodotto e delle sue componenti, insieme dei diagrammi UML dell'architettura ed i test di verifica;
						\item \textbf{product baseline:} specifica più dettagliata dell'attività di progettazione e definisce i test necessari per la verifica;
						\item \textbf{diagrammi UML:} diagrammi utilizzati per rendere più chiare le soluzioni progettuali utilizzate; si suddividono in:
						\begin{itemize}
							\item \textbf{diagrammi delle attività:} descrivono un processo o un algoritmo;
							\item \textbf{diagrammi delle classi:} rappresentano gli oggetti del sistema e loro relazioni;
							\item \textbf{diagrammi dei casi d'uso:} descrivono le funzioni offerte dal sistema;
							\item \textbf{diagrammi dei package:} descrivono le dipendenze tra classi raggruppate in package;
							\item \textbf{diagrammi di sequenza:} descrivono una sequenza di processi o funzioni;
						\end{itemize}
						\item \textbf{tecnologie utilizzate:} elenco dettagliato delle tecnologie impiegate.
						\end{itemize}

			\paragraph{Codifica}
				\subparagraph{Scopo}
					L'obbiettivo di questa attività è di normare la concretizzazione del prodotto attraverso la programmazione. Quindi gli sviluppatori, durante la fase di implementazione, dovranno attenersi alle norme sotto elencate.
				\subparagraph{Aspettative}
					 L’obiettivo è lo sviluppo del software richiesto dal proponente utilizzando le norme di programmazione stabilite in modo da:
					 	\begin{itemize}
					 	\item ottenere codice leggibile ed uniforme per i programmatori;
						\item agevolare le fasi di manutenzione, verifica e validazione;
						\item fornire un prodotto conforme alle richieste prefissate dal proponente;
						\item creare un prodotto di qualità.
					 \end{itemize}
				 \subparagraph{Descrizione}
				 	Il codice scritto dovrà rispettare e perseguire quanto stabilito nel documento \dext{Piano di Qualifica} con il fine di fornire una buona qualità del codice.
				 \subparagraph{Stile di codifica}
				 	Per garantire l'uniformità del codice, ciascun sviluppatore dovrà attenersi alle seguenti norme di programmazione:
					\begin{itemize}
						\item \textbf{indentazione:} i blocchi del codice devono essere indentati, per ciascun livello, con tabulazione la cui larghezza sia impostata a quattro (4) spazi. Ogni programmatore dovrà configurare il proprio editor di testo secondo questa regola (l'indentazione dei commenti non viene considerata);
						\item \textbf{univocità dei nomi:} metodi e variabili devono avere nomi univoci e che ne descrivano il più possibile la funzione dove la prima lettera deve essere sempre minuscola e, nel caso in cui il metodo/variabile sia una concatenazione di più parole, i programmatori devo attenersi al \glock{CamelCase}.
						Per quanto riguarda le classi si applicano le regole esposte in precedenza ad eccezione della lettera iniziale del nome che sarà maiuscola;
						\item \textbf{parentesizzazione:} l'uso delle parentesi è obbligatorio per la disambiguazione, sia logica che non, delle operazioni lineari, es. somma, concatenazione in AND, etc.; inoltre, si impone, nell'uso delle parentesi graffe per la definizione di classi e metodi, di concatenare la parentesi di apertura con la stringa di definizione separandola con uno spazio, per quanto riguarda le parentesi tonde esse non dovranno essere separate da alcun spazio e/o indentazione. La stessa politica viene applicata per costrutti di selezione ed iterazione. Sono ammessi, ove applicabili, costrutti veloci, per esempio  \textit{if} (\textit{(cond1) ? "stringa1" : var1; }) veloci in \textit{php}, con la possibilità di non utilizzare le parentesi graffe. Viene riportato un esempio per chiarificare ulteriormante le prescrizioni sopra citate:
						\lstset{language=Java}
						\begin{lstlisting}
							public class Classe {
								  private int var1;
								  private int var2;
									private boolean var3;
					        public void metodo() {
									    while( true ){
											    if( var1 && ( var2 || var3 ))
														  var1 += var2;
													else
													    var2 = var1 + var2;
										}
									}
									private int altroMetodo() {
									    if( var3 ){
											  	return var3;
											} else {
												  return var1 + var2;
											}
									}
						  }
						\end{lstlisting}
						\item \textbf{spazi e caporiga:} prima di ogni apertura e/o chiusura di parentesi graffa, tonda e quadra ci deve essere uno (1) spazio, fatta eccezione per le parentesi tonde di definizione di metodi. Inoltre, le variabili e simboli dei costrutti lineari dovranno essere separati da uno (1) spazio. Ogni chiusura di parentesi graffa per metodi, classi e condizioni necessita di un caporiga, fatta eccezione dei costrutti logici \textit{if-else} nei quali la condizine di \textit{else} deve essere separata da uno (1) spazo dalla parentesi graffa di chiusura del costrutto \textit{if}.
					\end{itemize}


		\subsubsection{Metriche}


		\paragraph{QC-2 Funzionabilità}
			\subparagraph{Scopo}
				Durante lo sviluppo si vuole monitorare la capacità del prodotto di soddisfare tutti i requisiti richiesti dall'utente.
			\subparagraph{Obiettivi}
				\begin{itemize}
					\item \textbf{appropriatezza:} viene richiesto che il prodotto metta a disposizione tutte le funzionalità richieste dall'utente;
					\item \textbf{accuratezza:} il prodotto deve riuscire a produrre risultati che rispettano l'aspettativa ed il grado di precisione richiesti;
					\item \textbf{interoperabilità:} il prodotto deve essere in grado di interagire ed operare con tutti i sistemi e vincoli specificati;
					\item \textbf{conformità:} il prodotto deve aderire a standard e regolamenti noti;
					\item \textbf{sicurezza:} i dati sensibili utilizzati e generati dal prodotto devono essere disponibili esclusivamente agli utenti e/o coloro che risultano autorizzati all'uso di tali dati.
				\end{itemize}
			\subparagraph{Introduzione alle Metriche}
				Per la funzionabilità si é deciso di utilizzare la seguente metrica:
				\begin{itemize}
					\item QM-PROD-1 Implementazione (IMP).
				\end{itemize}
			\subparagraph{QM-PROD-1 Implementazione (IMP)}
			\begin{itemize}
      			\item \textbf{descrizione: }
					La metrica IMP si utilizza per valutare l'avanzamento dello sviluppo delle funzionalità richieste;
				\item \textbf{unità di misura: }
					La metrica è espressa in percentuale;
				\item \textbf{formula: }
					La formula della metrica è la seguente:
					\(
						IMP = \frac{\# funzionalita implementate}{\# funzionalita proposte}\times100
					\)
				\item \textbf{risultato: }
					Il risultato della formula ha i seguenti significati:
					\begin{itemize}
						\item se il risultato è pari a 0\%, allora nessuna funzionalità è stata implementata;
						\item se il risultato è compreso tra 0\% e 100\%, allora parte delle funzionalità sono state implementate;
						\item se il risultato è pari a 100\%, allora tutte le funzionalità sono state implementate.
					\end{itemize}
			\end{itemize}
		\paragraph{QC-3 Affidabilità}
			\subparagraph{Scopo}
				Durante lo sviluppo si vuole monitorare l'affidabilità e correttezza, così come la sua tolleranza agli errori.
			\subparagraph{Obiettivi}
				\begin{itemize}
					\item \textbf{maturità:} il software deve essere in grado di evitare il verificarsi di errori e/o malfunzionamenti derivanti dalla sua esecuzione;
					\item \textbf{tolleranza degli errori:} il prodotto é in grado di mantenere un livello minimo di prestazioni predeterminate anche in presenza di malfunzionamenti e/o usi impropri di esso;
					\item \textbf{recuperabilità:} il software, in seguito ad un errore e/o malfunzionamento, deve essere in grado di ripristinare uno stato di usabilità in un arco di tempo definito e di recuperare eventuali dati persi durante il suddetto lasso di tempo;
					\item \textbf{aderenza:} descrive la capacitá del prodotto di aderire alle specifiche relative all'affidabilità.
				\end{itemize}
			\subparagraph{Introduzione alle Metriche}
				Per l'affidabilità si é deciso di utilizzare le seguenti metriche:
				\begin{itemize}
					\item QM-PROD-2 Densità errori (DE);
					\item QM-PROD-3 Complessità dei test di classe (CTCLA).
				\end{itemize}
			\subparagraph{ QM-PROD-2 Densità errori (DE)}
			\begin{itemize}
      			\item \textbf{descrizione: }
					La metrica DE permette di misurare in maniera precisa la tolleranza e correttezza di ogni componente del prodotto, mostrando attraverso una percentuale la sua stabilità;
				\item \textbf{unità di misura: }
					La metrica è espressa in percentuale;
				\item \textbf{formula: }
					La formula della metrica è la seguente:
					\(DE = \frac{\# test passati}{\# test condotti}\times100\)
				\item \textbf{risultato: }
					Il risultato della formula ha i seguenti significati:
					\begin{itemize}
						\item se il risultato è pari a 0\%, allora il prodotto non è stato testato o ha fallito i test al quale è stato sottoposto;
						\item se il risultato è compreso tra 0\%, e 100\%, allora parte del prodotto è stato testato e/o il prodotto ha passato parte dei test;
						\item se il risultato è pari a 100\%, allora tutte le parti sono state testate e il prodotto ha passato tutti i test.
					\end{itemize}
			\end{itemize}
			\subparagraph{QM-PROD-3 Complessità dei test di classe (CTCLA)}
			\begin{itemize}
      			\item \textbf{descrizione: }
					La metrica CTCLA trova il suo utilizzo nel monitoraggio dei test che coinvolgono il prodotto e le sue componenti. Essa permette di sapere se ci sono componenti non testate;
				\item \textbf{unità di misura: }
					La metrica è espressa tramite un numero intero;
				\item \textbf{formula: }
					La formula della metrica è la seguente:
					\textit{CTCLA = \# dei test che coinvolgono la classe}
				\item \textbf{risultato: }
					Il risultato della formula ha i seguenti significati:
					\begin{itemize}
						\item se il risultato è pari a 0, allora il prodotto o componente non è stato incluso in alcun test;
						\item se il risultato è maggiore di 0, allora il prodotto o componente è stato incluso in almeno un test.
					\end{itemize}
			\end{itemize}
		\paragraph{QC-4 Efficienza}
			\subparagraph{Scopo}
				È necessari misurare l'efficienza del prodotto in modo da fornire all'utente un software usabile, piacevole e veloce.
				\subparagraph{Obiettivi}
					\begin{itemize}
						\item \textbf{comportamento nel tempo:} garanzia di tempi di elaborazione accettabili da parte del prodotto;
						\item \textbf{utilizzo di risorse:} utilizzo non eccessivo delle risorse a disposizione.
					\end{itemize}
			\subparagraph{Introduzione alle Metriche}
				Per l'efficienza si è deciso di utilizzare la seguente metrica:
				\begin{itemize}
					\item QM-PROD-4 Risposta media (RM).
				\end{itemize}
			\subparagraph{QM-PROD-4 Risposta media (RM)}
			\begin{itemize}
      			\item \textbf{descrizione: }
					La metrica RM punta a misurare l'efficienza di elaborazione del prodotto in modo da fornire all'utente un'esperienza piacevole di utilizzo;
				\item \textbf{unità di misura: }
					La metrica è espressa millisecondi (\textit{ms});
				\item \textbf{formula: }
					La formula della metrica è la seguente:
					\(
						RM = \frac{\sum_{n=1}^{z} tempo di risposta in ms}{z}
					\)
					dove $z$ è il numero di misurazioni effettuate;
				\item \textbf{risultato: }
					Il risultato della formula ha i seguenti significati:
					\begin{itemize}
						\item se il risultato è indica il tempo di risposta medio del prodotto.
					\end{itemize}
			\end{itemize}
		\paragraph{QC-5 Usabilità}
			\subparagraph{Scopo}
			Si vuole misurare l'usabilità del prodotto in modo da fornire all'utente un'esperienza piacevole durante il suo utilizzo, nonché fornire un prodotto facile da apprendere ed utilizzare.
			\subparagraph{Obiettivi}
				\begin{itemize}
					\item \textbf{comprensibilità:} determina la facilità di utilizzo e di comprensione del prodotto e delle sue funzionalità da parte dell'utente;
					\item \textbf{apprendibilità:} definisce il livello di impegno richiesto, da parte dell'utilizzatore, per imparare ad usare il prodotto;
					\item \textbf{operabilità:} stabilisce il grado con cui il software riesce a mettere il suo utilizzatore in condizione di sfruttare il prodotto per i suoi fini;
					\item \textbf{attrattiva:} la proprietà del software di produrre un'esperienza d'uso gradevole per l'utente.
				\end{itemize}
			\subparagraph{Introduzione alle Metriche}
				Per la funzionabilità si é deciso di utilizzare le seguenti metriche:
				\begin{itemize}
					\item QM-PROD-5 Profondità dell'albero delle azioni (PAA);
					\item QM-PROD-6 Profondità dell'albero delle pagine (PAP).
				\end{itemize}
			\subparagraph{QM-PROD-5 Profondità dell'albero delle azioni (PAA)}
			\begin{itemize}
      			\item \textbf{descrizione: }
					La metrica PAA permette di misurare l'operabilità e la comprensibilità del prodotto da parte dell'utente. Essa misura il numero di azioni effettuate dall'utente prima di poter arrivare al suo obbiettivo;
				\item \textbf{unità di misura: }
					La metrica è espressa tramite un numero intero;
				\item \textbf{formula: }
					La formula della metrica è la seguente:
					\textit{PAA = \# delle azioni}
					dove ogni click corrisponde ad un'azione;
				\item \textbf{risultato: }
					Il risultato della formula ha i seguenti significati:
					\begin{itemize}
						\item se il risultato è pari 0 allora l'obbiettivo non è raggiungibile e/o l'utente non ne ha accesso;
						\item se il risultato è maggiore di 0 allora l'obbiettivo è raggiungibile in un numero finito di azioni.
					\end{itemize}
			\end{itemize}
			\subparagraph{QM-PROD-6 Profondità dell'albero delle pagine (PAP)}
			\begin{itemize}
      			\item \textbf{descrizione: }
					La metrica PAP permette di misurare l'apprendibilità, l'operabilità e l'attrattiva del prodotto dal punto di vista del cliente. Essa misura il numero di pagine visitate dall'utente prima di poter arrivare al suo obbiettivo;
				\item \textbf{unità di misura: }
					La metrica è espressa tramite un numero intero;
				\item \textbf{formula: }
					La formula della metrica è la seguente:
					\textit{PAP = \# delle pagine visitate dall'utente}
				\item \textbf{risultato: }
					Il risultato della formula ha i seguenti significati:
					\begin{itemize}
						\item se il risultato è pari a 0 allora la pagina obbiettivo non esiste o l'utente non ne ha accesso;
						\item se il risultato è maggiore di 0 allora la pagina obbiettivo è raggiungibile in un numero finito di passaggi.
					\end{itemize}
			\end{itemize}
		\paragraph{QC-6 Manutenibilità}
			\subparagraph{Scopo}
				La manutenibilità viene monitorata in modo da fornire un prodotto modificabile ed estendibile, in modo da poter estendere facilmente sia la vita del prodotto che le sue funzionalità.
				\begin{itemize}
					\item \textbf{analizzabilità:} determina la facilità con cui é possibile analizzare e localizzare un errore all'interno del codice;
					\item \textbf{modificabilità:} definisce la capacitá del prodotto di apportare una modifica o una estensione;
					\item \textbf{stabilità:} il software deve essere un grado di essere usato anche in caso le modifiche apportate siano errate;
					\item \textbf{testabilità:} determina la capacità del software di essere testato facilmente per fornire una validazione delle modifica apportate.
				\end{itemize}
			\subparagraph{Introduzione alle Metriche}
				Per la funzionabilità si é deciso di utilizzare le seguenti metricche:
				\begin{itemize}
					\item QM-PROD-QM-PROD-7 Complessità del codice (CCOD);
					\item QM-PROD-8 Complessità della classe (CCLA);
					\item QM-PROD-9 Complessità del metodo (CMET).
				\end{itemize}
			\subparagraph{QM-PROD-7 Complessità del codice (CCOD)}
			\begin{itemize}
      			\item \textbf{descrizione: }
					La metrica CCOD permette di misurare la chiarezza dei commenti rispetto la lunghezza del codice scritto;
				\item \textbf{unità di misura: }
					La metrica è espressa tramite un numero reale;
				\item \textbf{formula: }
					La formula della metrica è la seguente:
				 \(
				 		CCOD = \frac{\# linee commento}{\# linee codice}
				 \)
				\item \textbf{risultato: }
					Il risultato della formula ha i seguenti significati:
					\begin{itemize}
						\item se il risultato è pari a 0, allora il codice non è stato commentato;
						\item se il risultato è maggiore di 0, allora il codice ha almeno una riga di commento.
					\end{itemize}
			\end{itemize}
			\subparagraph{QM-PROD-8 Complessità della classe (CCLA)}
			\begin{itemize}
      			\item \textbf{descrizione: }
					La metrica CCLA misure il numero di metodi presenti in ogni classe e permette di valutare la complessità generale della stessa;
				\item \textbf{unità di misura: }
					La metrica è espressa tramite un numero intero;
				\item \textbf{formula: }
					La formula della metrica è la seguente:
					\textit{CCLA = \# numero metodi}
				\item \textbf{risultato: }
					Il risultato della formula ha i seguenti significati:
					\begin{itemize}
						\item se il risultato è pari a 0, allora la classe non esiste e/o non ha metodi che le appartengono;
						\item se il risultato è maggiore di 0, allora la classe esiste ed ha un numero finito di metodi.
					\end{itemize}
			\end{itemize}
			\subparagraph{QM-PROD-9 Complessità del metodo (CMET)}
			\begin{itemize}
      			\item \textbf{descrizione: }
					La metrica CMET permette di valutare la complessità di ogni metodo rispetto la sua modificabilità, ossia permette di capire se un metodo è una \glock{maschera} o esegue azioni rilevanti;
				\item \textbf{unità di misura: }
					La metrica è espressa tramite un numero reale;
				\item \textbf{formula :}
					La formula della metrica è la seguente:
					\(
						CMET = \frac{\# linee codice}{\# chiamate interne ad altri metodi+1}
					\)
				\item \textbf{risultato: }
					Il risultato della formula ha i seguenti significati:
					\begin{itemize}
						\item se il risultato è pari a 0, allora il metodo non esiste e/o non è stato implementato;
						\item se il risultato è maggiore di 0, allora il metodo è stato implementato ed potrebbe effettuare chiamate interne;
					\end{itemize}
					Più è alto il risultato meno è la complessità del metodo.
			\end{itemize}
%				 \subparagraph{Stile di codifica}
%				 	Per garantire l'uniformità del codice, ciascun Programmatore dovrà attenersi alle seguenti regole norme di programmazione:
%					\begin{itemize}
%						\item \textbf{Indentazione:} i blocchi del codice devono essere indentati, per ciascun livello, con tabulazione la cui larghezza sia impostata a quattro (4) spazi. Ogni programmatore dovrà configurare il proprio editor di testo secondo questa regola;
%						\item \textbf{Univocità dei nomi:} classi, metodi e variabili dovono avere nomi univoci e che ne descrivano il più possibile la funzione dove la prima lettera deve essere sempre minuscola e, nel caso in cui la classe/metodo/variabile sia una concatenazione di più parole, i programmatori devo attenersi al \glock{CamelCase}.
%						\item \textbf{Spazi:} prima di ogni apertura di parentesi graffa, tonda e quadra ci deve essere uno (1) spazio. Ogni chiusura di parentesi graffa per metodi, classi e condizioni va fatta andando a capo;
%					\end{itemize}
%
%
%		\subsubsection{Strumenti}
%			Di seguito elencati gli strumenti che verranno utilizzati nella fase di sviluppo.
%				\subparagraph{Chrome}
%					Browser web sviluppato da Google, basato sul motore di rendering Blink.
%				\subparagraph{Visual Studio Code}
%					Visual Studio Code è un editor di codice sorgente sviluppato da Microsoft per Windows, Linux e macOS. Include il supporto per debugging.
%				\subparagraph{Draw.io}
%					Strumento open source semplice ed  intuitivo per la creazione dei diagrammi UML.
%				\subparagraph{Bootstrap}
%					Raccolta di strumenti liberi per la creazione di siti e applicazioni per il Web. Essa contiene modelli di progettazione basati su HTML e CSS per le varie componenti dell'interfaccia.
%				\subparagraph{Apache Kafka}
%					Piattaforma open source di stream processing scritta in Java sviluppata da Apache Software Foundation. Il progetto mira a creare una piattaforma a bassa latenza ed alta velocità per la gestione di feed dati in tempo reale;
%				\subparagraph{Docker}
%					Tecnologia di containerizzazione che consente la creazione e l'utilizzo di container Linux. Nel progetto risulterebbe utile per l'instanziazione di tutti i componenti;
%				\subparagraph{API Producer, Consumer, Connect e Streams}
%					API consigliate per la produzione di componenti custom per Kafka;
%
%
\subsubsection{Strumenti}
		Per il processo di fornitura non sono stati individuati particolari strumenti da impiegare.
