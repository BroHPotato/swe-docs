\subsection{Sviluppo}
		\subsubsection{Scopo}
			Il processo di sviluppo definisce i compiti e le attività da intraprendere per ottenere il prodotto finale richiesto dal proponente.
		\subsubsection{Aspettative}
			Per una corretta implementazione di questo processo è necessario fissare:
				\begin{itemize}
					\item obiettivi di sviluppo;
					\item vincoli tecnologici e di design.
				\end{itemize}	
			Il prodotto finale deve rispettare i requisiti e le aspettative del proponente, superando i test definiti dalle norme di qualità.
		\subsubsection{Descrizione}
			Il processo di sviluppo, secondo lo standard ISO/IEC 12207:1995, si articola nelle seguenti attività:
				\begin{itemize}
					\item analisi dei requisiti;
					\item progettazione;
					\item codifica.
				\end{itemize}
			
		\subsubsection{Attività}
			Di seguito verranno analizzate dettagliatamente le attività menzionate nella sezione precedente.
			\paragraph{Analisi dei requisiti}
				\subparagraph{Scopo}
					Gli analisti si occupano di stilare il documento \dext{Analisi dei Requisiti v1.0.0}, il cui scopo è definire ed elencare tutti i requisiti del capitolato. Il documento finale conterrà:
					\begin{itemize}
						\item descrizione generale del prodotto;
						\item argomentazioni precise ed affidabili per i progettisti;
						\item casi d'uso rappresentati tramite diagrammi UML;
						\item funzionalità e requisiti concordi con le richieste del cliente;
						\item tracciamento dei requisiti individuati. 
					\end{itemize}
				\subparagraph{Aspettative}
					Creazione del documento formale contente tutti i requisiti richiesti dal proponente per la realizzazione del capitolato.
				\subparagraph{Classificazione dei Requisiti}
					Al fine di facilitarne la consultazione e comprensione, i requisiti saranno classificati ed identificati univocamente secondo il seguente schema identificativo:
					\begin{center}
						\textbf{R[Priorità]-[Tipologia]-[Identificativo]}
					\end{center}
					Dove:
					\begin{itemize}
						\item \textbf{R:} requisito 
						\item \textbf{Priorità:} ogni requisito assumerà uno dei seguenti valori:
						\begin{itemize}
							\item \textbf{A:} obbligatorio, strettamente necessario;
							\item \textbf{B:} desiderabile, non strettamente necessario;
							\item \textbf{C:} opzionale, relativamente utile o contrattabile in corso d'opera.
						\end{itemize}
						\item \textbf{Tipologia:} ogni requisito assumerà uno dei seguenti valori:
						\begin{itemize}
							\item \textbf{F:} funzionale;
							\item \textbf{P:} prestazionale;
							\item \textbf{Q:} qualitativo;
							\item \textbf{V:} vincolo.
						\end{itemize}
						\item \textbf{Identificativo:} numero progressivo per contraddistinguere il requisito, in forma gerarchica padre-figlio strutturato come segue: 
						\begin{center}
							\textbf{[codicePadre].[codiceFiglio]}
						\end{center}
					\end{itemize}
				\subparagraph{Classificazione dei casi d'uso}
					Gli analisti, dopo la stesura dei requisiti, hanno anche il compito di identificare ed elencare i casi d’uso. Ognuno di essi è identificato, in maniera univoca, secondo il seguente schema identificativo:
					\begin{center}
						\textbf{UC[codiceCaso].[codiceSottoCaso].[codiceSottoSottoCaso]}
					\end{center}
					Ogni caso d'uso oltre al codice di identificazione deve contenere, integralmente o parzialmente, i seguenti campi:
					\begin{itemize}
						\item \textbf{diagrammi UML:} diagrammi realizzati usando la versione 2.0 del linguaggio;
						\item \textbf{attori primari:} attori principali del caso d’uso; 
						\item \textbf{attori secondari:} attori secondari del caso d’uso;
						\item \textbf{descrizione:} breve descrizione del caso d'uso;
						\item \textbf{attori secondari:} attori secondari del caso d’uso;
						\item \textbf{estensioni:} eventuali estensioni coinvolte;
						\item \textbf{inclusioni:} eventuali inclusioni coinvolte;
						\item \textbf{precondizione:} condizioni che devono essere soddisfatte perché si verifichino gli eventi del caso d’uso;
						\item \textbf{postcondizione:} condizioni che devono essere soddisfatte dopo il verificarsi degli eventi del caso d’uso;
						\item \textbf{scenario principale:} flusso degli eventi, in forma di elenco numerato, con eventuale riferimento ad ulteriori casi d’uso.
					\end{itemize}
				
			\paragraph {Progettazione}
				\subparagraph{Scopo}
					L'attività di progettazione avviene una volta concluso il documento \dext{Analisi dei Requisiti v1.0.0}, in essa i progettisti hanno il compito di definire una soluzione soddisfacente del problema.
				\subparagraph{Aspettative}
					Realizzazione dell'architettura del sistema.
				\subparagraph{Descrizione}
					Questa fase si divide nelle seguenti fasi: 
					\begin{itemize}
						\item \textbf{tecnology baseline:} specifiche della progettazione del prodotto e delle sue componenti, insieme dei diagrammi UML dell'architettura ed i test di verifica;
						\item \textbf{product baseline:} specifica più dettagliata dell'attività di progettazione e definisce i test necessari per la verifica;
						\item \textbf{diagrammi UML:} diagrammi utilizzati per rendere più chiare le soluzioni progettuali utilizzate; si suddividono in:	
						\begin{itemize}
							\item \textbf{diagrammi delle attività:} descrivono un processo o un algoritmo;
							\item \textbf{diagrammi delle classi:} rappresentano gli oggetti del sistema e loro relazioni;
							\item \textbf{diagrammi dei casi d'uso:} descrivono le funzioni offerte dal sistema;
							\item \textbf{diagrammi dei package:} descrivono le dipendenze tra classi raggruppate in package;
							\item \textbf{diagrammi di sequenza:} descrivono una sequenza di processi o funzioni;
						\end{itemize}
						\item \textbf{tecnologie utilizzate:} elenco dettagliato delle tecnologie impiegate.
						\end{itemize}	
				
			\paragraph{Codifica}
				\subparagraph{Scopo}	
					In questa attività vengono stese le norme alle quali i programmatori devono attenersi durante l’attività di programmazione ed implementazione.
				\subparagraph{Aspettative}
					 L’obiettivo è lo sviluppo del software richiesto dal proponente utilizzando le norme di programmazione stabilite in modo da:
					 	\begin{itemize}
					 	\item ottenere codice leggibile ed uniforme per i programmatori;
					 	\item agevolare le fasi di manutenzione, verifica e validazione.
					 \end{itemize} 
%				 \subparagraph{Stile di codifica}
%				 	Per garantire l'uniformità del codice, ciascun Programmatore dovrà attenersi alle seguenti regole norme di programmazione: 
%					\begin{itemize}
%						\item \textbf{Indentazione:} i blocchi del codice devono essere indentati, per ciascun livello, con tabulazione la cui larghezza sia impostata a quattro (4) spazi. Ogni programmatore dovrà configurare il proprio editor di testo secondo questa regola;
%						\item \textbf{Univocità dei nomi:} classi, metodi e variabili dovono avere nomi univoci e che ne descrivano il più possibile la funzione dove la prima lettera deve essere sempre minuscola e, nel caso in cui la classe/metodo/variabile sia una concatenazione di più parole, i programmatori devo attenersi al \glock{CamelCase}.
%						\item \textbf{Spazi:} prima di ogni apertura di parentesi graffa, tonda e quadra ci deve essere uno (1) spazio. Ogni chiusura di parentesi graffa per metodi, classi e condizioni va fatta andando a capo;
%					\end{itemize}
%					
%				 
%		\subsubsection{Strumenti}
%			Di seguito elencati gli strumenti che verranno utilizzati nella fase di sviluppo.
%				\subparagraph{Chrome}
%					Browser web sviluppato da Google, basato sul motore di rendering Blink.
%				\subparagraph{Visual Studio Code}
%					Visual Studio Code è un editor di codice sorgente sviluppato da Microsoft per Windows, Linux e macOS. Include il supporto per debugging.
%				\subparagraph{Draw.io}
%					Strumento open source semplice ed  intuitivo per la creazione dei diagrammi UML.
%				\subparagraph{Bootstrap}
%					Raccolta di strumenti liberi per la creazione di siti e applicazioni per il Web. Essa contiene modelli di progettazione basati su HTML e CSS per le varie componenti dell'interfaccia.
%				\subparagraph{Apache Kafka}
%					Piattaforma open source di stream processing scritta in Java sviluppata da Apache Software Foundation. Il progetto mira a creare una piattaforma a bassa latenza ed alta velocità per la gestione di feed dati in tempo reale;
%				\subparagraph{Docker}
%					Tecnologia di containerizzazione che consente la creazione e l'utilizzo di container Linux. Nel progetto risulterebbe utile per l'instanziazione di tutti i componenti;   
%				\subparagraph{API Producer, Consumer, Connect e Streams}	
%					API consigliate per la produzione di componenti custom per Kafka;
%
%			
\subsubsection{Strumenti}
		Per il processo di fornitura non sono stati individuati particolari strumenti da impiegare.