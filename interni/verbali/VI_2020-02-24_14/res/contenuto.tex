\section*{Introduzione}

\subsection*{Luogo e data dell'incontro}
	\begin{itemize}
		\item \textbf{luogo:} videoconferenza su \glock{Discord};
		\item \textbf{data:} 2020-02-24;
		\item \textbf{ora di inizio:} 14:45;
		\item \textbf{ora di fine:} 15:30.
	\end{itemize}

\subsection*{Ordine del giorno}
	\begin{enumerate}
			\item richiesta di conferma per l'appuntamento con il proponente per il giorno 2020-02-26.
			\item decisione della lingua per realizzare la documentazione delle API.
			\item suddivisione dei ruoli per l'incremento settimanale.
  			\item varie ed eventuali.
	\end{enumerate}

\subsection*{Presenze}
	\begin{itemize}
		\item \textbf{totale presenti:} 6 su 7
		\item \textbf{presenti: }
			\begin{itemize}			
				\item Nicolò Frison;
				\item Giuseppe Vito Bitetti;
				\item Fouad Mouad;
				\item Mariano Sciacco (segretario);
				\item Alessandro Tommasin;
				\item Giovanni Vidotto.
			\end{itemize}
		\item \textbf{assenti: } 
			\begin{itemize}	
				\item Lorenzo Dei Negri (giustificato).
			\end{itemize}
	\end{itemize}


\newpage
\section*{Svolgimento}

	\subsection*{Richiesta di conferma per l'appuntamento con il proponente per il giorno 2020-02-26}
		Nel corso della settimana precedente, è stato richiesto tramite \glock{Slack} un appuntamento di persona con il proponente, per mostrare i progressi fatti e i cambiamenti eseguiti in materia di documentazione.
		\newline
		L'appuntamento era fissato per il 26 febbraio 2020, ore 16:00 in sede del proponente. A seguito dell'annuncio dell'epidemia che si è diffusa nel Padovano, il proponente ha preferito rimandare la data dell'appuntamento e ci ha proposto, in alternativa, di realizzare, nel corso della settimana, un \textbf{video} per mostrare il \glock{proof of concept} fino ad ora realizzato, così da stimolare lo \textit{Smart Working}.
		\newline
		La proposta è stata accolta dai membri del gruppo e pertanto, visto il buon avanzamento con gli incrementi attuali e valutata l'onerosità della proposta, si è deciso di realizzare un breve video dimostrativo di pochi minuti per mostrare la bontà del lavoro svolto fino ad ora.
		\newline
		Il proponente, infine, ha accolto di riceverci in data 5 marzo 2020, ore 15:00 in sede, a meno di diverse indicazioni aziendali.

	\subsection*{Decisione della lingua per realizzare la documentazione delle API.}
		È stata messa ai voti la decisione sulla lingua da utilizzare per le API, in modo definitivo, così da permettere a tutti di adattarsi nel corso dello sviluppo, in vista della documentazione da realizzare.
		\newline
		Tutti sono stati favorevoli all'uso della \textbf{lingua inglese} per realizzare le funzionalità e la relativa documentazione.

	\subsection*{Suddivisione dei ruoli per l'incremento settimanale.}
		È stato fatto un brevissimo riepilogo dell'incremento settimanale da portare a termine e sono stati aggiustati gli appuntamenti in base allo sviluppo dell'epidemia diffusa nel Padovano.
		\newline
		Si è deciso di proseguire, per questa settimana, con gli incontri da remoto utilizzando \glock{Discord} per effettuare delle videoconferenze. Sono stati assegnati i ruoli per la settimana in base alla disponibilità di tutti.

	\subsection*{Varie ed eventuali}
		Sono state fatte ulteriori considerazioni da parte dei membri del gruppo in merito al video da realizzare:
		\begin{itemize}
			\item la preparazione del video non comporta troppo onere di lavoro aggiuntivo dal momento che parte delle slide potranno essere riutilizzate per le presentazioni successive con i committenti;
			\item il video consisterà in una registrazione del \glock{proof of concept} fino ad ora realizzato unita ad una eventuale audio-descrizione. 
		\end{itemize}