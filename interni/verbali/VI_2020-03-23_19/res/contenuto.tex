\section*{Introduzione}

\subsection*{Luogo e data dell'incontro}
	\begin{itemize}
		\item \textbf{luogo:} videoconferenza su \glock{Discord};
		\item \textbf{data:} 2020-03-23;
		\item \textbf{ora di inizio:} 10:30;
		\item \textbf{ora di fine:} 13:00.
	\end{itemize}

\subsection*{Ordine del giorno}
	\begin{enumerate}
			\item discussione sulla valutazione e sulle correzioni ricevute nella revisione di progettazione;
			\item discussione sulla realizzazione del manuale utente e del manutentore;
			\item suddivisione dei ruoli per l'incremento settimanale e per la correzione dei documenti.
	\end{enumerate}

\subsection*{Presenze}
	\begin{itemize}
		\item \textbf{totale presenti:} 7 su 7
		\item \textbf{presenti: }
			\begin{itemize}
				\item Giuseppe Vito Bitetti;
				\item Lorenzo Dei Negri;
				\item Nicolò Frison;
				\item Fouad Mouad;
				\item Mariano Sciacco;
				\item Alessandro Tommasin;
				\item Giovanni Vidotto (segretario).
			\end{itemize}
		\item \textbf{assenti: }
			\begin{itemize}
				\item nessuno.
			\end{itemize}
	\end{itemize}


\newpage
\section*{Svolgimento}

	\subsection*{Discussione sulla valutazione e sulle correzioni ricevute nella revisione di progettazione}
		Si è discusso sulle correzioni ricevute nella revisione di progettazione, dove è emerso che la maggior parte delle segnalazioni fatte sono state ben comprese e di conseguenza si è individuato metodo più opportuno per sistemarle ma, dal momento che le segnalazioni ricevute sul \dext{Piano di Progetto} e sul versionamento dei documenti richiedono una maggiore delucidazione, si è deciso di richiedere un colloquio al professore per discuterne collettivamente. 

	\subsection*{Discussione sulla realizzazione del manuale utente e del manutentore}
		C'è stato un confronto interno per decidere se realizzare il manuale utente e del manutentore in forma completamente testuale, oppure con la realizzazione di alcuni video guida per ogni componente del prodotto. Si è optato per la seconda opzione considerandola più semplice, elegante e intuitiva per gli utenti e i manutentori. 
		
	\subsection*{Suddivisione dei ruoli per l'incremento settimanale e per la correzione dei documenti}
		È stato fatto un riepilogo dell'incremento settimanale da portare a termine suddividendo, nel modo più equilibrato possibile vista la completa disponibilità dei componenti del gruppo, gli obbiettivi da portare a termine. Inoltre verranno aggiornati i documenti in base alle segnalazioni ricevute, con la premessa di rivalutare le correzioni, non completamente comprese dal gruppo, dopo che sarà avvenuto il colloquio con il professore.