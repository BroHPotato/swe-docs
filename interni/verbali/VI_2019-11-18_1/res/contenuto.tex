\section*{Introduzione}

\subsection*{Luogo e data dell'incontro}
	\begin{itemize}
		\item \textbf{luogo:} Dipartimento di Matematica, Padova;
		\item \textbf{data:} 18 novembre 2019;
		\item \textbf{ora di inizio:} 14:00;
		\item \textbf{ora di fine:} 16:50.
	\end{itemize}

\subsection*{Ordine del giorno}
	\begin{enumerate}

		\item opinioni sulla scelta del \glock{capitolato} e sui capitolati presentati;
		\item decisione sul nome del gruppo e sul logo;
		\item raccolta delle presenze ai seminari di approfondimento;
		\item prime idee sulle norme di progetto e sulla documentazione da preparare;
		\item fissare la data per una nuova riunione del gruppo;
		\item varie ed eventuali.

	\end{enumerate}

\subsection*{Presenze}
	\begin{itemize}
		\item \textbf{totale presenti:} 6 su 7
		\item \textbf{presenti: }
			\begin{itemize}			
				\item Giuseppe Vito Bitetti;
				\item Lorenzo Dei Negri;
				\item Nicolò Frison;
				\item Mariano Sciacco (segretario);
				\item Alessandro Tommasin;
				\item Giovanni Vidotto.
			\end{itemize}
		\item \textbf{assenti: } 
			\begin{itemize}	
				\item Fouad Mouad (giustificato).
			\end{itemize}
	\end{itemize}


\newpage
\section*{Svolgimento}

\subsection*{1. Opinioni sui capitolati}

Si è discusso con ciascun membro del team riguardo a tutti i \glock{capitolati} presentati nella stessa giornata in cui si è svolta la riunione.
\newline
In generale, è stata espressa una certa decisione riguardo allo scarto di alcuni \glock{capitolati}, anche sulla base della complessità e della preparazione di tutti.

\subsection*{2. Decisione del nome del gruppo e del logo}

A seguito di un breve \glock{brainstorming}, è stato deliberato su decisione comune il nome del team e il relativo logo.

\begin{itemize}
	\item \textbf{nome del team:} Red Round Robin;
	\item \textbf{logo:} una rondine rossa con attorno una freccia rossa circolare.
\end{itemize}

\subsection*{3. Presenze ai seminari di approfondimento}

Tutti i membri del gruppo hanno confermato la presenza ai prossimi seminari di approfondimento dei capitolati.

\subsection*{4. Idee sulle norme di progetto e sulla documentazione}


A seguito di una breve discussione, si è deciso di concentrarsi con maggiore priorità sulle norme che riguardano il \glock{way of working}, la gestione della \glock{repository} e delle comunicazioni, così da rendere più agevole le operazioni iniziali di configurazione.
Nelle prossime riunioni si studieranno i punti in comune che riguardano le convenzioni comuni sulla documentazione e sul software, le nomenclature e gli strumenti da utilizzare.
\newline
Da parte di tutti si è concordato di collaborare sulla scrittura dei documenti utilizzando \LaTeX{} e usando come mezzo principale di comunicazione \glock{Slack}. Per la parte di versionamento è stato confermato l'uso di \glock{Github} con il relativo \glock{Issue Tracking System}.

\subsection*{5. Prossima riunione}

La prossima riunione è stata fissata come segue:
\begin{itemize}
	\item \textbf{luogo:} Dipartimento di Matematica, Padova;
	\item \textbf{data:} 21 novembre 2019;
	\item \textbf{ora di inizio:} 12:00.
\end{itemize}


\subsection*{6. Varie ed eventuali}

\begin{itemize}

	\item è stata consigliata la lettura di alcune guide circa l'utilizzo di \LaTeX{};
	\item entro la prossima settimana verrà eseguita una prima configurazione della \glock{repository} del progetto e di \glock{Slack}.
\end{itemize}


