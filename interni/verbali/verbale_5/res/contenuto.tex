\section*{Introduzione}

\subsection*{Luogo e data dell'incontro}
	\begin{itemize}
		\item \textbf{Luogo:} Dipartimento di Matematica, Padova;

		\item \textbf{Data:} 10 dicembre 2019;

		\item \textbf{Ora di inizio:} 14:15;
		\item \textbf{Ora di fine:} 18:00.
	\end{itemize}

\subsection*{Ordine del giorno}
	\begin{enumerate}

		\item Rilevazione e discussione di eventuali dubbi sulle \dext{norme di progetto};
		\item scelta del \glock{capitolato};
		\item prime idee sull'\dext{analisi dei requisiti};
		\item definizione di un primo contatto con il proponente;
		\item fissare la data per una nuova riunione del gruppo.

	\end{enumerate}

\subsection*{Presenze}
	\begin{itemize}
		\item \textbf{Totale presenti:} 7 su 7
		\item \textbf{Presenti: }

			\begin{itemize}			
				\item Giuseppe Vito Bitetti;
				\item Lorenzo Dei Negri (Segretario);

				\item Nicolò Frison;
				\item Fouad Mouad;
				\item Mariano Sciacco;
				\item Alessandro Tommasin;
				\item Giovanni Vidotto.
			\end{itemize}

		\item \textbf{Assenti: } 
			\begin{itemize}	

				\item Nessuno.
			\end{itemize}
	\end{itemize}


\newpage
\section*{Svolgimento}

\subsection*{1. Rilevazione e discussione di eventuali dubbi sulle norme di progetto}

A turno, ogni membro del gruppo ha esposto agli altri le problematiche riscontate nella stesura delle \dext{norme di progetto} a lui assegnate; queste potevano essere sia delle incertezze su come proseguire un particolare tema da trattare, sia delle scelte da prendere che richiedevano una decisione collettiva.
Ogni dubbio è stato discusso da tutti i presenti arrivando, in breve tempo, ad una risoluzione condivisa all'unanimità.
Le criticità principali riguardavano:
\begin{itemize}
	\item lo scopo di alcune attività dei processi primari;
	\item lo scopo di alcuni processi di supporto;
	\item l'istanziazione dei processi organizzativi;
	\item il tracciamento di requisiti e casi d'uso.
\end{itemize}

\subsection*{2. Scelta del capitolato}

In seguito alla partecipazione a tutti i seminari tenuti dai proponenti, facendo anche riferimento alla prima stesura dello \dext{studio di fattibilità}, il gruppo si è confrontato internamente per dare una valutazione complessiva di tutti i \glock{capitolati} proposti e fissare gli aspetti positivi e gli aspetti negativi di ogni progetto.
Al termine delle considerazioni di ogni componente, il gruppo si è trovato concorde nella scelta del \glock{capitolato} C6, ossia \textit{ThiReMa - Things Relationship Management}.

\subsection*{3. Prime idee sull'analisi dei requisiti}

Il gruppo ha analizzato collettivamente la presentazione del \glock{capitolato} scelto e gli approfondimenti tecnologici trattati dal proponente nel seminario tenuto in precedenza, con l'intento di evidenziare le parti critiche e/o non chiare da discutere con il proponente stesso durante il primo incontro.
Per cercare di identificare il maggior numero possibile di argomenti su cui avere un riscontro diretto, è stato deciso di procedere con una prima analisi dei requisiti e dei casi d'uso del progetto.

\subsection*{4. Definizione di un primo contatto con il proponente}

Il gruppo si è ritrovato concorde sul fatto che una delle maggiori priorità fosse quella di contattare il proponente, per fissare al più presto il primo incontro.
A tale scopo è stato deciso di inviare una mail di presentazione all'indirizzo di posta elettronica indicato durante il seminario di approfondimento, nella quale viene inoltre chiesta l'eventuale disponibilità ad utilizzate una modalità di comunicazione più diretta rispetto alle e-mail.

\subsection*{5. Fissare la data per una nuova riunione del gruppo}

La prossima riunione è stata fissata come segue:
\begin{itemize}
	\item \textbf{Luogo:} Dipartimento di Matematica, Padova;
	\item \textbf{Data:} 17 dicembre 2019;
	\item \textbf{Ora di inizio:} 14:00.
\end{itemize}

