\section*{Introduzione}

\subsection*{Luogo e data dell'incontro}
	\begin{itemize}
		\item \textbf{Luogo:} Dipartimento di Matematica, Padova
		\item \textbf{Data:} 2019-12-10
		\item \textbf{Ora di inizio:} 14:15
		\item \textbf{Ora di fine:} 18:00
	\end{itemize}

\subsection*{Ordine del giorno}
	\begin{enumerate}
		\item Rilevazione di eventuali dubbi sulle norme di progetto
		\item Discussione sulle norme di progetto rimaste in sospeso
		\item Prime idee sull'analisi dei requisiti
		\item Definizione di un primo contatto con il proponente
		\item Fissare la data per una nuova riunione del gruppo
		\item Varie ed eventuali
	\end{enumerate}

\subsection*{Presenze}
	\begin{itemize}
		\item \textbf{Totale presenti:} 7 su 7
		\item \textbf{Presenti: }
			\begin{itemize}			
				\item Giuseppe Vito Bitetti
				\item Lorenzo Dei Negri
				\item Nicolò Frison
				\item Mariano Sciacco
				\item Alessandro Tommasin
				\item Giovanni Vidotto
				\item Fouad Mouad
			\end{itemize}
		\item \textbf{Assenti: } 
			\begin{itemize}	
				\item Nessuno
			\end{itemize}
	\end{itemize}


\newpage
\section*{Svolgimento}

\subsection*{1. Rilevazione di eventuali dubbi sulle norme di progetto}

Si è discusso con ciascun membro del team riguardo a tutti i capitolati presentati nella stessa giornata in cui si è svolta la riunione. 

In generale, è stata espressa una certa decisione riguardo allo scarto di alcuni capitolati, anche sulla base della complessità e della preparazione di tutti.

\subsection*{2. Discussione sulle norme di progetto rimaste in sospeso}

A seguito di un breve brainstorming, è stato deliberato su decisione comune il nome del team e il relativo logo.

\begin{itemize}
	\item \textbf{Nome del team:} Red Round Robin
	\item \textbf{Logo:} una rondine rossa con attorno una freccia rossa circolare
\end{itemize}

\subsection*{3. Prime idee sull'analisi dei requisiti}

Tutti i membri del gruppo hanno confermato la presenza ai prossimi seminari di approfondimento dei capitolati.

\subsection*{4. Definizione di un primo contatto con il proponente}

A seguito di una breve discussione, si è deciso di concentrarsi con maggiore priorità sulle norme che riguardano il way of working, la gestione della repository e delle comunicazioni, così da rendere più agevole le operazioni iniziali di configurazione.
Nelle prossime riunioni si studieranno i punti in comune che riguardano le convenzioni comuni sulla documentazione e sul software, le nomenclature e gli strumenti da utilizzare. 

Da parte di tutti si è concordato di collaborare sulla scrittura dei documenti utilizzando \LaTeX{} e usando come mezzo principale di comunicazione \textit{Slack}. Per la parte di versionamento è stato confermato l'uso di \textit{Github} (Git) con il relativo \textit{Issue Tracking System}.

\subsection*{5. Fissare la data per una nuova riunione del gruppo}

La prossima riunione è stata fissata come segue:
\begin{itemize}
	\item \textbf{Luogo:} Dipartimento di Matematica, Padova
	\item \textbf{Data:} 2019-12-17
	\item \textbf{Ora di inizio:} 14:00
\end{itemize}


\subsection*{6. Varie ed eventuali}

\begin{itemize}
	\item E' stata consigliata la lettura di alcune guide circa l'utilizzo di \LaTeX 
	\item Entro la prossima settimana verrà eseguita una prima configurazione della repository del progetto e di \textit{Slack}
\end{itemize}