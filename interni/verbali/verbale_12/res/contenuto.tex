\section*{Introduzione}

\subsection*{Luogo e data dell'incontro}
	\begin{itemize}
		\item \textbf{luogo:} Dipartimento di Matematica, Padova;
		\item \textbf{data:} 2020-02-10;
		\item \textbf{ora di inizio:} 14:00;
		\item \textbf{ora di fine:} 16:00.
	\end{itemize}

\subsection*{Ordine del giorno}
	\begin{enumerate}
			\item decidere le domande da fare al professor Cardin riguardo l'analisi dei requisiti e al \glock{poc};
  			\item scegliere quali tecnologie utilizzare per la realizzazione del prodotto software;
  			\item individuare nei documenti le cose rimanenti da correggere segnalate dai professori;
  			\item organizzare una riunione con SanMarco Informatica;
  			\item varie ed eventuali.
	\end{enumerate}

\subsection*{Presenze}
	\begin{itemize}
		\item \textbf{totale presenti:} 6 su 7
		\item \textbf{presenti: }
			\begin{itemize}			
				\item Lorenzo Dei Negri;
				\item Giuseppe Vito Bitetti;
				\item Fouad Mouad (segretario);
				\item Mariano Sciacco;
				\item Alessandro Tommasin;
				\item Giovanni Vidotto.
			\end{itemize}
		\item \textbf{assenti: } 
			\begin{itemize}	
				\item Nicolò Frison (giustificato).
			\end{itemize}
	\end{itemize}


\newpage
\section*{Svolgimento}

	\subsection*{Decidere le domande da fare al professor Cardin riguardo l'analisi dei requisiti e al poc}
		Alcune segnalazioni fatte dal professor Cardin riguardo dei casi d'uso non sono state ben comprese, inoltre non è chiaro quali siano gli obiettivi da raggiungere per ottenere una \textit{technology baseline}.  Per questi motivi, si è discusso per scegliere le domande da porgli durante la videoconferenza fissata in data 2020-02-11 a partire dalle 12:30.

	\subsection*{Scegliere quali tecnologie utilizzare per la realizzazione del prodotto software}
		Sono state scelte in modo definitivo le tecnologie da utilizzare per la realizzazione delle componenti software del prodotto. In particolare:
		\begin{itemize}
				\item tra i \glock{time series db} proposti da SanMarco Informatica nel capitolato, si è scelto di usare \textit{Timescale};
				\item per mantenere i metadati degli utenti registrati si è optato per \textit{PostgreSQL};
				\item per realizzare le \glock{API} che si interfacciano con \glock{Kafka} per il prelievo ed invio dei dati da e verso i dispositivi, si è deciso di usare \glock{Java};
				\item per implementare il \glock{bot Telegram} si è deciso di usare \glock{Node.js};
				\item per la realizzazione della \glock{web app} lato server si è deciso di usare \glock{PHP};
				\item per implementare l'interfaccia della web app e il suo comportamento conseguente all'interazione con gli utenti, si è deciso di utilizzare \glock{XHTML strict},
				\glock{Bootstrap}, \glock{JavaScript}, \glock{Ajax} e \glock{jQuery}.
			\end{itemize}

	\subsection*{Individuare nei documenti le cose rimanenti da correggere segnalate dai professori}
		È stato fatto un riepilogo delle parti di documentazione da eliminare o modificare segnalate dai professori e non ancora sistemate, dopodichè ad ogni membro del gruppo sono stati assegnati compiti da svolgere atti a concludere le correzioni mancanti.

	\subsection*{Organizzare una riunione con SanMarco Informatica}
		È stato deciso di mandare una mail al proponente con lo scopo di fissare un incontro per la prossima settimana.
		\subsection*{Varie ed eventuali}
		Nulla da riportare.