\section*{Introduzione}

\subsection*{Luogo e data dell'incontro}
	\begin{itemize}
		\item \textbf{luogo:} Dipartimento di Matematica, Padova;
		\item \textbf{data:} 17 dicembre 2019;
		\item \textbf{ora di inizio:} 14:15;
		\item \textbf{ora di fine:} 18:00.
	\end{itemize}

\subsection*{Ordine del giorno}
	\begin{enumerate}
		\item verificare eventuali risposte dell'azienda;
		\item valutare se utilizzare Github Actions per la creazione degli artifacts e se tenerlo pubblico o meno;
		\item valutare l'uso di una DevBoard usata per le review dei documenti;
		\item valutare la struttura dei prossimi documenti e le assegnazioni per le divisioni dei compiti;
		\item iniziare la review dei documenti di studio di fattibilità e norme di progetto per trovare punti in comune mancanti (es: standard per la qualità) o cose non coerenti;
		\item varie ed eventuali.
	\end{enumerate}

\subsection*{Presenze}
	\begin{itemize}
		\item \textbf{totale presenti:} 7 su 7
		\item \textbf{presenti: }
			\begin{itemize}			
				\item Giuseppe Vito Bitetti;
				\item Lorenzo Dei Negri;
				\item Nicolò Frison (segretario);
				\item Fouad Mouad;
				\item Mariano Sciacco;
				\item Alessandro Tommasin;
				\item Giovanni Vidotto.
			\end{itemize}
		\item \textbf{assenti: } 
			\begin{itemize}	
				\item nessuno.
			\end{itemize}
	\end{itemize}


\newpage
\section*{Svolgimento}

	\subsection*{Verificare eventuali risposte dell'azienda}
		Non è stata ancora ricevuta risposta dall'azienda e quindi si svilupperanno i casi d'uso fino a che è possibile.
	\subsection*{Valutare se utilizzare Github Actions per la creazione degli artifacts e se tenerlo pubblico o meno}
		Vengono utilizzate le Github Actions per la compilazione dei documenti in Latex online, ovvero non bisogna fare il push nei branch dei file pdf, verranno compilati in automatico sul sito \\ https://artifacts.redroundrobin.site/ .
	\subsection*{Valutare l'uso di una DevBoard usata per le review dei documenti}
		Si continuerà ad usare Github come \glock{issue tracking system}.
	\subsection*{Valutare la struttura dei prossimi documenti e le assegnazioni per le divisioni dei compiti}
		Discussione sui documenti riguardanti l'analisi dei requisiti, il piano di progetto e il piano di qualifica
	\subsection*{Iniziare la review dei documenti di studio di fattibilità e norme di progetto per trovare punti in comune mancanti (es: standard per la qualità) o cose non coerenti.}
		\begin{itemize}
			\item definizione identificativi per i rischi;
			\item definizione nomenclatura di priorità per i problemi;
			\item definita nomenclatura assegnata alla documentazione dei verbali e al riferimento ad essi.
		\end{itemize}
	\subsection*{Varie ed eventuali}
		Settato branch di default da master a develop in github.