\section*{Introduzione}

\subsection*{Luogo e data dell'incontro}
	\begin{itemize}
		\item \textbf{luogo:} Dipartimento di Matematica, Padova;
		\item \textbf{data:} 21 novembre 2019;
		\item \textbf{ora di inizio:} 12:00;
		\item \textbf{ora di fine:} 16:00.
	\end{itemize}

\subsection*{Ordine del giorno}
	\begin{enumerate}

		\item impostare un \glock{way of working} per \LaTeX{};
		\item creazione e configurazione di \glock{Slack} e della email ufficiale;
		\item valutazione uso di \glock{Google Drive};
		\item breve raccolta e organizzazione del lavoro da svolgere;
		\item fissare la prossima riunione.

	\end{enumerate}

\subsection*{Presenze}
	\begin{itemize}
		\item \textbf{totale presenti:} 6 su 7
		\item \textbf{presenti: }

			\begin{itemize}			
				\item Lorenzo Dei Negri (segretario);
				\item Nicolò Frison;
				\item Fouad Mouad;
				\item Mariano Sciacco;
				\item Alessandro Tommasin;
				\item Giovanni Vidotto.
			\end{itemize}

		\item \textbf{assenti: } 
			\begin{itemize}	
				\item Giuseppe Vito Bitetti (giustificato).
			\end{itemize}
	\end{itemize}


\newpage
\section*{Svolgimento}

\subsection*{1. Impostare un way-of-working per \LaTeX{} }

Con il gruppo si è discusso di come organizzare il modo di lavorare su \LaTeX{}, ideando insieme un template di base su cui poter poi redigere tutti i documenti necessari al progetto. \
La documentazione verrà preservata nella \glock{repository} del progetto e versionata ad ogni modifica, facendo uso di \glock{Github}.

\subsection*{2. Creazione e configurazione Slack ed email ufficiale}

Insieme a tutto il team, si è configurato uno spazio di lavoro apposito per poter comunicare per via telematica con degli appositi canali differenziati per topic su \glock{Slack}. \
Il link al workspace di \glock{Slack} è il seguente: \href{http://redroundrobin.slack.com}{redroundrobin.slack.com} \
Analogamente, si è creata la email ufficiale del gruppo con il seguente indirizzo:
\href{mailto:redroundrobin.site@gmail.com}{redroundrobin.site@gmail.com}.


\subsection*{3. Valutazione uso di Google Drive}

È stato deliberato l'uso di un \glock{account Google} (redroundrobin.site@gmail.com) con cui sarà possibile inserire al suo interno documenti che riguardano note, appunti e bozze. A tal proposito, ogni membro del gruppo potrà accedere a questo spazio online indipendente e potrà condividere tutto ciò che può essere di interesse per gli altri. \
Anche in vista della realizzazione dei \glock{diagrammi UML} si valuteranno nelle prossime riunioni delle integrazioni online con la suite di \glock{Google Drive} che potrebbero fare comodo per il progetto.


\subsection*{4. Breve raccolta e organizzazione del lavoro da svolgere}

Con tutto il gruppo si è fatta una analisi approfondita dei documenti da dover preparare, mettendo in luce i principali punti su cui fornire maggiore attenzione e studio col fine di essere il più completi possibili nel progetto.

\subsection*{5. Prossima riunione}

La prossima riunione è stata fissata come segue:
\begin{itemize}
	\item \textbf{luogo:} Dipartimento di Matematica, Padova;
	\item \textbf{data:} 26 novembre 2019;
	\item \textbf{ora di inizio:} 14:00.
\end{itemize}
