\section*{Introduzione}

\subsection*{Luogo e data dell'incontro}
	\begin{itemize}
		\item \textbf{Luogo:} Dipartimento di Matematica, Padova
		\item \textbf{Data:} 2019-12-17
		\item \textbf{Ora di inizio:} 14:15
		\item \textbf{Ora di fine:} 18:00
	\end{itemize}

\subsection*{Ordine del giorno}
	\begin{enumerate}
		\item Verificare eventuali risposte dell'azienda
		\item Valutare se utilizzare Github Actions per la creazione degli artifacts e se tenerlo pubblico o meno
		\item Valutare l'uso di una DevBoard usata per le review dei documenti
		\item Valutare la struttura dei prossimi documenti e le assegnazioni per le divisioni dei compiti
		\item Iniziare la review dei documenti di studio di fattibilità e norme di progetto per trovare punti in comune mancanti (es: standard per la qualità) o cose non coerenti.
		\item Varie ed eventuali
	\end{enumerate}

\subsection*{Presenze}
	\begin{itemize}
		\item \textbf{Totale presenti:} 7 su 7
		\item \textbf{Presenti: }
			\begin{itemize}			
				\item Giuseppe Vito Bitetti
				\item Lorenzo Dei Negri
				\item Nicolò Frison
				\item Fouad Mouad
				\item Mariano Sciacco
				\item Alessandro Tommasin
				\item Giovanni Vidotto
			\end{itemize}
		\item \textbf{Assenti: } 
			\begin{itemize}	
				\item Nessuno
			\end{itemize}
	\end{itemize}


\newpage
\section*{Svolgimento}

	\subsection*{Verificare eventuali risposte dell'azienda}
		Non è stata ancora ricevuta risposta dall'azienda e quindi si svilupperanno i casi d'uso fino a che è possibile.
	\subsection*{Valutare se utilizzare Github Actions per la creazione degli artifacts e se tenerlo pubblico o meno}
		Vengono utilizzate le Github Actions per la compilazione dei documenti in Latex online, ovvero non bisogna fare il push nei branch dei file pdf, verranno compilati in automatico sul sito // https://artifacts.redroundrobin.site/ .
	\subsection*{Valutare l'uso di una DevBoard usata per le review dei documenti}
		Si continuerà ad usare github come issue tracking system.
	\subsection*{Valutare la struttura dei prossimi documenti e le assegnazioni per le divisioni dei compiti}
		Discussione sui documenti riguardanti l'Analisi dei requisiti, il Piano di Progetto e il Piano di Qualifica
	\subsection*{Iniziare la review dei documenti di studio di fattibilità e norme di progetto per trovare punti in comune mancanti (es: standard per la qualità) o cose non coerenti.}
		\begin{itemize}
			\item Definizione identificativi per i rischi.
			\item Definizione nomenclatura di priorità per i problemi.
			\item Definita nomenclatura assegnata alla documentazione dei verbali e al riferimento ad essi.
		\end{itemize}
	\subsection*{Varie ed eventuali}
		Settato branch di default da master -> develop in github.