\section*{Introduzione}

\subsection*{Luogo e data dell'incontro}
	\begin{itemize}
		\item \textbf{Luogo:} Dipartimento di Matematica, Padova
		\item \textbf{Data:} 21 novembre 2019
		\item \textbf{Ora di inizio:} 12:00
		\item \textbf{Ora di fine:} 16:00
	\end{itemize}

\subsection*{Ordine del giorno}
	\begin{enumerate}
		\item Impostare un way-of-working per \LaTeX{}
		\item Creazione e configurazione di \textit{Slack} e della email ufficiale. 
		\item Valutazione uso di \textit{Google Drive}
		\item Breve raccolta e organizzazione del lavoro da svolgere
		\item Fissare la prossima riunione
	\end{enumerate}

\subsection*{Presenze}
	\begin{itemize}
		\item \textbf{Totale presenti:} 6 su 7
		\item \textbf{Presenti: }
			\begin{itemize}			
				\item Fouad Mouad
				\item Lorenzo Dei Negri
				\item Nicolò Frison
				\item Fouad Mouad
				\item Mariano Sciacco
				\item Alessandro Tommasin
				\item Giovanni Vidotto
			\end{itemize}
		\item \textbf{Assenti: } 
			\begin{itemize}	
				\item Giuseppe Vito Bitetti (giustificato)
			\end{itemize}
	\end{itemize}


\newpage
\section*{Svolgimento}

\subsection*{1. Impostare un way-of-working per \LaTeX{} }

Con il gruppo si è discusso di come organizzare il modo di lavorare su \LaTeX{}, ideando insieme un template di base su cui poter poi redigere tutti i documenti necessari al progetto. \
La documentazione verrà preservata nella repository del progetto e versionata ad ogni modifica, facendo uso di \textit{Github}.

\subsection*{2. Creazione e configurazione \textit{Slack} ed Email ufficiale}

Insieme a tutto il team, si è configurato uno spazio di lavoro apposito per poter comunicare per via telematica con degli appositi canali differenziati per topic su \textit{Slack}. \
Il link al workspace di \textit{Slack} è il seguente: \href{http://redroundrobin.slack.com}{redroundrobin.slack.com} \
Analogamente, si è creata la email ufficiale del gruppo con il seguente indirizzo:
\href{mailto:info@redroundrobin.site}{info@redroundrobin.site}


\subsection*{3. Valutazione uso di \textit{Google Drive}}

E' stato deliberato l'uso di un \textit{account Google} (redroundrobin.site@gmail.com) con cui sarà possibile inserire al suo interno documenti che riguardano note, appunti e bozze. A tal proposito, Ogni membro del gruppo potrà accedere a questo spazio online indipendente e potrà condividere tutto ciò che può essere di interesse per gli altri. \
Anche in vista della realizzazione dei \textit{diagrammi UML} si valuteranno nelle prossime riunioni delle integrazioni online con la suite di \textit{Google Drive} che potrebbero fare comodo per il progetto.


\subsection*{4. Breve raccolta e organizzazione del lavoro da svolgere}

Con tutto il gruppo si è fatta una analisi approfondita dei documenti da dover preparare, mettendo in luce i principali punti su cui fornire maggiore attenzione e studio col fine di essere il più completi possibili nel progetto.

\subsection*{5. Prossima riunione}

La prossima riunione è stata fissata come segue:
\begin{itemize}
	\item \textbf{Luogo:} Dipartimento di Matematica, Padova
	\item \textbf{Data:} 26 novembre 2019
	\item \textbf{Ora di inizio:} 14:00
\end{itemize}
