\section*{Introduzione}

\subsection*{Luogo e data dell'incontro}
	\begin{itemize}
		\item \textbf{Luogo:} Dipartimento di Matematica, Padova;
		\item \textbf{Data:} 03 dicembre 2019;
		\item \textbf{Ora di inizio:} 14:00;
		\item \textbf{Ora di fine:} 18:00.
	\end{itemize}

\subsection*{Ordine del giorno}
	\begin{enumerate}
		\item Inizio stesura della struttura dei documenti;
		\item assegnazione dei compiti e suddivisione dei ruoli per tutti i documenti (ognuno opera su almeno tutti i documenti);
		\item raccolta e analisi più approfondita delle {\dext};
		\item prossima riunione.
	\end{enumerate}

\subsection*{Presenze}
	\begin{itemize}
		\item \textbf{Totale presenti:} 7 su 7
		\item \textbf{Presenti: }
			\begin{itemize}
				\item Giuseppe Vito Bitetti;
				\item Lorenzo Dei Negri;
				\item Nicolò Frison;
				\item Fouad Mouad;
				\item Mariano Sciacco;
				\item Alessandro Tommasin;
				\item Giovanni Vidotto.
			\end{itemize}
		\item \textbf{Assenti: }
			Nessuno.
	\end{itemize}


\newpage
\section*{Svolgimento}

\subsection*{1. Inizio stesura della struttura dei documenti }

Redazione della struttura delle \dext{norme di progetto} e dello \dext{studio di fattibilità} e su come configurare le sezioni all'interno dei documenti.

\subsection*{2. Assegnazione dei compiti e suddivisione dei ruoli per i documenti }

Sono state assegnate delle sezioni da sviluppare dei documenti riguardanti le \dext{norme di progetto} e lo \dext{studio di fattibilità}.

\subsection*{3. Raccolta e analisi più approfondita delle norme di progetto }

Brainstorming su regole da utilizzare per ciò che scriviamo, discussione sulla \glock{continuous integration} per il progetto, organizzare il funzionamento del \glock{way of working}. Decisione sul \glock{workflow} utilizzato sulla \glock{repository}, sulle \glock{milestone} di \glock{github} e sulla rotazione dei ruoli.

\subsection*{4. Prossima riunione}

La prossima riunione è stata fissata come segue:
\begin{itemize}
	\item \textbf{Luogo:} Dipartimento di Matematica, Padova;
	\item \textbf{Data:} 10 dicembre 2019;
	\item \textbf{Ora di inizio:} 14:00.
\end{itemize}
