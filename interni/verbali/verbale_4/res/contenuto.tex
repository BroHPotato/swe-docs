\section*{Introduzione}

\subsection*{Luogo e data dell'incontro}
	\begin{itemize}
		\item \textbf{Luogo:} Dipartimento di Matematica, Padova
		\item \textbf{Data:} 03-Dicembre-2019
		\item \textbf{Ora di inizio:} 14:00
		\item \textbf{Ora di fine:} 18:00
	\end{itemize}

\subsection*{Ordine del giorno}
	\begin{enumerate}
		\item Inizio stesura della struttura dei documenti
		\item Assegnazione dei compiti e suddivisione dei ruoli per tutti i documenti (ognuno tocca almeno tutti i documenti)
		\item Raccolta e analisi più approfondita delle norme di progetto
		\item Prossima riunione
	\end{enumerate}

\subsection*{Presenze}
	\begin{itemize}
		\item \textbf{Totale presenti:} 7 su 7
		\item \textbf{Presenti: }
			\begin{itemize}			
				\item Giuseppe Vito Bitetti
				\item Lorenzo Dei Negri
				\item Nicolò Frison
				\item Fouad Mouad
				\item Mariano Sciacco
				\item Alessandro Tommasin
				\item Giovanni Vidotto
			\end{itemize}
		\item \textbf{Assenti: } 
			Nessuno
	\end{itemize}


\newpage
\section*{Svolgimento}

\subsection*{1. Inizio stesura della struttura dei documenti }

Redazione della struttura delle norme di progetto e dello studio di fattibilità e su come configurare le sezioni all'interno dei documenti.

\subsection*{2. Assegnazione dei compiti e suddivisione dei ruoli per i documenti }

Sono state assegnate delle sezioni da sviluppare dei documenti riguardanti le norme di progetto e lo studio di fattibilità.

\subsection*{3. Raccolta e analisi più approfondita delle norme di progetto }

Brainstorming su regole da utilizzare per ciò che scriviamo, discussione sulla continuous integration per il progetto, organizzare il funzionamento del way of working. Decisione sul workflow utilizzato sulla repo, sulle milestone di github e sulla rotazione dei ruoli.

\subsection*{4. Prossima riunione}

La prossima riunione è stata fissata come segue:
\begin{itemize}
	\item \textbf{Luogo:} Dipartimento di Matematica, Padova
	\item \textbf{Data:} 10 dicembre 2019
	\item \textbf{Ora di inizio:} 14:00
\end{itemize}
