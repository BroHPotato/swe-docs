\section*{Introduzione}

\subsection*{Luogo e data dell'incontro}
	\begin{itemize}
		\item \textbf{luogo:} Dipartimento di Matematica, Padova;
		\item \textbf{data:} 2020-01-29;
		\item \textbf{ora di inizio:} 11:30;
		\item \textbf{ora di fine:} 12:30.
	\end{itemize}

\subsection*{Ordine del giorno}
	\begin{enumerate}
			\item rivedere la risposta del professore via email per la gestione della \glock{repository};
  			\item valutare le modifiche da implementare nei documenti;
  			\item organizzare il lavoro per i prossimi giorni in base alle disponibilità;
  			\item varie ed eventuali.
	\end{enumerate}

\subsection*{Presenze}
	\begin{itemize}
		\item \textbf{totale presenti:} 6 su 7
		\item \textbf{presenti: }
			\begin{itemize}			
				\item Lorenzo Dei Negri;
				\item Nicolò Frison;
				\item Fouad Mouad;
				\item Mariano Sciacco (segretario);
				\item Alessandro Tommasin;
				\item Giovanni Vidotto.
			\end{itemize}
		\item \textbf{assenti: } 
			\begin{itemize}	
				\item Giuseppe Vito Bitetti (giustificato).
			\end{itemize}
	\end{itemize}


\newpage
\section*{Svolgimento}

	\subsection*{Rivedere la risposta del professore via email per la gestione della repository}
		Ai fini di gestire la \glock{repository}, si è deciso di valutare una alternativa che permettesse di ridurre la complessità di astrazione dei branch. In particolare, il gruppo ha trovato una soluzione che fa uso dei \textbf{Git submodules} che permettono di separare con più \glock{repository} i componenti da integrare nel prodotto finale. Questo permetterà al gruppo di mantenere e sviluppare con maggiore flessibilità tutto ciò che è richiesto dal progetto, senza rendere troppo indipendente.

	\subsection*{Valutare le modifiche da implementare nei documenti}
		Sono state ricontrollate tutte le segnalazioni effettuate dal professore nella scorsa revisione e sono state valutate assieme le modifiche da implementare nei documenti. Alcune di esse richiedono una maggiore delucidazione, motivo per cui si è deciso di mandare una email per una richiesta di colloquio con il professore.

	\subsection*{Organizzare il lavoro per i prossimi giorni in base alle disponibilità}
		Sono state assegnate alcune attività per i prossimi giorni, in base alla sessione d'esami e alle disponibilità dei membri del gruppo.

	\subsection*{Varie ed eventuali}
		Sono state esposte alcune considerazioni a ridosso della scorsa consegna per capire eventuali problematiche che ci sono state.