\section{Valutazione capitolati rimanenti}
    \subsection{Capitolato C1 - Autonomous Highlights Platform}
       \subsubsection{Informazioni generali}
       \begin{itemize}
           \item \textbf{Nome:} Autonomous Highlights Platform;
           \item \textbf{Proponente: }Zero12;
           \item \textbf{Committente: }Prof. Tullio Vardanega e Prof. Riccardo Cardin.
       \end{itemize}

    \subsubsection{Descrizione}
        L'obiettivo di questo \glock{capitolato} è creare una \glock{piattaforma web} che è capace di ricevere in input dei video di eventi sportivi, come una partita di calcio o di tennis, e che riesca a creare autonomamente un video di massimo 5 minuti contenente soltanto i suoi momenti chiave (\glock{highlights}).

    \subsubsection{Finalità del progetto}
        Il prodotto finale, ovvero la \glock{piattaforma web}, dovrà essere dotata di un \glock{modello di machine learning} in grado di identificare i momenti più importanti dell'evento sportivo che le è stato inviato. Un fattore importante è che andrà scelto uno sport sul quale focalizzare la propria attenzione e sul quale verrà addestrato il modello di apprendimento atto all'identificazione automatica dei momenti salienti di un evento dello sport.
        Il flusso di generazione del suddetto highlight dovrà avere la seguente struttura:
        \begin{itemize}
            \item caricamento del video;
            \item identificazione dei momenti salienti;
            \item estrazione delle corrispondenti parti di video;
            \item generazione del video di sintesi.
        \end{itemize}
    \subsubsection{Tecnologie}
    Le tecnologie consigliate dall'azienda riguardano la tecnologia di \glock{Amazon Web Services} ed in particolare:
    \begin{itemize}
        \item \textbf{\glock{Elastic Container Service} o \glock{Elastic Kubernetes Service}: }è un servizio che permette la gestione di contenitori, altamente dimensionabile e ad elevate prestazioni;
        \item \textbf{\glock{DynamoDB}: }è un database non relazionale per applicazioni che necessitano di prestazioni elevate su qualsiasi scala;
        \item \textbf{\glock{AWS Transcode}: }è un servizio di transcodifica di contenuti multimediali nel \glock{cloud};
        \item \textbf{\glock{Sage Maker}: }è un servizio completamente gestito che permette a sviluppatori e \glock{data scientist} di creare, addestrare e distribuire modelli di apprendimento automatico;
        \item \textbf{\glock{AWS Rekognition video}: }è un servizio di analisi video basato su \glock{apprendimento approfondito}.
    \end{itemize}
        \subsubsection{Linguaggi di programmazione}
        \begin{itemize}
            \item \textbf{\glock{Node.JS}: }linguaggio ideale per sviluppare \glock{API Restful} \glock{JSON} a supporto dell'applicativo;
            \item \textbf{\glock{Python}: }linguaggio ideale per lo sviluppo delle componenti di \glock{machine learning};
            \item \textbf{\glock{HTML5}}, \textbf{\glock{CSS3}}, \textbf{\glock{JavaScript}: }linguaggi per la realizzazione dell'interfaccia web di gestione del flusso di lavoro, utilizzando un \glock{framework} responsive come Twitter.
        \end{itemize}
    \subsection{Vincoli del progetto}
    \begin{itemize}
        \item utilizzo di \glock{Sage Maker};
        \item l'architettura dovrà essere basata su \glock{micro-servizi}, suddividendo il progetto in tante funzioni di base denominate servizi; questi ultimi dovranno essere indipendenti fra loro;
        \item caricamento dei video da elaborare tramite riga di comando;
        \item console web di analisi e controllo degli stati di elaborazione dei video.
      \end{itemize}

    \subsubsection{Aspetti positivi}
    \begin{itemize}
    		\item per quanto riguarda le tecnologie interessate è presente molta documentazione;
    		\item il tema principale del progetto è stato accolto con molto interesse dal gruppo, che si è dimostrato incuriosito ad approfondire l'argomento;
    		 \item Il proponente fornisce attività di formazione sulle principali tecnologie AWS e \glock{wireframe} dell'interfaccia della console web di analisi e controllo dello stato di elaborazione dei video.
    \end{itemize}
    \subsubsection{Criticità}
    \begin{itemize}
    		\item il proponente non fornisce nessun \glock{data-set} per effettuare il \glock{training} dell'algoritmo;
    		\item buona parte del tempo verrebbe utilizzato per lavori ripetitivi, come la creazione di un data-set, e che non stimolerebbero il gruppo.
    \end{itemize}

    \subsubsection{Conclusione}
	Dopo aver valutato gli aspetti positivi e le criticità, si è deciso che questo capitolato, nonostante tocchi delle tecnologie piuttosto interessanti, richiede, secondo il gruppo, un'eccessiva mole di lavoro ripetitivo.
