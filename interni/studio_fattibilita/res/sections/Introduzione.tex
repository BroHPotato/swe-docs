
\section{Introduzione}
\subsection{Scopo del documento}
Il seguente documento ha l'obiettivo di descrivere brevemente ciò che ogni capitolato\ped{G} ha da proporre, elencando quelli che il nostro gruppo ha considerato come i loro aspetti più interessanti e le loro criticità. 

\subsection{Glossario}
Viene fornito un \textit{Glossario v0.0.1} per evitare possibili ambiguità relative alle terminologie utilizzate nei vari documenti. Nel documento sarà presente a pedice di quelle che riteniamo delle parole una '\textbf{G}'.   
\subsection{Riferimenti}
    \subsubsection{Normativi}
    \begin{itemize}
      \item \textbf{Norme di Progetto}: 
    \end{itemize}
    \subsubsection{Informativi}
    \begin{itemize}
      \item \textbf{Capitolato\ped{G} d'appalto C1 - Autonomous Highlights Platform:} \url{https://www.math.unipd.it/~tullio/IS-1/2019/Progetto/C1.pdf}
      \item \textbf{Capitolato\ped{G} d'appalto C2 - Etherless:} \url{https://www.math.unipd.it/~tullio/IS-1/2019/Progetto/C2.pdf}
      \item \textbf{Capitolato\ped{G} d'appalto C3 - NaturalAPI:} \url{https://www.math.unipd.it/~tullio/IS-1/2019/Progetto/C3.pdf}
      \item \textbf{Capitolato\ped{G} d'appalto C4 - Predire in Grafana:} \url{https://www.math.unipd.it/~tullio/IS-1/2019/Progetto/C4.pdf}
      \item \textbf{Capitolato\ped{G} d'appalto C5 - Stalker:} \url{https://www.math.unipd.it/~tullio/IS-1/2019/Progetto/C5.pdf}
      \item \textbf{Capitolato\ped{G} d'appalto C6 - ThiReMa - Things Relationship Management:} \url{https://www.math.unipd.it/~tullio/IS-1/2019/Progetto/C6.pdf}
    \end{itemize}
    
  

