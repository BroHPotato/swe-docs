
\section{Introduzione}
\subsection{Scopo del documento}
  Il seguente documento ha l'obiettivo di descrivere brevemente ciò che ogni \glock{capitolato}  ha da proporre, elencando quelli che il nostro gruppo ha considerato come i loro aspetti più interessanti e le loro criticità. 
\subsection{Glossario e Documenti esterni}
    Per evitare possibili ambiguità relative alle terminologie (che andranno indicate in \textsc{maiuscoletto})utilizzate nei vari documenti, verranno utilizzate due simboli:
    \begin{itemize}
      \item Una \textit{D} al pedice per indicare il nome di un particolare documento.
      \item Una \textit{G} al pedice per indicare un termine che sarà presente nel \dext{Glossario v0.0.1}.
    \end{itemize}   
\subsection{Riferimenti}
    \subsubsection{Normativi}
    \begin{itemize}
      \item \textbf{Norme di Progetto}: 
    \end{itemize}
    \subsubsection{Informativi}
    \begin{itemize}
      \item \textbf{\glock{Capitolato} d'appalto C1 - Autonomous Highlights Platform:} \url{https://www.math.unipd.it/~tullio/IS-1/2019/Progetto/C1.pdf}
      \item \textbf{\glock{Capitolato} d'appalto C2 - Etherless:} \url{https://www.math.unipd.it/~tullio/IS-1/2019/Progetto/C2.pdf}
      \item \textbf{\glock{Capitolato} d'appalto C3 - NaturalAPI:} \url{https://www.math.unipd.it/~tullio/IS-1/2019/Progetto/C3.pdf}
      \item \textbf{\glock{Capitolato} d'appalto C4 - Predire in Grafana:} \url{https://www.math.unipd.it/~tullio/IS-1/2019/Progetto/C4.pdf}
      \item \textbf{\glock{Capitolato} d'appalto C5 - Stalker:} \url{https://www.math.unipd.it/~tullio/IS-1/2019/Progetto/C5.pdf}
      \item \textbf{\glock{Capitolato} d'appalto C6 - ThiReMa - Things Relationship Management:} \url{https://www.math.unipd.it/~tullio/IS-1/2019/Progetto/C6.pdf}
    \end{itemize}
    
  

