

\section{Valutazione capitolato scelto}

	\subsection{Capitolato 6 - ThiReMa}
	
	\subsubsection{Informazioni generali}
	\begin{itemize}
		\item \textbf{Nome:} ThiReMa - Things Relationship Management
		\item \textbf{Proponente:} Sanmarco Informatica
	\end{itemize}
	
	\subsubsection{Descrizione capitolato}
	Sviluppo di un software che,  dopo aver ricevuto misurazioni da sensori eterogenei, li accumola in un database centralizzato. Questa applicazione viene poi completata da un servizio di dispatching per inoltrare in modo tempestivo le informazioni utili per gestire le azioni urgenti.
	I dati messi a disposizione dal database centralizzato dovranno essere suddivisi in due macro-categorie: dati operativi e fattori influenzanti.
	
	\subsubsection{Finalità del progetto}
	Creare una \glock{web-application}, che permetta di valutare la correlazione tra dati operativi (misure) e i fattori influenzanti. Tale applicazione si potrà focalizzare nella definizione di uno o più algoritmi per la successiva analisi dei dati al fine di essere in grado di effettuare delle previsioni sull’andamento dei dati stessi ed offrire, ad esempio, dei servizi di \glock{manutenzione predittiva}.
	Per ogni tipologia di informazioni rilevate dovrà anche essere possibile assegnare il monitoraggio ad un particolare ente. 
	Analizzando un determinato sensore, in base ai dati ricevuti, si può prevedere un deterioramento complessivo tale da generare una necessaria azione di manutenzione preventiva.
	La web-application dovrà essere suddivisa in 3 macro-sezioni:
	\begin{itemize}
		\item Censimento dei sensori e dei relativi dati;
		\item Modulo di analisi di correlazione;
		\item Modulo di monitoraggio per ente.
	\end{itemize}
	
	\subsubsection{Tecnologie interessate}
	\begin{itemize}
		\item \textbf{\glock{Apache Kafka}:} Il progetto mira a creare una piattaforma a bassa latenza ed alta velocità per la gestione di feed dati in tempo reale;
		\item \textbf{\glock{API Producer}, \glock{Consumer}, \glock{Connect} e \glock{Streams}:} \glock{API} consigliate per la produzione di componenti custom per Kafka;
		\item \textbf{\glock{PostgreSQL}, \glock{TimescaleDB}, ClickHouse:} Implementazioni database suggerite per contenere i dati relativi alle misurazioni, agli utenti e le loro informazioni di autorizzazione;
		\item \textbf{\glock{Docker}:} Tecnologia di containerizzazione che consente la creazione e l'utilizzo di container Linux. Nel progetto risulterebbe utile per l'instanziazione di tutti i componenti;   
		\item \textbf{Java:} Linguaggio di programmazione ad alto livello orientato agli oggetti e a tipizzazione statica;
		\item \textbf{Bootstrap:} Raccolta di strumenti open source per la creazione di siti e applicazioni per il Web.
	\end{itemize}
	
	\subsubsection{Aspetti positivi}
	\begin{itemize}
		\item Tecnologie già in parte conosciute dal gruppo con la possibilità di ampliarne le conoscenze;
		\item L'azienda mette a disposizione figure di diverso livello per rispondere alle varie esigenze del gruppo e per facilitare la creazione di ambienti di sviluppo e test;
		\item Consistente set di dati su cui testare l'applicativo;
		\item Interfacciarsi con l'harware tramite API;
	\end{itemize}
	
	\subsubsection{Criticità}
	\begin{itemize}
		\item Protocolli proprietari, la documentazione su di essi potrebbe essere limitata;
		\item Il capitolato richiede un'analisi dei dati più avanzata rispetto agli altri.
	\end{itemize}
	
	\subsubsection{Conclusioni}
	Il capitolato ha suscitato l'interesse del gruppo, dando la possibilità di ampliare tecnologie già in parte conosciute ed al contempo molto attuali, quali IoT e Big Data. C'è stato inoltre molto entusiasmo per la tipologia di web-application da sviluppare. Nell'insieme questo capitolato è stato accolto con forte interesse da tutti i componenti del gruppo.

