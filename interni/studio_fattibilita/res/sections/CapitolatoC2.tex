\subsection{Capitolato C2 - Etherless}
	 \subsubsection{Informazioni generali}
       	\begin{itemize}
			\item \textbf{nome:} Etherless;
         	\item \textbf{proponente:} Red Babel;
          	\item \textbf{committente:} Prof. Tullio Vardanega e Prof. Riccardo Cardin.
         \end{itemize}
	\subsubsection{Descrizione}
       Etherless è una piattaforma \glock{cloud} che permette, agli sviluppatori di fare il \glock{deploy} di funzioni \glock{JavaScript}, mentre  agli utenti finali di pagare per l'esecuzione delle suddette funzioni (si basa su \glock{CaaS}, ovvero Computation-as-a-Service).
       Una parte di ciò che viene pagato dagli utenti verrà trattenuta dal piattaforma stessa come compenso per l'effettiva esecuzione della funzione.
    \subsubsection{Finalità del progetto}
     Il progetto finale si propone quindi di fornire un servizio di \glock{CaaS} utilizzando la seguente struttura:
    		 \begin{itemize}
    			\item \textbf{Etherless-cli: }è il modulo con il quale gli sviluppatori interagiscono con Etherless. Deve supportare diversi comandi, ovvero: configurare il proprio \glock{account}, eseguire il \glock{deploy} delle funzioni, elencare le funzioni già presenti, eseguire una funzione e visualizzare i \glock{logs} riguardanti una specifica funzione;
    			\item \textbf{Etherless-smart: }consiste in un set di \glock{contratti smart} che gestiscono la comunicazione e il trasferimento di denaro (detto ETH) tra Etherless-cli ed Etherless-server;
    			\item \textbf{Etherless-server: }è il modulo che esegue le funzioni e che tramite Etherless-smart comunica con Etherless-cli.  Nel momento in cui viene restituito il valore di una funzione o viene lanciata una eccezione, Etherless-server emetterà un evento nella \glock{blockchain} che verrà ricevuto dalla Etherless-cli che mostrerà infine il risultato all'utente.
		\end{itemize}
    \subsubsection{Tecnologie}
     Il progetto si basa sull'unione di due tecnologie, ovvero \glock{Ethereum} e \glock{serverless}; più precisamente utilizza:
    		 \begin{itemize}
    			\item \textbf{Ethereum: }è una piattaforma che permette ai suoi utenti di scrivere applicazioni decentralizzate (dette \glock{\DJ Apps}) che usano la tecnologia \glock{blockchain};
    			\item \textbf{Ethereum Virtual Machine (EVM): }è una \glock{macchina virtuale decentralizzata} che esegue \glock{script} usando un \glock{network} internazionale di nodi pubblici;
    			\item \textbf{Blockchain: }è una struttura dati condivisa e immutabile; tramite questa struttura è possibile tenere traccia dei pagamenti effettuati in Ethereum;
    			\item \textbf{Smart Contract: }sono utilizzati per le interazioni tra attori e possono contenere denaro (\glock{Ether} o ETH), dati o una combinazione di entrambi;
    			\item \textbf{Gas: }è il carburante che permette alla \glock{EVM} di eseguire un programma;
    		 	\item \textbf{\glock{MainNet}, \glock{Ropsten}: }sono reti sulle quali vengono eseguiti i protocolli di Ethereum;
    		 	\item \textbf{Eventi Ethereum: }sono una delle parti fondamentali di Ethereum in quanto possono essere: il  valore di ritorno di uno smart contract, un innesco asincrono contenente dati oppure una forma di  immagazzinamento più economica rispetto ad uno smart contract;
    			 \item \textbf{Architettura \glock{serverless}: }è un metodo di creazione ed esecuzione di applicazioni e servizi che non richiede la gestione di un'infrastruttura (idea di \glock{BaaS}, \glock{Back end}-as-a-Service);
    		 	\item \textbf{\glock{AWS Lambda}: }è un servizio che permette di eseguire codice  in risposta ad eventi, senza effettuare il provisioning né gestire server;
    		 	\item \textbf{Serverless Framework: }è un \glock{framework} \glock{open source} che permette di sviluppare e distribuire applicazioni \glock{serverless};
    		 	\item \textbf{CloudFormation: }è uno strumento che permette di gestire risorse e per fare il deploy di infrastrutture;
    		 	\item \textbf{\glock{API Gateway}, \glock{AWS DynamoDB}, \glock{AWS S3}: }componenti che possono servire da supporto alle applicazioni \glock{serverless};
    		 	\item \textbf{Truffle: }è un ambiente di sviluppo per Ethereum, consigliato per creare un web server locale per gestire la parte \glock{front end} del progetto;
    		 	\item \textbf{Node Package Manager (NPM): }è un \glock{package manager} consigliato per installare etherless-cli;
    			\item \textbf{ESLint: }è uno strumento di \glock{analisi del codice statico} per identificare schemi problematici nel codice \glock{JavaScript}.

		\end{itemize}
	\subsubsection{Linguaggi di programmazione}
        	\begin{itemize}
        		\item \textbf{\glock{YAML}, \glock{JSON}: }sono due linguaggi che permettono di dare una struttura ai dati che poi verrà creata tramite \glock{Cloud Formation};
        		\item \textbf{\glock{Solidity}: }è un linguaggio tipato staticamente che supporta l'ereditarietà, librerie e tipi definiti dall'utente; viene utilizzato per scrivere smart contracts;
        		\item \textbf{\glock{JavaScript}: }linguaggio utilizzato per svolgere differenti compiti all'interno del progetto;
        		\item \textbf{\glock{TypeScript}: }linguaggio basato su \glock{JavaScript} che permette di definire la tipologia dei dati. Ne è consigliata la versione 3.6.
        \end{itemize}

  \subsection{Vincoli del progetto}
    	   \begin{itemize}
    			\item Etherless-cli deve poter essere installato con il comando bash:
    			\verb! npm install -g etherless-cli!;

    			\item dopo essere stato installato uno sviluppatore deve essere in grado di eseguire diversi comandi associati ai seguenti compiti: creare o entrare in un \glock{account} \glock{Ethereum}, rilasciare una funzione, eseguire una funzione già presente, eliminare una funzione caricata;
    			\item deve essere possibile fare degli upgrade agli smart contracts;
    			\item Etherless deve essere sviluppato usando \glock{TypeScript} 3.6 usando un \glock{approccio promise/async-await};
    			\item deve essere usato typescript-eslint insieme a EsLint durante la fase di sviluppo;
    			\item Etherless-server deve essere implementato usando Serverless Framework.
	 \end{itemize}


    \subsubsection{Aspetti positivi}
	    \begin{itemize}
    			\item dato l'interesse riscosso dalla tecnologia \glock{blockchain}, questo capitolato potrebbe essere un notevole arricchimento del bagaglio di conoscenze del gruppo.
    	   \end{itemize}
    \subsubsection{Criticità}
    	   \begin{itemize}
    			\item l'azienda ha sede all'estero e potrebbe fornire un supporto inferiore rispetto alle aziende che si trovano nel territorio nazionale;
    			\item il capitolato dopo un'attenta analisi non ha riscosso molto interesse.
    	   \end{itemize}
    \subsubsection{Conclusione}
	Dopo aver valutato attentamente il capitolato, il gruppo ha deciso di non considerarlo come prima scelta.
