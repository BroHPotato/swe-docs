\subsection{Capitolato C2 - Etherless}
	 \subsubsection{Informazioni generali}
       	\begin{itemize}
         	  \item \textbf{Proponente: Red Babel};
          	  \item \textbf{Committente: }Prof. Tullio Vardanega e Prof. Riccardo Cardin.
         \end{itemize}
	\subsubsection{Descrizione}
       Etherless è una piattaforma \glock{cloud} che permette, agli sviluppatori di fare il \glock{deploy} di funzioni JavaScript, mentre  agli utenti finali di pagare per l'esecuzione delle suddette funzioni (si basa su \glock{CaaS}, ovvero Computation-as-a-Service).
       Una parte di ciò che viene pagato dagli utenti verrà trattenuta dal piattaforma stessa come compenso per l'effettiva esecuzione della funzione. 
    \subsubsection{Finalità del progetto}
     Il progetto finale si propone quindi di fornire un servizio di \glock{CaaS} utilizzando la seguente struttura:
    		 \begin{itemize}
    			\item \textbf{Etherless-cli: }è il modulo con il quale gli sviluppatori interagiscono con Etherless. Deve supportare diversi comandi, ovvero: la configurazione del proprio account, eseguire il \glock{deploy} delle funzioni, elencare le funzioni già presenti, eseguire una funzione e visualizzare i logs riguardanti una specifica funzione;
    			\item \textbf{Etherless-smart: }consiste in un set di \glock{contratti smart} che gestiscono la comucazione e il trasferimento di denaro (detto ETH) tra Etherless-cli ed Etherless-server;
    			\item \textbf{Etherless-server: }è il modulo che esegue le funzioni e che tramite Etherless-smart comunica con Etherless-cli.  Nel momento in cui viene restituito il valore di una funzione o viene lanciata una eccezione, Etherless-server  emetterà un evento nella \glock{blockchain} che verrà ricevuto dalla Etherless-cli che mostrarà infine il risultato all'utente.
		\end{itemize}      
    \subsubsection{Tecnologie}
     Il progetto si basa sull'unione di due tecnologie, ovvero \glock{Ethereum} e \glock{Serverless}, più precisamente utilizza:
    		 \begin{itemize}
    			\item \textbf{Ethereum: }è una piattaforma che permette ai suoi utenti di scrivere applicazioni decentralizzate (dette \glock{\DJ Apps}) che usano la tecnologia \glock{blockchain};
    			\item \textbf{Ethereum Virtual Machine (EVM): }è una \glock{macchina virtuale decentralizzata} che esegue script usando un network internazionale di nodi pubblici;
    			\item \textbf{Blockchain : }è una struttura dati condivisa e immutabile;  tramite questa struttura è possibile tenere traccia dei pagamenti effettuati in Ethereum;
    			\item \textbf{Smart Contract: }sono utilizzati per le interazioni tra attori e possono contenere denaro (\glock{Ether} o ETH), dati o una combinazione di entrambi;
    			\item \textbf{Gas: }è il carburante che permette alla \glock{EVM} di eseguire un programma;
    		 	\item \textbf{\glock{MainNet}, \glock{Ropsten}: }sono reti sulle quali vengono eseguiti i protocolli di Ethereum. 
    		 	\item \textbf{Eventi Ethereum: }sono una delle parti fondamentali di Ethereum in quanto possono essere: il  valore di ritorno di uno smart contract,  un innesco asincrono contenente dati oppure una forma di  immagazzinamento più economica rispetto ad uno smart contract;
    			 \item \textbf{Architettura serverless: }è un metodo di creazione ed esecuzione di applicazioni e servizi che non richiede la gestione di un'infrastruttura (idea di \glock{Baas}, Backend-as-a-Service);
    		 	\item \textbf{\glock{AWS Lambda}: }è un servizio che permette di eseguire codice  in risposta ad eventi, senza effettuare il provisioning né gestire server;
    		 	\item \textbf{Serverless Framework: }è un framework open-source che permette di sviluppare e distribuire applicazioni serverless;
    		 	\item \textbf{CloudFormation: }è uno strumento che permette di gestire risorse e per fare il deploy di infrastrutture;
    		 	\item \textbf{\glock{API Gateway}, \glock{AWS DynamoDB}, \glock{AWS S3}: }componenti che possono servire da supporto alle applicazioni serverless;
    		 	\item \textbf{Truffle: }è un ambiente di sviluppo per Ethereum, consigliato per creare un web server locale per gestire la front end;
    		 	\item \textbf{Node Package Manager (NPM): }è un package manager congliato per installare etherless-cli;
    			\item \textbf{ESLint: }è uno strumento di \glock{analisi del codice statico} per identificare schemi problematici nel codice JavaScript.
    		 
		\end{itemize}     	
	\subsubsection{Linguaggi di programmazione}
        	\begin{itemize}
        		\item \textbf{YAML, JSON: }sono due linguaggi che permettono di dare una struttura ai dati che poi verrà creata  tramite CloudFormation;
        		\item \textbf{Solidity: }è un liguaggio tipato staticamente che supporta l'eredità, librerie e tipi definiti dall'utente. Viene utilizzato per scrivere smart contracts;	
        		\item \textbf{JavaScript: } linguaggio utilizzato per svolgere differenti compiti all'interno del progetto;
        		\item \textbf{TypeScript: }linguaggio basato su JavaScript che  permette di definire la tipologia dei dati. Ne è consigliata la versione 3.6.
        \end{itemize}
      
    Vincoli del progetto:
    	   \begin{itemize}
    			\item Etherless-cli deve poter essere installato con il comando bash:  
    			\verb! npm install -g etherless-cli!
    	  
    			\item Dopo essere stato installato uno sviluppatore deve essere in grado di eseguire diversi comandi associati ai seguenti compiti: Creare o entrare in un account \glock{Ethereum}, rilasciare una funzione, eseguire una funzione già presente, eliminare una funzione caricata;
    			\item Deve essere possibile fare degli upgrade agli smart contracts;
    			\item Etherless deve essere sviluppato usando TypeScript 3.6 usando un approccio  promise / async-await;
    			\item Deve essere usato typescript-eslint insieme a EsLint durante la fase di sviluppo;
    			\item Etherless-server deve essere implementato usando Serverless Framework.
    						
	 \end{itemize}
    
    
    \subsubsection{Aspetti positivi}
	    \begin{itemize}
    			\item Dato l'interesse riscosso dalla tecnologia \glock{blockchain}, questo capitolato potrebbe essere un notevole arricchimento del bagaglio di conoscenze del gruppo;
    	   \end{itemize}
    \subsubsection{Criticità}
    	   \begin{itemize}
    			\item L'azienda ha sede all'estero e potrebbe fornire un supporto inferiore rispetto alle aziende che si trovano nel territorio nazionale;
    			\item Il capitolato dopo un'attenta analisi non ha riscosso molto interesse.
    	   \end{itemize}
    \subsubsection{Conclusione}
	Dopo aver valutato attentamente il capitolato, il gruppo ha deciso di orientarsi verso un altro capitolato.
