\section{Conclusioni e motivazioni del capitolato scelto}

Dopo una attenta analisi di tutti i seminari di approfondimento effettuati dai relativi proponenti, si è deciso di optare per un argomento che racchiudesse il maggiore grado di interesse di tutto il gruppo. Basandosi sulle proprie conoscenze, inoltre, si è cercato di determinare un capitolato adeguato che potesse produrre una applicazione concreta nella materia di interesse proposta, con l'obbiettivo di raggiungere una buona qualità per il prodotto finale da consegnare.

Inizialmente, abbiamo preso in ampia considerazione il capitolato C4 (\textit{Predire in Grafana}), poiché richiede una analisi dei dati reperiti dalla \glock{server farm} facendo uso di algoritmi di predizione e, soprattutto, si appoggia su un software di monitoraggio molto utilizzato e \glock{open source}. Tuttavia, la poca competenza nella parte di \glock{machine learning} e l'idea di dover sviluppare essenzialmente una piccola parte di un sistema molto più grande e complesso con un unico linguaggio di programmazione, ci ha fatto convergere verso un argomento correlato che riprende l'analisi dei dati, ma in modo più avvincente. \\
Infatti, si è voluto optare per il capitolato C6 (\textit{ThiReMa}) che racchiude lo sviluppo di una web application in diretto contatto con il mondo \glock{IoT}, attraversando le più moderne tecnologie (quali \glock{Docker} e i database non relazionali) per giungere alla gestione remota di dispositivi fisici su larga scala.

L'idea che ha menzionato l'azienda proponente per questo capitolato, inoltre, ci ha particolarmente colpito: \textit{nascondere la complessità e ricavare l'informazione dai dati (grezzi)}. Da questa idea si è stimolato il nostro interesse di poter approfondire queste pratiche di sviluppo per la gestione e interpretazione dei dati, tali per cui l'uso di un sistema automatizzato e ampiamente connesso permetterebbe di semplificare di gran lunga il lavoro di reperimento dell'informazione. \\
Chiaramente, la nostra sfida si baserà sull'apprendere il più possibile queste nuove tecnologie (come \glock{Kafka}, \glock{Docker} e i \glock{timeseries DB}) così da realizzare un prodotto valido che soddisfi le aspettative e i requisiti richiesti dal capitolato, nonchè presentare tutta la documentazione aggiuntiva richiesta dal proponente prima e dopo lo sviluppo del software. \\
Concludendo, con la scelta questo capitolato si vuole ampliare il proprio bagaglio di conoscenza in questa materia, facendo luce sui modi di integrare le principali funzionalità del sistema, usando tecnologie moderne e di grande interesse per tutto il gruppo, senza trascurare gli aspetti formali del progetto.
