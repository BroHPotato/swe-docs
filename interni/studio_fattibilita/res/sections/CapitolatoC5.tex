\subsection{Capitolato 5 - Stalker}

	\subsubsection{Informazioni generali}
		\begin{itemize}
			\item \textbf{Nome:} Stalker;
			\item \textbf{Proponente:} Imola Informatica;
			\item \textbf{Committente: }Prof. Tullio Vardanega e Prof. Riccardo Cardin.
		\end{itemize}

	\subsubsection{Descrizione capitolato}
		Il proponente chiede la realizzazione di una \glock{mobile-application} al fine di poter tracciare, in forma anonima e non, il numero esatto di persone presenti all'interno di uno spazio fisico identificato da un insieme di coordinate geografiche.

	\subsubsection{Finalità del progetto}
		L’obiettivo è sviluppare un’applicazione in grado di segnalare, ad un server dedicato, l’ingresso e l’uscita dell’utilizzatore dalle aree d’interesse (basandosi sulla posizione attuale dell'utilizzatore del dispositivo) in due modalità, autenticata o anonima, a seconda delle esigenze.
		L’applicazione deve permettere le seguenti operazioni:
		\begin{itemize}
			\item recupero lista organizzazioni (refresh manuale e/o temporizzato);
			\item \glock{login} nell’organizzazione con eventuale autenticazione;
			\item storico degli accessi;
			\item visualizzazione in tempo reale della propria presenza o meno all’interno di un luogo monitorato e cronometro del tempo trascorso al suo interno;
			\item predisposizione di un pulsante ``anonimo'' che permetta di risultare presente in maniera anonima all'interno dell'organizzazione.
		\end{itemize}
		Le comunicazioni tra applicazione cellulare e server dovranno avvenire solo nel momento d'ingresso ed uscita dai luoghi designati. Il rilevamento della posizione può essere effettuato in due modi distinti:
		\begin{itemize}
			\item \textbf{\glock{dead recoking}:} dato un punto di partenza, la velocità, la direzione del movimento, il tempo trascorso e la distanza percorsa si può comprendere il punto di arrivo;
			\item \textbf{\glock{proximity sensing}:} la posizione del punto mobile è ricavata dalle coordinate di determinate stazioni che tracciano il segnale che viene trasmesso da esse (\glock{cell ID}). Ogni stazione ha un suo \glock{pattern} di segnale.
		\end{itemize}
		In genere con il \glock{GPS} si usa la trilaterazione, prendendo la posizione nota di due o più punti di riferimento e la distanza misurata tra il punto mobile ciascun punto di riferimento. La triangolazione permette di calcolare la posizione sulla base di angoli di arrivo (AOA) tra punto mobile e punti di riferimento e la distanza stessa tra i punti di riferimento. Ovviamente una precisione perfetta é difficile da raggiungere, perciò l'obbiettivo é un'approssimazione abbastanza precisa e dimostrabile della posizione. Oltre all'ottimizzazione del rilevamento viene richiesto di limitare il consumo energetico che il sistema può utilizzare in modo da estendere la durata di eventuali batterie presenti nel sistema.

	\subsubsection{Tecnologie interessate}
		Per lo sviluppo del server \glock{back end} sono consigliate le seguenti tecnologie:
		\begin{itemize}
			\item \textbf{\glock{Java}:} linguaggio di programmazione ad alto livello orientato agli oggetti e a tipizzazione statica;
			\item \textbf{\glock{Python}:} linguaggio di programmazione ad alto livello orientato agli oggetti adatto, tra gli altri usi, a sviluppare applicazioni distribuite, \glock{scripting}, computazione numerica e \glock{system testing};
			\item \textbf{\glock{Node.JS}:} è una \glock{runtime} di \glock{JavaScript} \glock{open source} multipiattaforma orientato agli eventi per l'esecuzione di codice \glock{JavaScript};
			\item \textbf{\glock{Continous Integration}:} continua disponibilità del codice prodotto tra gli sviluppatori;
			\item \textbf{\glock{Continous Delivery}:} gestione degli artefatti creati in produzione con relativi versionamenti;
			\item \textbf{\glock{Continous Testing}:} il codice sviluppato viene testato, tramite test unitari, prima di essere rilasciato, per fornire una base solida per lo sviluppo da parte di altri sviluppatori interni, e consente di capire cosa é funzionante e cosa no. In generale si vuole una copertura del 85-90\%.
		\end{itemize}
		Saranno inoltre necessari:
		\begin{itemize}
			\item protocolli asincroni per le comunicazioni \glock{app mobile-server};
			\item \textbf{\glock{pattern di Publisher/Subscriber}:} in questo \glock{pattern}, mittenti e destinatari di messaggi dialogano attraverso un tramite, che può essere detto \glock{dispatcher} o \glock{broker}. Il mittente di un messaggio (detto \glock{publisher}) non deve essere consapevole dell'identità dei destinatari (detti \glock{subscriber}); esso si limita a "pubblicare" il proprio messaggio al \glock{dispatcher}. I destinatari si rivolgono a loro volta al \glock{dispatcher} "abbonandosi" alla ricezione di messaggi. Il \glock{dispatcher} quindi inoltra ogni messaggio inviato da un \glock{publisher} a tutti i \glock{subscriber} interessati a quel messaggio;
			\item Utilizzo dell’\glock{IAAS} \glock{Kubernetes} o di un \glock{PAAS}, \glock{Openshift} o \glock{Rancher}, per il rilascio delle componenti del server nonché per la gestione della scalabilità orizzontale.
		\end{itemize}

	\subsubsection{Vincoli del progetto}
		Non sono stati espressi vincoli di sviluppo, ma è stata espressa l'esigenza di avere un rilevamento di posizione precisa e testata.

	\subsubsection{Aspetti positivi}
		\begin{itemize}
			\item Possibilità di ampliare il bagaglio di tecnologie conosciute per lo sviluppo di applicazioni mobile;
			\item il proponente non impone tecnologie specifiche per lo sviluppo del server o
			della \glock{UI}.
		\end{itemize}

	\subsubsection{Criticità}
		\begin{itemize}
			\item Progetto lungo da sviluppare;
			\item documentazione GPS complicata da interpretare;
			\item difficile determinare con massima precisione la posizione dell'utilizzatore.
		\end{itemize}

	\subsubsection{Conclusioni}
		Il capitolato ha stimolato l'interesse del gruppo visto le tipologie di tecnologie che verranno utilizzate e per l'applicazione che l'azienda intende farne. Il proponente chiede di fare diversi test per ottenere una stima il più precisa possibile sull'utilizzatore, dimostrando i risultati ottenuti. Dopo un'accurata riflessione abbiamo constatato che la dimostrazione dei test poteva essere molto dispendiosa a livello di tempistiche. Nonostante l'interesse il gruppo ha deciso, dopo un conteggio delle preferenze, di non scegliere questo capitolato.
