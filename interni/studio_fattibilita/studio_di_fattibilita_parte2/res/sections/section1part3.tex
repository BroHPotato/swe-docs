\subsection{Capitolato 6 - ThiReMa}

	\subsubsection{Informazioni generali}
		\begin{itemize}
			\item \textbf{Nome:} ThiReMa - Things Relationship Management
			\item \textbf{Proponente:} Sanmarco Informatica
			\item \textbf{Link:} https://www.math.unipd.it/~tullio/IS-1/2019/Progetto/C6.pdf
		\end{itemize}

	\subsubsection{Descrizione capitolato}
		Sviluppo di un software che,  dopo aver ricevuto misurazioni da sensori eterogenei, li accumola in un database centralizzato. Questa applicazione viene poi completata da un servizio di dispatching per inoltrare in modo tempestivo le informazioni utili per gestire le azioni urgenti.
		I dati messi a disposizione dal database centralizzato dovranno essere suddivisi in due macro-categorie: dati operativi e fattori influenzanti.

	\subsubsection{Finalità del progetto}
		Creare una web-application, che permetta di valutare la correlazione tra dati operativi (misure) e i fattori influenzanti. Tale applicazione si potrà focalizzare nella definizione di uno o più algoritmi per la successiva analisi dei dati al fine di essere in grado di effettuare delle previsioni sull’andamento dei dati stessi ed offrire, ad esempio, dei servizi manutenzione predittiva.
		Per ogni tipologia di informazioni rilevate dovrà anche essere	possibile assegnare il monitoraggio ad un particolare ente. 
		Analizzando un determinato sensore, in base ai dati ricevuti, si può prevedere un deterioramento complessivo tale da generare una necessaria azione di manutenzione preventiva.
		La web-application dovrà essere suddivisa in 3 macro-sezioni:
			\begin{itemize}
				\item Censimento dei sensori e dei relativi dati.
				\item Modulo di analisi di correlazione.
				\item Modulo di monitoraggio per ente.
			\end{itemize}
		
	\subsubsection{Tecnologie interessate}
		\begin{itemize}
			\item \textbf{Apache Kafka:} Piattaforma open source di stream processing scritta in Java sviluppata da Apache Software Foundation. Il progetto mira a creare una piattaforma a bassa latenza ed alta velocità per la gestione di feed dati in tempo reale.
			\item \textbf{Java:} Linguaggio di programmazione ad alto livello orientato agli oggetti e a tipizzazione statica.
			\item \textbf{Bootstrap:} Raccolta di strumenti open source per la creazione di siti e applicazioni per il Web. Essa contiene modelli di progettazione basati su HTML e CSS, sia per la tipografia, che per le varie componenti dell'interfaccia, come moduli, pulsanti e navigazione, così come alcune estensioni opzionali di JavaScript.
		\end{itemize}

	\subsubsection{Aspetti positivi:}
		\begin{itemize}
			\item Tecnologie già in parte conosciute dal gruppo con la possibilità di ampliarne le conoscenze.
			\item Consistente set di dati su cui testare l'applicativo.
		\end{itemize}
	
	\subsubsection{Criticità:}
		\begin{itemize}
			\item Protocolli proprietari, la documentazione su di essi potrebbe essere limitata.
		\end{itemize}
		
	\subsubsection{Conclusioni:}
		Il capitolato ha suscitato l'interesse del gruppo, dando la possibilità di ampliare tecnlogie gia in parte conosciute. C'è stato inoltre molto entusiasmo per la tipologia di web-application da sviluppare.

